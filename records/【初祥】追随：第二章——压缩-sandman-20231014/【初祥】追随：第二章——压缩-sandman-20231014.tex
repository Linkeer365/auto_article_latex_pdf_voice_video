\documentclass{article}
\usepackage[utf8]{inputenc}
\usepackage{ctex}
\usepackage{url}

% support Chinese Chars %
\newfontfamily\urlfontfamily{FandolSong-Regular}
\def\UrlFont{\urlfontfamily}

\title{【初祥】追随:第二章——压缩}
\author{sandman}
\date{20231014}

\maketitle
\url{https://www.pixiv.net/novel/show.php?id=20843000}
\newpage

\begin{document}
\CJKfamily{zhkai}

% : comm which is only poems edit in-use %


\Large

她看了看表,距离这一次排练结束还有一个小时的时间。丰川祥子皱起眉头,举起手:“各位辛苦了,我们先休息五分钟。过一会儿再来。睦,干得很好。若麦,海玲,你们的表现实在是令我惊喜。初华你今天状态不太对,别着急,慢慢来。今天失误主要是我的问题,耽误了各位的时间十分抱歉。我会努力调整一下的。”说完话,她转过身,无视了睦关切的目光和初华探询的问候,一个人走向盥洗室。



她走到水池前,看着镜中的自己。镜中人的脸苍白得像个幽灵,暗红色裙子下的后背早已被冷汗覆盖。她咬住牙齿,深呼吸了几下,头变得更晕了,腹中仿佛有一条不安分的泥鳅在咬。还不是时候!她拧开水龙头,用力洗脸。你还不能倒下。只是晚了几天而已!她用力拍打自己的脸颊,让自己睁开双眼。视野周围好像有一圈黑色的框,就连站都站不稳,要是有个温暖的被窝——不能软弱!她用双手按住小腹,努力想要遏制住那条狂暴的泥鳅。排练还有一个小时,场地费已经交了,不能在这种时候停下——



“——小祥!小祥!”有人在摇晃她的肩膀,她转过头,熟悉的紫色双眸正关切地望着自己,耳朵里有铜铃在响。初华?她什么时候进来的?被她注意到了吗?“你哪里不舒服?”



“没什么,大概是早上吃冰激凌吃坏肚子了。”她熟练地挤出一个微笑,初华……自己已经欠了她太多了,不能再让她担心,“问题不大,一会儿就好。”



“小祥……”初华别过头,似乎想到了什么,“那个……抱歉,虽然现在时间快到了,但我其实还有点累,可不可以再让我休息五分钟?”



“好啊,没问题。”祥子点点头,“你先出去吧,这里空气不好。不用担心,我马上就出来了。”



“……好吧,我去和海玲她们说一声。”初华离开了,很好,看来她还没有注意到。祥子靠住墙,无声地滑下,坐在地上。双腿已经彻底失去了力气,真不是时候。



“该死的……为什么做女人要受这种罪……”



咚咚咚。



祥子抬起头,有人在敲门。一个她熟悉的没有波动的声音响起了:“祥,你很痛苦。”这不是个问题,睦只是在陈述事实。



“不当紧,马上就好。”她咬住嘴唇,扶住墙,站起身。



“……你应该休息的。你的状态不好。”



“不需要你管!”该死的,为什么睦总是得这样说话?她一瘸一拐地向外走去,“我已经好了——”



然后,她的意识沉入了黑暗。



\newpage





{\centering\section*{第二章——压缩}}





“检查结果出来了,”医生看着手中的报告单,由于他戴着口罩,初华分辨不出他的表情,“营养不良,再加上压力过大和疲劳导致的月经失调,最主要的问题是由于长期营养不良导致的低钾血症。她在节食吗?你们和她是什么关系?她的父母呢?”



鬼知道。初华把几乎脱口而出的回答咽了回去。海玲开口了:“乐队里的朋友罢了。”



“是吗?”医生淡漠地扫视了海玲一眼,“总之你们最好尽快找到她的父母,或者有谁知道她的医保在哪里也行。我是负责救人的,但是收账单的可不会管这些。祝你们好运。十点的时候护士会来查房,所以你们最多可以待到九点,到时候会有人叫你们出去的。”



初华走出医院,睦一言不发地跟在她身边,若麦早就借口有事不知所踪,海玲也在初华的劝说下回家了。初华转过头,看向身边五官如人偶般精致的少女。睦长长的白色睫毛微微颤抖着,仿佛在诉说着主人无法出口的心事。初华叹了口气,她还记得睦在病房里坐在祥子床前,一言不发地握着对方的手的样子。睦的眼睛一眨不眨地盯着病床上昏迷不醒的祥子,仿佛要把那身影烙印在自己视野里一样。现在初华看清楚了,祥子苍白的手臂上青筋纵横,额头上依然覆盖着密密麻麻的汗珠。虽然处于昏迷之中,但她的表情因为痛苦而扭曲了。初华叹了口气,祥子一定以为自己掩饰的很好。她是个心高气傲的人,很难接受别人的同情或者怜悯。哪怕是自己也不行。可是睦……



她摇了摇头,把这些阴暗的想法甩出脑海:“小睦,这么晚了,你家里会担心的吧?我把你送回去吧。”



“不用,会有人来接。”睦摇摇头,“初华,去过祥家了吗?”



初华微微一愣,在她和睦那场并不能说友好的交谈之后,两人都心照不宣地对当天的谈话内容避而不谈。她并没有预料到睦会主动提起此事:“……还没有,我想还不是时候。”



睦先是沉默了一小会儿,接着再次开口了:“你应该去一次。”



睦转身离开了,留下她孤身一人走在车水马龙的大街上。初华抬起头,路灯的橙黄,液晶屏的湛蓝,霓虹灯的品红充满了她的视野。一轮满月正透过喧闹的光火,把苍白的光洒在她身上。今日的夜空依然一片漆黑。今日的群星依然对她隐去自己的身影。



她做了个梦,梦中她回到了那个她出身的小岛。她站在那座古朴典雅的庭院门口,向内张望。庭院鲜少有人造访,自从她记事以来,庭院的主人第一次在岛上现身。头戴阳帽,手捧人偶的蓝衣女孩正神色淡然的向着屋内走去。女孩的琥珀色眼睛被埋没在帽子的阴影里,全无笑意。夏日的艳阳透过树荫的缝隙,在她们身上投下无数细小的光斑。微风吹拂,金黄色的蝌蚪在她们四周游走。是了,这是她的少女时代,她第一次遇到小祥的那一天。



三角初华是个不服管教的野孩子,她喜欢青草的芬芳,夏蝉的啼鸣,大海的壮阔,但是不知为何,明明身处人群之中,明明与他人一同欢笑,可是内心里却有一个洞,让她在半夜辗转反侧。她的父母只是以为她吃坏了肚子。于是她面对他们的问候露出开朗的微笑:没事的,我很好。但是胸中的空洞却总是越来越大。有什么话语想要诉说却无人倾听。有什么感情想要涌出却不知其名。有什么事情想要做到却不得要领。终于在她八岁那年,在日语课上,她学到了,这种感觉就是孤独。这就是荒唐的真相——她,三角初华,岛上胆子最大,人缘最好的女孩,居然是孤独的。她有那么多朋友,亲人,师长,却无一人可以派遣缓解她的孤独。这种事情是错误的,这种事情不可能正确,于是初华继续戴着阳光开朗的微笑,浪迹于山野之上,同龄人之中,装作自己的内心其实完好无损。



不知为何,冥冥之中她有了一种错觉,眼前的女孩和自己是一样的。



三角初华并不是个主动的人。她知道如何在人群中隐藏自己的不合群。她知道如何顺着他人的意见让自己显得无害而可亲。她知道如何用微笑的假面来博取他人的欢心。她从来都是站在人群外围,听着他们的意见,随声附和他们的决定,并为他们出谋划策,效犬马之劳。她知道自己的异常,她知道自己的不合理,她知道自己的错误。但是她还是再一次主动伸出了手:“你好啊,你是外地人吗?你叫什么名字?”



蓝发女孩看着她,先是露出困惑的眼神,接着转过头:“爸爸,有个奇怪的小孩和我说话!”



事后她从女孩的父亲口中得知了女孩的名字是丰川祥子。他们是来岛上避暑的有钱人家,那所庭院就是他们家的。虽然是有钱人,但是祥子的父亲似乎和初华想象中的那种铺张浪费一掷千金的花花公子有所不同。他友好地把初华迎进家里,接着告诉祥子不能这样随便叫别人“奇怪的小孩”,哪怕真的很奇怪也不行。初华听了这话只好厚颜无耻地哈哈一笑,随后看到客厅里的钢琴顿时两眼放光。在此之前她最多只在电视上见过这“乐器之王”,实际见到还是第一次。祥子看着她的眼神似乎若有所思,没等她父亲发话,自己便跑到钢琴前,弹了起来。原来祥子的父亲是个玩爵士乐的钢琴家,小祥子也弹得一手好钢琴。祥子的脸上面无表情,手指却在键盘上上下翻飞。激昂,骄傲,愤怒。虽然眼前的女孩没有说话,不知为何,初华却仿佛能够从钢琴声中感知到对方的炽热的内心。一曲终了,祥子跳下钢琴凳,依旧面无表情,垂下眼睑,鞠了一躬。初华按照往常父母的教导连连鼓掌,却不知为何感到自己的脸颊上有泪水留下。祥子默默地走到她身边,掏出手绢帮她擦干。



那天下午,初华带着祥子在岛上随意撒欢。爬树,挖蚯蚓,抓独角仙,采蘑菇,唱歌。伟大的小岛女王,不可一世的三角初华陛下使出浑身解数,带着她新抢来的花姑娘好生逍遥,讨对方的欢心——这是初华脑海里打出的广告语。十分不幸的是她直接把这话在爬树的时候说出了口,因此不得不对着满脸好奇的祥子解释花姑娘和逍遥的意思。到了晚上,两个人都精疲力竭,气喘吁吁。祥子的父亲便把她们带回家,给她们吃自己亲手做的大米布丁。吃完了以后两个人带了块野餐布,找到一片开阔的山丘躺下,看着头顶的满天繁星。银河从天边倾泻而下,无声地落入海中。初华看着星空,不知为何一阵旋律在她的脑中响起,她不由自主地低声唱了起来:



Roter Mond überm Silbersee

Feuerglut wärmt den kalten Tee 

Kiefernwald in der Nacht

Und noch ist der neue Tag nicht erwacht

Kiefernwald in der Nacht

Und noch ist der neue Tag nicht erwacht 

Sterne stehn hoch am Firmament

Solche Nacht findet nie ein End 

Dieses Land wild und schön

Und wir dürfen seine Herrlichkeit sehn 

Dieses Land wild und schön

Und wir dürfen seine Herrlichkeit sehn 

Grauer Fels Moos und Heidekraut 

Weit entfernt schon der Morgen graut 

Fahne weht wei und blau

Das Gras schimmert unterm Morgentau

Fahne weht wei und blau

Das Gras schimmert unterm Morgentau



“星空,很美呢。”祥子低声说道。“在东京的时候很少能看到这么多星星。”



“那是因为光污染吧。”



“光污染?”



“就是地上的光太亮,导致星星的光被盖住。就像拍照的时候如果闪光灯开得太大,照片边缘就会不自然地亮起来。”初华吸了口气,“小祥,你知道吗?天上的星星虽然看起来彼此之间很近,但是其实相距甚远呢。就连离我们最近的比邻星,也和我们有着四光年以上的距离,天上的那片银河中的许多星星距离我们有着几万光年以上的距离,就连我们可以看到的其他星星大多数也是在这个银河系之中。所谓光年,就是光走一年的距离。光的速度是每秒30万千米,比新干线还要快四百万倍以上。虽然群星在我们眼中仿佛如此与彼此相近,但其实它们之间所隔绝的距离足够让任何人类都无法在其间航行。我们只是这个银河系当中的小小居民,距离我们最近的其他星系——仙女座大星云——和我们有着230万光年的天堑。我们看到的这片星空也只是过去的残像,其中的许多星星,大概早已经死了吧。”



“初华你……很喜欢星星吗?”



“说不上吧。”初华别过头,“我不知道。我觉得,过去我是喜欢的,但现在……”



“现在怎么了?”



“那和我的生活太遥远了。”好像有某种阴暗的浊流在她喉中流淌,她的心脏不安地在胸中抽搐,仿佛要从她嘴里跳出来一样,“就算再怎么看星星……我也只能生活在这个岛上,也只有我一个人去看。”她别过头,注意到祥子担忧的眼神,赶紧让自己露出笑脸,重新看向天空,“不说这些奇怪的话了,小祥你看,那颗星星就是织女星,它是全天第二亮的恒星,然后那个是牛郎星,再加上天鹅座十字顶上的天津四,这个就是所谓的夏季大三角。”



“不对哦,”初华回过头,海风捎来了野花的芳香,夏蝉的叫声汇成一曲宏大的交响。她突然意识到祥子的脸离自己很近,近到她可以感受到对方打在自己脸上的呼吸,“我和初华是一样的。我们都是一个人。”



三角初华是个不服管教的野孩子,她喜欢青草的芬芳,夏蝉的啼鸣,大海的壮阔,但是她最喜欢的,就是头顶星河的浩瀚无垠。当她抬起头,望向万里无云的夜空,被星辉环绕,她感到自己仿佛也离开了她土生土长的这小小天地,置身于无尽的宇宙之中。岛上的人没有一人可以理解她的这一喜好,就连父母也只能露出无奈的苦笑。那样的话不是很孤独吗?他们这样疑惑地问道,对此她也不知道该怎么回答。人类的世界太过吵闹,太过无趣。明明周围的一切都生机盎然,她却总是觉得自己的生活中充斥着一股腐败的恶臭,让她止不住地心烦。只有在天上,在群星之中,她才可以找到解脱。虽然孤独,但她却终于听到自己的心声——在这里,我可以拥有自由。而现在,眼前的女孩正在对她说着无法理解的话语。三角初华别过头,再次看向星空。她突然意识到自己其实已经很久没有看过星星了。虽然喜欢星星,虽然喜欢天文,但是不知何时起,自己就低下了头。是在恐惧吗?还是在愧疚呢?还是单纯看到不属于自己的耀眼而感到痛苦呢?



“是吗?”她喃喃自语道,“小祥你也和我一样吗?说实话,在岛上我也有不少朋友,但是我从来没有想和他们一起看星星。每当我看星星的时候我都觉得自己好渺小,好微不足道……我不想让他们知道我的这一面,因为他们不会喜欢这样的我。可是即便如此……”



“是啊,”祥子握住她的手,一股暖流从对方的手上流入了她的身体,“即便如此,你也在追随着星辰呢。初华你明明这么小却知道这么多,是很了不起的人。就算别人不喜欢这样的你,我也会喜欢初华。”祥子转过头,再次看向群星,“看着这片星空,真的感觉自己很渺小……‘生有七尺之形,死唯一棺之土,唯立德扬名,可以不朽’。虽然曹丕这话说的是人,但是对星星也适用吧。明明已经死去的星星发出的光创造出了它们活着的假象。真好啊……”她叹了口气,“初华,再过两天我就要回东京了。”



“啊……”她当然是知道的,祥子不可能永远待在这里,但是——“你以后还会再来这里吗?我还可以再见到你吗?”不对,她要说的不是这些,“如果我去了东京可以找到你吗?”



“当然了,初华你唱歌很好听吧,说不定你能成为歌手呢。那样如果你到了东京我们就可以一直在一起了,在那之前我会等着你的。”



“那就……一言为定!”



“一言为定!”祥子与她相视而笑。两人再次看向星空,初华感觉某种前所未有的情感正在胸中雀跃,她却不知其名。明明和往日一样,看着星空,她却没有感到那种令她安心的孤独感。她看向身边,发现祥子也在看着她。



“初华?”



“怎么了,小祥?”



“别忘了,就算我们不在一起,每个夜晚,我们都望着同一片星空。”哪怕我们都是一个人,但我们的心也会一直和彼此一道。



自那天起,她就追随着对方,如同海员追随着北极星一般。她努力地锻炼歌喉,精进学业,终于成功来到了东京。她忘不了那个和善,温柔,但是却和自己一样孤独的女孩。她忘不了在那片星空下对方鼓励的话语。她忘不了与对方双目相对时,胸中那种莫名的悸动——仿佛只要孤独的自己和孤独的她在一起,就会不那么孤独。然后,她抵达了此刻。



电话响了,她从回忆中抬起头,打开手机:“喂?这里是三角。你说什么?她从医院里失踪了?”



一辆列车从她身边的高架桥上飞驰而过,哐当作响,在三角初华看来,这是她世界粉碎的声音。



\newpage



她首先联络了海玲。根据监控显示,祥子自己在病房里换了身衣服,然后堂而皇之地从住院部的大门走了出去,中间没有受到任何阻拦。医院里人来人往,大概护士们早就习惯了病人们这样随意地住院和出院。门口的护士站的值班护士看到祥子的时候也只是向她问了声好。因为提前垫付了住院费,所以也没有医生找她的麻烦。然后丰川祥子走出医院大门,混入人群,不知所踪。她没有联络经纪人或者乐队里的其他人,也没有认识她的人再见过她。丰川祥子就这样简单地走过那扇门,然后在这座巨大的城市中人间蒸发。她简单地和海玲交流了一下情报,接着请求对方帮自己寻找祥子。海玲似乎在她提出之前就已经预判到了她的请求,于是痛快地答应了。随后她看着手机上的另一个号码,陷入了犹豫。她知道对方对于祥子有着十分特殊的意义,但是她也知道祥子有意向对方隐瞒了自己的处境。要联络那个人吗?她咬住牙齿。



电话又一次响了起来,她看了眼来电显示,是若叶睦,她接通了:“小睦?你在哪?小祥从医院失踪了,你有见到她或者收到她的信息吗?”



“没有。”睦的声音似乎有些颤抖,“我要去祥的家里。”



“你觉得她会回家吗?”



“不,”睦的声音提高了,“但是我要去那里结束这一切。”



她沉默了,过了一会儿,她再次开口了:“小睦,你真的知道你在说什么吗……你真的明白,如果你做了那件事,你会怎么样吗?”



“……我明白,但是这是唯一的选择了。”睦的声音平静了下来,但是初华却莫名地从对方的声音里听到了一种悲哀,“因为祥,已经坏掉了。”



\newpage



丰川祥子漫无目的地走在大街上,周围的行人似乎注意到了她魂不守舍的样子,自然而然地给她让路。已经不是再去考虑别人的眼光的场合了。毕竟死人没有体面可言。从医院里逃出来了,接下来做什么呢?她不知道。回家吗?回到那个腐烂,阴暗,肮脏的巢穴,和那个形同行尸走肉的男人一起——那样的地方也可以被称作家吗?上学吗?她的脑子里一片空白,这种状况又怎么能学得进任何东西呢?去排练吗?两只手依然软绵绵的使不上力气,这样的她去了也只是亵渎音乐。已经完蛋了,失败了!她们会知道的,她们已经知道了!她们知道多少?失败了失败了失败了失败了失败了失败了失败了失败了失败了失败了失败了失败了失败了失败了失败了失败了失败了失败了失败了失败了失败了——



她摸了摸自己的脸,她意识到了,自己并没有在哭,反而在笑。真是可笑,面对这样的结局她居然还笑得出来。



“果然,到头来也没有成为人类啊。”



她抬起头,她突然意识到自己眼前的景色有些熟悉,樱花树枯黄的叶片在秋风中摇摆着,桥下的轨道在夕阳的余晖下闪烁着金色的光芒。是了,这就是她和高松灯第一次相遇的那座铁路桥。灯……她咬住嘴唇。她怎么会忘掉那个不善言谈,但是又可以写出打动人心的话语的人?她怎么会忘记她那骄傲绽放的重要之人?她怎么会忘掉因为自己的傲慢而注定无法再会的知己?她是知道的,在和她分开之后,灯已经有了新的归宿。她看着灯和mygo的众人一点点磨合,一点点成长。她也听说了灯挽回素世的那场传奇性的演出。灯已经变得不一样了。她变得更勇敢,更有行动力,更体贴他人。不要留恋!你只会让她伤心!灯的未来没有自己,这样也好,忘掉她,这样的话灯就会获得幸福。



自分はここにいない 居場所がないつて



自己不属于此没有容身之所



言い闇かせてるだけ



只是这样告诉自己



ここではないどこかに行きたい



想要前往这里之外的某个地方



思い込んで慰めてる



说服自己寻求着慰藉



她牵起裙子,鞠了一躬。舞会开始之前礼节是必要的。她无视了周围路人诧异的目光,伴随着心中响起的旋律,舞动起来。跳跃着,奔跑着,如同逐火之蛾一般。好久没有跳舞了呢。Ave Musica,赞美音乐;Ave Mujica,赞美身处狂欢节的一丘之貉们。这首歌是是她为灯创作的第一首歌。她其实后来知道了,那并不是歌词,只是灯意识流一般涂画在本子上的呓语。但是看到那些话语的一瞬间,旋律自然而然地在她脑海中显现出来。没有归宿,无法融入人群,无法成为人类,故而想要逃脱,在那时,她在歌词里看到的,其实是自己吧。





見つけたくつて下ばかり見ていたから



因为想要寻到 一直低头看往脚下



気づかない



并未发现



つまずいてぶつかつて  よらめく



不断跌倒不断受伤步履蹒跚



世界が揺れる



世界摇摇晃晃



靴ひもがまたほどけた



鞋带又松开了啊



睦,她的青梅竹马,一直跟随在她身边的半身。在离开睦之前,她就一直担心着对方软弱的性格也许会被人利用。那孩子也是,变得坚强起来了。之前只是沉默地跟在自己身边,现在主动加入Ave Mujica也是在担心自己吧。一直以来都默默地支持着自己,可是自己却一直都在用那样的态度伤害她,迁怒她。就算想要和好也……她转了个圈,双手举过头顶。她调整呼吸,让自己的身体和脑中的音乐同步。转圈的时候要先转动身体再迅速扭头,这样才不会眩晕。真的是无可救药的蠢货啊,明明都这样被自己对待了,却还是不肯放弃自己。她这种人,怎么配得上——





みんなみたいに友達できたけど



明明像大家一样交到了朋友



みんなといるのに独りみたいな



明明和大家都在一起却还如同孤身一人



みんなみたいに生きたいのに



明明好想像大家那样活着



初华。她闭上眼睛,踮起脚尖,旋转起来。音乐加快了,她必须化为暴雨。注意脚步,不要慌乱!初华……



她还记得,在那个岛上,那个好奇的,热情的,开朗的,忧郁的,感伤的金发女孩和自己的初遇。孤独的弹着钢琴的自己遇到了孤独的仰望星空的她。于是她们可以一起孤独。她还记得那天,自己失魂落魄地从Ring逃出来,抬起头,在大屏幕上看到初华那双美丽的紫色眼睛的感情。那是她的憧憬,她的挚爱,她想要拥有的一切。她看着与自己约定再次相遇的友人顺利出道,光鲜亮丽的姿态。初华的身边有着一个她不认识的女孩,两人似乎情投意合,默契无间。可初华的笑容一如既往地令她心脏躁动不安。她应该为自己友人的事业感到认可的,她应该发自内心地为对方献上祝福的。可是某种阴暗的感情占据了她的心,她眯起眼睛,努力地想要按压下那侵凌的黑,但她意识到了。



那是嫉妒。



太阳距离我们8分20秒,比邻星距离我们4年,织女星则是25年,天津四是1670年,仙女座大星云则来自230万年的久远过去。群星来自过去的光火照亮了夜空,让它们虽死犹生,和自己这种虽生犹死的行尸走肉完全不同。自己到头来也是那个男人的女儿,在这一点上无可救药的一致。为了那点可怜的自尊,为了她那荒唐的理想,她居然开始憎恨初华,哪怕对方从未伤害过自己。于是她厚颜无耻地利用了初华。她知道初华喜欢自己,知道只要自己一声令下,就可以让初华为她献出一切。哪怕初华早已和其他人一起摆脱了孤独,不再需要自己,但她依然可以依靠那虚伪的残像轻而易举地毁掉初华的人生。于是她找到了初华,提出利用对方的事务所的资源和睦的名气,再加上自己的运作,组成所谓假面乐队的Ave Mujica。但那说是乐队,倒不如说是在对于丰川家的报复心操作之下绝望的狂想。就凭一个金属乐队,就凭她这个肉体凡胎,怎么可能玩得过那些耕耘其中几十年的老人?身体的崩坏也只是自作自受的报应而已。到头来她也不过是和长崎素世一样,自我中心虚与委蛇,为了自己的私欲玩弄他人的人生,最后沦落至此的渣滓。就算如此堕落,就算如此抛弃尊严,就算如此地作践自己也无法达成目的,真是失败的人生。到头来不过是她一个人在自我感动罢了。



如果飞行是一种坠落的话,坠落是否是一种飞行呢?她向着前方伸出手——



\newpage



人間になりたい



“啊,好想成为人类啊。”

\end{document}
