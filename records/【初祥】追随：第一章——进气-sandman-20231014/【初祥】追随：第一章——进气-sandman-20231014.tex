\documentclass{article}
\usepackage[utf8]{inputenc}
\usepackage{ctex}
\usepackage{url}

% support Chinese Chars %
\newfontfamily\urlfontfamily{FandolSong-Regular}
\def\UrlFont{\urlfontfamily}

\title{【初祥】追随:第一章——进气}
\author{sandman}
\date{2023-10-14}

\maketitle
\url{https://www.pixiv.net/novel/show.php?id=20842990}
\newpage

\begin{document}
\CJKfamily{zhkai}

% : comm which is only poems edit in-use %


\Large

“所以说,现在怎么办?”三角初华转过头,乐队的贝斯手八幡海玲两手抱胸立在门口,面无表情。在她的身后,祐天寺若麦用手遮住了口鼻,不知是在阻挡气味还是在掩盖她脸上若有若无的笑意。初华站起身,若叶睦站在她的身旁一动不动,一如既往地捉摸不透。在她们四个人面前,狭小逼仄的起居室里散落着空酒瓶和易拉罐,电视机的屏幕一闪一闪的,偶尔会照亮男人苍白的脸庞。男人的头不自然地歪到一边,右边的额角不自然地肿了起来。他胸口的衣服被口中溢出的白沫和呕吐物覆盖,散发出一股酸味。一只顽强的苍蝇不知何时飞进了屋子,盘旋了几圈之后停留在了男人涣散的瞳孔上,爬行着,探求着。初华转过身,目光扫过她的队友们,最后停留在了门口的储物柜上。一个穿着红色洋装的人偶坐在那里静静地看着她们,无神的眼睛里似乎有微光在闪烁。太阳已经沉入了地底,晚秋的夜空中浓云密布。远方依稀可以听见火车在高架桥上的低吟声,在三角初华听来,那声音却仿佛雷鸣般震耳欲聋。





{\centering\section*{第一章——进气}}





她曾经不止一次感到过这种莫名的疏离感。身处这座巨大、冰冷、陌生的钢铁都市之中,看着摩天大楼上的巨大液晶屏在黑夜中闪现出异样的光彩,被忙碌的车流嘶鸣着吐出污浊的废气包裹,她被两种截然相反的冲动所支配——一种让她想要沉溺于其中,另一种则仿佛本能般提示着她其中的危险。她想起在过去,她和双亲在一座偏僻的海岛上居住,那时的她尚是个孩童。她曾经在一次潜水时沉迷于五彩斑斓的珊瑚,而险些葬身于水母之手。这座城市就是那片珊瑚——妖异,诱人,而又带有致命的危险,仿佛随时都会把她一口吞下。



三角初华讨厌东京。



\newpage



她掏出钥匙,打开防盗门,走进屋子,把帽子挂在门口的挂钩上。她把墨镜从头上摘下,随手扔在门口的高柜上。接着掏出手机,打开选歌界面,轻轻点击了几下。不一会儿,科特·柯本低沉的嗓音便开始从她的耳机中传来:



Underneath the bridge, Tarp has sprung a leak



And the animals I've trapped, Have all become my pets.



她没有开灯,作为观星者,她的双眼早已适应了黑暗。借着窗外的纸醉金迷的余光,她一个人驾轻就熟地穿过昏暗的房间,走进厨房,从冰箱里取出昨天剩下的土豆烧肉,放在炉子上,打开燃气灶。看着青色的火焰在黑暗中舞动着,舔舐着锅底,她感到莫名的安心。这是一件五六十平米的单人公寓,对一个人来说这点面积算不上奢侈,但足以让人觉得空旷。一个厨房,一间卧室,一套单人桌椅摆在起居室面对着窗户,既是餐桌也是工作台。窗户下面窗框向内延伸,上面铺了些毯子权作是长椅,长椅上面扣着一本书。她走近窗户,捡起那书,书依然停留在她上周翻开的那一页,是《查拉图斯特拉如是说》的“夜之歌”:



一种饥饿发生于我的美里。我想伤害我照耀着的人们;我想抢掠我曾给与赠品的人们:——我如此地想作恶事。



当别人想握我的手的时候,我却缩回我已伸出的手;我迟疑着,如急倾的瀑布迟疑一样;——我如此地想作恶事!



我的丰富沉思着这种报复;我的孤独诞生了这种恶念。



我给与时的幸福因给与而死去;我的道德已经厌倦了它自己的丰满!



常常给与的人有失去羞涩的危险;因为这人的心与手,终于会因分赠而生出一层硬厚的皮。



我的眼睛不再为请求者之羞惭而流泪;我的手皮变成硬厚的,不能感觉到受施者的手之战栗。



我的眼泪和我的心之柔嫩何往了呢?啊,给与者之寂寞啊!发光者之沉默啊!



她叹了口气,从一旁的柜子上取下一片书签,插进书里,放到窗台的另一侧,和它那些遭遇了相同命运的兄弟姐妹一起——《荒原》《古今和歌集》《白居易诗集注》……每一本她都只看了不到一半就扔到了一边。工作和学习的压力实在是让她难以集中精力在书上。定时器响了,她走进厨房,关了火,拿了两块抹布,把土豆炖肉端进客厅,又走回厨房,从冰箱里取出一袋速冻裙带菜撕开,拿了一双筷子,回到餐桌前,开始一个人享用她迟来的晚餐。



And I'm living off grass, and the drippings from the ceiling.



It's okay to eat fish, Cause they don't have any feelings.



在她偶像出道之后,学业和工作的压力使得她往往不得不靠便利店的半价便当打发日子——虽然这听上去有些荒谬,功成名就的偶像居然依靠半价便当充饥,但刚来东京一个人捉襟见肘的生活不可逆的塑造了她的消费方式。话虽如此,不管有多忙,她依然每周末起码给自己做一顿饭——一是为了让自己手艺不生疏,二是为了调解自己的营养配比,最后则是让自己铭记。她用筷子夹起一块牛肉,皱起眉头——哪怕是放在冰箱里,三十小时的时间足够让油脂因为氧化而染上些许馊味,土豆也因为反复加热而彻底融化成了一坨烂泥。她又吃了几口,犹豫了一下,最后还是把剩下的倒进了垃圾桶。大晚上吃这种东西实在是折寿。她转过身,把锅放进水池,加了几滴洗洁精泡上,随后又拧开龙头,给自己倒了杯自来水,坐到窗前,一边小口喝着,一边看向窗外那个灯红酒绿的花花世界。



Something in the way, hmmm~ Something in the way, yeah, hmmm~



Something in the way, hmmm~ Something in the way, yeah, hmmm~



她不喜欢东京的水,这里的水太硬,充斥着人类的冷漠与不近人情。她也不喜欢东京的空气,这里的空气太浑浊,飘荡着人类的傲慢与自命不凡。她尤为厌恶的是东京的夜晚,地上的光火过于耀眼,以至于遮蔽了天上的星辰,仿佛在夸耀人类的虚荣与自我中心。



明明身处人群之中,却仿佛孤身一人。



她的手机屏幕亮了起来,是经纪人的短信,祝贺她今天的演出成功,同时已经开始考虑未来sumimi和ave mujica的联合公演。听起来棒极了,反正上台的是她而不是对方。屏幕又一次亮了起来,是父母的定期联络:“吃饭了吗?工作怎么样?有没有看上眼的男朋友?”她叹了口气,在脸上堆起笑容,发了条语音告诉他们一切顺利,无需挂念。接着把手机扔到一边,再次看向窗外。



我从过去开始,就感到自己有点偏离这个世界。



她出生在一个偏远的小岛上,生性顽劣,是当地出了名的野孩子。山上和海里没有一处可以阻止她的脚步。她的家乡并不大,在岛心的小山上便可以看到环绕着她的小小世界的大海。她喜欢听着海浪冲刷岩石,日复一日周而复始;她喜欢闻着海风的咸腥味,生死在这气息中流转;她喜欢站在海边极目远眺,看着远方的海天融为一色,出行的白帆消失在地平线上。她的父母是岛上土生土长的渔民,老实巴交,恪守本分。她的一生原本会和其他这样的女孩一样:长大,成人,结婚,生子,作为渔民度过平凡的一生。这是她原定的命运,直到那天。她遇到了她的贵人,借由对方的指引,三角初华第一次注意到了夜空。然后,她看到了,繁星满天,倒挂的银河如同瀑布般倾斜而下,仿佛流到了她的心里。



\newpage



“丰川同学呢?”海玲看向祥子,对方已经在不知何时换好了便服。



“我想先冷却一下演出的激情。所以要搭电车回去。”



三角初华开口了:“那我也——”



“初华,你身为名人的自觉还不够,我很困扰。”祥子打断了她的话,“那么我先告辞了。”祥子离开了,留下了她,还有她的一丘之貉们。



东京看不到银河,她也从没去看过东京的海。她能看到的只有窗外,在霓虹灯的海洋中,有黑色的星辰若隐若现,陪伴着这黑暗中落寞的异乡人。



“果然,还是没办法让她信任吗?”她喃喃自语道,把杯中的水一饮而尽,“你到底怎么了,小祥?”



\newpage



她是被早上的闹钟吵醒的。她歪过脑袋,看了看自己的手机,四点五十,天空依然一片漆黑。她爬起身,刷了牙,然后用牛奶冲了碗麦片——富含膳食纤维和维生素,身为偶像要时刻注意自己的饮食管理——囫囵吞下肚。接着穿上外衣,换上跑鞋,出了家门,开始例行的晨跑。



五点,盛夏将尽,东方的天空依稀露出了鱼肚白,一轮苍白的下凸月挂在天边。这座城市尚未醒来,然而它的循环系统已经开始了工作。几个清洁工,低着头,戴着口罩,从她身边经过。一家家庭餐厅的店主正在推开卷帘门,他的老婆则站在一边,两手叉腰数落着他。一辆白色的面包车正停在一家便利店门口,两个店员正在从上面把一箱箱啤酒搬进店里。穿过冒着蒸汽的井盖,依然空无一人的大街,这座都市的大多数人依然在睡梦之中,醒来的大多是这样忙碌着的不被人正视的底层人。然而就是这样无数的底层人撑起了这座城市的运行。他们看守你们的家门,打扫你们的居所,把食物和水送给你们,治好你们的疾病,承担你们的抱怨。她虽然在学校里名声良好,左右逢源,可是她却觉得和这些人更加亲近。只是他们永远不会进入她的世界,正如她无法进入他们的一般。



在那座熟悉的公园门口,她放缓了脚步。虽然说是公园,其实也就是一块用围栏围起来的铺了路的草坪。几个帐篷立在公园里鹤立鸡群的几棵树下。她走近帐篷,一个穿着老旧的棕色呢子大衣的老头正坐在一个折叠椅上,面前烧开的水壶正在咕嘟作响。水壶被放在一个改造成了火盆的油漆桶上,白烟从壶嘴里袅袅升起。老头伸出两只手,靠近水壶,想要从蒸汽里获得些许暖意。听到脚步声,老头抬起头,见到是初华,他顿时笑了起来:



“三角小姐,又在晨跑啊。”



“山田先生。”三角初华也露出微笑,蹲在老头身边,“醒的还是那么早。”



山田和其他这里帐篷里的人一样,都是流落在东京的无家可归者。初华并不知道山田的名字,他只是让她叫自己山田。这里的人没有名字,没有身份,没有过去,没有未来。由于初华时常在经过这里的时候和他们聊天,顺带帮他们跑个腿,搬点东西,久而久之他们也和她熟络起来。他们并不知道她是偶像,也对她的学生身份毫无兴趣。在这里她只是三角小姐,是那些无名无根之人中的一员。山田过去在石川县当一家服装公司的老板,在上一轮经济危机的时候破产。当时的他已经五十多岁,找不到工作之后就当了流浪汉,一路漂泊到东京,平时靠着拾荒和从便利店拿过期便当维生。初华偶尔会给他带点水果作为伴手礼——一般的食物他是不肯收的。他平日里最大的兴趣就是织毛线,前几天还坚持要把自己织的帽子送给初华,初华费了好大力气才拒绝了他。



“可不是吗,人老了就是这样。”山田抱紧双臂,打了个哆嗦,“天也冷起来了,以前这种天气要是有点烧酒喝,那真是比什么都强。可惜年轻的时候太糟蹋身体,得了痛风连酒都不能喝。”



“大早上的喝什么酒啊。”初华笑道,“要不要帮你去买点红茶茶包?那个喝了也暖和。”



“我喝不来红茶。”山田从帐篷里取出两个杯子,倒了两杯水,“三角小姐可真是,大清早的不多睡点觉,而是来和我们这帮闲杂人等唠嗑。”



“最近有什么事吗?”



“……古川他……前几天走了。我们也不知道是哪天,发现他的时候人都僵了,就在他那个帐篷里,穿着他那身灰西装。”山田叹了口气,“他家里人还来找他了。明明十几年都没见了。现在大家知道了,古川年轻的时候好像是京都大学的高材生呢,毕业后也在大公司里上班,后来被裁员潮影响丢了工作,不好意思和家里人说,每天就穿上西装,打好领带,离开家门之后一个人来公园里坐着。他那时候也四十多了,找不到工作,就这么瞒了家里半年,最后因为没有钱了才被戳破。他老婆自然是带着孩子和他离婚了。我们平时看见他总是穿着那身西装还嘲笑他假正经,现在人走了……”山田终于克制不住,哭了起来。初华没有说话,只是轻轻抚摸着眼前的老人的脊背,过了半晌,老人冷静了下来:



“三角小姐,说实话,你和我们这些人不一样,我们已经是行尸走肉了,你还是个年轻人,你这种年纪应该好好念书,理应在游戏厅和同龄人嬉戏打闹,在课堂里谈笑风生,而不是在这里……怜悯我们……”



“那并不是怜悯,”她制止了老人,“我并不是因为怜悯你们才来这里的。我来这里和你们是一样的。”



“三角小姐,”老人冷笑起来,“说胡话也要有个分寸,你看看你,光鲜亮丽,衣服一尘不染,你还有个家,还有可以回去的房子,你甚至还有属于自己的名字。你拿什么和我们这种人比?你就当我一个老头乱发脾气,可是我们是流浪汉,你又算是什么?”



“我是什么呢?”她闭上眼睛,秋风吹过,落叶在她脚边旋转起来,不知要去往何方。她是什么呢?偏远海岛上出生的渔民之女吗?东漂追梦的行动家吗?功成名就的当红偶像吗?挥霍青春的做梦之人吗?“我只是一个背井离乡的落魄诗人罢了。”



\newpage



高松灯是在演出结束的后台里第三次见到了三角初华。爱音已经和素世叽叽喳喳地离开了,立希则被乐奈缠着去买抹茶巴菲,在她一个人在休息室的时候,门被推开了,三角初华走进房间,自然而然地坐在了她的身边:“演出很成功呢,恭喜。”



“啊……谢谢……”赶紧找点什么可说的,“初华小姐……也来看演唱会了吗?”



“是啊,”初华露出爽朗的笑容,“我可以说是小灯你的粉丝呢?”



“……其实我之前就很奇怪,为什么初华小姐知道我的名字呢?”



“……啊,说来有些尴尬,我还没和你说过呢。”初华打开手机,接着灯看到了那个她无比熟悉的网页,“其实我早在你之前的乐队的时候就已经关注你了。这个就是你对吧?crychic的主唱?”



\newpage



人们往往觉得跟踪是一件很困难的事情,但那大多是因为他们对于自己的目的毫无隐藏,对于对方情报不足。她走进更衣室,换下演出服,接着拿起背包,走进洗手间的一间隔间,关上门。她打开背包,取出绿色的大衣套在自己的白T恤外面,接着把裤子脱掉,换上了一条灰色的运动裤。然后戴上黑色的假发,再配合一张土里土气的红色针织帽,她走出隔间,看着镜中的自己。戴上美瞳和方框眼镜之后,sumimi的当红偶像三角初华消失了,取而代之的是一个陌生的略显羞涩的女高中生,看上去就像那种班里不起眼的学习委员。她把换下的裤子放进背包,背回背上,推开卫生间的大门,走向电车站。她借口独自不舒服提前半小时离开了彩排现场,虽然很对不起海玲他们,但是她必须得到真相。她走进电车站,随便找了张长椅坐下,看着电子钟上的数字变换,等待着。半小时之后,丰川祥子出现在了电车站入口处。她站起身,伸了个懒腰,自然而然地跟在祥子的身后二十步远的地方,上了站台。整个电车途中她一直待在另一节车厢,一边玩着手机一边用余光盯着不远处形单影只,心事重重的蓝发女孩,直到对方下车。赤羽站,那就是对方下车的地点。



到底是什么时候起疑的呢?初华闭上眼睛。大概就是从她接到那个电话的时候开始的吧。



“拜托了……初华……让我……忘记一切吧。”电话那边的声音熟悉而又陌生。这的确是小祥的号码,但是这是她从来没有见过的小祥。在她印象里,小祥总是自信,意气风发,温柔,如同太阳一样鼓舞人心的存在。电话那头的少女听上去破碎,迷茫,绝望,愤怒。她不能理解。她看着小祥用力抱紧自己的胳膊,防御姿态,小祥在害怕。有什么会让她害怕?她看着小祥一个人走进办公室,面对着那些精于世故的大人们口若悬河,妙语如珠。她感到陌生,彷徨,心痛。小祥变了,她不再是记忆中那个无忧无虑,天真烂漫,温柔体贴的大小姐了。她不知道如何面对这样的小祥。小祥很害怕,所以小祥想要保护自己。可是为什么?为什么要推开她的手?为什么要对她隐瞒?为什么要在自己面前掩饰她的痛苦?



她来到东京之后并没有去过祥子的家,但她是知道的,祥子给她发过照片。那所器宇轩昂的庄园距离赤羽站起码有一个小时以上的步行时间。有什么情况不对劲,小祥不应该在这么晚的时候孤身一人去那里,赤羽附近的治安她有所耳闻,而且房价也和小祥的身份不符合。小祥为什么会在那里下车?



她低下头,看向手机。crychic,这个名字让祥子感到痛苦吗?在那一天,她提起这个名字的时候,她有意放慢了语速。祥子别过了眼睛,那就足够了,祥子身上的变化和无疑crychic停止活动有关。在那之后,crychic的主唱,鼓手,还有贝斯组成了新的乐队,小祥则沉寂了一年,直到最后才找到自己,和睦一起发起了Ave Mujica的企划。高松灯告诉了她crychic的解散经过,果然,是小祥提出的退出导致了以她为核心的crychic分崩离析。睦也是小祥的青梅竹马,和自己不同,她一直陪在小祥身边。睦可能没有注意到小祥身上的异样吗?不对,这是不可能的。她虽然没有见过睦几面,但是从小祥的口中,她足够了解这位沉默寡言的青梅竹马强大的洞察力和敏锐的直觉。睦不可能注意不到小祥的异样,但是她却保持着沉默,仿佛什么都没有发生一般。不对劲。这只有两种可能:1. 睦认为沉默是她能做的最好的事情; 2. 睦被祥子要求沉默。睦绝对不会在不知道原因的情况下沉默不语,因此无论哪种可能,睦都应该知道事情的真相。愉快的假面舞会结束了,她必须找到若叶睦。



\newpage



“小睦。”三角初华看向眼前人偶般面无表情的少女,“你收到我的信息了。”



“初华说想要在这里见面,所以来了。”睦喝了一口芒果汁,声音没有任何波动,仿佛早有准备一般。初华眨了眨眼:



“最近你和小祥在月之森过得怎么样?”睦的眼睛睁大了,虽然没有出声,但是初华察觉到了她的动摇。



“……”睦没有说话,但她的沉默已经说明了一切。果然,不在月之森了吗?高松灯之前表演的时候穿过羽丘的校服,她在羽丘?果然当时应该再问问灯的。



“我去了赤羽站。”初华漫不经心地转过头,用余光打量着睦,依然没有反应,“小祥在那里下了车。”



“……你不应该知道的。”睦的声音有些颤抖。



“是的,但是我还是知道了,就像我知道了crychic的事情一样。小祥的状态很危险,这点连祐天寺都能看出来,可是你却一言不发,你到底知道多少?”



“我不能说。”



“是因为她的要求吗?果然。”初华叹了口气,“不论何时都不肯屈服,不知妥协,哪怕死也不肯低下她那高傲的头。那就是我们的小祥。换一个问题吧?你为什么要冒着站在聚光灯下的风险加入她的这场人偶戏?”



“因为祥快要坏了!”睦的语气突然变得急迫起来,“她退出crychic的时候我就在她的身边,她一个人和丰川家决裂的时候我也在她的身边,她现在哪怕把自己弄坏也要进行ave mujica的时候我还在她的身边。可是我什么都做不了!我救不了祥,我只能陪着她一起下地狱!可是你,初华你不一样!你可以救她,你应该去救她!因为就算祥她自己也不知道,她一直相信着你!”



“是啊,我们都爱着她,不是吗?可是即便如此,她也不肯依靠我们。但是我一个人是做不到的。因此,我需要你,小睦。”



“……”睦闭上了眼睛,神情扭曲起来,仿佛在和什么斗争。



“但是,我不会强迫你做你不愿意的事情。”睦又一次惊讶地睁大了眼睛,“如果你认可了她的决意的话,那就这样吧。”初华站起身,虽然很失望,但也只能到此为止了,进一步对睦施压只会让她自己更快暴露,而且也于事无补,不如先退一步留下一个好印象,“如果你回心转意的话——”



“我答应你。”睦打断了她的话。“我来帮你,拜托了,救救祥。”



\newpage



她一个人去了那座屋子。几只硕大的蟑螂在她推开门的时候从门上跑了出来。她不耐烦地把这些顽强的生灵拨到一边,走进屋子。虽然是在白天,没有开灯的出租屋里依然一片昏暗,但她的眼睛瞬间就看清了门口堆积如山的易拉罐和酒瓶。虽然瓶瓶罐罐不少,但是地板竟然意外的干净——主人显然花了不少力气来维持这个小家形式上的整洁。她抬起脚,小心地绕开地上的陷阱,走近客厅。在她的左手边,几张薄薄的毯子铺在地上,一条简易的帘子组成了居住者与外界唯一的阻隔。右手边的柜子上,一个身着红色洋装的人偶正盯着她露出诡异的笑容,人偶的衣服虽然有些磨损,但是依然整洁。显然人偶的主人依然对它关照有加。她从来搞不懂为何小祥对那个人偶情有独钟,小祥说它很像小睦,但是在初华看来睦的内心可比人偶炽热无数倍。她的目光扫过地上堆积的垃圾,茶几上散落的餐盘,闪烁不止的电视机,最终停留在了紧闭的窗帘前匍匐着的男人身上。



男人的呼吸很浅,仿佛失去了意识一般。初华还记得男人,在过去,在她和小祥第一次在岛上相遇的时候,男人就在小祥家的屋子里。男人是个开朗的人,喜欢音乐,爱开玩笑,对自己的老婆和女儿都一往情深。她还记得男人牵着自己和小祥的手在山上奔跑,自己教会男人用叶子吹口哨,男人给自己吹了一首舒伯特的小夜曲。那是多么欢快的时光啊!那是多么美好,转瞬即逝,一去不返的她和小祥的少女时代啊!此刻的男人在地上瘫成一团烂泥,从喉咙里不时发出可怕的呼噜声。男人的眼眶和四肢都因为缺少运动和过度饮酒浮肿起来,皮肤也皱皱巴巴的,失去了弹性。地上的酒瓶在诉说着男人无可救药的堕落,而空气中断断续续的喘息声则是男人沉沦的挽歌。三角初华闭上眼睛,这就是小祥身处的世界,空气当中弥漫着尘埃,霉菌,以及垃圾腐败的臭味,如同甜腻的泔水,令人作呕。孤身一人躺在这坚硬的地板上,与蟑螂,呕吐物,火车的汽笛声为伴,甚至找不出第二套可以更换的私服,更不要说在乎自己的外貌。她的人生就要和这样的一个酒鬼一点点消磨掉,浪费在每天的打工,收拾垃圾,帮男人洗衣服,让男人活下去上。她曾经那么的才华横溢,意气风发,现在的她却栖身在这样一件狭窄,阴暗的斗室之中,和这个男人一样,一点点死去。她本不必这样的……她本不应这样的!



不知何时,三角初华的拳头握紧了。她华是个偶像,但在那之前她是个乡下长大的野孩子。她喜欢运动,喜欢看星星,喜欢爬树,喜欢游泳。她的视力可以轻易地在东京找到一颗四等星,而她的身体足够让她背着五公斤重的书包一口气跑完十公里。她伸出手,眼前的男人衰老而脆弱,只要她用自己的双手环绕住对方的脖子,一用力,他甚至没有反抗的机会,就会在睡眠中死去。



冷静下来,她告诉自己,眼前的这个男人并不是小祥痛苦的根源,不要迁怒于他。



可是?



可是她明白,如果眼前的人不复存在,小祥的生活可以变得更好。没有无穷无尽的抱怨,没有不知尽头的酗酒,更没有那消磨生命的负担。没有了他,小祥才有可能再次如过去一般发自内心的开怀大笑。



但是这是小祥的父亲,杀人是不对的,更何况这是你最重要的友人的亲人。



三角初华摇摇头,从很早开始,她就知道了,世界上并没有对错,有的只有目标和实现它们的手段。有效,不有效,这才是唯一的判别标准。



但是小祥会伤心的。



可是如果他还活着,小祥只会更伤心。想想看吧,一个酗酒的,情绪化的,没有工作的父亲可以做出什么。谁知道小祥的衣服下面有没有尚未愈合的淤青?谁曾注意到小祥强装的冷静之下那破碎的内心?谁可以结束她用自尊掩盖的无边的苦难?她的拳头颤抖起来,她咬住自己的嘴唇不让自己出声。这个畜生……



杀了他杀了他杀了他杀了他杀了他杀了他杀了他杀了他杀了他杀了他杀了他杀了他杀了他杀了他杀了他杀了他杀了他杀了他杀了他杀了他杀了他杀了他杀了他杀了他杀了他杀了他杀了他——



三角初华转过身,走出大门,离开了公寓,向着电车站走去。她的步伐歪歪扭扭,仿佛失去了灵魂一般,终于,她跪倒在地,发出无声的嚎叫,在她看来,这叫声几乎要撕碎自己的心脏。

\end{document}
