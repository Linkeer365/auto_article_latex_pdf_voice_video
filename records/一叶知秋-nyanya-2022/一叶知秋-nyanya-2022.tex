\documentclass{article}
\usepackage[utf8]{inputenc}
\usepackage{ctex}

\title{一叶知秋\footnote{Click to View:\url{}}}
\author{nyanya}
\date{2022}

\maketitle
\newpage

\begin{document}
\CJKfamily{zhkai}

% : comm which is only poems edit in-use %


\Large

喜欢秋天的八成原因是喜欢落叶,剩下两成均分给流糖的烤红薯、刚出炉的糖炒栗子、捧在手心里的热饮、棕色的毛茸茸外套营造出来的小鹿一样无害的感觉、凉爽得恰到好处的天气和久未见面的人。



但还是先说说落叶。校园里有很多长得枝叶繁茂的树,到了秋天就簌簌地脱发。一些落叶卡在篮球场的铁丝网格里,一些铺在路边绿化带的小灌木上,一些粗心大意地掉在我的头上。我会特意挑选有落叶的地方走路,大多数都被扫到了路的两边,我踩上去,干枯的落叶沙沙地在鞋底碎裂成更小的碎片,而更多的是刚落下没多久的肥厚叶片,踩上去厚墩墩的。落叶堆一向疏松,站在里面迈步,会感到仿佛被落叶埋起来似的。



于是秋天的我大多数都是这个形象——穿着毛呢长裙和棕色外套的少女,低着头呆呆地沿着路边踱步,在落叶堆里深一脚浅一脚地前行,时不时捡起叶子捏碎在手里,或者撕成更小的小片就丢掉。在这样的日子里我少有观察路边的行人,比起行人的面貌和穿着,地面上的一切更加吸引我。



我看到扎了口的大大的玻璃丝袋子,我轻轻地推了推它,它的内容物比看起来要更加轻盈,于是我知道那是落叶暂时的陵墓。而它让我产生另一些冲动,譬如在堆得高高的厚厚的落叶堆上躺下,在秋日高远的天空下望着漂浮得很高很高的云朵,听着从千里之外吹过来的风。我的躯体会在落叶堆上微微下陷,一些叶片碎裂开来,是大地的心跳声。



落下的叶子是死去但未完全死去的树,树比我活的时间要更加长久,这只不过是它漫长生命里的一次别离——而我是生命更短暂也更没用的人类。

\end{document}
