\documentclass{article}
\usepackage[utf8]{inputenc}
\usepackage{ctex}
\usepackage{url}

% support Chinese Chars %
\newfontfamily\urlfontfamily{FandolSong-Regular}
\def\UrlFont{\urlfontfamily}

\title{恐怖}
\author{莫泊桑}
\date{18840518}

\maketitle
\url{https://www.bilibili.com/read/cv34129001/}
\newpage

\begin{document}
\CJKfamily{zhkai}

% : comm which is only poems edit in-use %


\Large

温热的夜缓缓降临。



女士们还待在别墅的客厅里,男士们则身处门前的花园中,他们围着一张摆放茶杯和小酒杯的圆桌,或默然静坐,或倒骑着椅子抽雪茄。



夜色渐浓,周围逐渐晦暗,雪茄头像一双双小眼睛,正闪烁着猩红的微光。有个人刚刚讲述了前一天发生的可怕事故:在河对岸一众来宾的注视之下,有三男两女竟在河里淹死了。



此时,G将军开口了:



或许,这样的事能让人感到可怕,但并不“恐怖”。



“恐怖”这个古老的词语,其含义远远超过可怕。刚才讲的那件可怕的事故,让人触动,让人惊愕,让人震惊,但不能让人感到恐怖。心灵的震颤,那些可怕的死亡场景,还不足以使人感到恐怖。“恐怖”必须来自一种神秘的怖骇,一种超自然的、不合常理的惊惧感。一个人即便在最悲惨的状况下死去,也不会引起恐怖。战场并不恐怖,流血也不恐怖,最暴力的犯罪行径也不怎么恐怖。



但是,我这里有两段亲身经历,它们曾让我感受到“恐怖”。



那是在1870年战争期间。我们部队穿过鲁昂,向奥德梅尔桥撤退。部队被打得还剩两万号人,那是两万个溃不成军、士气衰落又精疲力竭的人,准备撤往勒阿弗尔接受整编。



大地被茫茫白雪覆盖,夜幕惨淡而至。我们从前一天起就粒米未进,一门心思只顾着撤退逃命,因为普鲁士人离得并不远。



诺曼底惨白的原野上,那些零星散落在农庄周围的阴暗树影偶尔显露出来。整个大地在黑暗、阴森、沉重的夜色下向远方伸展。



凄凉的暮色中,只能听到队伍一阵又一阵哀怨疲惫的抱怨声,像牲口一样乱哄哄。在数不清的脚步声中,夹杂着饭盒和军刀模糊的撞击声。士兵们浑身脏兮兮,很多人衣衫褴褛,腰沮丧地弯下去,背也无奈地弓着。我们在雪地中拖着根本拖不动的步子,蹒跚赶路。



那一夜,奇寒透骨。手刚一触碰钢制的枪托,皮就会粘在上面。而穿鞋已经成了受罪,我时常看见有年轻的士兵脱去鞋子赤脚走路,雪地上便会留下带血的脚印。这样坚持一段时间后,我们想在田野上休息片刻,但只要一坐下去,就再也站不起来了。每个坐下去的人就是死人。



我们把那些虚弱的士兵丢在身后。他们已经耗尽了体力,原以为等僵硬的腿休息一会儿就可以立即上路。但是,等他们已经凝滞的血液在冻僵的肉体中彻底停止循环流动后,一种无法抑制的麻木就会让他们动弹不得,把他们钉在地上,合上他们的双眼,在一瞬间停止他们那早已劳累过度的人体机能。他们的额头向着膝盖渐渐下沉,但又不会马上倒下,因为他们的腰和四肢硬得像木头一样根本动不了,无法弯曲,也无法伸直。



而我们这些强壮的人还能向前走仅仅是依着惯性。在黑色的夜空下,在严寒的雪地中,在冰冷的死亡原野上,奇寒侵入我们的骨髓。悲愤、失败和绝望早把我们的精神压垮,被遗弃的痛苦让我们陷入末日、死亡和虚无。我们被逼到了咽气的前一秒。



这时,我看见两名士兵押着一个小个子男人的胳膊,那人上了年纪,却没有胡子,模样有些古怪。



士兵以为抓到了一个间谍,便来报告军官。



“间谍”这个词立刻就在蹒跚前行的士兵中传开了,他们立刻把这个俘虏团团围住,一个声音高喊:“枪毙他!”这时,所有那些原本疲惫沮丧、只能倚枪而立的士兵们,突然亢奋起来,进入了兽性的愤怒中,这种愤怒把整个人群变得杀气腾腾。



我想开口表态,因为那时我已是营长。但此刻谁都不买长官的账,说得不好,连我也可能被他们杀了。



其中一个士兵对我说:



“这家伙跟踪我们三天了,见人就问炮兵的情况。”



我试着审问这个男人:



“你是做什么的?你想要什么?你为什么跟着部队?”



他咕哝了两句算作回答,但却是听不懂的方言。



这真是个古怪的男人,他的肩膀较常人更窄一些,目光并不安分,在我面前又表现得惶惶不安。我当时毫不怀疑,就认为他是个间谍。他看起来上了年纪,身体也很虚弱。他在偷偷地打量我,神色中透露出一点愚蠢和撒谎的迹象。



周围的士兵们大喊道:



“靠墙站!靠墙站!”



我对士兵们说:



“你们能确保俘虏的安全吗?……”



话还没说完,一阵激烈的推搡就把我掀翻在地。随即,我看见那男人被愤怒的士兵们抓住,摁在地上拳打脚踢,又被拖到路边,扔在树旁。最后瘫倒在雪地上,奄奄一息。



他立刻就被枪毙了。士兵们刚向他射完一粒子弹,就立即重新装填,再射一枪,仿佛躁怒的野兽。他们在尸体前排成行,争先恐后地射上一枪,就像在灵柩前排队洒圣水一样。



突然,一个声音高喊:



“普鲁士人!普鲁士人!”



随即,我看到军队开始溃乱奔逃,喧闹声响彻天际。



原来,向这个流浪汉开火所引发的恐慌已经吓坏了开枪者们,他们还没明白这误以为敌军来袭的慌乱就来自他们自己,便一个个仓皇消失在黑暗中,各自逃命去了。



我独自待在尸体前,两个士兵因为职责所在,也留在了我身边。



他们抬起这堆被打得血淋淋的烂肉。



“搜搜看。”我对他们说。



我从兜里掏出一盒军用火柴递过去。一个士兵为另一个照明。我站在两人中间。



搬弄尸体的那个士兵大声说:



“上身穿蓝罩衫、白衬衣,下身穿长裤和皮鞋。”



第一根火柴熄灭了,我们划亮了第二根。



士兵翻着死者的口袋:



“一把牛角柄小刀、一块方格子手帕、一个鼻烟壶、一截绳子、一块面包。”



第二根火柴熄灭了,我们划亮了第三根。



士兵在尸体身上搜摸了一会儿,最后宣布:



“就这些了。”



我说:



“扒了他的衣服。可能会在贴身的地方找到点什么。”



为了让两个士兵能同时行动,我亲自给他们照明。在火柴的闪熄之间,我看见他们把死者的衣服一件件扒掉,露出这个尚有体温却血肉模糊的死人躯干。



突然,一个士兵惊讶地说:



“见鬼,指挥官,这是个女人!”



我无法描述当时在我心里涌动的是怎样一种古怪、悲痛又恐怖的情感。我无法就这样相信,便亲自跪倒在雪中,检查这堆不成人形的血肉,验明:这真是个女人!



两名士兵目瞪口呆,十分气馁,只等我发表意见。



但我也没了主意,不知该怎样去设想。



此时,两人中的队长吞吞吐吐地说:



“也许,这是一位母亲,她是来找她那当炮兵的儿子的,这位母亲一直没有她儿子的消息。”



另一个附和道:



“很可能是这样。”



惨不忍睹的场面我也曾目睹过一些,但那次,我竟为她而恸哭。在那个奇寒的夜晚,在那片黑暗的原野,在这个被残杀的母亲面前,“恐怖”降临在我的心上。



而去年,在审讯弗拉泰考察团的幸存者——一个阿尔及利亚步兵的时候,我也有过同样的感觉。



弗拉泰上校经由沙漠前往苏丹,途中要穿越图阿雷格人的一大块地盘,那里是一片沙漠,从大西洋延伸到埃及、从苏丹蔓延到阿尔及利亚。图阿雷格人是这片沙漠海洋里的海盗,就像过去那些在海上肆虐的民族一样。



为上校带路的向导们,属于瓦尔格拉地区的香巴部落。



有一天,他们在沙漠中搭建营地,而那些阿拉伯人声称水源还有点距离,他们应该带上全部骆驼去找水。



只有一个人提醒上校,他会被出卖;但弗拉泰上校根本不信,并和工程师、医生以及几乎所有军官一起随驼队出发了。



他们在水源附近被杀害,所有骆驼都被劫走了。



瓦尔格拉阿拉伯办事处的上尉留守营地,他成了幸存骑兵和步兵的指挥官,于是他们开始撤退,并抛下了行李和食品,因为没有运输的骆驼。



他们在这片无边无际且没有一丝阴凉的荒僻沙漠中赶路,从早到晚忍受着烈日的暴晒。



一个部落前来归顺他们,并带来了蜜枣。但枣被下了毒,几乎所有法国人都被毒死了,包括最后那位军官。



只有骑兵中士波贝甘和香巴部落的几个步兵活了下来。还剩两头骆驼,但一天夜里,它们和两个阿拉伯人一同消失了。



一发现那两个人带着骆驼逃跑以后,幸存者们就明白他们只能靠自相残杀活下去了。于是,他们相互保持着超过一个步枪射程的距离,在烈日下一个接一个,分头走在松软的沙地上。



炙热而平坦的沙漠上时不时会掀起一个个小沙柱,从远处表示,这是有人在沙漠里行走。



然而,某天早晨,其中一个士兵突然向斜刺里插过去,靠近他旁边的士兵。所有人都停下来观看。



被饿红了眼的士兵逼近的那个人并没有逃走,而是趴在地上,瞄准来人。当他认为已达射程之内时,就开了枪。对方没有被击中,继续向前走,同时端起枪,一枪就打死了他的战友。



杀人者把死者切成肉块,接着,其他人从四面八方跑过来,寻求自己那一份。



然后,这些无法妥协的盟友们又一次拉开了距离,直到下一次杀戮时才会聚在一起。



他们靠分食人肉挨过去两天,之后,饥饿再度来袭。杀死第一个人的士兵,这次又杀了一个。他像个屠夫一样,再次切割尸体分给同伴们,自己只留下一份。



食人者们就这样继续撤退。



最后,那个法国人波贝甘,是在救兵到达的前一天,在一口井边被杀死的。



你们现在明白我所感受到的“恐怖”了吧?



这就是那天晚上,G将军给我们讲的故事……


\end{document}
