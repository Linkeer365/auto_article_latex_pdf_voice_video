\documentclass{article}
\usepackage[utf8]{inputenc}
\usepackage{ctex}
\usepackage{url}

% support Chinese Chars %
\newfontfamily\urlfontfamily{FandolSong-Regular}
\def\UrlFont{\urlfontfamily}

\title{Grey}
\author{nyanyanyanyamii}
\date{2023}

\maketitle
\url{https://catcorework.readthedocs.io/zh/latest/原创故事(SFW)/【百合】Grey.html}
\newpage

\begin{document}
\CJKfamily{zhkai}

% : comm which is only poems edit in-use %


\Large



{\centering\section*{(一)}}





我遇见她的那一天正值傍晚,一个适合小孩子拿着糖瓜和烟花跑闹嬉戏,恋人手挽着手在凉爽的湖边散步的时间段。我在院子里面看见她:坐在破旧的木板秋千上,小手攥着粗糙的绑绳,一身白色连衣裙仿若透明。这个形容不是我色心大发,毕竟我也不过是个普通的女高中生,梳着普通的学生头,过着普通的日子。可她真的是,白蝴蝶一样的姑娘,在周围穿梭来往的小孩和裹在墨蓝校服里的乏味躯体之中,让我清楚地感受到现实和梦境的撕裂感,就是每天早上我从宿舍封闭的的床罩里醒来,梦境的余韵和铺天盖地张牙舞爪的现实共存于我这一个躯体中的感觉。我看见她就想起来蕾梅黛丝裹着床单,飞到天上去的那一幕。



透明的,轻盈的,属于天空而不属于地面的。



在后来我和她相处的许多天里,我唤她猫猫。不是她名字里有这字的谐音,也不是她长得真像那种有温热躯体柔软毛皮讨人喜欢的小动物。只因为她给我讲了一个故事,讲完我们就互换了称呼,成了朋友。至少我单方面看来是这样。



再回到那个院子,那个秋千,那条白色长裙。我本来不过那灰黑河流里的一颗水滴,却因那抹白色驻足。她招了招手,我不由自主地走过去。时至今日我回想起这天依然记得她随着秋千飞扬的白色长裙,透出的若隐若现的纤长双腿,还有这天的光线、温度,夕阳的阴影,许许多多,我都一直记在心里,准备往坟墓里带。



她问我,这么晚了你怎么不回家。



我说高三生留学校留得晚,我还是不上大晚自习,才能这个点出来。



然后听到她有点惊讶的笑声,呀原来你高三了啊,我还以为是个初中生。



这种话我见怪不怪了。我长得小,从初中就是不争的事实。为此我特意喜欢上了灰黑色的穿搭,据说黑色显老,是年轻女孩子大多偶尔会穿但绝不常用的颜色,只有我趋之若鹜,在大街上五颜六色的人流里做一个污浊的泥点。



我问她叫什么名字,她不答话,反而问我。我又反问回去,她自顾自说:我今天早晨去跑步的时候遇见一只猫。



于是我问她那是只什么样的猫,橘猫还是布丁猫,流浪猫还是家养猫。



是只白猫。额头一侧的毛都快掉光了,露出斑驳坑洼的粉红色的皮藓。她一边说,一边用手指头比划自己的额头那儿。她的额头光滑圆润,像白煮蛋的表面。她说她遇见那只猫时跑到了六千米的最后一圈儿,双腿软绵绵地没半点力气。那只猫停下来,冲她喵喵叫,一声接一声。



让我想象接下来的画面吧:出来晨跑的少女,运动衫汗湿了贴在年轻紧实的肉体上,伫立在那两分钟,气都没喘匀就扭头往小卖店跑,心里只念着那猫不要就这样扭头消失,让她颤动不齐的心跳付诸东流。



少女手里拿着火腿肠跑回原处时,太阳穴应该已经开始抽动作痛,此时朝日还未升起一个肉眼可见的角度。少女应该不是很认路,破旧校园里的道路又几乎处处相同,她应该徘徊了有一段时间才找到锈迹斑斑的铜绿色自行车管,然而周围并没有猫,一如既往。



流浪猫向你讨食,等你带了吃的回来,它却消失不见。这种感觉大抵不是很好,如同一个人把他的悲惨倾诉给你,却不要你救,空留下满满的遗憾和悔恨。但是这时少女看见了上锁的铁门,锁链仅容一手通过,那一头的荒凉庭院里远远地坐着那只白猫,像一尊雕塑,在缓缓上升的清晨的阳光中,应该莫名地触动了少女的泪腺。



所以少女把肠衣剥开,将食物丢进铁栏里的时候,应该是流泪了。和周围戴着防霾口罩机械地绕圈跑的人不同,和骑着共享单车刷着学校的体育软件的人不同,她因为一只流浪猫而感到伤心。



就在这时候她打断我的想象。她说不是的,我说不准眼泪流下来的一刻我是为自己难过,还是为那只猫难过。正如白居易和琵琶女,沾湿了江州司马青衫的泪水,怕更多地是为他自己而流,而不是红颜不复的女子。她说她冲着铁栏那边喊“猫猫,你要活下去”的时候,感觉就像是跟自己做一个约定。



所以那天在薄暮的余晖里,在柔和的夜色即将吞入炙热的、一切都暴露在光明之中无所适从的白昼时,秋千上的女孩和我交换了彼此的名字。她让我叫她猫猫,这个叠词从唇间吐出来软软的,温和又有点乖张的魅力,而且是完全属于自由的。我接受了,并在她下了秋千,走向巷子尽头的时候告诉了她我的名字。



魏莱。我说。



你——说——什——么?猫猫隔着巷子冲我喊。



魏——莱——我的名字,魏莱——



猫猫笑了。隔着一条窄巷的粘稠湿热的空气,一条窄巷的喧闹和不安,半条窄巷的光和半条窄巷的暗混沌地混合,我很难分清那是个怎样的笑容。可我没来由地觉得她在笑,正如她扬起的白色裙角,正如把我裹在蚕蛹里的灰黑色长袍。光与暗,白与黑,在黄昏这个时刻交汇。白吞噬黑,黑吞噬白,黑与白融为一体。



你的名字。她说,是我没有的东西。





{\centering\section*{(二)}}





那之后的一周里,四处寻找猫猫成了我放学后必做的一件事,这个总是身穿白色连衣裙、笑容模糊不清的少女,似乎比街边吱吱作响地流着油的烤串和河里鳞光闪闪的小鱼更有吸引力。我仔细观察过,她穿的裙子不是每天一样的,要么是胸口缀了个蝴蝶结,要么是裙摆多一层蕾丝,总是白的这点没有变。猫猫会出现在各个地方,有时候在秋千那儿,有时候坐在低矮的墙头上望着快要黑掉的天空,有时候躺在谷堆上像一个思索着乡村烦恼的普通女孩。我总是会在一番寻找后来到她身边,在隔她一米七五的位置坐下:这应该是接近猫咪的适当距离,不会太近而危险,不会太远而疏离。



我和她关系进展飞速,远快于理综试卷上那个红色数字的变化。我们从一开始的拘谨,到后来无话不谈。猫猫是个比外表有趣一万倍的姑娘,当然她外表已然够光鲜,只是撑起这皮囊里的内涵还差了点。和她聊天的感触就像同古人游,焚琴煮鹤都觉得爽快。我很快迷恋上这种感觉,同时变得愈加孤僻:那些无趣而虚伪的同龄人,每天谈吐低俗的笑话,藉以学校的硬性契约维持脆弱的关系,在我眼中不及猫猫一根头发丝。



可是猫猫只会出现在黄昏,白天和夜晚的交界处,朦胧而模糊不清的时间点,一天中最短暂的时光。夜幕降临,坐在秋千上的她,倚在墙头边的她,仰面躺在谷堆上的她,无论和我聊得多开心,都会站起身来,拍拍不存在的灰,挥手道别。



她道别的时候喜欢叫我名字。她说,魏莱,再见。这个说法没来由地让我感觉很开心,就像我们结下了未来再见的约定,就让我还有那么一丁点蜘蛛丝的希望再苟活一天,撑过漫长无眠的黑夜昏昏欲睡的早上闷热无聊的中午乏味枯燥的下午,到黄昏时刻,这个柔媚而漂亮的时间。这让我感到明天我还找得见她,墙头不在就是在秋千,秋千不在就是在谷堆,或者是教室后堆砌得高高的废用桌椅,或者是破败的布满灰尘和蜘蛛网的小储物间。我在不断的寻找中是安心的,因为我坚信这个白裙子的少女一定会在某处等我;我在不断的寻找之中又是焦灼的,因为我知道晚遇见她一秒,就离黑夜更近一秒。



所以,我要如何留住她?



自古以来人之为人,欲壑难填。有了一米七五的距离,总会贪求一米六五的,就这样十厘米十厘米地缩减到零,恨不得要变成负数才能真切感受对方存在。说到底人是有生存焦虑的,由此演变出角色焦虑分离焦虑等种种焦虑。英语早读考单词,写到那个anxiety我手下圆珠笔无意识地多划几道,粗糙惨白的试卷徒劳地凹下去一块儿。我成绩不怎么好,也就那样,平日里听听歌写点文字,在老师眼里那种还蛮聪明却不认真努力的学生。但好在我做人中规中矩,从不做什么事惹大家都不开心,因而就做一个看似没目标没理想只想过好自己小生活的中等生,混着日子。



我平日读罗素,读康德,读弗洛伊德,读黑格尔也读叔本华。家里没给买智能手机,平日零花钱多用来买CD和画册,没有什么其他的高端途径去接近哲学,靠的也就是一本本泛黄的旧书本。可能哲学读多了,我看起来不是那么平易近人;也可能是哲学读多了,我看起来总是一副懒懒散散不爱这世界的模样。其实可能原本是不爱的,也许喜欢但不怎么爱,不讨厌也就够了。但是遇见猫猫后,我敢说我爱这世界爱到要死。



同龄人没法跟我谈哲学,这话倒不是想说出多少优越感,只是志趣不同必不相合,做个酒肉朋友还可以,深了去我就感到无法自如。他们为分数和认可挤破脑袋,拼命挤过独木桥来证明自己脱颖而出,从而实现最完好的自我。而我很早就意识到我无法和我的同龄人一样,在强加于我的学业上获得成功这件事里找到任何快感,它触不到我的G点,但那些疯疯癫癫的书可以,猫猫可以。



所以想象我们相识之后的几个傍晚——一米七五的距离之间,有趣的灵魂彼此试探。我们聊了三个晚上的《纯粹理性批判》,想象人的认知可以越过星空的边缘,模糊受限的边界;那之后又聊叔本华,要么庸俗要么孤独,人生得失。旁人路过我们大概很难想象两个正值花季的少女谈论的不是偶像与兴趣,烘焙与日常,而是人生和理想的碎片与解读。我多次想再近一步,可当宛若透明的猫猫冲我露出宛若透明的笑,我就一下子慌得不知所措。



终于有天我们的话题在黄昏之刻走过一半时渐渐偃旗息鼓,空气由顺畅的流动而变得暧昧不清,那些暧昧因子钻进我的鼻腔让我嗅到了机会的味道。于是我趁机问起她的住址,问她是否有机会可以上门拜访。



可人是从何而来,到哪里去呢?猫猫暧昧不清地笑。



她这一句话似要把话题终结了。我躺在草堆上望着渐渐褪去昏黄、被墨黑浸染的天空,想着不过多久就会出现无尽的繁星,在浩渺的夜空之下人的孤独感将愈演愈烈,而那时猫猫将不在我身边。



我和猫猫谈起,那所有的专业里我想学建筑。想听巴特农神庙的低语,想抚摸山花与檐壁,看着哥特的玫瑰窗被侵略者的铁蹄踏碎。想要去构筑空间,在那样的一个我做出来的空间里,或许有我的容身之地。



猫猫只是在笑,没有说那些父母和老师都会说的或是拒绝或是恐吓的话,没有表现出周围同学的敬仰与钦羡之情。她只是笑,笑容淡淡地在脸上浮着,似乎转瞬之间就会消散于虚空。那是我曾经在书上见过的,圣母一样的微笑。





{\centering\section*{(三)}}





我在那之后仍然没有得到猫猫的住址。那天晚上我们各自躺在草堆上,望着康德也曾望过的头顶的星空,各自想很多很多事情,彼此间却沉默而不发一言。我听说过最好的关系应当是这样的,当你不说话她也不说话时,你们两个人还能友好相处。大抵就是我和猫猫这样,可我总觉得少点什么,这点“什么”湮灭在夏日傍晚潮湿的雾气里,消逝在沉默而温厚的空间里,是我永远怀念的夏天的缺失部分。



那缺失部分将永远缺失,像拼图上丢失的一块。我的人生大概就是纯白的地狱吧,可就算是这样的人生,缺了一块也还是会隐隐作痛。这事暂且不提。



翌日我上学的路上,见到学校花圃里种的那两株绣球突然开了。她们俩苟延残喘过了冰冷的冬季,又苟过了潮湿的春天,迎来盛夏的时候终于肯把自己绽放给这没什么好意的世界看。也许是要向花证明,这个世界并非教科书上照本宣科写的那么美好,大多数时候总是公平而且努力总有回报,我折了两朵偷偷夹带回去。这两朵花瓣上浮着一层浅浅的蓝紫色的花儿被我精心保护了一天,想等到晚上送给那个白裙子的少女。



我想她的裙子配上花儿一定很好看。纯净到透明的白色,美得有些过分了,我想要猫猫看起来更像是一个实体,而不是仿佛一阵风吹过来就会消散的幻影。



可是那天晚上,猫猫不在那儿。



接下来的一个星期,我们约会的小广场熙熙攘攘,独缺她一人。



那天我在广场上一直徘徊到天黑,站在烧烤摊旁边吃了一串又一串的烤鱿鱼,直到烧烤摊主收摊。摊主大叔可能看我是个学生模样,临走还叮嘱我早点回家,别遇到坏人也别让爸妈担心。我笑着答应了他,顺带祝他生意兴隆,他看起来很高兴。



鱿鱼卷曲的须子,嫩滑Q弹的肉配上浓浓的酱汁,这两个元素适合一场大雨。我没来由地突然这样想,就像某个人,就像猫猫的想法突然钻到了我脑袋里。顷刻天边惊雷,乌云滚滚而来,倾盆的水滂沱之势浇上大地,浇上毫无防备而失魂落魄的灰黑色的我。



濡湿的我,站在朵朵伞花之中,站了一会,然后扭头慢慢地走回家去。至今为止我记得每一滴雨水的温度,记得水滴沿着我敞开的领口滚进贴身内衣,冰冷的雨滑落温热的锁骨,隐没在下面肉体的曲线里。我闭上眼,想象猫猫融在雨水之中,想象她散落在撒满夜空的繁星中,又掉在我眼睛里化了。



我回去之后,该吃饭吃饭,该睡觉睡觉,一如往常。食堂不变的饭菜,在床罩里醒来时的、梦与现实的撕裂感。自打升入高三之后我很少做梦,往常我会梦到那些或富丽堂皇的建筑或质朴简真的房子,梦到雕像,梦到油画,梦到在颜料里涂抹得很温柔的世界。而现在我虽然不能确切地记得我做了什么梦,但我知道梦里是有猫猫的。我醒过来的那一刻,噗嗤撕裂的那一声,我能捕捉到她稍纵即逝的气息,像毛茸茸的猫尾巴撩拨着我的心弦。



整整一个月,我没再见到猫猫。



取而代之的是烤串廉价却诱人的香气。鱼排牛排米排,皮蛋面筋脆骨肠,羊肉串里脊串鸡肉串,烤蘑菇烤地瓜烤鱿鱼。我每天去她的地方等她,买一串烤串,用烧烤蘸料的味道在嘴间散开的这段时间逛遍她可能会出现的每一个地点。我绝不相信猫猫会去别的地方,所有我也从不在他处浪费力气:这个如同被困在世间的白色幽灵般的少女,这里是她的地方,她只会在这里出现。



在这一个月里发生了许多事情,不仅是课桌上面明显变厚的卷子,还有我明显增高的成绩。每逢考试后发下来的榜单也好,出成绩之后四处乱传的写着数字的小纸条也好,我的名字最后那一栏分数,总是不断地上涨。我从无人过问的中等也许偏上等一些的学生,竟然在某次大考之后有幸跻身优等生之列。对于我来说这没什么变化,或许只是班主任缓和一点的语气,舍友试探性的交流,课上增多的提问点名。



只有我知道这真正意味着什么。我成绩提高的唯一原因,不是我在见不到猫猫的日子里伤心欲绝于是化悲愤为力量决心发愤图强,只是我深埋在心里的一些潜藏因子,在那段与猫猫相处的短暂时光里被引得活了过来:我开始试图与题目对话了。



这开始感觉很奇妙——一边做物理题一边笑的那些人,或者是一边写解题步骤一边自言自语的那些人,被称为天才也被称为疯子。在外人看来我还比较正常,只是在模拟测验的午后,铺在教室的一片阳光中,会久久地盯着不会做的那些题目发呆。一直看下去,能把自己沉进粗糙的打印纸当中,将纸纤维分崩离析,墨迹抽离。然后那些书本上的公式自然而然地浮现在脑子里,腾到纸面上,与题目交涉。它们煮酒温茶,在无关人类的天上无声交谈。我始终认为所有的天人对话都是扯淡,正如懂得人类语言的只有人类,天只会和另一个星球的天对话。



最后我写下答案。起初偶尔会写下鬼画符一样的东西,到后来正确答案越来越多。我不知道我是只是一个单纯想得太多的理科生,还是一个信了巫蛊的迷信少女,但是我的成绩开始提高。唯一无效的是语文,也许古人不屑于看这些拙劣的文字,会化作一股青烟,飘回那个可以凭空飞行的时代。



我想念猫猫。



当我意识到,我多渴望她是一只真正的猫。这样我就可以把她抱在怀里吸她,撸她,摸她的小耳朵小额头小肚皮小后背,搓到她整只猫都舒坦成一摊。然后一起舒舒服服地晒太阳,晒一天——这时候,我发现我已经彻底爱上了她这个人,不是爱和她聊天,是爱她这个人。



可是现实里没有能晒一天的太阳,现实只有能和我对话的题目,干燥的越来越热的天气,永远忙不完的事情。傍晚的烤串,入夜了点点升起的萤火虫,她不会出现的秋千,她不会出现的谷堆,她不会出现的墙头,她不会出现的小巷。



她不会出现的,我作为一个高三生的,平凡普通再日常不过了的生活。





{\centering\section*{(四)}}





所以读到这里,你应该猜得到,在那架遭受风吹雨打而日益破旧的秋千上再次看到猫猫的时候,我是有多么地激动和惊喜。



或许那天是心情最为低落的时候。说要低落也谈不上,只是淡漠的生活中一丝丝的阴霾,遮住了本来就不会出现太阳的天空。



当天体测。说到底从上学开始,我最讨厌的课程是体育课没有之一。小的时候,当别的小朋友在操场上奔跑,尖声大笑,因为解散的游戏时光而欢快地扎堆时,我抱紧怀里的人体解剖学,靠在冷而硬的石柱上,木木地看着头骨的结构种种。说到底,学校还是按某些人喜欢的形状去培育学生,小孩子就是要跑、跳,叽里呱啦地乱叫,而不能提前拥有中学生的忧愁,趴在洒了一把阳光的教室桌椅上,怔怔地看风吹动的鹅黄色窗帘。



而长大之后,成绩开始占据我的头脑。在成人的字典里,这个词语约等于自由。而那时的我竟荒唐地相信,如萤扑火,最终发现不过是自取灭亡,被沉重的枷锁牢牢缚住,痛苦到无法自拔地度过了一段没有光的日子。所以之后的我开始不上大晚,开始逃避一些有的没的,学着过得更舒服一点。我知道很多人恨铁不成钢,说你看那孩子之前聪明得要命,后来呢,不努力还是泯然众人矣。



但那又能怎样呢?



体育测验已经是之后不会再经历了的,至少是高中的体育测验。我就要毕业了呀。



说来可笑,那时我十六七岁光景,竟错误地以为可以掌控自己的人生,仿若占星术一样,认为自己可以窥见之后命运的运行轨迹。我认为至少生活的很大一部分都是被我玩弄在鼓掌之中的——直到猫猫出现那天。她白色得近乎透明、每每让我怀疑此刻肉身是否存在的身影,打破了我费尽几十年调出来的中庸灰色。恰到好处的黑和恰到好处的白平衡,在白色瓷盘上抹开,变得均一。灰色是我舒适的安全区,可白色美到惊人,我无法移开视线。



所以那天我再见到猫猫的时候,我着了魔一样地向她走去。我打破了刚刚好的一米距离,用沾着雷雨刚拍打过的泥水的板鞋鞋底踏碎了凝滞的空气。我知道我会毁了这样均衡的博弈,毁掉那些由璀璨如星辰的哲学思想填补的夜,但我想要碰触她的感情如此强烈,以至于我将那一切都抛在脑后。



现在回想起来,她没有躲开。



我漾满体温的怀抱,生涩紧张的力度,雨水刚过的泥土气息和新鲜的汗水混合发酵——也许还有嘈杂的人群说话声,蝉微弱的鸣叫,在矮墙那边悄悄露出脑袋的流浪布偶猫,我因为没吃晚饭而揪起来疼的胃,因为奔跑而微微加速跳动的心脏,因为那比雷雨还要猛烈得多的滚烫爱意而变得滚烫的唇。



我写不下去了。





{\centering\section*{(五)}}





之后——我们发生了什么?



我很期待和你讲一讲那之后发生的事,如你所愿会发生的事。被打破的黑与白在新的平衡点上合二为一,透明的白色少女被沉重的黑色少女拉回地面上,不可承受的生命之轻与重经过调谐都变得可以共存。我想和你说起这件事,说起猫猫时她不再是我生命中的一段奇遇,而是我生活里不可或缺的一部分,是每天早上醒来时往右侧一摸就能搂进怀里的温度。



我很期待我们会有未来。像我的名字,魏莱。未来尽是未来之事,可要是有她,我觉得没关系。淡漠地微笑着、仿佛浮于尘世之外的、不知由何方而来的神秘少女,以及每天过着一成不变生活的、想得要比同龄人多得多、却在心里暗暗渴望璀璨和危险的女高中生,这样的搭配听起来有些奇怪,可未尝不行。倘若我们活不过歧视、排挤、抵触与生活的不易,那便逃亡,手牵着手,逃去任何地方都在所不辞,在路上丢掉我十几年来获得的一切都甘之如饴。倘若我们活得到老,那就一起在下雨的夜里窝在床上相拥取暖,身边一只养了十几年的白猫,会钻进被窝里喵喵叫。



我很期待同她一起流浪,一起颠沛流离。也想和她一起过日子,柴米油盐,焖一锅腊肠煲仔饭,开锅的刹那香气扑鼻,温柔地氤氲房间里的每个角落。



我很期待去进一步了解猫猫。想知道她更喜欢柠檬味的酸奶还是西柚味的,喜不喜欢德芙那样甜腻得有点过分的巧克力糖作为情人节礼物,在七夕的时候更想收到花束还是冰淇淋筒。想知道她进门是先迈左脚还是先迈右脚,洗澡的时候先抹沐浴露还是先洗头发,睡觉的时候喜欢向右侧躺还是左侧。我期待去了解这些,这些无足轻重的小事,想把我记忆里那个飘忽的白色身影补全。



但是。



我接下来要给你讲一讲那之后发生的事。



她在我亲吻她的那个午夜,饮下一杯毒药,刚刚好不多不少是致死的量。想想那是她第一次提出同我对饮一杯,给我的杯子里面斟满雪碧,小小的气泡在透明的液体里浮上来,炸开身躯,像自爆的兵士。



自那之后,我再也没见过猫猫。她应该是故意离开的,如果不离开我,未免对我太过残忍了些。我不愿意去猜测那具干净透明的身躯要遭受怎样的坏死与溃烂,只是等着新闻发布的那一天。



可我始终觉得,猫猫是代替我去死的。





{\centering\section*{(六)}}





翌日下午,我从班主任手里要回了我的志愿单,用那支磨秃了的黑色水笔,重重地划掉了我本来想去的,那个艺术与理性交织的神秘学科。然后随便填了个什么专业,具体是什么,我也不是记得很清楚。我只是记得那天我走出教学楼的时候,墙体投下的黑色阴影和天空洒下的炽热白色阳光一同将我吞没。黑和白在我眼前无限延伸,交融成一片朦朦胧胧的灰色:这灰色将伴我度过余生。

\end{document}
