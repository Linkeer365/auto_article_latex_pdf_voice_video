\documentclass{article}
\usepackage[utf8]{inputenc}
\usepackage{ctex}
\usepackage{url}

% support Chinese Chars %
\newfontfamily\urlfontfamily{FandolSong-Regular}
\def\UrlFont{\urlfontfamily}

\title{【希海】但愿月不明}
\author{TP02型}
\date{20240203}

\maketitle
\url{https://www.pixiv.net/novel/show.php?id=21510951}
\newpage

\begin{document}
\CJKfamily{zhkai}

% : comm which is only poems edit in-use %


\Large



{\centering\section*{1}}





上午的时候下过一场大雨,青石板铺成的官道都滑溜溜没法好好走人,何况面前这条树枝层层遮掩,满是泥泞的小路。



椎名立希随意摸出几块铜板付车钱,接下来的路她比谁都熟悉。将鞋换作适宜淌水的木屐,紧紧草绳充当的腰带,短衣短褂,外搭一条竹笠。与其说是利落不如称为狼狈。唯有腰间别着的一把肋差,一把打刀展示主人并非随处可见的浪人剑客。



立希小心翼翼地扒开茂密的杂草,躲开盘踞的树根。约莫两小时过去,立希终于看见仿佛要被森林埋藏住的几间小屋,小屋围出局促的院落。隐约能看见丝丝缕缕的青烟。立希知道,那人又在趁自己出远门时煮酒喝。



这时候敲正门肯定不会有人来应,她借院落附近的小山丘,俯冲加速,顺低矮的篱笆墙,三步之内蹬进去。木屐落地发出沉闷的响声。煮酒的青年女性像丝毫没有感知到一般,继续给火炉扇风。立希右手搭在刀柄上,压低身形,按照老师说过的一切去行动。距离越来越近,还差一截手臂远时,陡然拔刀向对方脖颈砍去。



精钢铸成的刀刃能将空气一分为二,刀与刀相碰的声音撕开宁静月夜的一道豁口。八幡海铃扔掉蒲扇,轻巧地卸劲,别开立希的打刀:“差点去见我祖父。”



“但凡你愿意屈尊来给我开门,也不至于是这种情况。”立希无奈地叹气,面前这人明明是自己的剑术老师,却很少表现出师长应有的威严,那副笑脸,抑或波澜不惊的背后究竟是什么。



刹那间海铃的短刀已经回到原位:“先不说这个,你最近进步很大嘛。”



“已经三个月,我去了一趟江户。今年没有下雪。可惜。”立希摘下竹笠,继续汇报,“城东新开一家道馆,人满为患,最近人们都抢着要学一点防身的技术。”



“你试过了吗?”



“交手两三回,虽然不如那个人,但很有几分本事。”



“哦,”海铃掏出热酒用的木桶,倒满一层热水,“怪不得,刚刚那种持刀方法是这么学来的——每次游历都有收获,这样很好。”



立希坐在半人高的木凳上抿一口温热的酒,粮食发酵后的味道和酸涩的青梅混在一起,立希不禁咋舌:“你泡酒的技术还是一点长进都没有。”



海铃并没有为她冒犯的话语生气,只是笑:“这不是特意在等你。好了,你还是输我一招,把东西给我吧。”



立希不耐烦地点头,说到底自己为什么要跟这人定下“谁输就要给别人一件随身物品”的约定?只出账不进账的生活谁受得了。她放下短褂里的小盒子。海铃刚打开就连声叫好,果然是那家的腌青梅泡酒好喝。你等着,明天给你露一手。



对方毕竟是老师,立希深知无法拒绝,不善嘴上功夫的她只想用刀剑解释许多问题,可她的老师乃是曾经的剑豪八幡海铃,刀势稳扎稳打,出手极细极密。立希跟她请教已有四年,除去游历的一年,剩下的时间都在琢磨海铃的剑招,也只敢觉得习到皮毛。



如今海铃还未动出山的念头,偶有出手也绝非凡物。立希的心境也被一次次失利催促成长,换常人来早想打道回府,可她有不得不精进自身的理由。



海铃隔老远招呼她别惦记吹冷风,烫好的酒可以进来慢慢喝。海铃看向窗外,明月高悬于空中,院落如撒满盐般明亮:“你回来的时候总是盈月。”



立希放好刀:“是吗?我没多注意。”



海铃笑笑,摊开一份江户附近的地图:“不说这些,告诉我你看到的情况。”



立希一五一十向她说,讲故事一般。江户内城守卫依旧森严,四年多前那起兵变的痕迹了然无踪,城门外聚集一些流民。向外走,一直到凌川,周围几座小城,好几家店的柜台上蒙层细密的黄沙,买不到米或者鱼。



海铃一边听一边不住地圈画。按理来说,她本应不关心这些事,安稳过好自给自足的生活,偶尔应付一下不知天高地厚的挑战者即可。立希打量她的动作,呷口凉酒,等待海铃的问题。海铃笔尖直指大河边的小城:“如果让你……”



这地方我去过。土地肥沃,农业发达,又依山傍水,恐怕城里还有几个成规模的粮仓。”立希皱眉,“我不会一直围它,把补给线拉得太长。我可能会一鼓作气攻下来——周围是很多森林,城中房屋也基本是木质结构。尝试用火攻。”



海铃扬起手,杯中酒尽数泼在那座城上:“用旁边的河怎么样?”



立希思索片刻,摇摇头:“太冒险,洪水不分敌友,而且它带来的后果远比想象中严重。”



海铃似乎是认可她的观点,卷好地图示意今天到此结束。平日在林中,每日除开练刀与基本功,便是研习晦涩难懂的兵书。用兵一事,皆是诡道。立希仅有的兵法功底,只是当年丰川祥子提点过的几招,不说也罢。她时常怀疑找到一位除开刀技什么都教的老师。



每年冬日中旬,立希总会被打发出去游历三个多月,这期间的风土人情都要好好记住,一一汇报给她。海铃看穿立希的疑惑:“今天的问题是不是太简单?”



“的确,没有工商和礼俗相关的部分。”



“哈哈,所以我只是想告诉你——最原本的道理。”海铃喝尽壶中冷酒,“情报。如果你不曾去过那儿,不可能这样清楚利落地判明攻城计策,对吧?”



“派出探子,收买军士,反间计。面对势均力敌的对手,要不惜一切代价弄到有价值的情报。”海铃并未继续往下说,拍拍手兀自去洗脸。



立希心有所悟,微微向她俯身表示受教。她提刀出门,站定在月光下,拔刀轻盈挥舞,好似要斩断所有的月光。海铃教她的招式,像是乐曲一般被她弹奏出来。她不知道海铃有没有在看,只是专注于手中的刀。



似有月光相助,今日挥刀时总觉得抓住海铃的几分轻盈。先前读过本名字业已被时间蚕食的古书,提到有些剑豪用招“天心月圆”,极满极盈,对手抓不住破绽,亦无处可逃。立希至今为止各处踢馆,见过最接近这般评价之人,也就是寥寥几位,其中当然有海铃。



月亮逐渐下移,立希走到院边,一盆凉水,从头往下浇一遍。这里可不是什么桃花源,没有瀑布可以冲澡。



又是一月训练过去,海铃保持每天上午敦促立希练习基本功,下午看她用木刀砍稻草人,晚上喝酒询问她如何用兵的节奏——当然不忘每十天打发她去采买。立希每天比幕府将军还忙碌,一到晚上睡着时就像尸体。今日亦是,虽然为盈月之日,她却没有丝毫观赏的闲心。寝至半夜,她却突然惊醒。窗户大开,月光尽数洒入,甚至不用点灯。



立希让眼睛适应黑暗,翻身下床。她听见院中传来兵器破空的声音,不用想都知道是哪位。她还没走近,海铃已察觉到来人。就算如此,海铃并未停下手中的剑招。那是立希未曾见过的刀法,连绵不绝,暗藏杀招,却能看出后继无力,戛然而止。



立希久久回味刚才的情景,直到海铃招呼她走近。她笑着说,你也看见了吧?相当精妙,的确是绝技。



“嗯……但我总觉得它不完整。”



惊喜与诧异爬上她的眉头:“没错。这套剑招我早在十余年前就学到,可习至现在,总有一部分动作……”



“竟然还有让你学上十年都无法精通的招式?”



“我认为那是一种条件上的缺失。”海铃怔住,突然说,“既然让你看完了,那我就教给你。”



海铃握紧刀柄,苍白的光芒投射出她的脸,翠绿色、宛如某种鸟类的眼睛。立希鬼使神差地拔刀。天际闪出第一缕晨光时,海铃解下刀鞘,捡起脚边的酒葫芦:“你学得很快。”



立希这才从震悚感中回过神,月夜下的自己更像一位追寻剑道的疯子。海铃看着她,眼神又好像穿过她去看别人:“都说练剑如练心——只有心中如湖面般平稳的人才能斩出那一刀。但我认为这太过片面,追求必杀的痴人与疯子,手中也握有刀的真理。”



当然,我并不是鼓励你去钻牛角尖。我只想说,执着一些,有目标是好事。海铃补充道。接着她转身就走,“啪”一下扣上房门。立希和衣睡到晌午。海铃在院中忙活,“叮叮当当”敲敲打打,似是在布置什么东西。看到她来,海铃头也不抬:“来得正好,去帮我拿洒扫的工具来。”



立希照办,蹲下来观察她的动作,你这是在干什么?



她的语调没甚起伏:“你已经可以离开这里。”要求突然,却也并非无法预知。“你说过,你是兵变那时逃出来的——虽然叛乱的结果只有那帮人才知道,可我觉得,丰川幕府大概是要命不久矣。原本的继承人失踪,新上任的将军能力也不足以稳固人心。”



“所以,你的意思是——”



“立希,你的能力不应被埋没,这是最好的时候。”



“那你呢?”立希下意识问。



海铃从怀中掏出布袋:“我也不会坐以待毙。”



她打开袋子,里头是一长一短两把刀,刀鞘通体漆黑,间有闪烁银光的金属环点缀。抽刀出鞘,利落的切割声令立希心中一惊。海铃默默注视立希虚虚挥刀:“有这个的话,说不定哪天你能斩断月亮。”



的确是两把好刀,适合已然成熟的武士。立希郑重地道谢,二人一起将面前的神龛擦拭干净。被当作武士的伟大者的源赖朝静静坐于其中,身穿的盔甲就像鬼兵俑。她们之间没有那么多繁文缛节,因此只是举起刀拜了几拜,敬过一遍酒。立希咬到杯底的青梅,这回是甜的。



结束这一切后,海铃带她去耳房,杂七杂八堆积不少柳条箱。她故作深沉地说,你知道为什么我们一直都不缺钱吗?”立希摇头。



“每个找我挑战又失败的人,我就让他们留下一件随身物品——嗯,不少值钱的。甚至还有人直接把佩刀都留下来,说要拜师之类的话。”她顿住一会,“我让他们走,并且别忘记把刀留下。”



这话听得立希心里有些不自在,剑豪唯一的徒弟的身份大于自己的能力(她自认如此),海铃游刃有余的样子让她觉得目标高不可攀。她很想问海铃,问身边唯一的理解者:我有让你骄傲吗,能挺直脊背吗?最终没能问出口。



海铃搬出木箱,空空如也。她拍拍脑门,又拿来药葫芦和干粮一类的东西塞满一半:“我们可以一起走一段路。”



“你终于要出山?”



“没错,应一位故人的邀请。”海铃一副欲言又止的样子。立希没多在意,耳房如同她说的一般,收集不少珍贵的宝物。她一边哀叹这些人的不自量力,一边拿出些钱票。



海铃抱怨:“这些人老是送些花瓶书法之类的东西附庸风雅,武士明明是刀尖舔血的行当。这也是他们输给我的原因。”



立希心有戚戚焉。尽管满是嫌弃,海铃依旧找出一枚玉佩递给她,那人说是在江户最负盛名的神社赐福过,能保平安,你收下吧。立希不想欠她人情:“……你不自己用?



海铃露出饶有兴味的表情:“哦,我有更好的。”



立希无言以对。二人吃完迟来的午饭,起身向山下出发。海铃锁上院落的门时,立希不住往回瞄几眼。她本以为自己是坚定的不留恋过去的人,到头来心里却涌起一阵逝去的悲伤,一如她离开江户那一年。



“我们还会回来吗?”



“可能,我无法保证。也许不是‘我们’。”海铃拍拍她的肩,“就像浦岛太郎的故事一样。”



到达渡口的时候是黄昏时分,最后一趟渡船已看不见踪影。二人不得不找一家就近的旅馆投宿。海铃永远会多给旅店的伙计一点钱。



喝过一点粗茶,立希倒在床上,月光透过窗棂走进来。十六的月亮与十五的月亮都是一般晶莹,真是很有意思的事物,晦明变化,人们据此制定历法。海铃突然说,你决定去哪?



立希回答她要去找灯,世道不安稳,去哪都让人不放心。海铃沉默许久:“我说过,我不希望你的才华被埋没——你可以成为将帅,或者大名。只要你想的话。”



“你那故人还真是来头不小。但是,我早就定好目标……等我办完这些,给你写信如何?”



海铃点点头,不置可否。很少有这般悠闲的夜晚,立希撑起身子关窗,她可没有对月赋诗的闲心,闭上眼,多少遵循师生间的礼仪,睡得规规矩矩。可某人倒是毫无顾虑地靠过来,头碰头,肩并肩,不知东方之既白。



起早赶船出发,一位要往北,一位要往南,临别时都没说什么。海铃挥挥手:“嗯……祝你、心想事成。”





{\centering\section*{2}}





椎名立希日夜兼程,平安京无论何时皆是灯火通明。她按一年前寄来的短信寻觅学者的白塔。它耸立在城东一角,周围少有商号集市,仅有几家茶店旅馆,以及书屋和住宅。立希不由自主放慢脚步,这里连门房都没有,一切都很安静,几只长尾山雀窥探着来人。主街上到处可见的浪人于此销声匿迹,是明白这里不会有油水可捞吗?



这个疑问霎时间被解答。刀柄握在手中,却来不及拔出,身后传来森森寒意,再往前半步,即会身首分离。立希看向树杈,上面的山雀甚至都好好立在原地。像忍者一样的人发话,问她从哪里来。



不带敬语和请词,说话习惯随意,看来是隐于闹市中的高手。立希模仿八幡海铃,仅凭只言片语来揣摩对方。



立希无意隐瞒:“我来此地意为寻到一位修行的学者——姓高松,名灯。敢问阁下……”



刀刃逼近几分:“来找灯?”



难道是灯的仇人。她想。估摸二人的实力差距,如果现在出刀,大概能断这刺客一臂,到时再从长计议——思考尚未中断,传来少女熟悉且诧异的喊声:“乐奈……唉?”



对峙中的二人保持同一个奇异的姿势向上看,身着浅色皮袄,头发长至肩部的灯与她们对视。冬日的微风吹过三人,长尾山雀亦不见踪影。



总之是被邀请进塔喝杯热茶。这里应该是从事某项专门研究的学者的居住地,可静悄悄的,只听见她们的脚步声。灯说,冬天的时候,其他人一般都会获准回故乡探亲。立希慌忙道歉,刺客嘟囔着,不怀好意——



立希想给这人脑门上来一拳。灯推开门,映入眼帘的即两座摇摇欲坠的知识之塔,真乃塔上加塔,塔内有塔。灯说来不及收拾了。多余的那人轻巧地绕过重重障碍,步伐就像猫一般。



灯大梦初醒,连忙向立希介绍,她是要乐奈,忍者世家的女儿,总之一直跟着自己,像要时常照顾的守卫。



立希心中并不以“要乐奈”,而是以“野猫”来称呼对方。她看着急忙端茶的灯,感慨道,头发变长好多。



“嗯,这里不愧是大城市,有很多闻所未闻的典籍,完全看不完。”答非所问,却是立希熟悉的高松灯。没来由的释然聚集在她心中,四年多了。立希一口饮尽淡茶,里头似是添入些龙眼与盐,口味奇特。



尽管觉得对不起灯,可时间宝贵,立希阐明来意。一路走来,但见不少流民行居附近,城中有军士戒严——以及身居此地,可能多有不知,闹市中尽是些奸诈的浪人子弟,治安大不如前。



综上所述,随时会爆发战争。就看是中部以东的农民先坐不住,还是那些重兵在手的大名想要夺权。



灯愣怔在原地。立希不免感到无奈,她们都是历经叛乱之人,距死亡仅一步之遥,丰川幕府气数将尽,不是一位武士,一位忍者,一名研究天文历法的学者可以解决的。不过,灯是强韧的人。她很快意识到立希所言不虚,遂留来人暂住几夜,开始打点行装。



乐奈不知何时消失不见,只在原地留下个小纸人。不出半炷香的功夫,她闪身出现,手边多提黑色布包。



“你也要一起?”



“嗯,城内没什么意思。”



灯劝诫武士:“请不要怀疑乐奈……这几年一直都是她义务巡逻此处。”



“这地方是我原先的……”有风吹过,油灯陡然熄灭。三人不发一言,外头传来神社报时敲响的悠长钟声。



起程时立希考虑繁多,最好向江户方向走近些,马车要选用最普通的,另用一辆拉运行李。即使如此,刚出城三十里,就有几个不知天高地厚的浪人拦车。立希抽刀收刀,一息之间,地府又多几位孤魂野鬼。温热的血圈画周围的树木,她身上却未曾沾到丁点。灯心惊胆战,好像没有料到立希没给他们留下丝毫活路。



武士只是想起老师的一句话,那也是在剿灭山贼的过程中听见的。那人说:“好人不长寿,祸害遗千年。”



她学到的第一堂课是要扔掉无用的怜悯,刀剑无眼。



与学者不同,忍者只是侧着脑袋:“这样很有趣。”



灯看上去都快哭出来。立希摸摸后脑勺,不知道该怎么将自己在格斗中学到的道理告诉她,如果能继承一点老师的口才就好。马车行至大道,终于没那么颠簸,迎面而来许多拉货的骡车,大都驮着大米与面粉,偶有运盐的官车,守卫一个个凶神恶煞,盔甲上有丰川家的标志。这时暂且不用担忧山贼浪人,立希端坐在马车内,靠窗的座位让给灯。路上尘土飞扬,越往北走,天气越冷。立希迷迷糊糊,想着到下座城镇一定得买件羽织,既有风度又有温度。



她怀抱着刀,小睡二十分钟,间或听见马蹄铁与铃铛的声音。灯关切地问她这几年立希去干了什么。立希曾试图模仿长崎素世的来信,引经据典,文采飞扬,最后也只好承认村野武夫的本质,匆匆写一些日志般的短信,三言两语交代自己吃什么去了哪里。



她沉吟一刻:“学到一点防身的技艺,活下去够用了。”



“立希真是很努力啊……我跟着乐奈学了好久忍者的技巧,遇到那些浪人还是只能考虑逃跑的事。



“没关系。每个人都有每个人擅长的领域,灯的研究怎么样?”



提到这个,她的脸上绽放出笑容:“我发现很多有启示性的东西!祥子留下来的那份历法就快要——哎,抱歉。”



立希弹一下刀鞘:“我无所谓的。”



一直沉默的乐奈突然说,她听见奇怪的声音。立希探身出去,她相信忍者的感知力。她注意到前方的大车,四面都蒙上黑布。此时她们已拐上官道旁边的路,仅容一辆车通行。立希坐上驾驶位,放慢车速,一边前进一边观察。对方斗笠上的家徽勾起她的兴趣:明黄色却不显张扬的花朵,镶在大片湖蓝色中。她深感熟悉,一时间却想不起来。立希发呆的间隙,突然听见钩索破空的声音,紧接着即是马匹受惊嘶叫,以及树枝折断的声音。



灯还没来得及反应,乐奈一个闪身翻出去,扒在马车顶上。立希尽全力拉稳嚼头,才看见面前的大车已濒临侧翻。立希一边感慨山贼的装备精良,一边悲愤地抱怨这山崩地裂的运气。



她提起刀,稳住马车。眼见不知侍奉哪家大名的武士已与歹人缠斗许久,她选择绕至侧面,先解决几个看到她们的人——他们手中无不是带刀鞘刀柄的武士刀。哪怕腰带再破烂,刀鞘再黯淡,将它带出去已是有家主的象征。立希还保存那两把从丰川家得到的刀,尽管它们已经陈旧不堪。



轻巧地拔开直刺来的刀刃,刀背挡住一道斜劈。对方堪堪防住胁差,小腹处已被打刀划破出大洞,不能给小孩子看的画面。传奇中描绘的一击制敌的场面,放在这种混战中已然是天方夜谭。



立希不由自主将对方与海铃作比较,刀法迟滞,后劲不足,反应也是如枯死的树木般缓慢。不出一会,立希飞奔到两辆车前的地方,帮助陷入苦战中的侍从们。这更坚定她心中的想法:两方都是受过训练的武士,而非浪人。她本还有些好奇押运的是何物,现在只剩卷入黑吃黑现场的无奈。



她环顾一圈,没看见领头者。她很快听见男人的咒骂,循声望去,歹徒抓住一位粉色头发的女性。几个家丁打扮的人喊她“掌柜”。立希稍有吃惊地看着全程隐身的老板,她似乎一直单独骑马跟着货物。女人甩来恳求的眼神,大概是把立希当成什么天降紫微星。



立希颇为无奈,的确要先解决这帮“山贼”。这事情不难办,她先询问对方的要求,叽里呱啦一大堆。立希听得心烦,无非就是权力者争夺倾轧的轮回,不知又要毁掉多少人的生活。她轻轻挥手。所有人都没反应过来,比粉红更鲜艳的血色占据所有人的目光,匕首在乐奈手中仿佛变作剔骨刀。立希看得脊背发寒,冷汗直流。乐奈似乎擅长在一瞬间内理解人的情态含义,作为代价的是宏观感知与人际交往的缺失。



立希和女人在尸堆中握手,哪边的人都有。立希不耐烦地说:“举手之劳。”



“哪里,帮大忙了——鄙人小姓千早,名爱音。是在这块地方干行商的小老百姓。”



“什么,行商——那你知道一种糕点吗?绿色的,一开始吃进去有点苦,但是很快就会变甜。”



灯伸出手:“啊,乐奈……那个是我从传教士的住地搞到的,本土应该没有吧。”



千早爱音拍拍手,你们找对人了,本人小本生意,原先一直卖些海外来的酒品饮食。她招呼家丁搬来有一个脑袋大小的盒子,里头用耐冲撞的布绒包上一层又一层:“这就是您要的那种糕点——它好像被称为一种茶的名字。”



乐奈满心满眼都在礼物上。立希更在意神秘的车,以及侍从们的家主,她注意到商人的眼神不住往黑布里头瞟。



她小声问:“你听见的声音是脚步声?”



“不。更明显的是金属碰撞的声音。”



如果只是金属原料,或者普通的工具,何必遮遮掩掩。立希状若无物地问,你们要去哪里?爱音斟酌词句,说要去春名。那是一座江户旁的中型城池。



海铃教她的地理知识终于派上用场,她脱口而出:“这是长崎家的地盘啊。”



“难不成恩人们也要去那儿?这可是个好选择,长崎家近几年异军突起,家大业大。”



的确有这样的消息,素世接任家主后继承上一代的遗志,吞并好几个周围的家族。人们都说“筑前的白梅”重现于世,也有人说丰川幕府养虎为患——可只有极少数人明白,素世不可能对丰川拔刀相向。想到这里,苦闷的浪潮席卷立希与灯的内心。



爱音有自己的算盘,这趟行程实在困难重重,如能捞到几位保镖也是极好的。殊不知立希也正思索如何印证猜想。二人诡异地一拍即合。



行至下一座城镇,立希替灯去买书,顺带到驿站寄信。她多付些钱,嘱托用最好的马。信送给长崎素世,简短说明她们的情况,以及前来投奔的想法。她按照灯给的书单,去一家从里到外散发灰尘味的商号,如豆般的灯火映照出满柜的书卷。她一边咳一边清点书目。



爱音正张罗喂给马的草料,比人吃饭还讲究。车大咧咧停在院中。敲过不知第几更钟时,万籁俱寂,立希等周围几人都睡熟,翻身下床,从窗户那边走,顺外墙向下,如猫般着地。她没有点灯,让双眼先造应黑暗。挤进车内,并非是她想象中的情景:仅仅只有几垛稻草,仿佛在嘲弄来人。



好巧不巧,外面响起由远及近的脚步声,以及侍卫不满的咕咙声。侍卫举灯四处照照,一无所获,咒骂声更甚。立希待脚步声远去,才敢从最里头的稻草中探出头,庆幸自己没带油灯来。



她赶忙下车,轻手轻脚,本想马上回去假装无事发生,可与生俱来的固执令她调转脚步。很快,停在公共区域的篷车引起她的注意。它的布置太过普通,油布都是陈旧的黄色。立希慢慢靠近马车,地上散落细碎的残骸一类的东西。她俯身捡起一片,在月光下闪烁暗金色的光芒,竟有些像黄瓜花的花瓣——毫无疑问,这是谷壳。



立希心下一动,谁会吃这样好的麦子?她轻揭开油布。商人押运的东西原原本本暴露在她面前。立希看着寒光闪闪的甲胄,每一件都是上好用料。对方,或者说素世的意思昭然若揭,既然有派出自家武士陪同护送,不说是其授意,至少也是从中有所谋划。



历代幕府的法令都说:私筑、私藏、私运甲胄都乃死罪。原因无他,至重弩广泛应用前,身着重甲的士兵一往无前,刀剑难以在瞬息之内斩断铁条。



立希深呼吸,再深呼吸,装作一切尚未发生一般退出去,斜靠在旅馆外的一角。明月依旧,过往却不再。之后的行程,立希尽职尽责,点灯守夜。爱音亦没有亏待几位恩人,好酒好菜招待着。她似乎乐于与灯交流。立希偶尔听到一些,总是灯眉飞色舞地讲有异国风情的冒险故事:一衣带水的邻国的“侠客”,大漠中的驼队,传说中出海的剑豪宫本武藏。灯像一位说书人。



立希得到件上好的羽织,还有节日才会用上的振袖。她收好,叠放在箱中,拿出一卷兵书细细琢磨。海铃的教诲早已烂熟于心,还有一句话久久回荡在耳边——她说过,不希望我的才能被埋没……如果我真的有才能,那么……她想到椎名真希,浪人口中的传奇。



临近城下,在最近的酒铺稍作休整。千早爱音揣来一口袋白银:“这算是封口费,感谢你的守口如瓶。”



椎名立希感受着袖口藏的短刀的冰冷触感,当然不难发现,立希本也无意隐瞒。



“长崎家的现任家主,是我的一位故友。我更想当面向她问明白。”



爱音点点头:“这样最好。”



等离开武士的视线范围,爱音才长舒一口气。立希那充满探寻的目光,天生具备压迫感。反应过来时浑身都在发抖,发现似乎有人偷看过货物,爱音几乎要晕过去。私自运甲乃是砍头大罪,如若被对方发现少了货物,也逃不开死亡的命运。爱音亦不想同那帮五大三粗的侍卫说话,她的梦想历来被认为只是小孩子的玩笑,摆在所有人面前的只是冰冷的,即将开战的现实。



所幸立希不知怎的选择沉默。爱音经营商号多年,难得看见有眼力见的武夫,无奈中又有几分感激。



守门的士兵反复检查文牒和信上的印章,放她们从侧门进,直通家主住宅。对立希来说,遇见丰川祥子只需质问一顿,遇见长崎素世还真棘手,这是认知方式上的问题。



负责通报的侍卫拿走爱音手中的委托状。几人百无聊赖地等在门外。灯与爱音局促不安地东走走西走走,乐奈的眼神跳跃在房梁和墙上的挂画之间。立希害怕她下一秒会出现在屋顶。



没过多久,出来两名侍女,恭恭敬敬地示意她们解下武器。爱音交出把镶金戴银的匕首,立希看她步伐虚浮,可怜这刀只能当作撑场面的道具。乐奈甚至还摸出吹火筒和六角钉。



立希思索半天,尽管屋内升起火炉,她仍披上蓝绿色的羽织。素世看上去与四年前并无多大差别,岁月的痕迹仿佛就那么被湖水吸走。见到立希,她停下手中的笔。



“你来得比我想象中早。没来得及准备宴会为你们接风洗尘。”



立希简单作揖,道声“好久不见”。素世指指旁边的坐垫。二人相对正坐。有人端上两杯浓茶。似是为打消武士的疑虑,素世浅浅品味几口:“晚上再给你准备酒吧。”



立希从中尝出生姜的味道,除此之外,应该还有银杏和薄荷。她想,这种茶给自己的粗舌头喝太过浪费。



“你跟我说过,正跟从天门斋修习武士之道?如果能够学有所成——”



“请等一下。”立希深觉这人像验收课业的家长,怎么,当过几年家主把人都腌入味?“你就直说,需要我们做什么。”



素世的眼神偏移开来,立希终究不会明白其中蕴含的些许眷恋。她说,如你所见,情况一直不好。”



“不好。”立希重复她的话,“按你的意思……是决定站在丰川那边。”



素世轻咬下嘴唇,面对人生岔路她总是举棋不定,力图两全,可每次都事与愿违。上任家主,也是她的母亲,离世前忏悔自己对她的疏于照料,更加遗憾留给女儿一盘散沙的岛内格局。对此素世只是轻轻握住母亲的手。



“我没有悲悯天下的闲心——武士遵循家主的教诲。你若决定了,最好马上去做。”立希摊开双手,手掌向上,“我也有要求。借助长崎家的力量,我想找到一位浪人。”



“我当然会不遗余力地帮你……但乱世之中,寻人如同大海捞针。”



立希知道她在担心什么:“我明白的,毕竟我也为之努力好几年,杳无音讯。”



“晚饭后,我会派人去找你。”素世终于笑起来,如释重负一般,“等开春时,我会预留一批最好的青梅酒,办一场正式的宴会为你们接风洗尘。”



立希想起那壶口味相当一般的青梅酒,她都快忘记酒本身应有的醇厚了。所以她说,我很期待。挺奇怪,立希本来以为这人会抓住自己一番盘问,到头来却都有些无话可说,兴许是她们并未适应身份与处境的变化。



立希突然觉得冷,连打好几个喷嚏。灯连忙解下外衣披在她身上。她刚想拒绝,看到那真挚的眼神又只好放弃。



侍女引她们去各自的卧室。素世大概只与灯聊了些无关紧要的事,见人下菜碟是她的拿手好戏。



立希盘腿坐在与客栈最高规格无异的客房,感慨素世的本事。春名一带,依山傍水,土地肥沃,盐可以直接从江户调运过来。素世上任后第一件事,就是征召民工疏通河道,拓宽运河。



她悬腕提笔,武士的腕力惊人,其字力透纸背,好像不适合写信,但八幡海铃不怎么在意细节。她继续写——展信佳,有一段时日不见……简要说些行程,报个平安,倒像汇报任务。



“每年都会离开数月,如今不过一月不到,却觉得百无聊赖。”立希学着市集中替人代写书信的文人们补充道,“许是因为处境上的不同,前路不明,不知何时能够再见。不过,待紧要之事结束后,期冀能同你一起品酒赏月。”



立希越读越觉得不自在,又添上一句:“当然,你赠予我的两把佩刀,实是珍品,感激不尽。”



另寻一张纸,又誊抄一遍,洋洋洒洒占据大半版面。她叫来侍从,嘱托快马加鞭去送,海铃一人独行,恐怕早就见到故人,把酒言欢。立希叹气,吹熄桌台上的油灯,披衣出门。侍卫问起,只说趁清凉月夜,在城中转转。脚步却不由自主拐向素世待客的房间。



还没走近,即听见爱音惊讶的声音。倾耳细听,断断续续听到“船只”“出海”“粮草”等词。素世的声音则染上几分威胁的味道,音量也拔高不少:“如千早小姐所见——城中物资价格飞涨,不解决这点,您那艘船不太可能扬帆起航。”



紧接着是漫长的沉默。立希识趣地翻出窗户,沿屋檐爬下去。上次前去江户,官府下令宵禁,无福享受夜市。她左绕右拐,顺道在心中记路。日常的夜市麻雀虽小,五脏俱全。刚回过神,她手中多几块糯米点心。虽然不服气,的确要承认素世治理有方——光看这样,谁能不去期待明天。



立希继续闲晃,看过几盏花灯。古树下有两人眉飞色舞地讲故事,大多都是编排某位将军,臆想天皇哪段秘史。今天却是那位天门斋的一次战斗:那个著名的浪人剑客,向当世无双的豪杰拔刀……听说激战数十回合,进进退退,纠缠如泥中沙石,难解难分。刹那间寒光一闪,似天幕大开,原来,浪人手中的刀竟被斩成两截——却看剑豪……亦卷刃裂口,不可复用也。



“天门斋当真如同天门,坚不可摧啊……”



“不过那位挑战者也只是略输一招,再来一次结果还说不定呢。”



立希只是默默听完,故事不知几分真假,但主角是她寻找的人无疑——以浪人的身份游历四方,挑战剑豪的椎名真希。真希的事迹在武士中也是传奇,她极少数的败绩有一笔由海铃留下。立希紧紧皮制的挂带,刀鞘轻轻颤动。正因如此,我才想要战胜她。立希想。



她昂首挺胸地离开夜市。一回去就看见愁眉苦脸的爱音,她正怀抱小木箱,跟着侍女挑选房间。立希明白,萍水相逢要变成朝夕相伴,不知素世用什么手段将人拉上贼船。





{\centering\section*{3}}





不管身在何处,椎名立希都保持修行中的作息,天不亮就起床练习空挥,吃过早饭开始读书。长崎素世找到她,拜托她去盯着新招来的步兵。



二十岁上下的年轻人大多血气方刚,少把别人看在眼里,更何况一位衣着朴素的武士。监督训练的第三天,尝到立希拳脚功夫的人已经超过半数。有人拿木刀和她决斗,不出三招直接连人带刀一起扔出兵营。



直到没人怀疑她的能力,她才找来几名伍长,交代每日规定的训练内容。



立希收到海铃的回信,她应在更南方一些的区域,信封中有几片风干的樱花。春名虽以春天作名,春日却迟迟不来拜访。海铃的笔迹较她更为随意,笔走龙蛇,轻盈自在,会有人愿意收藏的水准。



她也只简单写明行程,抱怨南方酸甜口的菜式,最后用一种近乎预言的口吻,说她们终会见面。立希左看右看找不到依据,只好当成一句海铃式玩笑。



素世验收训练成果时颇为满意,调出一队出城巡逻。这几天立希经常看到车队进城,路上尘土飞扬。素世喝加盐的茶:“已经开始了,你不知道吗,小规模,不到二百人的冲突。依我看只是争夺地盘——亦足以让人提高警惕。近日爱音跑前跑后,兴建简易的民居。“



“那些石料是为了收容各处来的流民?”



“显而易见。正好修建城防工事也需要工人。”素世的眉头总算舒展开一些,“小灯设计出一种用料简单的住宅,不过只有雏形罢了。”



“我还看见丰川家的使者。果然,如果南方有人起兵反抗,你就是他们最大的祸患吧?”



“我这样选择是否正确……”



“嗯,也不是不行。的确以前就是爱自找麻烦。”立希环抱双手,“不过,祥子那家伙不是离开丰川家了?”



“即使这样,过去的情义亦没有消失。”



“反正,作为武士,我会支持你的选择。”立希带队去监督住宅的施工,一边修建一边完善房屋图纸,从中可以看出工期之紧张。她又借了笔墨,在一堆散乱的木头石块中写信,对远方的境况表示关切,诉说对战争的隐秘担忧。爱音几次想偷看,每次都换来武士令人毛骨悚然的一瞪。



工人们用绳索拉动运河上满载石块的船穴,一边拼命往前一边放声歌唱。不成曲调,词也尽是听不清的语气词。立希却听得有些恍惚,旺盛的生命力流过她的心头。春天也正慢慢前进,笼罩更往北的地方。



八幡海铃的回信久久不到,先传来的是开战的消息,立希还没来得及欣赏春名城中心盛放的樱花,素世的脸色像白角瓜,家臣在她身边围成一圈。前来汇报的军士说,是从丰川那边来的消息。



出乎所有人意料,并非是工匠或农民掀起叛乱。有人大声念出信件后半段——原来南方不知何时兴起一门修习的“法术”,领头人声称自己乃是盈月意志的代行者。他们供奉一位类似辉夜姬的大神,都从月亮上来。与辉夜姬的故事不同的是,这位神只在满月时才会现身为代表盈月的神明。本国神话,讲究处处有灵,一花一草间皆寄居它们的庇护神。可单象征满月的神,并未有人听说过。四处跑商,见多识广的爱音一拍脑门:“密、密教?像是那帮海对头来的传教士一样。”



不管怎样,教徒们已经迅猛地占据南部的一席之地。领头的代行人更加神秘,只在盈月的几天出现,也从未露出过真面目。



大致交换完情报,立希只是在琢磨沙盘,向南几百里远的凌川小城需加固城防。作为前哨站,守住它的意义重大。



暂时没她什么事,她上街随意走动,心中不免一阵复杂的心绪。令人心弛神往的和平并未停留太久,只要本国维持四分五裂的现状一天,战争的车轮就不会停转一天。沉睡她记忆深处的是一幅全城戒严,火光冲天的图画。



转瞬间,她只是看到摊位上的木制小鸟。机关翼一开一合能飞上半柱香时间,机巧总是给人们带来新奇玩意儿。立希没什么犹豫地买下两只,灯一只野猫一只。



去的时候野猫正削木棍,灯面前摊开满桌的草稿纸,每张都密密麻麻写满无数算式。武士没有系统进修过数理天文,只觉比最晦涩的兵书还难懂。她轻咳几声:“休息一会吧。”



灯如梦初醒地叹气。立希问,我打扰到你了吗?“不、并没有……立希有什么紧急的事吗?”



立希举起手中马兰草串成的绳子,灯的眼中放射出光芒,连忙接过木头和金属织成的鸟。东奈摸来摸去:有趣。”



灯试探性抬它的翅膀,就连羽毛的尾端也根根分明。



“我就要暂时离开这里,很快回来。”立希唯唯诺诺地说,不太敢去看灯的眼睛。



出乎她意料的是,灯并没有露出过分哀伤的眼神。学者露出理解般的笑容:“没关系,我会等着的。我知道你在做应当去做的事情……我只是在想,我没办法帮上忙。”



得到灯的理解,对她来说是意外之喜。她胡乱挥手:我觉得你不会喜欢我要去做的事。灯坚定地摇摇头:并不是如此,这就跟翻越一座山一样。只有坚定的人才能到达山顶。



听完这些,立希动身出发。她试图理解,想了半天只觉得和平需要力量去守护,唯有强者才能看到和平。





{\centering\section*{4}}





“有这样两个选择。”长崎素世指着面前摆满蓝与黑两种颜色的木片的地图,“一是去最南的那条线上驻扎军队,这就意味着最先面临冲击;二是去守——这两座城池间的交通枢纽。”



立希拾起指甲盖大小的蓝色木片:“这里?可真小啊。”



“毕竟它只是作为车马歇脚的地方。周围有山有河,较远处还有不短的狭谷。”



“地形相当复杂,也很重要。”素世点头肯定她的意见,询问她的选择。立希问素世,她能带走多少士兵。素世摸摸她自己的手背,若有所思:“显而易见。”



立希敏锐地察觉到对方的言外之意,虽不服气也只得认命:“就按你心里的数字来。我有一个请求,他们都必须得是乐于听从命令的小伙子。”



立希看见丰川家的老弱病残,气不打一处来。临走之前,她回头问素世,灯会怎么样?素世先是一愣,提起祥子留下的半套历法,如果灯能完成它,对农事生产相当有帮助。立希并不关心具体内容(而且还有丰川祥子要素),她只需知晓灯有事可做,过得舒心即可。她清楚素世挂念旧情,却不好说会宽容无用之人。



战前的准备不像书中那样能被一笔带过,立希清点兵装,长戈矛戟;藤条串起铁片编织成的护甲整齐码放在箱中。海铃三番五次提起护甲的重要性:本国势力众多,金属稀少且分布不均,人口更不用说——身居穷乡僻壤的大名,跟管理几个农庄没什么不一样。正因如此,保护好仅有的土地,合理运用资源是将领的必修课。立希实战经验不多,此刻也只好亦步亦趋跟着老师的话前进。



出发的前一天,她竟然收到海铃的回信。她说舟车劳顿,未能找到时间。武士的老师对战争的态度反而明朗,说有人的地方就会产生压迫和反抗,鼓励立希抓住机会大显身手,之后宕开一笔去写沿途的风景。樱花开得极早,现今反倒要凋谢;雨总下个不停,这封信寄出去时,星星都被阴云遮蔽,看不见一丝光,这使人想起万户的故事——我刚把信交到驿站,雨以像要杀死行人的气势倾泻下来。因此,信封上大概还沾上我这里的些许水汽。



这算什么跟什么。立希想,颇有些不服气,难不成这人天生喜欢混乱。正好此刻尚有闲心,她拿来纸笔,写江户精细的料理。以及,木头削成的鸟儿。她看向墙边陈列的几把木刀,它们材料相同,成品的用处大相径庭。立希无意识地想:天上飞过大雁,可以说是萧散美丽的画面。可如果空中到处都是刀剑那未免太过可怕。



她摇摇头,在结尾补上几句,希望对方能顺带帮自己打听关于南方近日兴起的密教的事。



椎名立希刚到就忙个不停,只剩半边屋顶的议事厅聚集许多大名派来的使者,他们大多在此换车前往江户。立希不得不空出手招待。对于这位年轻的武士,大家又好奇又恐惧。立希会在面对满屋吵闹的家丁时,用刀柄狠敲桌角,摆出标志性的警示眼神,下一秒刀刃就要弹出。但人们又能从她身上看出与原先不同的感觉。



立希先唤来几位泥瓦匠修补屋顶,从随身携带的行李里抽出灯的最新成果:经过多次修改的避难居房,在此处进行小范围实验刚刚好。立希叮嘱工匠办事,自己换上粗布短衫,解下胁差,跑到骡马街旁边的茶馆,要两大碗散茶。她就着乐奈塞给她的抹茶点心,一人占据一张四人桌。



所见之景皆是东水马龙,各种装束的行人匆匆走过。不知不觉一碗茶即将见底,终于有人朝她这边走来。立希慢悠悠抬头,为首那人和她一样身穿短衣,破腰带系挂一柄打刀。双方都不需询问各自来意。立希快速打量这一小拨浪人。对方显然被她天不怕地不怕的态度气到,拿刀柄猛砸旁边的桌。



立希颇觉无奈,收保护费的恶霸,恃强凌弱还弄出道理来了?她刚起身,浪人们一边喊叫一边俯冲过来。紧接着是钢铁劈上硬木的闷响。定睛一看,原来是武士用刀鞘挡住浪人的攻击。立希臂见从侧面冲撞来的敌人,并无慌乱,右手搭上还未出鞘的打刀,挥出完美的半圆形,恰到好处地在对手身上留下不至毙命的伤口。刀鞘上扬,抬开白刀,转瞬间,对方的左肩迸开一朵鲜红的花。



两回合下来,已经无人敢再上前。立希举起刀,迎着看客和对手的眼神慢慢走。立希终究渴求一场属于真正的武士的战斗,像是说书人口中的真希与海铃:暮夜月明,各持两柄名刀利刃,没有阴谋与偷袭,只剩下最直接最残忍的碰撞。就算侥幸未死,见证者以及介错人也会让你的时间定格于此。



立希复又看向不停后退的对手,估摸着他们会向哪边逃跑。她吹几声长短不一的口哨,暗处窜出几名全副武装的士兵,顷刻即擒住闹事之人。立希收刀,拍拍手:“先带那几个受伤的去包扎,剩下的全部当街头闹事抓起来。”



周围的人们不知是否该用“白吃黑”来形容这出戏码。她用过来命令般的眼神,转身就走。似乎是文职人员的中年男人慌忙站出来宣布,下不为例,绝不姑息。各地大名治理领地,大多是甩手掌柜,家臣们制成各种计策,交由他定夺即可。



毕竟当权者乃武夫,靠一柄刀打天下,可能连城池的名字都无法拼写。立希本想效仿前辈,但她骨子里还是流淌着领导者的血,根本闲不下来几天。茶馆的事给她启发,她亲自调出几十名精英,让人穿淡黄色的短袍,各赏好刀一把,整日定时定点在城中活动。如遇上闹事的劫匪浪人,先行控制住,带回关押。水利,经济,外交那些精细的方面她不明白,她只是践行些极微小的道理。



乱世中的浪人,逍遥自在惯了,除开一身蛮力两手不勤五谷不分。立希收缴他们的武器钱财,凭最朴实的道理——吃饱穿暖拿捏人心。城防之类的事百废待兴,犯事的浪人们被安排去修房铺路,待遇只在一般工匠上扣去伙食、住宿费用。忙碌三月有余,某日清晨,一位士官敲响她的门,递来两封信。信上都盖有长崎家明黄色的印记,这是“重要且加急”的含义。士官说,有封信是别人寄到春名,要给椎名大人的。



立希听懂他的意思,大概是素世代由转交过来的。她忙不迭切开信封,抖开其中一张,熟悉的笔迹让她长舒一口气。



海铃提起盈月教的事已传遍大江南北——首领似乎是女性,还有几位护法一类的角色,实力强劲而且直指丰川将军的首极。连年战乱和庄稼歉收,催化这场革命,看来她们打算向北边进军。



立希阅读这些字句,总觉得对方像避讳什么。模糊的不安在她心中爬行。最后,海铃照旧聊起月亮。



“如果你是在清晨收到这封信,甚是遗憾。盈月之日实乃美不胜收。我周围尽是忙碌之人,无心抬头欣赏……和平的目的即为让更多人能够坐下来好好看看这个世界。这当然只是一个粗浅的例子,你应该能明白此中真谛。请代我向长崎氏问好。”



立希几乎快想象出某人就着月光写信的模样。她来不及细细揣摩海铃的话,赶紧拆开下封素世的来信。家主的命令,要求立希快马加鞭回春名,领精兵救援前线。立希只觉头晕眼花,但又不得不去做。去驿站拔白马一匹,行李得扔在原地,等他们用车拉回来。



立希刚推开议事厅的门。素世的声音弩箭般射过来,不过目标并非向着她:“什么,你说急行军?那几万步兵难道是长翅膀飞过来的?中部地区的家主是连猫都不如?”



劈头盖脸的问题直接打蒙士官。小伙子支支吾吾老半天:对方宣称是……盈月的“法术”。



“他们要么就是走小路,要么就是抄近道过了森林,我不相信会有法术这东西——去清点千早押来的那批武器。”



士官行礼后匆匆离开。立希哑然却并不诧异,素世的真面目可是比立花訚千代还要可怕的存在,形容她是江户的金木犀,不如说她乃雷神之女。素世慢慢品茶:“多有失态。”



“不用在意我。您有何事请、尽管吩咐。”立希依然不习惯用尊敬的语气说话。



“如你所见,形势一旦发展起来就像坂上走丸。中部的大名比纸人还脆弱。”



“嗯。也许对手的实力亦不容小觑。所以,需要去做什么?”



“领兵去支援凌川。如果你准备好,就去码头找千早。”素世不再多说任何一个字,立希试图找到什么却无迹可循。她点点头,不带留恋地向外走,只听家主仿佛闷在被中的声音:“我没能给你选择。”



立希在心中想,都是同样的身不由己。



爱音百天聊赖地靠在木箱上打盹。立希觉得这人一点危机感都没有,轻手轻脚接近某人,抽刀“咚”一下砍在她身侧。爱音一跳三米高,情急之下直接去解腰边的号角。看清来人后她放下手,重重叹口气:“喂,我说——”



“我来找你拿东西。话说,那是你的船吗?”爱音点头又摇头:“不,它暂时还不是我的。”



总之二人钻进船舱,爱音搬出几个重重上锁的箱子。武士嗅到空气中弥漫的奇怪气味。它刺激人的鼻腔,让人想咳嗽。她曾在灯的房间闻过相似的味道。



爱音拍拍她的肩膀:“这可是我能找到的最好的一批货,你要小心再小心不能沾水受潮,也不可以摔到它。就这样。”



立希瞅她不可一世的得意样子,忍住动手的欲望:“这里面到底是——”



“到了你就知道了。”



立希不多问,系好褡带,准备去城中道场练习挥刀。爱音连连感叹:“我的确羡慕规律的生话,但到处冒险的日子更适合我。”



“留在这里,你开心吗?”立希停住脚步问。



“各取所需。而且,如果在打仗的话,去哪都不方便。”





{\centering\section*{5}}





椎名立希想起高松灯先前制作的小道具,她摸出圆柱状的东西,放在右眼上,按灯交代的那般闭紧左眼——透过镜片看到的凌川周边,比用肉眼观察清晰数倍,不过大部分都只剩下模糊的轮廓。她兀自感慨学者的厉害,连带自己也高兴不少,一扫几日的阴云。



盈月信徒们大多扮成黑衣黑靴,影子一般的存在,伴随秋天到来,夜晚正在延长,这身打扮适合隐藏。对手的确尽力避免正面冲突。



立希替掉守将。战场出现面生的将领,足以令对方警惕,这也可以说是素世的考量。



果不其然,对方从没发起过大规模的攻击。昨天黄昏时郊野有过一场遭遇战,百人间的纯粹肉搏触目惊心。立希疑惑对手黑袍轻甲的打扮,为机动性牺牲过多保护能力,于是很快丢盔弃甲。她久久注视敌人逃窜的羊肠小道,挥挥手即鸣金收兵。如果是五年前血气方刚的自己,大概会考虑追上去。八幡海铃稳健的行军风格多少感染她。



这天刚卸下武器装备,斥侯给她念流传在中南部地区的盈月教的传说,诸如在满月时会出现的超级士兵,实乃神的使者;代行人以及她的几位守护者,不能直接接触凡尘土地,必须戴上样式不同的面具云云。



立希一边写信一边饶有兴味地听:“这至少比那些传教士的故事有意思,不是吗?但是关于面具的事还挺有价值……有没有画像一类的东西。”



斥侯准备充分,掏出几本坊间统传的连环画,指出各自的角色,顺便读出拗口的名字。立希颇为震撼,她只能在传教士聚居的地方听见这类生词:“看来我还是修习不够。”



她打发斥侯出门,收下小册子,对其中传奇式的故事嗤之以鼻。她顺意寻一个小书箱,将画册和其他杂七杂八的战报扔进去,送回春名当作情报。



下一步,立希批示城中工匠筑坝引水,占卜结果显示可能会有大雨来到,尽管水攻对需要以战养战的对手来说是下下策,还是没人可以摸清敌方大将的脾性。几名士官几次提议主动出击,以实击虚。立希不以为意,敌人从南边杀至此处,明显迅捷但根基不稳,一定是她们先沉不住气。



没过几日,立希看见对方的阵地正缓缓前移。她认为此乃千载难逢的良机。斥侯快马加鞭贴近观察,飞鸽传书告诉她:督战的似乎并不是神秘的代行人,而是金发的女性, “守卫”中的一位。



她思索。总不能只是看着对方逼近而坐以待毙,她环视凌川的街道,心下一动,仿若回到某个挑灯苦读的夜晚。她派遣一队士兵驻扎在流经城池的大河边,这如按往年经验来谈,夏秋之交经常干旱少雨,届时全城用水皆依托井与河,对方若察觉这点,很可能去截断上游。



立希仔细设想对方的行动。其后几天,河边的军队三番五次抓住探子,与她所想的大致相同。战线一天天收缩,立希摸不透对方的含义。



寻个平常日子,打两壶薄酒,躲在城中不起眼的亭子。面前是小桥流水,亭子是四角飞檐,雕刻飞鸟走兽,景色怡人。她深知自己东走西顾,疲于奔命,也不过是为那个赌气般的目标。可事到如今,她会怀念曾经和平的日子。不仅是某人的草庐,还有十余年前的江户。



丰川幕府存活与否,于她心上不过是飞鸿一点,她最终明悟:我不过想要证明自己绝非别人的影子,才想和更多人一同见证这事实。



立希缓缓抽出刀,摆出演练过无数次,却未曾实战过的起手式。刀尖丝毫不避让地指向全副武装的将领。立希凝视对方的黑衣,看不清他的神色。她深呼吸,脚尖点地,猛地发力向前冲,直直劈砍下去。武士刀在近身格斗中吃不到多少结构上的优势,设计上的面面兼顾变成拖累主人的致命短板——因而,如果能在三招之内解决,我才有把握赢。立希暗想。



对手显然被她突如其来的冲刺打个措手不及,挥刀去挡。生为攻击产生的武器,一旦陷入防御就落了下风。立希向下压刀,直到将对方逼至退无可退,像回弹的树木一般收刀,趁架势不稳时砍中对方小腹。蓝绿色的羽织霎时染作鲜红。立希用刀背敲对方的脑袋,一击踢下“台”。



她双手握紧刀柄,面朝敌军阵型,没有向后退半步。



“下一个。”她这样说。



事态演变至此实属无奈,立希考虑过敌将使用火攻的可能,为此特意安排士兵守河。她不断后撤防线,集中兵力,严阵以待。敌人当真投来密密麻麻的火石,宛如火流星。与此同时,城中也有几处粮仓商号起火,里应外合。她虽做两手准备,压制火势仍需时间。更何况立希想起仓库中的危险品。来不及怨天尤人,她调队去城中各处灭:“那些人呢?火枪手!”



“和步兵一起在城外……十里左右的位置阻拦对方的大部队。”



“能空出手来吗?!”



“恐怕很难……”



“那我要从哪里找人手,变出来?”立希拿起象征守将身份的令牌,“你,派几个人去离这里最近的城池求援,然后把现在还在特命的士兵抽十分之七给我。”



走到街道上,本该漆黑的深夜在火光映照下亮堂得宛若白昼,本该静谧的时光被喊杀声填满。她要做的事情只剩下一件。打头阵的军士眼见立希三步之内即让对手见了血,纷纷萌生后退的意思。



金发的守护者。指出一位不怒自威的武士向前叫阵。立希举刀至与耳际平行的位置。对手摆出起手式的几秒钟,她已大致判断好情况:脚步虚浮,手中的刀也比自己短上一节。



她尽力拉开距离,用刀的前端抵住对手的攻击。刀尖斜滑,轻巧地卸掉死力,紧接上挑,从立希体内迸发的迅猛力量崩开刀刃,转瞬间另一把短刀已击中大腿。立希没有去看倒下的对手,一刀背直接撞碎他的门牙。虎口隐隐作痛,羽织快破成布条,看不见的地方还有两道小伤口。无伤大雅,就当是某人没来得及给自己的毕业测试。她想。



敌阵骚动更甚。立希身后的军士则是不停叫喊,鼓劲示威。立希无一去在意名誉与礼节之类的东西,她只想尽早见到长崎家的旌旗。



武士用两把刀才堪堪挡住力士手中的短锤,哪怕体力已经见底还是要放开步子跑,灵活是仅剩的优势。简直像只在人身边打转的飞虫。她不无自嘲地感叹。



直到双方都显露出疲态,刹那的机会才暴露出来。刀尖直指心口。为了追赶灵巧的对手特意换好的轻甲成为伏笔,刀刃并未感到阻滞,直将力士捅个对穿。撕裂一层层组织,分不清是血还是眼泪的东西打在她脸上,竟比拳掌更痛。立希收回刀,在周身画一轮明月,以示对亡魂的宽慰。



一时间,万籁俱寂,凌川四处的火势已被扑灭殆尽。为首的敌将终于开口,声音婉转柔和,又深藏几分威严:“椎名将军果真年少有为,气宇不凡……像您这样的人应该更明白局势如何,何必在一棵快枯死的槐树上久留。”



“胡杨树的确千年不倒,千年不腐。谁又能保证……”立希一边大口喘气一边回答,“谁又能保证和平。”



对方抬起手,指向高悬空中的月亮,另一边的天际已翻出鱼肚白。这大概是时间快到的表现。金发女人看上去还想说什么,但有人拉住她。立希才注意到那位一直躲在阴影中的将军。她戴着式样不同的面具,只能看见上半张脸。



那人摇摇头,不知耳语哪种好话。金发的将领长长叹气:“看在椎名将军如此英武的分上,暂时撤退吧——我希望您能考虑我们诚心的提议。”



立希似是得到解脱,刀往旁边一扔,直直摔在冰冷的、鲜红的仿若花海盛开的石板路。清晨的寒风不带感情地吹拂过所有人,血肉的铁锈气息灌满她的鼻腔,恐若百年后那位著名的“人斩”能与她感同身受。



领头的将士大喝一声,七手八脚来抬人去应急处理。许多人都是第一次亲眼见证武士的一对一决斗,不由呆愣在原地。此后许久,这场原本无关紧要的攻城战,成为长崎势力范围下口耳相传的史诗故事。可此时萦绕在立希耳中的,只不过是由远及近的号角声,那是城外鸣金收兵的信号。





{\centering\section*{6}}





依然是报时的钟声叫醒椎名立希,手臂有些抬不起来,腹部像被人砸一拳般疼痛。她慢慢调整呼吸,自己大概是已经回来了……那么过去多久?



她在房间里绕圈。没过多久,有军官推门进来,向她作辑行军礼,引人出去见长崎素世。素世手边有两杯冷茶,面前是一盘残局。



立希拘谨地正坐,牵动伤口也不太敢直言。素世满头黑线的神情真乃不怒自威。



“看见你气色还好我就放心了。如果没有你,情况可能会很糟糕。”素世叠好手中的匿名信,招呼侍从换上新茶,“你寄来的情报让我对战况有新的预计,而且,猜猜她们寄来了什么?”



“不会是……”



“议和的请求。”



“你会答应吗?”



“她想先从一个月开始,通过交换人质来顺延日期——如果是不久之前的我,可能会坚定地拒绝。但现在,一切都要从长计议。”



片刻后,素世又像自嘲般地感慨,竟然如此害怕我,难不成以为我手底下皆是那些虎狼猛将?立希总觉得她在挖苦自己。



“总之,家臣们已经为此吵得不可开交。你又怎么认为?”



立希不可抑制地回想起那天晚上,自醒来之后,她觉得自己的思绪像有一小块永远留在那儿;八幡海铃若遇上这般角力,会有多么漂亮的表现。椎名真希也是。



“那个晚上她们没有下死手 …我要说的就这么多。”



素世冲她点头,我明白了。立希得到梦中的假期,一时不知该如何是好。她拿赏银请高松灯和要乐奈吃饭,包下全城最好的酒楼,光点心就挤满好几个三层食盒。可惜灯只是一边呜咽一边不住喝水,光看着立希都觉得渴。乐奈只吃甜味的点心,风卷残云一般,正菜正眼不瞧。



她完全不知该如何安慰灯,只好像年长者照顾弟妹一样给她夹菜。最后八热八凉,十六道精细料理全赏给手下人。



立希又打一瓶好酒,买一件和原先那条几乎一模一样的羽织。冬日已然不远,还得裁些棉衣棉裤。此时此刻,她就像万千平民百姓,操虑琐碎的事情。



她习惯自力更生的生活节奏,不去练习的日子躲在家中砍柴生火,洒扫庭院。素世一直在帮她打听真希的事,可总是一无所获,也许是对乱世灰意冷,寻到哪个世外桃源隐居起来。



长崎家正与盈月教交换人质和土地,以此换取延伸至凛冬的和平。只是素世只字不提合作。立希无法置喙这一切。城中开始启用灯编撰的历法,在这历法中所标注的冬日来临之际,立希收到海铃阔别已久的信。



信中难得关心她的身体状况,语气之郑重令她怀疑是不是他人仿冒,不过海铃的字迹神韵独特,她对此最为相熟。



“不知你是否记得我同你说过的‘好人不长命,祸害遗千年’?我更加深这般印象,似乎坚守正道总求不得善果——即使如此,看见你仍坚持,我由衷感到欣喜。近日我受人所托,将去往他地,暂且不用与我传信。”



不管经历的战争多么波澜壮阔,某人的来信总是从平淡细微之处入手,好似一切都回到它们诞生之初,于混沌中一视同仁。拜此所赐,立希还能找到现实的锚点。她阖上眼,在丝丝缕缕的阳光下打盹,心中执念的火焰都变平和,自己想要的东西(几乎)已聚在身边。



平静存在的唯一意义似乎是被打破。椎名立希起个大早,长崎素世派人请她过来。她不情不愿地走在阴冷的早晨。侍从为她开门,紧接着,立希脸上显出与撞见死人复活没两样——素世旁边还有一位一袭浅葱长衣,头戴斗笠,身挂两柄武士刀,个子稍比家主矮一些的女性。立希只瞥一眼即知来者何人——这不是亲爱的敬爱的天门斋,自己的老师八幡海铃吗?来这里住宿还是打尖?



海铃向她微笑致意,轻轻鞠躬:“好久未见,你气色还不错。”



立希不顾礼节地敲脑门,别是冻出幻觉来。素世的眼神一直停留在立希身上:“那么,我想不用再多费口舌——有八幡的加入,实乃如虎添翼。”



立希的心情多少平复一些,向二人施礼。海铃脸上笑意更浓。素世留二人吃茶,立希实在提不起兴趣,借口要去训练跑路。刚出门就听见肚子在抗议,她找一家小酒馆,一壶烫酒,放在小锅里炖煮的鱼羹,奶油色的汤中上下翻滚着大块鱼肉,配上厚蛋烧,这可比什么竹笋柑橘泡茶有味多了。



立希一个人大快朵颐,总觉不对劲。海铃可没在信里提到她要来投奔长崎家。更何况,哪里的大名会放跑赫赫有名的天门斋?立希转念一想,总不可能是为了照顾学生,顺便验收课业吧——喂你,别太得意忘形。海铃能做许多更有意义的事。



正在胡思乱想的当口,窗外传来小孩子欣喜的喊叫声。她向外看去,纷纷扬扬的雪花宛如白色的眼泪。她哈出白气,不由得用酒杯来暖手。店主生起火炉,开水壶不停吐出水汽。所以说,海铃伴随着今年的第一场雪到来,冬日的使者啊。



立希跑去道馆练习五百下挥刀,与几名瓶颈中的弟子交手。尽管天气寒冷,汗水浸透短衣。她刚坐下休息,旁边有人递来水。立希含糊地说谢谢,一抬头看见海铃正对自己笑。



“啊,你们聊完了?”



“长崎家的家主,远比我想象的年轻。而且,说不定比那些老油条还聪明。”



“这话可不能当她的面说。“我知道。更令我惊讶的是她的忠心耿耿。”海铃颇为遗憾地叹气,“你找到一个不错的归宿,我由衷为你高兴。”



“哎、额……谢谢——那么你呢?”立希背过身,突然觉得尴尬,不敢直视海铃的眼睛。



“我当然一切都好。在浪人中,大概算是最富裕的。”



立希无法把“浪人”和“有钱”两种概念划等号,总之没缺胳膊少腿就是对亡命之徒最好的嘉奖。海铃又说,要不你带我去熟悉下城内?立希下午本打算誉写名家字帖,但为了海铃这些事情都得往后让让。



海铃就像曾经哪个夜晚,漫不经心地甩出一道道问题,有些是立希早已烂熟于心的典籍内容,有些是需要思虑的应急对策。街上行人熙熙攘攘,再走上半小时复归寂静,又过一会,能听见船工的号子。码头周围除开投机的商人,就是船工和传教士。立希专注看那艘“属于”千早爱音的船.船帆远远看去像鹰的翅膀,甲板宽阔,上架六门大炮,舱内似能塞下半城居民。



海铃注意到岸边两位席地而坐的中年男人,面相一看即知不是本国居人。左边那个环抱人脸大小的陶罐,另一人手执稍小些的瓦罐,内中盛水。他们面前聚集约莫十余人,规规矩矩地排队,应该是在等待赐福一类的东西。男人口中念念有词,日文夹杂着别国语言,搭弓射箭般一句一顿。



立希问身边人,你听懂他说什么了?



海铃点头,我有个故人也爱说这句话——“我已历经一切必要之战,我已行尽应行之路,我已守住己身之道”。



“有点莫名其妙。”



“是吗?我倒觉得挺好明白的。”海铃笑呵呵地否定她的意见,背过身向回走,看不出什么留恋。这种与世界的膈膜感,不知怎的让立希安心下来。她说,请等等我。海铃于是就停下来,从她的袖口里钻出一道黑影。立希刹住脚步,差点以为是突然的测试。



一只浅蓝色,脑袋呈水滴状,两只突出的眼球占据一半头颅的爬行生物直愣愣盯着她。



“如果吓着你了,我很抱歉。这是我向一位传教士朋友讨来的,它很安静,而且不需精心照料。”



“长相倒是不吓人……它有名字吗?”



“没有。我不擅长给东西起名。我不习惯这些标记类的事。”



谈话间,壁虎又钻回去,不知缩进何处。它只是探出头,轻轻扰乱二人的世界。



“有一段时候没见,我简直不知道要怎么面对你。”



海铃不动声色地说,按平常就好。话是相当轻巧,立希从逃难的武士变作长崎家的得力部下,能喝到海外来的葡萄美酒,身份的倒置让她很不自在。远方的圣人教诲说:“一日为师,终生为父。”海铃却更像她生命中的一块界碑,时不时回望总能看到。



立希闷闷地说:“你是我的老师。”



“哦——老师。有你这样的徒弟,我甚是欣慰。海铃摊开手,“毕竟。我也是第一次去教某个人。”



这之后谁都一言不发,立希手中那盏灯笼忽明忽暗,在冬风的吹打下好不凄凉。她们似乎要走进温和的良夜。立希将灯笼从左手换到右手:“你那只壁虎还挺好看,是那里得到的珍宝?”



“半年前,我救过被逃兵打劫的传教士们。他们一开始拿出来很多钱,但是我并不需要。”海铃若有所思,“这只壁虎挺像我们的。”



立希在心中反驳,它仅仅像你而已,绿松石一般的眼睛,隐约还有天空的蔚蓝,一袭轻衣,总在旁观世界。



“所以,你那位故人——”



“我本也只是暂时落脚,而且理念不合。”海铃轻描淡写地概括。通过谈判和让步达不到和平的最终目的,我想试试出一份力。立希呢?”



海铃的话也不错,力量是维护和平的前提。立希想了想,人们并不全是为了和平走进这场战争的:“是吗?毕竟你来得大过突然。”



“如果我说我想快点见到我的好学生,你相信吗。”



立希久久凝视对方没甚波澜的眼睛,气不打一处来,不由得抬手下力气猛推把某人。海铃的步伐只是短暂凌乱一下:“你进步很大。”



“啧……你总不会在原地踏步吧?”



“哎呀哎呀,一天不练,自己知道;三天不练,阎王知道。”海铃笑得畅快,仿佛立希刚讲完一段完美的落语笑话,“等我们哪天一起见过奈落,也就知道几斤几两了。”



“能不能别诅咒我们。”



“终有一天,对吧?终有一天。”



“我还是希望那一天可以慢点到来。”



她们穿过临近闭市时间的夜市。已有不少士兵提着灯笼上街巡逻,他们挥动手中的棍棒,驱赶还没来得及回家的人们。海铃小跑起来,买下最后两串三色丸子,递一串给立希。紫白绿三种颜色整齐排成一列。海铃顺手捞起摊位上的小罐,丸子伸进去再滚一圈,刚好用完剩下的糖稀。深秋的晚上可真够冷,糖稀迅速变干变硬。她丝毫不在意这些,慢慢咬掉一个。



“令人怀念的味道——哦,吃起来有点像葡萄夹心。”



立希学她尝一个白色的,没涂糖稀的话,外皮吃上去就是糯米的本味,红豆的沁甜冲刷舌尖。



“你还挺悠闲。”



“你怎么就没学来我这一手‘苦中作乐’?”



“大敌当前最好还是存有一点危机感。”立希开始教训自己的老师。对此海铃只是伸伸懒腰,露出腰间一长一短两把武士刀。立希的心里自始至终盘旋一道问题,她善于忍耐。两人再没说话,直到她将海铃送到家宅门口。她最终还是没忍住:“说来我一直想问你——浪人很少会用两把刀,我遇见你时就已经……”



她做好不会得到答案的准备,可惜总是事与愿违。海铃平静地说,正如传闻,我侍奉过许多家主,他们大多都亡于战争的车轮下。



“你希望长崎家能、带来和平吗?”



“和平能给所有人以归宿,我想应该没人会拒绝。”



海铃迈开腿,缓缓走过来。她的身形在立希眼中慢慢放大,遮蔽夜晚。海铃靠得过近,立希甚至没来得及后退。她轻笑几声,藏在袖口的手飞速伸出来,按按立希的手心。再之后她又自然而然地抽身离去,挥手权当告别。



立希一路小跑,根本不带停。一到家,她确认门窗紧闭,摊开拳头,手里是张薄如蝉翼的纸,这种纸常用于情报传递。



“一直都有人在看。”简短却没头没尾的一句话,立希却觉得腰部的伤口隐隐作痛。





{\centering\section*{7}}





八幡海铃的生活相当平常,早上吃过一碗裙带菜汤,春名特产的豆腐直接滑进胃中。她满足地放下筷子,紧紧腰带。



她坐马车到离城两百里远的峡谷,到时已经日上三竿。海岭背条行囊,慢悠悠沿行人的道路上山。此处三面环山,只留一条供小型马车驶过的通路,骑兵需要下马行走。脚下的路不一会就陷于茂盛的草中。她小心翼翼地绕开藤蔓与树根,站在山顶眺望四周。周围并不高耸,能获得的视野有限。她眯眼细望,勉强能看见最近的城池的轮廓。



海铃到达那座小城时,早过午饭时间。她穿过好几条街道,本就不大的交通中转定位的城池被翻个底朝天。即使如此,她也找不着几家正常开业的餐馆。人们大多行色匆匆,不停有马车疾驰过主干道。士兵感觉比城中民居还多,组成细密的网兜住前线。间或有些浪人认出她,赶忙向她低头致意。



海铃买到几个豆包,一边吃一边物色马车。出乎她意料的是,战时的混乱似乎并未波及到此地,很少有浪人在闹市中拔刀相向,军士们也只顾巡逻和时不时盘问路边店家。修理农具的铁匠告诉她,这都多亏椎名大人,她先前带队驻扎过一月,与当地的民众约法三章。偶尔见到的一袭黄衣的武士,也是她设立的监督军队行为的人。



海铃点点头,发出感慨:“老人家,你觉得如果多一些椎名大人这样的将领,战争会结束吗?”似乎仅仅是在自言自语,不奢求回答。对方放下锄头,拿起钉耙和铁皮:“好人不长命啊。一任又一任的将军,只爱那些祸患。”



“嗯。好人总是不长命。”



海铃转身离开,在城门搭上马车。是匹好马,步履如飞。她得以在月亮升起前回春名。立希正缩在院落里喝闷酒。尽管医馆告诫她少食辛辣少饮烈酒,终究还是苦闷侵占心头。



最近的冲突只是小打小闹,她认为这是山雨欲来前的风波。有人监视自己的不安全感让她失落。这时有人轻扣门扉,立希谁也不想见,遂未起身开门。没过多久,敲门声渐弱。



立希本想洗把脸而后睡觉。谁料醉醺醺的大脑适时发出警报,她凭本能一个侧身,骨碌碌滚出几尺远。沉重的撞击声在她原先站定的地方响起。



海铃干脆地扔掉手中的木刀,畅快地笑起来:“不错嘛。”



立希冷汗直冒,幸好这家伙没拿真刀来,不然自己可能命丧于此。



“一天天的怎么这么闲。”



“没有。”海铃无辜地说,“我白天去了前哨线上的城池,大致侦察它周围的地形。”



“你自己一个人?”



“我擅长独自行动。”



立希有种想摔酒杯的冲动:“如果你想摆脱被监视的处境……”



立希太过在意这件事。身为合格的一邦之主,保持警惕明明是基本功,可她只觉得思绪陷入泥潭中。海铃没回话,默默注视她。壁虎又从她的衣袖里爬出来,学主人一并关注她。换别人可能高低酿成惨案。立希仍保有相当的理智,估摸一下情况,好像打不赢,也只能依着她来。



海铃问,要不要我去随便做点什么给你。得到许可后,她端来热粥,探身看杯中物:“梅花下酒虽清冽,但是不要贪杯。”



“五十步别笑百步……”



“先不说这个。今天我老想起万户。”



“万户?是那个故事里面,想靠爆竹去月亮上看看的人?”



海铃点头:“尽管不知真假,倒的确是不错的故事。”



“我一直以为你是结果论者,毕竟嗯,你好像对很多事情都毫无关心。”



海铃面无表情地说:“倘若你所言非虚,还能安安稳稳站在这——开玩笑的,多休息会吧。”



立希很想得到问题的答案:“所以,过程中的努力也同样重要,哪怕目标没有达成?”



死缠烂打的醉鬼最难对付,海铃扶住对方摇晃的脑袋,用安慰的语气说没错,只要自己问心无愧就好。问心有愧的诲铃觉得自己这话毫无说服力,不过立希得到肯定就很是满足:“努力不会白费,我有一天可以赶上……对吗?”



“你的进步我一直看在眼里,总有一天你会战胜我。”



“你对我太有信心了吧。”



“因为我希望你能心想事成。”海铃慢慢品味冷掉的清酒,从喉咙凉到心底。立希愣住,转而开始喝闷酒,大口大口地喝,抬起头,又哭又笑的。



海铃见立希已然睡熟,轻手轻脚溜去厨房,生火热酒。剩下半瓶陈酿全稀里糊涂进她的肚子。她并非贪恋杯中物,只是有个俗气至极的理由:借酒壮胆。堂堂天门斋,说书人口中传奇故事的主角,竟沦落如此境况。



海铃本云淡风轻,可今日把酒言欢后,心中总有未知的迟疑。她熄火,提起酒葫芦上街。暮夜月明,没有一丝声音。海铃感受着周围,没有异样的被凝视的感觉。



几两酒没对剑豪造成实际影响,她健步如飞,左转右拐,进一条小巷。巷中一家药铺仍点着灯。店内的地下室,除开药材弥漫的苦香,还有另一阵茉莉花的味道。正中端坐一位金发女性,粗一看像是掌柜家的女儿,若仔细观察,能瞧见额角一条刀疤,手上遍布硬茧,衣服上并没有深棕色的汤药痕迹。



“你们真的决定好了?”海铃没有走近,远远问一句。



“按照她的意思,就在初诣后第二个盈月之夜,组织对那个地方的进攻,借此倒逼——”



“我明白,不用再说下去。”



“八幡将军……这其中也许会有残酷的牺牲,但是拨开迷雾就能看见月亮。”



“你说的这些我都明白。本着契约精神,我也会协助你们直到最后。三角将军。”



三角初华被对方突如其来的严肃与挣扎镇住,犹豫半天才点点头,双膝并拢,向这位愿意秉持大义的武士敬礼道谢。





{\centering\section*{8}}





自从听闻某个著名的能剧剧团要来春名,千早爱音在床上滚完好几圈都没能睡着。长崎素世借此机会宴请江户的部分达官显要。料想当日盛况爱音开始担心自己能否一睹年轻的天才女演员真容。她闷思苦想,待神社的钟敲过三遍,终于有了想法。



素世放下手中书卷,一脸诧异:“你要借我那块银牌——虽然它放着是吃灰,但为什么?”



爱音比划着形容自己的想法。素世若有所思:“我的确不喜欢去看这些东西,但是……啊,我其实可以给你,但是需要你帮我一个小忙。”



爱音满口答应,想着这狐狸女人终于也肯当一回人,自己能亲眼看到主演的英姿。素世轻笑几声,挥挥手召来几名侍女,带爱音先行离开。



半柱香的功夫,素世支开所有剩下的侍从,对着空无一人的房间说辛苦了,请出来吧。



灰白色的影子从天花板上闪身下来。没等家主发问,忍者已经递来小纸卷。素世仔细阅读其中内容,反复向乐奈确认细节。后者惜字如金,只点头或摇头。一连串问题过后,她脱力般倒在椅上。



乐奈终于肯开口:“我交到的第一个友人是灯,所以我不明白。如果有人和最好的最亲密的友人一定要拔刀相向的话——这样是正确的吗?”



素世没有看她:“这问题还真是棘手。从情理的角度来说,我不希望任何人步入死局……万事万物若都能用那些道理解决就好了。”



“她是好人,我不希望她死。”



“好人不长命呐。但是,也会有转机。”素世的眉头舒展开一些,“我会把决定一切的权力交给那个人,这是我最后的礼物。”



门外传来杂乱的脚步声,乐奈识趣地躲起来。素世很快看见另一个自己——如果不是标志性的虎牙,谁认得出面前这人是千早爱音。



素世笑了笑,很合身,也很适合你。爱音想当场挖开地板钻进去,或者直接和这只狐狸拼了,合计所谓“小忙”是影武士的活。



“一举多得。”素世拍拍手,“感谢你的忠诚。”



能剧大多取材于传说故事,佐以传统乐器,演绎传奇式的内容,与传教士口中的“戏剧”有相似之处。最大的不同在于能剧演员都会戴上样式不同的面具,大多是经过夸张化的人脸,男女老少皆有。



爱音强迫自己的思绪回到演出上:义隆王与义仓王,为争夺继承之位反目成仇的故事。在幕府时代看这个颇有奇妙的预言性质,正如兄弟俩两败俱伤的结局一般,如今的天皇不过是名面上的人偶。不过,天才主演与伴奏的知名三味线演奏家都展现出不凡的表现力,一举一动牵动着观众的心。



坐在最顶上小房间内的爱音将一切尽收眼底弹奏三味线的乐师站起身准备参与谢幕,领头的主演即将解开宽大的演出戏袍——怎么突然扔出去什么东西?不好,烟幕开始在场内蔓延,主演没摘面具,反而拿出锤子与柴刀。近距离目睹这场面的人们作鸟兽散,一位不幸的大名已然身首异处。



爱音吓出一身冷汗,又听见由远及近的上楼梯的声响。她忙打开侧墙上的暗门,脚底抹油开跑,同时在心里诅咒某本该经受此劫的某人。



“嘭”一声摔在地板上,来不及整理衣角,踉踉跄跄地向前猛冲。出门就到车水马龙的街道上,她本以为万事大吉,谁料有人正在街边屋顶上窜下跳,颇像晃荡绳索。定睛一看,竟是那位乐师,深紫色的目光和她相撞。



爱音暗叫不妙,拔腿就跑,再回头,只见对方手中三味线应声被从中扯成两截,赫然弹出短刀一把。她情急之下摸向腰间,奢华的刀鞘只是为了撑场面,唯一会使的东西仅剩下吹火筒——谁会在大街上纵火?!



没办法,爱音没命地跑,逐渐离开人群聚集的地方。她在心里大喊“我不是长崎素世”,可又有谁会相信。死之前,希望能再看一眼我的爱船,还没能亲自掌舵,扬帆出海。



刀刃挥出的破空声响起,却并没有疼痛,反而是金属相撞的刺耳悲鸣。爱音惊魂未定地睁开眼,只见黑色的、“伟岸的”身影挡在前面。



椎名立希用力抬开对方的短刀:“你还愣在那里干什么——”



生死时速。立希集中精神与对手过招,“砰当”的声响持续十个来回。立希惊觉乐师不是三流刺客,还有几分真本事。因而不能耗下去。她像文艺作品中的杀手一般砸下烟幕,向爱音逃跑的方向追过去,推着没跑几步即气喘吁吁的商人一路狂奔。不到半个时辰,爱音却觉得自己跑遍整座春名,仍然不敢松懈,直到看见一小队士兵才停下脚步。



立希没好气地看一眼仰躺在地上的某人。



军人们开始骚动:“家主大人……”



“啊,原因很复杂,先把我们送回去。”



没几步路,立希本不想乘车,但爱音恐怕此生都不想再跑步。没等爱音问,她开口说:“家主大人让我来的,她考虑十分周全。”



“周全……呕,我感觉看见我的祖母……”



“好歹活下来了,待会记得把当时的情况梳理一下,这可不是小事。”



立希看向窗外,骚动在民众之间裂变式传播,她本人也不免露出忧心忡忡的表情,她在想更遥远更隐秘的事,希冀其并非真实。



处理完擦伤和扭伤,爱音看上去总算不是刚从死人堆里爬出来的状态。素世亲自来找她们,从演出的戏目到刺客的神态,细细问了个遍。她扬手嘱托厨子为影武士准备一桌好菜,说论功行赏,打发人回去。最终留立希和她面对面。



“你也相当辛苦。”



“这是我应该做的。”



“那么,直入正题。看来盈月的教徒已经渗透进春名,以前从未发生。你对此怎么看。”



立希没有第一时间回话:“……一介武夫的言论,有何采纳之处。”



“为家主排忧解难也是家臣的职责所在——历法的编写靠观察万物之间的联系,从天象到脚边一只蚂蚁,偶尔向小灯学一学,如何?”素世一边说一边往外走,“你要是想好了,就来找我。”



一年的最后一个月到了尾,立希终究没去找素世。出手意料的,素世也未曾派人来请,兴许是明白武士跟野牛一样顽强执着,不服气的时候说什么都不听。





{\centering\section*{9}}





长崎素世最后还是召见椎名立希,拜托她动身去江户送信。立希应允,临出门时却被叫住。素世说,因为是相当重要的东西,请和八幡一起去。不知她是何用意,等立希走到海铃家,某人已经站在门口,百无聊赖地注视像要落雪的天空。见到立希,她打个哈欠说赶快走吧,倒似她去找对方一般。



江户如同许多著名的国都,风雨飘摇,几易其主,也算是充满创伤的都城——史书记载,百年前的霸主一把火烧掉大半建筑;再近一些,就是六、七年前的事。现如今,大概还能找到焦黑的废墟。



外面的地界战火连天,江户城内倒是显出一股祥和。立希略一思索,突然意识到,再过没几天就是新年。丰川家又交出一批战俘,求得停火期延长到明年年初。毕竟是新年,打打杀杀之类的事放在之后再说。她偷偷看海铃的侧脸,平静且淡漠。她放下心,毕竟是新年。



觑见丰川将军的路上,海铃像突然想起来一样问:“走之前我教给你的那招,有好好练习吗?”



“当然,但是没什么成果,依然是残招。难不成——”



“没有。我试过很多次,总是差那么一口气。”海铃畅快地笑起来,“很多时候我也跟你一样,目标是在前面,无论怎么跳都够不到。”



她的眼神真挚,仿佛在说,这样能让你好受一些吗。被这般的湖水安抚,立希想生气都做不到,只好颇不服气地说,总有一天,我们可以做到。



将信件交给内侍代为传达,得闲可以在江户街头游荡。迎面走来些衣衫褴褛的浪人,人声鼎沸。海铃走在稍前方的位置,迎着人潮向外。在这鱼龙混杂,喧闹不已的混沌之地,立希终于明白素世的真意。每天都有无数人死在刀剑枪炮之下,无论是大名还是奴仆,死亡一视同仁。立希定定地凝望海铃的背影,只消一刀就可了结的心魔屹立于此。然而她伸出手时,手中仅有只木头雕成的雨燕:“喏,这个送你。”



海铃收下并道谢:“话说怎么这么突然。”



“你陪我去个地方,不远——好,跑起来。”



海铃露出被耍一样的表情,立希的心情也仿佛照到阳光。她们最终停在一幢木质的二层小楼前——不过是海铃的想象,因为这间小屋只剩下大致的框架,倾斜和横倒的房梁构成几组几何体。



立希轻轻说:“这里是给贵族的女儿练习器乐的地方,曾经是。”



“在兵变那晚被烧毁了?”



“没错。我正想跟你说一些相关的事情。”海铃表示洗耳恭听。



“我们虽然在有经验的车队都不会去的深山老林见面,但之前我是实打实侍奉主人的武士,那位是丰川家原先的继承人丰川祥子。”立希语气怀念,似提起一位老友。“她与我们见过的纸老虎都不一样,博学多才,饱读兵书。这算是我生命中第二道幻影,当我以她为目标时,我觉得一切都在往好的方向发展。”



这之后的事情像织田信长的生平,在势力鼎盛时期,死于胞弟的一场阴谋。丰川家的旁系意图夺权,在城中布下天罗地网,主动挑起战争。不过,祥子周围的侍从自觉组成某种类似近卫队的角色,硬是打开一条生路——沿着宫殿的水道乘船离开江户。等城池远到看不见,人们才停下,于此分别。



“到现在为止,我连关于她的一丁点消息都不知道。她成为了永远的幻影。”



“看得见又摸不着的,的确是幻影。谢谢你和我讲这些。”遥望天际,夜幕逐渐压下来,冬天的夜晚来得快且迅猛。周围行人逐渐散去,只余二人,这会是最后的机会。立希的右手放在刀柄上,旋即放下:“海铃……作为你的学生,能让我‘进言’一句吗?”没等人回话,她继续说:“普天之下,还有很多风调雨顺,和谐安定的地方。随便什么,都比在这里要好——所以,你完全可以……”



海铃迅速贴过来,食指抵在立希唇上:“我和你的目的,本质上是一样的。请相信,就算是道路有差别,高悬在我们头上的明月如旧。”



“谁对谁错,交给时间来判断。在此之前,就先各显神通——一起回去吧。”海铃迅速抽离,“而且我不会坐以待毙。我还想多看看我想看的东西。”



明明是宣战般的言论,立希却莫名觉得安心。她再次回头看那栋小楼,那座残碑。路过城中著名的寺庙,令人想起孩童的吟唱:“砥园精舍之钟声,奏诸行无常之响。



娑罗双树之花色,表盛者必衰之兆。



骄者难久,正如春宵一梦;



猛者逐灭,恰似风前之尘。”



长崎素世用笔尖轻点桌台,似乎在等待解释。立希不动声色地提醒她,试图暗算天门斋约等于以一敌百,而且,她希望自己能以武士的方式来了结这件事。素世沉默很久,她早就决定将有关的一切交给立希处理,作为权力者的仁慈。



她叹口气,不再纠缠:“对方原来一直在推进议和的事宜,但是最近,我收到探子的消息,她们正筹划针对西线的攻城战,看来也是做了两手准备。”



“有具体时间吗?”



“大概是在新年之后。”



立希心下一动:“一定会是盈月之夜。然后,我们还可以用灯的历法来预测。”



“倒也是个不错的提议,但不能保证完全准确。值得一试。”



又交流一些治安上的问题,素世突然说:“其实,不管你用什么方法我都会支持你。因为我要的只是最后的结果。抱歉,无论过程多么感人肺腑,到头来可能皆是一场空。”没等立希回话,她又继续说起神社周围的巡逻队安排。刚刚的一切好像突如其来的玩笑,无可奈何。



立希吹着街头刺骨的寒风,只觉迷惘。一旦有不想面对的事横屹在未来,时间就如急流般前进。素世作为家主,在本城最大的神社主持新年的仪式。



其实先非是老一套流程,立希昏昏欲睡,千早爱音早和要乐奈溜走。一直到八幡海铃跃身上马,拉开足有半人长的战弓,甫一松手,箭镞正正好好地穿透仅一指大小的纸人。立希撑开眼皮欣赏其英姿,跟着观众们一起叫好。仪式结束后,大家大都去做属于新年的平常计划。爱音一瘸一拐地走回来,真不知道这人哪来的活力,身上的伤都不好全。



她拍拍手,我们应该来许新年第一个愿望。



写这张纸上,然后投进去就行了?



爱音一边点头一边几笔写成,折两折扔到箱中。立希对着白纸沉默三分钟,想实现的事情太多,一时令人难以抉择。海铃不知何时凑过来:“随心而动吧。”



她还穿着量身定制的战袍,却丝毫不像英勇的武士。立希不想被她揣摩,连忙胡乱写下刚从心中浮现的几个字,反正大概也没人相信所谓心想事成。海铃抖抖手中的纸片,故意不给她看,扔进箱中后摊开手:“你的愿望是?”



“等价交换,我们得同时说。”



海铃点头:“那我先告诉你——我希望善人都能好好活下去。”



“朴素的愿望——我倒是觉得战争应该早些结束。”



“一百个人里头恐怕有八十人和你一样。”海铃笑出声,“顺便,你还挺相信我啊。”



“我可没有办法去求证。新的一年应该修习更多制胜的能力。”



海铃感慨,你也太严厉,到底谁才是老师。立希让她赶紧去把衣服换回来。二人对节日气氛都没什么感触,往年隐居山中时,不过是趁此机会买些年糕之类的小零嘴,洒扫院落完事。城中大张旗鼓庆祝新年仅让人有些不习惯。



立希躲在家里,避开喧闹。没过多久,素世的命令下来,把人赶去西线最外的城池。一出门,听见爆竹的闷响,悠扬的琴声,多少拖住过路人的脚步。立希却近乎跑起来,她深知时间不多。当看到同行人还有爱音时,她的心情陡然变沉重。



爱音连忙摆手:“额,虽然我不能充当铺路的壮丁,但给你打后手应该还算分内之职。”



立希没说多余的话。爱音估计自己被丢下车的风险降低至零,忙不迭拉上车帘:“然后,我还有一句话要代为传达——‘我把定夺一切的权力暂且移交给你。’很没头脑啊,毕竟是长崎大人下的命令。”



“这话你可不能给……算了,我明白。”



爱音斜眼看她,嘟囔着真是莫名其妙。



城中气氛肃穆,到处都显出战争临头的感觉。立希问过爱音,确认历法中下一个盈月之夜的时间。她又要一匹白马,捎上两壶好酒,包几盘牛肉,疾驰在出城的道路上。八幡海铃被安排在战线最外头的堡垒里,而且到处都安插有忍者,未尝不是素世的考量。



立希只是冷漠地想,只要海铃愿意,再来一百位忍者都无法拦住她。



守门的军士见是椎名将军,连忙放人进去。海铃百无聊赖地阅读《平家物语》,看到立希站在门口,喜形于色:“怎么,今天倒有时候跑这么远?”



立希发现这人大难临头还能这么悠闲,顿时有种真心喂狗的错位感。她慢慢挪进来,放下酒肉:“我只是觉得,好久没能和你这么说话了。”



海铃盘腿坐着,合上书,似笑非笑:“只要你想,只要我想,我们还能有很多机会。”



立希没有回话,只是闷闷地大口喝酒:“不,不是这样的……我们比谁都明白。”



海铃没有碰酒杯,还在山上的时候,如若武士喝醉,胡言乱语说些疯话,总是老师去熬粥,负责照顾酒鬼。她仍保留着这般习惯。



三大碗体量的好酒下去,老虎也要醉三分。海铃看她一直不说话,自己倒也很有耐心。灯火如豆,在月夜中飘荡,一如亡命之徒浮萍似的命运。



立希呆愣地凝视海铃,这时她才发现自己的老师有张不逊色城中艺妓的脸,蓝绿色的瞳孔蕴含沉寂。她追求的便是如此:所有人都只注意一个人本身的能力,而不去看她的皮囊,也不关心她的人脉。立希是崇拜,乃至于敬爱天门斋的。对方倾囊相助,教会她立足于世的方法。于情于理,她都不想走向那必然的结局。



立希移开眼神,轻轻说出历法中的日期,又补充道:“如果满月在此之前到来……乱军之中,你不必留手保我。”



“如果刚好是这一天?”



“就在城外森林里相见吧。我会等你到天明。”



海铃默默点头,她听不见立希的声音,也感受不到自己的存在。此时谁都不像人间的武士,而是从阎王手下爬上来的妖怪。逢魔之时明明已经过去,那股气息却始终不散。恍惚间,一切都像要变作尘埃。





{\centering\section*{10}}





立希取来纸,悬腕提笔写上触目惊心的一句话,对叠后连同令牌一齐交给爱音。后者哑然,猜不透武士的意思。



立希捡起墙边黑色的长条布包,系在背上:“如果我爬回来,你就当什么也没发生——如果我一去不归,你就随便处理它们。”



爱音捏紧手中象征调兵权利的令牌,挽留的话尚未出口,硬邦邦的冰凉东西正中她脑门。立希头也不回:“这个也能卖不少钱。”



摊开手掌,玉佩散发着幽光。爱音彻底无话可说,收拾好东西即往军营跑。



立希稍绕些远路,以免中途和八幡海铃碰上,料想那人也会是一幅关心学生的云淡风轻的样子,明明走上岔路。



海铃等在约定的地方,随意坐上一截木桩。她听见来人的脚步声,回头看立希。武士在心中呼喊“海铃”,嘴上却说:“恐惧。”



哪怕戴上面具,依然能看出海铃在笑:“果然,听你这么说还是觉得太过火。”



她站起身,面具被扔在脚边。立希注意到她腰间的木刀,不由得叹息:“为何不能再相信我一些。”



立希解开布包的褡扣,从中甩出两把虽稍有年头,但仍能看出做工精细的好刀:“你总是告诫我说,不能过于骄傲自满。”



“你的话不错,我还说过,不管面临怎样的战斗,都应将对手当作毕生的宿敌对待。”海铃提起刀,“我们本来不该这样的。”



立希一怔,似是有什么触及她的心弦。她深呼吸,迎着月光拔出刀,刀锋凛冽,寒过冬夜的厉风。



“那算是嘲讽吗?星点也敢对明月——”



海铃不说话,或是觉得无话可说。武士快速后退半步,躲避可能的杀招,箭步猛冲。海铃巍然不动,直至刀刃逼向脖颈,“嗵”一声别开进攻,向左后方跳去,一下拉开距离,刀尖直指咽喉。立希侧身躲开,向后仰的同时朝下劈去——只来得及砍到空气。



分明是下过大雪的寒冬,短衣被汗水侵湿,贴在后背上就像钢铁。明月逐渐上涌,枝叶无法阻挡它。今晚乃是盈月,与高松灯预测的结果丝毫不差。立希感觉到联系,疯狂的想法侵占自己的大脑:灯改进的,是丰川祥子留下的历法。她终于有一种近似胜利的喜悦,说出去的话谁也不会明白,但她还是喊道:“是她吧……丰川家的长女,所谓‘代行人’。”



海铃先是诧异,后又说,你已这般笃信,也不需我再度确认。这下换她出手。



立希大气都不敢喘。只见老师做出标准的起势,却断绝对手的逃路,逼立希正面接下一击。立希将刀刃横亘于身前,早早迈开步子,主动撞上对方的刀。不带其他技巧的硬碰硬。一接触即掀起能卷走人的气流。立希不得不用尽全力抵住对方,她捕捉到海铃的重心变化,



一瞬由刀快步后撤。明晃晃的白刃往她脸上划出一道血痕。两、三滴血溅至海铃眼角,倒像红的痣。



人在精神高度集中时,痛觉会被麻痹。海铃凝望学生脸上的伤口:“你的进步果然很大,说不定真的可以——”



她识趣地住嘴,一切关于未来的谋划都为时已晚,一切声音都成为其中一人鸣响的丧钟。突然,二人听见号角声。恍忽间,立希以为有孩童在哭泣。大概是爱音或者祥子选择出击,势要杀出胜负来。立希眯眼观察几片月光,亮堂的宛如白昼。立希又明了一些事:“选择在月圆之夜进攻,想必也不是因为‘赐福’……你们准备好一大批火枪手,对吗?”



海铃就一边喘气一边笑:“啊啊,丰川大人的意思又被你看穿。你真是很厉害的人——等等,你是我的学生,那我也是相当不错的老师。”



立希还真没话反驳。短暂的对话结束,又是一轮刀剑相交。这次立希试探性地加入独创的招式,尽管不甚成熟,拿出来吓吓人业已足够。刀尖一点,在海铃的左手臂上留下两道血痕。



“其实,其实两个月以前,我就收到命令,要……”



“为何现在才动手?”



“在江户的时候——我和你说起我的过去的时候,我真的想过我们也许可以一起向同一目标前进。”



海铃移开眼神:“抱歉,让你失望了。”



立希嗤笑:“没关系,我该这么说吗?我的意志无法改变事实。”



海铃双手持刀,武士训练的刻苦如实反映在刀上,她在心里回答:并非是你自认为的那样,在过去,甚至是不久的将来,你的存在深刻影响周围人。想说的话说不尽,想做的事做不到。对二人来说都是如此。



逐渐,刀刃间不再针锋相对,而是步入某种共振。作为彼此的理解者,出刀收刀的方式也在走向默契。隐隐听见远处鸣枪的声音,不知何时,武士们身上皆是大大小小的伤口。立希忘记在心中计数,究竟缠斗多少回合?有没有超过椎名真希?待到可用的招式尽数倾泻完毕,立希有些站不住,还是逼自己直视海铃。



“究竟是为什么……”海铃举起武士刀。立希霎时明悟其中深意,只剩下那一式,不是吗?



二人摆出同样的起手式,回忆起即将动身离开的夜晚,命运交错,相似的目的终将她们引至对立面。奇怪的是,原本的阻滞感荡然无存,招式自然接续下去。海铃心下了然:原来这是需要二人合力才能施展完全的力量。而本该合力斩向敌方的刀法,被用于杀死故人。



真正的胜负只在一瞬:立希击碎海铃腰间的玉佩,海铃正好刺穿立希的腰。她感受到钢铁刺入血肉,绞断组织的黏稠。人类的生命,最终还是如谎言般脆弱。



海铃的双眼被血糊满,因解离而消失的痛觉正在缓缓流回她的身体,左臂抬不起来。立希倚靠在枯树下,一张嘴,血泡争先恐后地涌出。她已说不出话。海铃无法听见愤怒的指责,太安静,太安静了。



海铃也不管立希听不听得见,自顾自说:“说书人皆爱添油加醋,当年那一战,你姐姐仅三十回合就折了刀。你已经要——强过她。”



她慢慢走,脚步踉跄,一点一点挡住月光:“现在说已经太晚,但我相信轮回转世——我一直想告诉你,你应该多相信自己一些。”



一切?



“是的,一切。”



这算对我的安慰吗,抑或惩罚?



“我一直想告诉你,我本来的愿望……”



对不起,已经听不清了。



只见模糊的光也变微弱,海铃站在自己身前,举起手中,来自学生的馈赠。八幡海铃俯下身,替武士合上双眼。她最后看到的不过是深紫色眼瞳中一道悲伤的影子,那是她自己的影子。



海铃用刀支撑住自己,全身都像被捶打过一遍。不知不觉间,立希已经成长太多——在凌川时她就知道这事。如果再给她二十年,甚至是十年,她一定能做到一切想做的事。未来已然不存在。



又不知道过多久,不远处传来脚步声:“八幡将军,暂时撤退,敌军的守将调度来别地的——这是?”



三角初华没有摘面具,她仍是“悲伤”。海铃悠悠看她一眼:“显而易见,不是吗?”



初华的脑中闪过“战鬼”这个词,听过二人故事的她对这你死我活的景象颇为遗憾,转念一想又是必然。



海铃看上去比事外人还淡然,她站起身,借条干净的布擦拭血迹:“请帮我个忙。“



初华点点头。海铃笑起来:“哦,感激不尽。”



她俯下身,示意初华帮自己扛起武士。立希搭在海铃身侧。



“没有见证人,因而只好对阵人来收尸。”



“真的不用我来吗?”



“没关系。”



都说醉鬼和死人,切忌不要背负他们,他们远比山石沉重。海铃却觉步履轻盈,似有神助。





{\centering\section*{终 }}





千早爱音出色地完成守将的任务,不过她更挂念某个临阵脱逃的武士,心里朦朦胧胧的,隐约预测到结局。



经此一役,长崎素世终于同意休战和解。她得以见到“代行人”的真面目,与她猜测的相同。



“我做了许多无谓的事才坐在这里。”



“你觉得遗憾吗?”



“如果是以前的我,说不定现在就会杀掉你……”素世自嘲地叹气,“但是,现在说什么都没用了。”



丰川幕府独木难支,恐怕将在不久后覆灭,由本来的继承人接管大权。这片充满创伤的土地,迎来百年间离和平最近的一瞬。爱音“如愿”见到自己的偶像,被偶像追杀可谓是独一份经历。与此相比,她有更现实的事要考虑。



八幡海铃睡到日上三竿,春社刚过去,参与祈福的祭典是她这两个月来唯一一次出门。原因无他,单手做不来的事情实在太多,她又不想麻烦长崎家的侍从照顾自己。这天海铃突然想到,她该不会是在置气?结论让剑豪“噗嗬”一下笑出声,半只脚迈进坟墓的人,怎么还在推托责任。



晃荡着空空如也的左袖,用筷子夹起新鲜的鱼获。今天非同寻常,端来食盒的侍女没有马上离开,而是放下一封信:“这是千早大人托我给您的。”



海铃老半天才将名姓对应上人脸,自己同她没打过交道,不知有什么话要如此弯弯绕绕。她用木筷较粗的一头按住信件,慢慢将其展开。海铃本以为会是写满字的长信,结果上面就一句话,字迹力透纸背,笔走龙蛇。



海铃轻声读:“我们本来可以做到的。”



那么,这就是结尾了?我教给你的文法,半点没用上。得知学生瞒着自己的话,海铃心下释然,想着我也有没来得及告诉你的事情——希望好人长命之类的全是谎话,不可能实现的事,你是最好的证明。



她真实的愿望,既渺小又宏大,违背自然的规律,因而不可能实现。海铃抬头,仿佛与那时的身影重叠。



二人同时说出那个愿望:“但愿月不明。”



向着世界宣告。将一切和盘托出后,她终于觉得好一些。春日和煦的阳光照进来,像鹿身上新生的绒毛。海铃第一次萌生要出去走走的想法。她深呼吸,绕着院落走过几圈,额角渗出细密的汗水。天气逐渐热起来,新生命诞生的季节已经到来。



海铃闭上眼,试图冷静下来。思绪开始奔流:战争年代,最强者方能见证和平。强者固然应承担更多的责任,承载一位死者的重量固然不是难事——好了,就这么办吧。我会成为你的眼睛,替你见证最终的和平。在那之前,我不会去见你,我不会去怀念你。直到我准备好这份最终的礼物之前。



在那之后呢……在那之后,让我来请你喝一杯酒吧。借着月光,你不要嫌弃我仍然生涩的技艺。毕竟,毕竟那是我不擅长的事。





{\centering\section*{FIN}}





1:“袛园精舍之钟声……”一段节选自《平家物语》开篇。



2:江户在百年前被烧毁,neta自织田信长火烧比睿山。



3:长崎素世多少参考了立花訚千代的形象。



4:万户的故事很大可能是后人杜撰,年代已不可考。特此提出以免误会。



至此,感谢你的观看。希冀这些文字多少能触动你的心弦。

\end{document}
