\documentclass{article}
\usepackage[utf8]{inputenc}
\usepackage{ctex}
\usepackage{url}

% support Chinese Chars %
\newfontfamily\urlfontfamily{FandolSong-Regular}
\def\UrlFont{\urlfontfamily}

\title{【Mygo】唯有河流流淌}
\author{TP02型}
\date{20231010}

\maketitle
\url{https://www.pixiv.net/novel/show.php?id=20821251}
\newpage

\begin{document}
\CJKfamily{zhkai}

% : comm which is only poems edit in-use %


\Large

象征高二结束的那个春假,天气晴朗,包括MyGO在内,一切都处于正轨。不过高松灯却在此时遇上许多人一生都不会有的经历:被警察找上门。



回过神来,灯母已经准备好两杯果汁,担忧地关上房门。餐厅的长桌,一位JK一位便衣警察相对而坐。灯手心发冷,后者注意到她额头上的汗珠,问她身体是否还好。灯说,没关系,你有什么事吗……



警察咳嗽几声:“只是想问高松小姐一些问题——您认识照片上的人吗?”



灯终于抬头,努力辨认两张照片上的男人。其中一张是官方的证件照,一个还算注重形象,保养良好,仍能看出些许青年时代气质的中年男人;一张是看上去模糊不清的摄影,背景是发黑的木地板,光线昏暗,男人仰面朝天倒在地上,后脑勺的暗红色是血吗?一圈粉笔绘成的白线圈住男人的身体。灯一瞬间想了许多,最终只是确定,这就是晨间悬疑剧里经常出现的那种尸体。



警察收回照片,再次向她重复一遍问题。



“有点眼熟……但一下子想不起来。”



警察例行公事地说:“这个男人是您的朋友,丰川祥子小姐的父亲……前天晚上九点左右,我们接到了丰川小姐的电话。”



警察耐心地给灯解释当时的情况,循循善诱一般。可惜灯心不在焉,左耳进右耳出。警察最后说:“根据我们采集到的监控录像,当时只有您和另一位丰川小姐的友人到过现场。虽然这起事件我们已经基本确定是一场意外,但仍然有一些细节需要您的补充。”



灯听到这话,开始努力回忆,就像有人用一柄名为记忆的重锤敲她的太阳穴,一锤将她砸回再次与丰川祥子相遇的时候。



惊心动魄的MyGO结成史话,在大家的努力下以一种近乎喜剧的方式结束。半年后,摇摇晃晃站起来的人也可以跑两步,一辈子的未来也清晰可见。好事都全在这时候发生。少数令人遗憾的事全都可以甩在脑后,如果灯是一位洒脱的人的话……



某日,祥子收到佑天寺若麦的礼物:几张电影卡。祥子不知道该以怎样的心态去回答若麦,直到现在她仍没有把握判断这人的想法——总不能是要和我出门约会吧?



总之祥子收下礼物,不过电影卡被甩在一旁,整整一个月无人问津。原因无他,祥子是大忙人,从学校到乐队,从打扫房间到排练舞台剧,祥子忙到恍惚,既没有时间也没有闲钱去观看屏幕上他人的故事。事情的转机发生在祥子打工回来的晚上。她推开门,酒瓶摇摇晃晃,生活垃圾忘了顺路带出去。她慢慢走近堆满杂物的桌子,拿起一瓶冰镇矿泉水。醉鬼吐一地,清理起来相当要命。祥子生平第无数次觉得窒息,兴许是神的旨意,这次她没有选择合眼进入梦的世界。她转头就走,只带小手提包。



等祥子压完半小时马路,她才想起来应给自己找一个去处。那个家至少今晚绝对不会回去,自己也不想因为小事麻烦队友和前队友。她不带希望地摸了摸手提包里面,发现若麦送的那几张电影卡竟然还躺在里面。她对着昏黄的路灯端详卡片,确认地址和时限,向一个月前的自己说谢谢。



晚上十点以后的电影基本属于午夜场,全都是叫不出名字的工业化烂片,或者是一些早年cult片的集合重映。祥子除了价格之外什么也没看,随便选择一场最晚结束的电影,钻进影厅。平日里看着气派的地方此时只有她一个人,屏幕上的观影须知只放给她看。祥子很久都没有这种奇妙的感觉,就好像仍有人仅仅只为自己来考虑一样。



祥子随便选一个位子,一开始还打算欣赏一下许久没享受过的乐趣,不过这种一看开头就知道结尾的三段式作文,平平无奇的对白,让祥子疲劳的身体更加呼唤睡眠。半小时过后,祥子宣告失败,闭上眼遁入梦乡。鼻腔里难得没有充斥酒精与酸腐的气味,座椅算不上豪华却也舒适,BGM更是成为催眠曲。祥子难得睡一个好觉,没怎么做梦,做梦也是梦到春天,还有一棵盛放的樱花树。



这之后电影院成为祥子秘密歇息的地方,反正那个父亲除了喝酒要钱之外,丝毫不顾女儿跑去哪受了什么罪。祥子精打细算用卡里头的钱,生平第一次没后悔招来若麦。至于午夜场电影,能看到什么完全是运气,偶尔祥子会撞上重口味的B级片,让人迷惑的青春苦情烂片。甚至偶尔,只是偶尔,会有误入的普通观众,百无聊赖地看完电影起身就走,奔赴自己的生活。而祥子此时的人生经历已经远超普通人,远超许多JK电影主角。



这天祥子进来看《年少轻狂》,无聊的前辈们为了找乐子压榨后辈的故事。看到一半,她不知不觉睡着。不到一小时就感受到有人在轻轻推她。她睁开眼,半梦半醒间看见自己身上盖着一件薄夹克,手边还有一杯热拿铁。面前电影刚好放到片尾字幕,难不成是清洁员……祥子刚想道歉,抬眼一看,一个熟悉的人影:高松灯。



祥子差点吓得把咖啡甩飞出去,她迅速把夹克塞回灯手中,并且觉得这是一场真实的梦。但是灯怯懦地扶住祥子的肩膀。祥子感受到小腿传来一阵酸麻。



灯问:“祥子ちゃん,需要我扶你出去吗……”



祥子轻轻摇头,又坐回去。灯默默往下走,坐在离祥子大概三排远的位置。祥子能很轻松地看到她的小动作。不过灯一直是很安静的人。又过几分钟,祥子估摸着清洁员要来,于是站起身。灯也跟着她站起来,但是没跟着她一起走。祥子路过她旁边,把夹克还给她,自始至终只是轻轻和她说了一句:谢谢。



回家后祥子第一次没在意睡在角落的父亲,她踢倒门边的易拉罐,开灯走进厨房,一遍遍端详起电影卡。之后祥子去电影院的次数明显增多,以至于祥子自己开始担心电影卡的消耗。不过午夜场的电影往往廉价,祥子想着:有一天就算一天吧。不停为别人的故事砸钱。



其实也不是每次都能遇见灯。但就像是仓鼠跑轮一样,有回报总比没回报好。毕竟,不是所有人都拥有这种不幸的自由。如果遇上灯,两个人其实也不会坐并排,一般都隔上两三排,祥子往往都可以一边看电影一边观察灯。灯何苦来这看午夜场电影呢。她总是想,却也没有机会问。



直到祥子发现,她们之间的距离,在一场又一场难以形容的cult片中缩短。有时候放映的电影没那么无聊,祥子也会断断续续看完。灯转头看她,没想到祥子今天醒着。她很快转过头,装作一切都没发生。祥子觉得这窘迫令人惬意,会让人想起小田和正的歌来,那首《時に愛は》是怎么唱的来着:“你对我是如此深信不疑,就这样静默着,就这般注视着我。”



某次,灯做出耗费她一年勇气的选择:坐在和祥子同排,但是隔上三个位置的地方。这种距离感实在是令人尴尬难熬,可确实也是她们最好的前进方式了。祥子没有赶人,也没有起身离开。她想起自己在音乐教室时拒绝灯的事情,现在的自己比那个时候要成熟许多。也许是因为,自己知道高松灯是不会放弃的人。



灯买两罐柠檬苏打摆在二人中间,好像诱饵。祥子观赏屏幕中的男男女女,约莫半小时后,灯已基本放弃。可是祥子却和她希冀的一样,轻轻拿过苏打,依然对她说了声谢谢。灯又不知道该拿什么去面对人家,赶忙道歉逃跑一气呵成。



宛如河流一般的距离感依然横贯二人中间,不过祥子觉得这就是最好的选择。每当她隔好几个座位看到灯,总会遗忘几小时前自己身处的现实。等她回家,再次开始考虑星期几去丢不可回收垃圾。



某日午夜场是一部B级片,压抑许久的主角掏枪闯入社区,毫不留情也毫无理由地向路人开枪,甚至连相识的人都不放过。这种电影丝毫没有欣赏的价值,但奇异般与祥子对上电波。祥子想,我绝不会做这么冲动的事,可是……



灯迷迷糊糊的其实快要睡着,但她看见祥子的沉思的样子,直觉告诉她这是很重要的时候。她问祥子在想什么。这是几个月来她们第一次完整搭上话,由灯主动。祥子愣住很久,一句“你不用管”卡在喉咙里半天说不出来,最后她慢慢开口:“你不觉得这个男主角有点疯狂吗?”



“啊,确实。毕竟都杀人了。”



“有时候也想过,自己如果面临相同的压力,会做出什么事。”



“这样……”



“我没法像娱乐作品一样快意恩仇啊。”祥子笑道,“你不用摆出那一幅担心害怕的表情。我不喜欢。”



“我有个有点讨厌的亲人,大概是那种、无能为力的感觉。每次看见他,我会觉得很没办法,他看见我估计也一样。我总是想,如果我们中有一位能够离开对方的话,也许对所有人都好。”



“那为什么……”



“灯,难不成所有希望的事情都能实现吗?



二人一如既往分别,一如既往没有道别,各自走各自的路。灯心里有很多想法,可惜面对祥子一句也说不出来。她深知自己不像祥子一样坚韧,甚至可能仍然被祥子看不起。因此说不出那句你可以试试信任我吗。不过,灯也想为祥子做到一点力所能及的事。祥子藏着的那些事她多少已经察觉到一些,也不觉得那样就会开始厌恶祥子。



所以,她突然决定去拜访丰川祥子。



很容易就向队友打听到祥子家的地址,似乎随着时间过去,祥子带来的伤口也愈合的差不多。灯搭晚上的电车过去,穿过一条散发着难以言喻味道的河流,走过十几节阶梯,就来到那栋三户建门口。她慢慢上楼,听到“咚”一声,像是什么东西摔下去的声音。



灯来到祥子家门口,发现竟然没关门。直觉让她快跑起来,几乎是冲进祥子家。祥子看到不速之客,先是一惊,很快平静下来。祥子关好门。灯看见无法形容的画面:很多酒瓶但依然干净的家,有一股酸腐气味的空气,面前的地板上仰躺着一个男人,好像还有血迹从男人后脑勺渗透出来。



灯下意识想冲过去扶起男人,谁料祥子伸手拦住她。



“祥子ちゃん、这是……”



“请不要再往前了。这是我的请求。”祥子没有看她,但是声音相当隐忍低沉。



灯被这股气势吓到,祥子恐怕比退出CRYCHIC时还要强硬。她停下脚步:“那……”



“哈,你跟我来。我们出去说吧。”



祥子带灯出去,不忘关门打反锁。她们慢慢下楼。灯跟在祥子身后,亦步亦趋。祥子最后停在路边,三户建投下的一片阴影里。灯觉得站在阴影中的祥子像受伤的独角仙,让人不自觉想伸出手。灯先开口:“祥子ちゃん,那是你的父亲吗?”



祥子点头,又用恳求一般的语气说:“但是我现在不想聊他的事情。”



灯也点头回应她。灯突然意识到,这是CRYCHIC解散以来,她们第一次有机会面对面聊天。她向祥子表明来意。祥子埋怨她为什么老是这么突然,在电影院看见你的时候是这样,现在也是。灯跟她道歉。祥子哑然失笑:我说这话不是为了听你道歉!



祥子给灯说话的机会。灯想了半天,看看天再看看祥子的脸,问祥子最近过的怎么样。



祥子更是觉得好笑,这人老是关心一些不合时宜的事。她就用一种云淡风轻的语气解释起自己最近的行程。没提Ave mujica,只说有事要忙。



灯一直在想楼上的男人,担心从他后脑勺中渗出的血会滴在她们两个头上。她坚信,祥子绝对不是会杀人的那一批,更何况对象还是自己的父亲。可是……



我是祥子ちゃん的朋友,是朋友,就要相信对方。灯最后想。此时祥子也定定地盯着灯,抛出一个问题:“在想些什么呢,这么出神。”



灯说:“在想祥子ちゃん到底是怎么看我的……因为,我一直没有问过祥子ちゃん你的想法,一直在做自己的事情……直到现在才反应过来,我很抱歉。”



祥子一瞬间没能做出回应,她说:“不用担心这个。我如果讨厌一个人,我会直说的。”



“祥子ちゃん。”



祥子没回话,握住灯的手,灯手心也很冷,像是漂流来的企鹅。祥子说:“灯,你不能像我这样……”祥子说的确实是真心话,她不过是个人生前十几年生活充实无忧无虑,而后稀里糊涂被塞了一大堆阅历的女孩,手握着珍宝过活的普通人。实在覆水难收。



“算了。灯,时间不早,需要我送你回去吗?”



“唉、啊……不用。话说,祥子ちゃん真的不需要我来帮忙吗?”



祥子笑出声,这是今晚灯看到的她最真实的笑容。于是灯一时间一句话都说不出来。祥子说:“不用。你自己的人生还摆在那里,灯!”



祥子道别后,转身就走,她一次也没有回头。灯有一种以后不会再看见祥子的错觉。她也走得很着急,就连有个会让人侧目而视的女人从身边走过都不知道。她越走越快,好像这个地方从根本上来说就留不住她。穿过狭窄的单行道,很久才有几辆车驶过的马路,灯终于呼吸到外面的空气。她回过神来,可依然不懂祥子的行为,祥子的想法。凡事都有一个出发点,那究竟是怎样的契机,祥子做出这样的选择。灯不明白。又有一个声音告诉她,不明白的你要去问,就像一把上膛的手枪,不开枪就没有存在的意义。



但是,那颗子弹不过是一颗空包弹,弹壳徒劳地落下。灯踏上回家的道路,在路上她看到一只受伤的独角仙,蹲下来端详一会。这东西在喧闹的市中心不可能有。独角仙费力向前爬,一直爬进灯看不见的阴影里,这之后她就不清楚受伤的虫子的命运。灯搭倒数第二班电车回家,下车之后才觉得重新回到现实世界。路边便利店放着极度卑劣少女的《サイデンティティ》:“即使会忘记,即使会失去,我也要高唱再去拼搏,因为你说过,只差一点了,在那之前我要努力活下去,看到那时的屏幕,略微笑了。”



灯努力感受着那一瞬间的心情。回过神来,警察已经摊开他的笔记本,等灯回答自己的问题。灯看看警察,又看看桌上的果汁:“我去的时候,祥子ちゃん的父亲在喝酒。他一个人喝了很多。一直到她提议下楼去走一走的时候,他还在喝酒。”



“我明白了。在这个时间段,你和丰川小姐好像关注过什么东西?”



“啊、我们在看一只受伤的独角仙。之后我把她送上去,那个时候祥子ちゃん的父亲还活着。”



警察沉思一会,将灯的话如实记录下来。灯不停深呼吸来排解心中那种异样的呕吐欲。道德观念告诉她说谎是不对的,但是她又想到正错有时候没那么重要,祥子告诉她要做出自己不会后悔的决定。



警察合上本子,向她道谢,一直到最后都没有喝一口果汁。灯目送着他离开,终于开口:“那么,祥子ちゃん会怎样呢……”



警察叹口气:“这起案子已经没什么反转的可能了,我们的确更加担心死者女儿的问题。但是听说她甚至是主流出道乐队的核心成员。所以我觉得,应该不会有什么事。”



“原来如此。”



“有句话也不知当说不当说。也许那个嗜酒的男人死后,丰川小姐说不定能过上更好的人生。”



说完这话,警察再次向灯道谢,头也不回的离开她家。灯坐回原来的位置,父母都没回家。此时此刻只有灯一人,品味这份负疚和期待并存的情绪。她想起那个晚上,不止是和祥子的那番对话,还有最后,似乎和一个熟悉的人影擦肩而过。灯有个模糊的想法,也许之后还有人去找过祥子,或者祥子请来那个人。会是谁?这个问题已然不重要。



经纪人把唐突拜访的警察带到会客室,端上两杯浓茶。三角初华对经纪人道谢。警察开门见山:“不想过多占用三角小姐的时间。你有见过这张照片上的男人吗?”



初华看了看,点头说:“嗯,他是我朋友的父亲。”



“那么他在前天晚上不幸摔死的事,你应该也听说过……”



初华喝口茶对他微笑:“我知道了,我会把我知道的情况告诉你,请给我一些时间来回忆。”



初华收到祥子的信息时,大概是晚饭后两小时。初华事先同祥子说自己这时候一般没有安排,话外之音是想让祥子方便找自己聊天说话。没想到今天竟然成真,初华连忙回复。祥子那边已读来的很快。初华细细阅读信息,祥子说希望初华能来自己家。Ave mujica起步已经半年,初华也比其他人更加了解祥子的家庭状况,但无论如何,被邀请还是第一次。



初华先感到一阵揣揣不安。祥子告诉她,是和父亲有关的事。初华想起那个男人的脸。她开始收拾东西,跟staff说今晚要赴小祥的约。她不想搭公司的车。不到五分钟就赶上电车,祥子家离最近的电车站都有好几条马路要穿过。初华准备好全套变装,遁入人群。



她在祥子家附近看见高松灯。后者脚步如风,丝毫没有停下打招呼的意思。初华甚至差点没能留意到她。等初华回头,灯已经快走到她看不见的地方。她隐约看见灯迷茫的表情,心里那份不安又加重了些。她想,兴许是小祥在和父亲吵架。



她敲门,祥子请她进来。她看见打扫的还算整洁的室内——地板上仰躺着祥子的父亲。直觉让初华快步走过去探他的呼吸。结果不出意料,初华按住祥子的肩膀,急促地说:“小祥,怎么会……”



祥子直直看着她的眼睛。金色的瞳孔里藏着很多初华没法理解的思绪。初华又问:“是小祥你做的吗……怎么可以为了这种人。”



祥子坚定地摇头:“不是。我也明白的,初华。这只是意外而已,他是摔死的。”



“啊……”初华觉得,这真是适合醉鬼的死法。但是又意识到其中的苦闷。“那你要怎么办?”



“我叫你来,就是为了这件事。”祥子轻轻握住初华的手,“我已经打算报警。但在那之前,我们一起去Livehouse吧。”



初华还没弄明白出什么事,自己已和祥子走在去电车站的路上。初华不住偷看祥子的侧脸,试图找到一点她们未来的去向。祥子在第八次发现这束目光时回望过去:“在干嘛啊?”



初华连忙说,没什么。祥子转过头,于是初华也没法再看。两个人路过环绕这一片的河流,与其说是河流,倒不如称呼它为水沟:生活废水与塑料垃圾在上头漂浮,空气中弥漫着淡淡的洗衣粉味。祥子云淡风轻地提起,这条河叫涉谷川,我们现在走的这座桥叫惠比寿桥。



初华听见乐福鞋敲在桥板上的“咚咚”声:“想象不出这座桥和惠比寿的关系。”



“那是当然,你以为从四国到北海道,整个日本有多少座惠比寿桥吗?”



初华当真认真思考这个问题,她说我不知道。祥子泄气,说我也不知道。片刻后祥子突然反应过来,说自己忘了扔生活垃圾。



初华笑出声,就好像一切没发生过的那种令人眷恋的笑容。祥子看着这个笑容,会想起十年前在海岛上的时光,虽然短暂,可它确实存在过。祥子大声叹气。初华问她怎么了,祥子说有点后悔,为什么没带相机出门呢?



初华说,无所谓,以后我们有的是时间拍照片。小祥想要合照还是单人照?



祥子推一把初华,说我哪里表示要照了,你这借题发挥的本事是跟若麦小姐学的吗?



初华苦恼一阵子。祥子倒是在想自己的事,涉谷川静静流淌的样子让她发散很多思绪。如果乐队、人生就像一条河流,想要站稳也太难。不幸的事(幸运的事)一票接一票的来,别人都是稍显湍急的溪水,而自己像在大海里浮沉。



很快她们来到电车站,坐上最后几班电车去Livehouse。Livehouse周围的街区依然很热闹,这就是不夜国日本。初华觉得,刚刚在涉谷川那块的经历像一场不真实的梦。



初华对staff说:请给我们准备一间练习室。



兴许是sumimi的人气使然,staff愣怔一会才将钥匙递出去。初华跟她道谢,推开门,复又锁上。祥子坐在房间的最里面。初华拨开谱架,拉住厚重的窗帘:“要开窗吗?”



祥子没有回答。初华打开窗户,一点点风跑进来,祥子耳边的发丝微微扬起。初华拎一把椅子坐在祥子面前。没有人先开口,没有人表示出一丁点意见。初华其实很少有这种宁静的时刻,可惜现在并不能享受。二人面前的小圆桌上摆了一次性纸杯,抽纸,烟灰缸之类的东西。祥子伸手去拿纸杯。初华看见祥子心不在焉的动作,下意识拦住她,从随身携带的包里拿出一罐宝矿力递给她。祥子接过来,倒满两个杯子。初华道谢的话语卡在喉咙里,作为当红偶像兼任Ave mujica的MC担当,她生平第一次感受到绝对的沉默,不允许她说一些轻飘飘的话。因为在已发生的事实面前,一切话语都已经失去意义。



初华也不打算演一位颇觉惊讶的无辜朋友,可是无尽的思绪都只能弥漫在空气里,慢慢散开。外头吹进来的风悄悄变大一些,祥子按住桌沿,刚想开口,隔壁传来一阵琴声,有乐队在排练Heavenstamp的《Morning glow》:“你还没有做出承诺,但是那摇摆不定的表情,我看不懂……”



对于已经主流出道的乐队门把们来说,技巧当然漏洞百出,但是人们依然愿意为了这样的曲子这样的演出耗费时间精力,并且真情实感去爱她们。不知为何,祥子复又想起CRYCHIC。在一切都已发生之后想起CRYCHIC,祥子一时不知道怎样面对初华。于是本来想说出的话,到嘴边又变成一声叹息。



初华似乎看穿祥子的意思,她既没有表示出无奈也没有跟着一起长叹。祥子跟她说,这是意外,她不会为了那种人放弃自己的人生。所以初华自然而然选择相信她,你如果连自家的队长、十几年情谊的青梅竹马都不相信,你还能相信谁呢?于是就靠着这种小孩子赌气的想法,初华义无反顾(但也稍有担心)的轻轻越过圆桌和椅子,还有风,抱一下祥子。片刻后,祥子用点力气从初华怀里挣脱出来。



初华说,我带了手机。小祥,能和我拍一张合照吗?对,就在这里……嗯,要不要改变一下角度……



祥子默默点头,站起身跟在初华后面。初华伸出手臂,想让祥子挽住。祥子摆摆手,换一个靠在初华身上的姿势。初华把ipone当单反用,延时摄影,立好手机,隔音的泡沫板墙是最好的背景——3、2、1,哎,被一次性纸杯与烟灰缸以微妙角度支撑的手机,面朝下扣在桌面上,响起拍照完毕的“喀嚓”声。



初华“唉”地叫了一声。她又把手机立起来,偏过头去问祥子要不要再来一次。祥子刚想回答,staff敲门说“打扰了”。她说预约的一小时已经到了,还要再续时间吗?初华准备开口,祥子抢先她一步说:“不需要了,谢谢你,我们这就走。”



初华默默整理好手提包,这回换她跟在祥子后面。二人出门的时候,听见旁边那个乐队刚好唱到曲子的高潮:“信号又变成红色,时间停止,这是我一生中最珍贵的地方,如果这个世界结束,现在将会变得很好,是只有我这么想吗?”



初华看看时间,差不多快到电车停运的时候。她决定至少要把祥子送到电车站。她们原路返回,依然经过涉谷川和惠比寿桥。到进站口,初华本来想陪祥子坐一会,她还有很多事情想问,可是少了一点勇气与时间来实践。谁料经纪人打来电话,告诉她有一些紧急的事项需要商谈,公司会派车去接她。祥子用关心的眼神看过来,初华连忙说没什么事,工作上的问题罢了。



初华想说:我希望小祥也能关心自己一点,能依靠别人一点……因为可以信赖的人,绝不是只有自己。



初华说:“小祥真的不需要我来帮忙吗?”



祥子说:“不用担心,这是我自己的事。我相信你,但我不想把你也牵扯进来。”



初华点头说好,跟祥子说了几句很隆重的道别,交代三次有什么事请尽管跟我打电话。祥子不停说好好好。一直到初华快要离开祥子的视线,祥子环视周围。末班电车时刻,不处于市中心通勤线路,除了她们之外没有人。所以她深呼吸,大声喊:“初华——”



祥子很少这么喊初华,没有加上任何后缀。初华有些疑惑地回头。祥子又补上一句:“谢谢你,拜托你了……”



等初华在视野里完全消失不见,祥子才站起身整理需要精洗的旧衬衫,慢慢往电车站外面走。她没跟初华说,其实自家并不在电车月票的通勤地点,到了站还要走二十分钟。祥子一个人走在归家的道路上,离太阳升起、也就是“morning grow”还有一段时间,离她真正要面对的现实也还有一段时间。那个男人依然在家里的地板上,那块地方估计会留下永远都洗不掉的血渍。那种暗红色的东西会渗进最深处,好像藏在心中的虫子。



那么,现在也就不想去考虑这么多了。今天,见到很久都没能好好说上话的灯,陪初华拍了一张不成功的照片。回过头去,发现遗憾远比自己想象的多,能弥补遗憾的时刻,也来的远比自己想象的简单和沉重。很多事情是一句“对不起谢谢你”就可以了结的,她们只是在寻求这个。



但是,我偏偏就不能只给予这个。祥子想。她站在闸道口,栏杆放下,昭示着末班电车即将驶过。她就那么站着不动,电车驶来之前,她听见风吹起像海浪一样的声音。



经纪人给二人都端上一杯乌龙茶。初华思索一会警察提出的问题。警察没打扰初华的思考,他拿出笔记本,等待一个答案。初华一口喝去半杯茶:“我想起来……我确实在那晚去见了祥子,而且也看见了小灯。但是我没有和她打招呼,因为小灯走得太快。然后,我和祥子一起去Livehouse练习——”



“是哪一家?”



初华报出名字:“记录应该还在那里,你们可以去查。”



“我知道了。那么三角小姐你到丰川家的时候,丰川先生和她的女儿正在做什么?”



“那个时候,小祥的父亲实在是喝了太多酒。我进门的时候他已经喝到睡着,小祥在一旁收拾酒瓶子。看到我来,我们稍微聊了几句,就打算出门去练习。”



“还有吗?”



“小祥的父亲睡得实在太死,我们出门的声音都没能吵醒他。”



警察似乎是把所有细节都记下来,然后他又不厌其烦地确认一些时间上的问题,甚至还有祥子那个时候穿了什么样的鞋子。初华报以模糊的回答。半小时后,警察向她道谢。经纪人送走警察,忙不迭问初华这是怎么回事。初华出神地看窗外,繁星点点,些许微风,很多谎言,一切都让她回到那个夜晚。她说:“好朋友的父亲不小心摔死了……嗯,喝酒喝到摔死,所以,没什么好说的。”



“真是不幸。”



“是啊,真是不幸。”



这之后的几天,三角初华突然想去找高松灯。虽然只是堪堪聊过几句,初华也能隐约察觉到灯做出跟自己差不多的选择。而且无论如何,现在她都没法好好面对祥子。灯是一个像昆虫的孩子,你若存心去找她,很大概率扑空。因此,初华带好两张天文馆的豪华套票,坐在羽丘门口的咖啡馆里等人。她戴口罩和墨镜,却依然被一些女高中生认出来。初华跟她们打招呼,对她们说谢谢。不过眼睛一直停留在校门。没过多久,灯慢慢走出来。



初华叹口气,迅速结账出门。灯似乎一直在路上找什么东西。初华从后面赶上她,问:“小灯,在找些什么?”



灯吓一跳,看清是初华又很快安静下来。她说:好巧……”



初华摇头,拿出门票,她无意隐瞒:“啊,我是来找小灯的——有一点事情,我觉得只能跟小灯说。我们去天文馆吧。”



灯揣揣不安地跟在初华后面,摇摇摆摆但步伐坚定,让初华想起帝王企鹅。那自己突然有点像这只企鹅的姐姐一类的角色。她摇摇头,将乱七八糟的想法赶出去。



二人坐在因为工作日而没什么人的星象电影馆。初华闭上眼睛,这里面的景象她已经看过无数次。灯坐在旁边,安静的像睡着一样。初华对她说:“小灯,我听说你在学校里参加了观星社团……真好啊,星星之类的东西。”



灯小声回答:“如果是星星的话,初华小姐应该懂得比我更多才对。”



“我也只是个兴趣使然的爱好者。”初华忍不住笑出声。“电影要开始了。”



良久的沉默,好似要走进一个温和的良夜。灯看看漆黑的屏幕,又看看邻座初华若有所思的脸庞,她轻轻开口:“初华小姐、在你们那边的祥子ちゃん是什么样的?”



初华想:果然,灯比我更加勇敢。



她说:“没什么大的变化。既然你问了第一个问题,那就由我来问第二个咯?”



初华巧妙回避灯直击死穴的问题,开启她自己的回合。不过她们的问题都很轻,大多是关于对方生活方式上的,而且两人都没怎么提到祥子。此时此刻,沉重的过去被抛开,初华想要延伸到稍微长一些的地方。



电影快要结束,旁白介绍起土星星环的前世今生。在沉寂到让人想起《星际穿越》的BGM中,初华问出最后一个问题:“小灯你……为什么和我说了差不多的话呢?”



灯的声音很闷,像一个身处黑暗中的人:“因为,我希望所有人……大家、都可以幸福。想和大家一辈子在一起。”



初华听见很多细碎的声音,像是河流在流淌:“我也是。”



\newpage



本文在写作的时候,并未有一只独角仙受到伤害。感谢你阅读到这里。涉谷川与惠比寿桥是来自绿团的单曲《涉谷川》。《Morning grow》也是还算不错的Jrock。

再次感谢你!

\end{document}
