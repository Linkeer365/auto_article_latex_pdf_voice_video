\documentclass{article}
\usepackage[utf8]{inputenc}
\usepackage{ctex}
\usepackage{url}

% support Chinese Chars %
\newfontfamily\urlfontfamily{FandolSong-Regular}
\def\UrlFont{\urlfontfamily}

\title{散步}
\author{莫泊桑}
\date{1880}

\maketitle
\url{https://5165.org/wenxue/mobosangduanpianxiaoshuoxuan/3970652730.html}
\newpage

\begin{document}
\CJKfamily{zhkai}

% : comm which is only poems edit in-use %


\Large

当勒若老爹,拉布泽公司的记账员,从店里出来的时候,眼睛被夕阳的光芒照得有些迷糊了,所以他站了一小会儿。他整天在煤气灯那黄色的灯光下面工作,那个地方离商店后面有些距离,在一个狭窄的深似井底的院子里。四十年来他的白天一直是在那间小屋子里度过的,里面非常昏暗,即使在盛夏的时候从十一点到三点之间不点煤气灯的话也很难看清楚。



小屋里总是潮湿寒冷,窗户上有一个洞,它正对着一条发着陈腐气味的下水道。四十年来,勒若先生每天八点钟就抵达这所“监狱”;然后一直在那儿待到晚上七点钟,弯腰对着账簿,用那种忠实店员的勤奋作风记着账。



现在他一年有三千法郎的收入,初入公司的时候,他每年工资是一千五百法郎。他一直过着单身生活,因为他的收入不允许他有一个妻子这样的奢侈品,他从来没有享受过什么东西,也没有什么欲望。然而,有时他也厌倦了这种连续而单调的工作,他就有了一个柏拉图式的愿望:“上帝啊!只要我每年有一万五千法郎的收入,我的生活就舒适多了。”



他从来没有过过安逸的生活,并且,除了每月的工资,他从来没有任何其他收入。他的生活平平淡淡,没有任何波澜,也没有希望。每一个人做梦的能力,在他平庸的雄心里从来没有得到进展。



二十一岁的时候,他成了拉布泽公司的雇员。以后,他一直没有离开过。一八五六年,他失去了他的父亲,接着在一八五九年又失去了母亲。从那以后,他的生活只发生过一件小事,就是他在一八六八年的时候搬了一次家,因为他的房东要提高他的房租。



每天早晨一到六点钟,他的闹钟就像咔哒咔哒的链条一样发出可怕的声音,让他从床上猛跳起来。然而有两次,这个装置却出了毛病,一次是在一八六六年,还有一次是在一八七四年;他从来不知道为什么。他每天穿衣服,铺床,擦椅子和房顶,做完这些事情要花去他一个半小时。然后他就出门,在拉于尔面包店买一个面包,这家商店已经换了十一个老板,从来没有改过店名,他会一边走着去办公室一边吃面包。



他的全部生活都耗在那间狭窄昏暗的办公室里了,里面糊着同样颜色的墙纸。他进公司的时候是一个年轻人,作为蒲吕孟先生的助手,他的愿望就是能够接替他的职务。



他早已接替了他的位子,于是也没有什么指望了。



旁人短短几年的生活中总有许多值得回忆的地方,像突如其来的事情,甜蜜的或者悲惨的爱情,冒险的旅行,所有事情都是在自由生活中偶然发生的,而这些事情对他来说是完全陌生的。



日子、星期、月份、季节、年,对他来说都是一样的。他每天在相同的时间起床,出门,赶到公司,吃午饭,回家,吃晚饭,然后睡觉。从来没有什么行动和思想打断这些有规律的单调而相同的动作。



从前,他经常用他前任留下来的那个小圆镜子,看着自己亚麻色的胡须和鬈发。现在,在每天晚上离开公司之前,他会在那面镜子前端详自己花白的胡子和已经秃了顶的脑袋。四十年已经过去了,长久而又迅速,枯燥得像是一整天都悲伤的日子,简直就像失眠者的漫漫长夜!四十年来,他什么都没有留下,一个回忆也没有,甚至自从他的父母去世后,连一个厄运也没有。总而言之,什么都没有。



这一天,勒若先生站在公司的门口,看着落日的余晖有些目眩;他并没有回家,而是决定在晚饭之前去稍微散下步,这种雅兴一年中仅有五六次。



他走到林荫大道上,这里人潮在绿树下流动着。这是一个春天的傍晚,一个温暖和令人愉快的傍晚,让人心中充满了生命的快乐。



勒若先生用他那种老年人式的碎步走着;他心中充满了欢乐地走着;世界平静祥和。他走到香榭丽舍大街了,接着继续走,看到那些年轻人轻松地跑动,他更加活跃了。



整个天空像是燃了起来;地平线上的凯旋门在辉煌的背景下站立着,好像一个被火围起来的巨人。等他走到这座巨大的建筑物跟前时,这个年老的记账员觉得自己的肚子饿了,然后他就走进一家酒馆里去吃晚饭。



有人招待他坐在店外人行道边的座位上,他叫了一些羊肉、色拉和芦笋;勒若先生很久没有吃过这么好的晚餐了。他在他的干酪上浇了一小瓶勃艮第葡萄酒,吃完饭后又喝了一杯咖啡,对于他这不是常有的事,最后他喝了一小杯白兰地。



等他结完账后,他觉得非常有朝气,甚至有点激动。然后他自言自语道:“多么愉快的晚上啊!我要一直走到布洛涅森林的入口为止。这对我是有好处的。”他动身了。一首过去他的一个邻居经常唱的老曲子总是萦绕在他的脑子里。



林子新绿时,



情人向我语:



我盼至爱来,



同往花荫下。



他不停地哼着这首曲子,哼完一遍又一遍。夜色已经降临巴黎上空了——一个无风而暖和的夜晚。勒若先生沿着布洛涅森林大道向前走着,并且看着那些经过的马车,它们都亮着灯,一辆跟着一辆,可以瞥见里面有一对一对的情人,女的穿着浅色连衣裙,而男的穿着黑色礼服。



那是恋人们组成的长队伍,在一个星光闪耀而且温暖的天空下兜风。他们快速不断地来来往往,坐在四轮马车里面,静静地互相搂着,沉溺在幻想之中,沉溺在欲望之中,沉溺在相互拥抱的颤抖之中。那些热烘烘的阴影像是充满了漂浮着的吻。一种温存的感觉充满了空气。所有那些四轮马车里都充满了温柔的情人们,所有这些人都被相同的想法和相同的目的弄得极其兴奋,好像散发着一种烦扰、微妙的东西。



最后勒若先生走得有点累了,于是他就在一条长凳上坐下来,注视着那些载着爱情的马车一辆跟着一辆驶过去。而几乎立刻就有一个女人走向他,并且在他旁边坐了下来。



“晚上好,先生。”她说。



他没有回答。她接着又说:



“让我来爱你吧,我的可人儿;你可以看得见我是很可爱的。”



他回答说:“夫人,您认错了人。”



她伸起她的胳膊挽住他,同时说道:“快点吧,现在;别傻了,听我说……”



他站起来走开了,心里有些不痛快。没走几米远,另外一个女人又走到他身边问道:“您愿意在我身边坐一会儿吗?”他说:“您为什么要过这种生活?”



她站在他面前,并且声音变得嘶哑和愤怒起来,大叫道:“好吧,无论如何不都是为了给自己找乐子。”



他坚持用柔和的声音问道:“那么,是什么让你这样做呢?”



她咕哝道:“我总要活下去,愚蠢的问题!”接着她就哼哼唧唧地走开了。



勒若先生有些茫然地站在那里。其他女人从他附近经过,和他说话,邀请他。他感觉好像被一种不友好的东西包在黑暗中。



他又在一条长凳上坐了下来。那些四轮马车依然隆隆驶过。“我不应该到这里,我完全被弄得烦躁了。”他暗自想着,然后他开始想着经过他眼前的这一切:这些被玷污或者激情的爱情,所有这些被售卖或者给予的亲吻。爱情!他几乎不认识它。在他这一生,他只熟悉两个或者三个女人,他的收入要求他必须过那种平静的生活,他想到他从前过的生活,那是和别人都不同的,是如此单调,如此哀痛,如此空虚。



有些人的确很不幸。忽然一下,仿佛他灵魂上的面纱被人撕开了,他感到了无止境的痛苦,他存在的单调乏味:过去,现在和未来的苦楚;他最后的日子和最初的一样,无论在他前、后或者四周一无所有,心里一无所有,任何方面都一无所有。



车流依然不停地经过。在那些快速经过的敞开式的马车里,他依然能看见一对对静悄悄地互相搂着的人,在他眼前显露又消失。他感觉全人类都沉醉于充满了喜悦、快乐和幸福的世界在他跟前流动着。而他则孤零零地站在一边看着。明天,他依旧是孤零零的,永远孤零零的,比其他任何人都要孤独。他站了起来,刚走了几步,突然感到疲倦了,如同他刚刚步行了很长一段旅程一样,他在挨着的长凳上坐下来。



他在等待什么?他还指望什么?什么都没有。他想到当一个人年老的时候,回到家里,看到许多小孩时一定非常开心。当一个人被那些自己养育的人围绕,爱你,关心你,会告诉你一些可笑和愚蠢的小事,让你心里感到温暖,让一切得到安慰,那么这时的暮年也是让人高兴的。



然后,他想起了自己的那间空房间,干净而忧愁,除了他自己从来没有人进去过,于是一种悲痛的感觉充满了他的心灵,那个地方,在他看来,比他的那间小办公室更让人伤心。



从来没有人去过那儿,也从来没有人在里面说过话。它是死的,沉默的,没有人说话的回响。好像这些墙还保留着一些曾经住在里面的人的影响,保留着几分他们的姿态、形象和言论。那些被幸福家庭住着的房子比不幸的人住着的房子更快乐。而他的房子没有他人生的记忆。后来,想到要返回那个地方,孤零零地躺在自己的床上,再次重复那些每天晚上的种种行为,这个想法让他很害怕。好像要远远逃离这个险恶的房子和那个必然到来的时刻,他站了起来,沿着一条小路走到树林茂密的地方去了,在那里他坐到草坪上。



他听见了一种在他周围,他头上,四面八方,连续的,巨大而混杂的隆隆声。它混杂了种类数不清的噪音,一种不确定的和有节奏跳动的生命:那是巴黎的气息,像一个巨人在呼吸。



太阳已经升得很高了,在布洛涅森林上面照下一层光浪。有几辆马车开始四处流动了,一些骑着马的人出现了。



一对情人在一条人迹罕至的小路上走着。突然间,那个年轻的女人抬起脑袋,看见了树杈上有件褐色的东西;她有些吃惊和不放心,于是举起手,大叫道:“看!那是什么?”



随后她尖叫了一声,就倒在了她同伴的怀里了,他只得让她躺在地上。



警察被喊来了,他解下了一个用自己的吊裤带自缢的老人。



调查表明他死在前天晚上。那些从他身上找出来的文件,表明他是拉布泽公司的记账员勒若。



他的死亡被归因于自杀,动机不明。或许是一种突然、疯狂的行为吧!




\end{document}
