\documentclass{article}
\usepackage[utf8]{inputenc}
\usepackage{ctex}
\usepackage{url}

% support Chinese Chars %
\newfontfamily\urlfontfamily{FandolSong-Regular}
\def\UrlFont{\urlfontfamily}

\title{【初祥】追随:序章——脱缰之马}
\author{sandman}
\date{2023-10-14}

\maketitle
\url{https://www.pixiv.net/novel/show.php?id=20842982}
\newpage

\begin{document}
\CJKfamily{zhkai}

% : comm which is only poems edit in-use %


\Large

那年夏天,大概是因为疫情的缘故,客服工作的密度比以往高了两到三倍。通话开始,已经形成肌肉记忆的敬语脱口而出。人们总是以为自己应当在对话时保持尊敬,但那并不适用于客服。急躁的诘问,狂怒的辱骂,还有那些吐字不清无法辨别的小声嘀咕。接通,问候,安抚,忍耐,安抚,询问,解答,请按1来为我的工作打个好评,再见。然后从头开始。



我所工作的客服部在一幢老旧的十层办公楼的七层租了这间办公室。空调机不知道坏了多久,空气当中总是弥漫着一股刺鼻的怪味。自从我在这里工作以来就没有见到过有人修过。我曾经试着和我的上级汇报此事,但他的答复是:“我知道了,让我打完这个电话。”随后便杳无音信。在夏天的时候室内温度可以达到稳定的32摄氏度,每天下班之后我那件可怜的唯一拿得出手的洋装的领口便会被盐渍染黄。但是这些无关紧要。人穷不能志短,这是我的父亲过去时常和我说的话。一个人如果失去了尊严那就连人都算不上。我还有家可以回,有学可以上,有饭可以吃。世界上有三分之一的人连饭都吃不饱,我生活在日本,起码可以喝上干净的水。我所经历的没有什么大不了的,只是平平无奇的不幸。我是丰川家的女儿,我可以忍受这些。我怀抱着这样的信念咬紧牙关。我怀抱着这样的决意走上现在的道路。





{\centering\section*{序章——脱缰之马}}





下班之后回到不到三十平米的出租屋,过去被称为父亲的男人一如既往地窝在电视前,身体弯得像一颗不堪重负的歪脖子树。电视屏幕里的光照亮了他胡子拉碴,略显浮肿的脸。丰川祥子皱起眉头,他看起来今天气色不错,起码地上只有三个瓶子。她努力地在脸上堆起笑容,虽然皮鞋有些硌脚,但是家里的地板如果弄脏了袜子之后会更加麻烦:“我回来了。”



“哦,小祥你回来了啊。”男人听到了她的声音,眼睛里先是有两道精光亮起,接着又黯淡了下去,“今天在学校怎么样?”



“还好吧。”和过去的每一天都一样,一个人上学,一个人在钢琴教室练琴,一个人透过窗户看着触手可及却已经无法再接近的她,一个人躲着其他人离开。“都说过了,别再喝酒了,对身体不好。”



“就喝两三瓶,你就让我一下吧。”男人露出苦笑,“我也就只有这点指望了。你妈那边怎么说?”



怎么说?她需要告诉这个人她根本就进不去丰川家的老宅,也不想进去吗?这个活在幻想当中的,依靠酒精欺骗自己的可怜虫,需要让他的生活更加糜烂吗?“她还是不愿意见你,但是……”



“但是?”眼睛里又开始闪光了。



“但是她给了我这些钱……”你们这帮骗子!你凭什么不给我们解决——



“太棒了!”男人没有注意到她的窘迫,一把将她手中的钞票夺过,“我就知道她心里还是有我们的。小祥,”也许是酒精的作用,他突然激动起来,一把抓住祥子的手,哭了起来,“我没有用啊,我没有用!要不是我的话,你也不至于……”



“别说了,父亲大人。”烦躁,胸中仿佛有烂泥在翻涌,“这是我当初自己的选择。我会支持您的。”鼻涕和眼泪粘在袖子上,又得洗了,真恶心。



“小祥,你等着,我很快就会找到工作的。到时候我们有了钱,你也就不用在外面忙那么久,我们也可以重新和妈妈在一起。”男人放开她的手,用力擤鼻涕,“我其实已经有眉目了,我以前的一个老朋友,和我说他女儿最近想要学钢琴,让我去试试。他是个好人,他肯定可以帮到我们……”



电视机上的喜剧演员依然在喋喋不休,可是祥子已经听不见了。她自然而然地屏蔽了周围的声音。在过去这是无法想象的。她生性敏感,听力敏锐的她可以轻松地在地铁站的一片嘈杂中发现卖唱艺人最微小的失误。这种天赋让她过去必须维持绝对的安静才能入梦。现在的她躺在坚硬的地板上,身下只有两条薄薄的被单,不远处的高架桥不时传来火车的声音,父亲晚上从来不关电视,但是这些都与她无关了。刚刚搬进这里的时候,天花板上爬过的蟑螂曾经让她彻夜无眠。这些棕色的小生物带着它们的两根触须,足足有她的两节手指那么长,在衣柜里,桌子上,镜子前,乃至于男人的空酒瓶中不时现身,但这些已经不再让她惊恐了。男人在晚上从来不关电视——他往往是看着电视失去了意识。他的呼噜声时断时续,有时仿佛震耳欲聋的雷鸣,更多的时候则是挣扎的喘息。空气当中弥漫着尘埃,霉菌,以及垃圾腐败的臭味,如同甜腻的泔水,令人作呕。她曾经试图在男人入睡之后关掉电视,但是第二天早上起来的时候,电视依然是开着的,某位不知名的演员正在推销一款她闻所未闻的洁厕灵。男人瘫软在桌子上,好似一条硕大的鼻涕虫,肩膀上的微弱起伏才可以勉强证明他的生存。



她闭上眼睛,不知不觉中,她睡着了。她渴望安息,可是自从她搬到这里之后她的睡眠往往多梦。今天也不例外,在梦中,她仿佛回到了自己的少年时代,一个腼腆的女孩牵住她的手,引着她向前。梦中那人的手温暖异常,自然而然地令她心安。女孩带着她走到一座桥上,此时正值晚春,花瓣从树上飘落,如同细雨般散开。女孩用温柔的声音唱起了一首遥远的歌,恍惚间,光芒从女孩的背后溢出,仿佛天使的双翼。



她醒了,身体下面的地板一如既往地硌得她的背生疼。头晕脑胀,脑袋上仿佛有小锤在打。电视机一如既往地开着,一个画了浓妆的女人正在情绪激动地喊着什么。男人如同一个破烂的玩偶一样在地上四仰八叉,地上又多了两个新的空易拉罐。空气当中飘着一股刺鼻的氨气味,她爬起身,转过头,不出所料,男人的裤子湿了。她叹了口气,窗外正好有一辆火车经过,仿佛有巨龙在低吟。太阳还未升起,借着凌晨的微光,她关了电视,然后忍着恶臭,把裤子连带内裤从男人身上拽下,接着蹑手蹑脚地打开门。住在这种地方的人大多数对邻居没有什么挑剔,但是她的教养告诉她不应该因为自己的不便制造过多的噪音。她从柜子里找出脸盆,穿过走廊,脚下的木板因为发霉而颤抖着吱嘎作响。所幸洗手间距离这里并不远,她打开电灯。电灯咳嗽了几声,点亮了。虽然房东会偶尔过来打扫,但是地板上的黄色似乎已经渗入了瓷砖里,这幅景象对她已是平平无奇的日常。她打开水龙头,把裤子泡进盆里,抓了一小把洗衣粉,用力揉搓。她还不擅长干这种事情,她的手还不擅长——抛弃那些无谓的自怜,你可是丰川祥子,这种事情不算什么!她咬住牙齿,手上变得更用力了,仿佛要把裤子撕烂一般。



从赤羽站到羽丘需要大约三十分钟的车程,从家里出发到车站需要额外走十分钟的路。换言之如果她想不迟到的话六点四十就应该出门了。她把裤子从水里捞出来用力拧干,挂在绳子上,然后洗了洗手和脸——哪怕身处牢房之中也要维持优雅和尊严,这是那个女人告诉她的。她在镜子前面端详自己,这真的是她自己吗?这真的是她所认识的丰川祥子吗?一头蓝灰色的长发无力地垂在她的身后,琥珀色的眼睛里目光涣散而无神,两个大大的黑眼袋出卖了她的疲惫,苍白的脸上看不出一丝笑意。她用力扭曲脸上的肌肉,想要挤出一个笑容。不对劲,还是看上去不够开心。她把手指伸进嘴里,用力向上勾——还是不对劲,还是不够开心!她叹了口气,终于放弃了。转而开始扎头发,接着用化妆盒简单地遮盖了一下眼袋。剩下的部分就只能祈祷其他人不要注意到她就行了。她换上校服,把垃圾打包,接着拎起书包,走出了家门,又是新的一天开始了。



在地铁上,她从书包里取出昨天晚上从便利店淘来的面包边。在夏日的高温下,哪怕只是经过了一晚,面包也已经开始发酸了。她曾经考虑过不吃早饭来省钱,但是那样的话整个上午她都会因为低血糖听不进去课。她只好遵循过去从来没有重视过的医疗建议,把每天晚饭的钱省出来一半用来买面包边——往好处想,起码她还有饭吃。取出水瓶,冰冷的自来水将口中残留的有些发硬的面包边冲下喉咙。她还活着,她还没有放弃。



“予独爱莲之出淤泥而不染,濯清涟而不妖,这句话出自周敦颐的《爱莲说》。周敦颐是宋朝人,写这篇文章是为了借物喻人,通过赞美莲花的品格来赞美这样的一类人。莲花的这种不因外界环境改变的品格,是君子的特征……”后藤老师在教室前面喋喋不休地说着,祥子却在忙着奋笔疾书。汉学课的这部分内容她早就事先预习过了,比起听课她更愿意去利用上课时间赶数学作业。每天放学之后除了少量的练琴时间,就是要在六点去托内利保险公司的客服部打工,下了班之后疲惫地回到家,她实在是没有意志力在那间陋室里写完全部的作业。她最开始的时候也试过让那个男人把电视的声音调小一点,在家里用折叠桌写作业,但是没有用。自从离开丰川家之后那个男人仿佛失去了全部的魂魄,除了喝酒之外就靠盯着电视消磨时光,在那么小的空间里电视的声音根本无处躲避。作业的质量更是一团乱麻。起码在课上,老师的声音还可以当成白噪音来屏蔽掉,但是在那个空间里的一切都让她烦躁。根本没有办法集中。不要去想那里,不要去想那些不愉快的事情,可恶,可恶——



“丰川同学?后藤老师叫你过去一趟。”为什么?已经下课了吗?什么时候?她麻木地把作业本推到一边。羽丘女子学校是以升学率为招牌的学校,她选择这所学校的主要原因就是为成绩优异者提供的足以抵消学杂费的奖学金。但作为代价,羽丘的课业强度比花咲川高出了整整一倍,甚至要稍微强过月之森。她只能利用上课那些捉襟见肘的时间写作业和预习。过去她虽然因此被谈了几次话,但是看在她成绩尚可的份上,从来没有人真的因此为难过她。但是当她今天走进办公室的时候,她注意到了,后藤老师显得欲言又止,这是她第一次在后藤的脸上看到这种表情。



“后藤老师。”她站住了,不知为何紧张起来,“出了什么问题吗?”



“丰川同学,坐。”后藤指了指对面的椅子,又咽了口唾沫,“是关于你的奖学金的事情……”



“怎么了,老师?”



“就是,怎么说呢?”后藤局促地笑了,“丰川同学,你知道,我们学校的奖学金名额并不多,因此竞争一向很激烈。”实际上每个年级只有三个人的份,“因此,校方建立了严格的考核制度,来确保这笔钱总是能够抵达需要它,同时配得上它的人手中。简单来说,你需要在校内的所有主要考试当中拿到80以上的偏差值,这你是知道的。”



“是的。”她叹了口气,“这是当时我申请奖学金的时候写在纸上的条件。”



“我知道你一向是个刻苦的好孩子,可是,我从很多老师那里都听到了这样的评价,你最近似乎在学业上没有以往那么用功了。有些老师甚至觉得你在糊弄作业。老师真的很不愿意这样说,可是,唉,怎么说呢?丰川同学,学生的天职是什么,你是明白的吧?”



“……是学习。”



“对的,就是这样,是学习。学校启发孩子,学校引导孩子,学校让孩子们做好步入社会的准备。在学校里,你获取知识,结交朋友。老师知道你的处境比较艰难——”不对,不要同情我!“——可是你呢?你真的把这件事当成自己人生中最为重要的事对待吗?”



“……我会努力的。”



“这不是努力不努力的事情,丰川,事情不是你说了努力你就会去努力的。最重要的是你的态度。而坦白地讲,我看不到你对于学习认真的态度。你最近的几次小考的问题很大,很多不该犯的小错误都犯了,而且是在所有学科上都这样。这种状态是不对的。如果这样的话……学校可能不得不考虑在明年停止发放你的奖学金。”



\newpage



“素世在组新的乐队。”睦一如既往地面无表情。“她还拉拢了我。”



这是命运对我的傲慢的报复吗?她想。要失败了,怎么办?没有奖学金的话根本没有钱付平时的学杂费。如果可以让父亲不再喝酒——不行,她根本没有办法从那个男人那里抢走钱。每月的抚养费一打进账户就被他喝个精光。就算省下来那些钱也根本不够支付。光是靠着东京都的私立高中学费补助只能付清学费,平时的伙食费和房租都是依靠奖学金和打工勉力支撑。就算如此,父亲还总是趁着自己不注意偷偷出门买酒,商店自然乐见其成。就算是把家里的钱藏起来也无济于事——他会打白条。而现在,素世还在惦记着那个crychic……



“这是对我的报复吗?”



“怎么办?要去月之森吗?”



“我怎么可能——”冷静下来,睦没有恶意,她只是想要帮助你们。别迁怒给她。她把自己按回座位上。



“她的乐队,最近要办live了。”



“那样的话,就只好把话说明白了。”买票要花600日元,不当紧,不过是两天的饭钱罢了。



\newpage



在一个理想的世界里,每个人都可以心平气和地在交流时维持自己的体面,每个人都可以清楚地明白他们的情绪于事无补,并意识到迁怒只是徒劳地让更多人受伤。而我们并没有生活在那样的世界里。大多数人在打电话过来的时候往往已经是孤注一掷,他们的耐心早已在和机器的博弈中丧失殆尽,剩下的只有无处安放的怨气和狂怒。她的电脑显示屏顶端贴着一张纸条,那就是客服部的五字箴言:“温良恭俭让”。



“这里是工号0080为您服务。请问您的问题是什么?”



“你们这些见死不救的狗东西,为什么不能赔钱?”这种情况已经算是比较好的,虽然话说的难听,但是只要用程式化的方式应答就行,很快对方就会失去热情。



“我们当时合同上明明白白写着的,应该可以赔。”高级知识分子,这种比较难缠,他们往往自信,聪明,对你的每一个逻辑漏洞紧咬不放,那个女人过去就是这样的人。步步为营,稳扎稳打,一点点谋取自己的利益。处理一个这样的电话往往需要三十分钟以上,但是工钱却绝不会因此而提升。



“小姑娘,帮他们说话赚的钱多吗?从我们这些老实人手里抢钱有意思吗?”这种是最为麻烦的,聪明,而且直入主题,他们从一开始就不是为了获得解决方案,而是通过人身攻击发泄怨气,通过诉诸道德来缓解压力。客服的服务守则要求你维持解决问题的态度,不能主动挂断电话,而对方则借此不断拖延时间,最后带着满足感离去。



“这位客人,我们还是说一下您遇到的问题吧?您的保单号是多少?”



“004276XXXX,你还没回答我呢?你一天能赚几个钱啊,这么卖力?”温良恭俭让。



“这种问题不属于我们业务的范畴——”



“那么换一个问题吧,你的声音听上去挺好听的,你上司的那东西味道如何啊?肯定尝过了吧?”



“我们不负责回答这种私人问题。”温良恭俭让,温良恭俭让。



“哦呀,开始摆谱了。没办法,我们这种没文化的人实在不知道该怎么说话,您是不是看不起我了。”



“这位先生,请问您还有什么关于您保单的问题吗?”莲,出淤泥而不染,濯清涟而不妖——



“你的语气听起来有点不耐烦啊,你这是什么服务态度?我都还没生气呢。对了,你在这里工作的钱你拿去干什么了,是不是买屎吃啊?要不这样,我拉一泡邮寄给你,你也不用买了……”



温良恭俭让。



\newpage



“丰川啊,”山田摘下眼镜,擦了擦汗,“这已经是你入职以来第三次被投诉了。按理来说我们应该开除你的。”



“是的。”失败了。停下来,不要去想。没有钱付房租,会被人知道,需要他人怜悯,需要别人的同情——



“但我也知道你的困难。就从你之后六周的薪水里扣一半吧。”



“感激不尽。”温良恭俭让。



\newpage



回到家中,家里依然一片漆黑,电视机上的男人喋喋不休地说着,地上的易拉罐变得更多了,房间里空无一人。她环视四周,挂在墙上的西装和屋子里的皮鞋不见了。男人大概是出去了,只是不知为何还没有回来。这的确是某种不同寻常的状况。她打开手机,并没有新的信息或者来电,男人似乎认为完全不需要解释自己的去向。她叹了口气,关掉了电视,展开折叠桌。起码可以趁着现在写一下下周交的作业,这样周末打工的时间也可以更多一点。如果他就这么消失了怎么办?没有理由的,她脑海里突然浮现出这样的念头。男人是个怎样的人呢?



小祥,来吧,跟着我一起弹。啊,你可真是个小天才,你喜欢钢琴吗?好吧,那就跟我一起学吧。



小祥,你爹我年轻的时候过的日子很辛苦,幸好遇到了你妈,然后有了你。你是我们俩一生的贵人,所以给你取名叫做“祥”,代表吉祥,美满的意思。希望你可以和这个名字一样一声都吉祥美满,这个名字还是我挑的呢,怎么样?很有文化吧?



小祥,虽然你生在丰川家,但是我希望你不要忘了,世界上也有很多生活不如我们的穷苦人。你一定要记住,面对这些人你要心怀怜悯,因为不是所有人都能像你一样拥有这样幸福的童年。你现在所拥有的一切……都是因为丰川家。



小祥,人穷不能志短,哪怕有一天,如果实在不幸的事情发生,你不得不过上穷困的生活,你也要记得有一件事是不能放弃的,那就是人的尊严。人正是因为有了尊严才和其他动物不同。钱可以再挣,但失去的尊严……不会再回来。



小祥……家里现在的情况比较复杂,接下来的日子很不容易,爹现在实在是没有钱,你还可以选择去和你妈求一下情,让你和她一起过……



小祥,爹没有用,没有用啊……



她睁开眼睛,肩膀酸痛地仿佛要裂开一般,手机正在响。一个陌生的号码的来电。她看了眼时间,凌晨两点。桌子上摊开的作业本上密密麻麻地写着字,可是自己居然完全失去了写下这些字的记忆。她又看了一眼手机,电话已经挂断了。有三个未接来电,都是同一个号码。没等她有时间思考,电话又响了起来,是睦,她接起电话:“睦。”



“涩谷,xxx大街,xxx号,他在那里。”



“……我明白了。”没有任何多余的话语,她已经明白了对方口中的“他”是谁。一股寒意爬上了她的脊背,“周围有别人吗?”



“他声音很大。”电话那边的声音停顿了,仿佛在迟疑些什么,“祥,多珍重自己。”



“……谢谢。我问题不大。”她揉了揉依然酸痛的肩膀,站起身,看来这下是落枕了,也不知道这屋子里还有没有膏药,“我这就去找他。”她挂断了电话,站起身,电话又一次响起了,还是那个陌生来电。她皱起眉头,最后还是接通了电话,“是谁?”



“是丰川祥子小姐吗?”



“什么事?”



“这里是东京警视厅,令尊现在和我们在涩谷的TGW物产门口,您可以来这里一趟吗?”



\newpage



虽然在她抵达TGW物产的时间已经接近凌晨三点,但是在门口还是聚集了不少看客。人头攒动,空气中漂浮着好奇,兴奋,以及对于闹剧的饥渴。她的心脏狂跳起来,她几乎想要转身逃走。冷静下来。现在有多少在看?会不会有羽丘的同学?不太可能。月之森的呢?更不可能了?那些人的家长呢?她在羽丘没有什么名气,就算有羽丘的家长也不会认出她。月之森的人大概不会在凌晨三点的涩谷徘徊。冷静下来!会有记者和狗仔队嗅着新闻的腐臭味过来吗?丰川家大概会摆平他们。做好你的工作!她深吸了一口气。当初真应该让初华教会自己爬树的。她向后退了几步,找了一个高一点的台阶,稍微踮起脚尖——幸好周围的不少人也因为花粉症戴上了口罩,不然她的口罩会显得很不合群——果然,在人群的正中间,那个灰头土脸的男人正带着一块牌子,跪在门口,周围的几个警察一边和他说话,一边努力喝退想要凑近的人群。果然,该来的还是回来。她又吸了口气,拨开人群:“借过一下,我是家属。”



她的声音并不大,却仿佛在喧嚣的河流中投入了一块巨石。议论声减弱了。人群分散开来,无数道目光——质疑的,好奇的,嗜血的,讥讽的——刺向她,勾住她的皮肤,撕扯她的血肉。你还戴着口罩,你是丰川祥子,不要让他们看到你的软弱。成为莲花。她挺起胸,向前走去。一个圆脸的矮胖子最先注意到了她:“我是三森警员,您就是……祥子小姐吧?”



她点点头,对方很体贴,没有提到那个和TGW相近到无法让人不去联想的名字。但是这大概已经无关紧要了。她的目光越过人群和警察,停留在那个蠕动着的男人身上。男人穿着他那套不知道多少年没穿过的洗的发白的旧西装,在他的脚边有着一滩散发着刺鼻的酸臭味的呕吐物,不知何时沾到了他的那条西裤上。男人正在一边痛苦一边吐出一些支离破碎的语句:“……我和你说,他们从来都瞧不起我。你们不知道。他们就觉得我是个臭搞音乐的,什么也不懂。他们那些人都是一伙的,什么丰川家,可笑!我和你说……你知道吗?可耻啊!可耻!”他突然歇斯底里起来,“我真是个可耻的畜生!可是那些人比我还要可耻!他们还说我,他们才是真正的冷血动物——”



“父亲。”她轻轻地扶住男人的肩膀。男人转过头,看向她。起初,他那污浊的双眼里依然充满了困惑。接着,仿佛灵光乍现,他的神情扭曲了,他的嘴唇颤抖起来,他的声音哽住了,他终于看到了她,“我是来接您回家的。”



“……小祥……你也是吗?”男人垂下了眼睛,肩膀颤抖起来,“……果然,果然!你也瞧不起我对吧?你和他们一样!”他猛地站起身,但是脚下一个趔趄,摔倒了。他挣扎着想要再次站起来,但瘫软的四肢阻止了他,“要不是因为你……要不是因为你!都是因为有了你之后,我才废掉的!你和他们是一伙的对不对?”



她死死地盯住他,盯着这个和她一起生活了十六年,此刻看起来无比陌生的男人,她张开嘴,想要说些什么,嗓子有针在扎。该死,不能在这种时候失声,别人还看着呢。审判,这里就是审判庭,他们正在宣读你的罪名,你无路可逃,必须说点什么:“……你在说什么呢?混蛋老爹?”



男人的表情变了,恐惧爬上了他的眉梢,他转过身,仿佛想要从厉鬼那里逃开一般,在地上蠕动着,“不要——不许你那样看我!你凭什么那样看我?就因为你有才华就可以这样看你爹?我养你这么大也只是养了个白眼狼!你和那个女人一样!你不是我女儿!离我远点!离我远点……”



\newpage



她并不知道自己几点才回家的。她只知道在她进门的时候东方依稀露出了鱼肚白。男人已经失去了意识。是警察把他们送到家里的。矮个子的三森没有说话,但是从他的眼神中,她读出了同情。她原本想要自己把那摊失去意识的烂肉扛上二楼,但是三森不允许站都站不稳的她再多做些什么。他们把男人扔在屋子里,三森留下了一盒醒酒药,没有拿钱就离开了。她关上门,筋疲力竭地瘫倒在门口。男人的眼泪,鼻涕,乃至于呕吐物残留在她的袖子和裙子上,湿漉漉的贴在身上,可是她却并不想脱掉身上的衣服。起码她还活着,想想莲花,这都算不了什么。



“真恶心。”



\newpage



她打开花洒,卫生间里有浴缸,只是大概有十年没有人在里面泡澡了。排水口的周围凝固了一圈黄色的尿渍,花洒颤抖着,咳嗽着,断断续续地喷出热水。这里的温度调节完全是报废的,水温只有两种——冰冷或者滚烫。她过去会通过自己的手去感知水管中微妙的压力平衡,找到那个合适的水温。但今天没有必要,没有心情。凉水就很好。



那年夏天,大概是因为疫情的缘故,客服工作的密度比以往高了两到三倍。通话开始,已经形成肌肉记忆的敬语脱口而出。人们总是以为自己应当在对话时保持尊敬,但那并不适用于客服。急躁的诘问,狂怒的辱骂,还有那些吐字不清无法辨别的小声嘀咕。接通,问候,安抚,忍耐,安抚,询问,解答,请按1来为我的工作打个好评,再见。然后从头开始。



冰冷的水打在她的头发上,身体上,她打了个哆嗦,头依然很痛,但是距离上班还有不到两个小时,她必须让自己保持清醒。空气中弥漫着一股发霉的味道,天花板上的墙皮被水泡了以后脱落了,如同坏死的人皮。在这样的环境中更要保持冷静和乐观的态度。



我所工作的客服部在一幢老旧的十层办公楼的七层租了这间办公室。空调机不知道坏了多久,空气当中总是弥漫着一股刺鼻的怪味。自从我在这里工作以来就没有见到过有人修过。我曾经试着和我的上级汇报此事,但他的答复是:“我知道了,让我打完这个电话。”随后便杳无音信。在夏天的时候室内温度可以达到稳定的32摄氏度,每天下班之后我那件可怜的唯一拿得出手的洋装的领口便会被盐渍染黄。但是这些无关紧要。人穷不能志短,这是我的父亲过去时常和我说的话。一个人如果失去了尊严那就连人都算不上。我还有家可以回,有学可以上,有饭可以吃。世界上有三分之一的人连饭都吃不饱,我生活在日本,起码可以喝上干净的水。我所经历的没有什么大不了的,只是平平无奇的不幸。我是丰川家的女儿,我可以忍受这些。我怀抱着这样的信念咬紧牙关。我怀抱着这样的决意走上现在的道路。



衣服堆积在水池里,她唯一的一套从丰川家带走的可以掩盖她窘境的裙子,现在上面沾满了泥土,酒气,还有那个男人的呕吐物。哪怕是现在墙上依然有该死的蟑螂在爬。蟑螂的蛋白质含量很高,但是因为城市里的蟑螂总是携带病菌所以不可食用。冰冷的水打在她脸上。保持冷静。想想莲花。出淤泥而不染。想想你的工作。温良恭俭让。



我知道我的父亲很受打击,我知道他希望用酒精缓解他的痛苦。他的所作所为恰好证明了他只是一个普通的人类,就和这地球上的其他七十亿人一样。为情所困,易于迁怒,逃避现实,软弱无能。就和我一样。彻头彻尾的失败者。



你在这里工作的钱你拿去干什么了,是不是买屎吃啊?



不要动摇。我到底是为了什么才——要承担作为女儿的责任。从那天起,我就不断意识到自己的软弱——甚至不惜利用和伤害朋友,为了那点可怜的自尊——



莲藕的内心是空的,因此里面充满了污泥。



软弱无能软弱无能软弱无能软弱无能软弱无能软弱无能软弱无能软弱无能软弱无能软弱无能软弱无能软弱无能软弱无能软弱无能软弱无能软弱无能软弱无能软弱无能软弱无能软弱无能软弱无能软弱无能软弱无能软弱无能软弱无能软弱无能软弱无能软弱无能软弱无能软弱无能软弱无能软弱无能软弱无能软弱无能软弱无能软弱无能软弱无能软弱无能软弱无能软弱无能软弱无能软弱无能软弱无能软弱无能软弱无能软弱无能软弱无能软弱无能软弱无能软弱无能软弱无能软弱无能软弱无能软弱无能软弱无能软弱无能软弱无能软弱无能软弱无能软弱无能软弱无能软弱无能软弱无能软弱无能



你不过是个学生而已,你能够背负起他人的人生吗?



为什么还活着啊,弱小的祥子。



“为什么还活着啊,臭女人。”



\newpage



“你看上去变瘦了。”睦在进场的时候和她说。



“是吗?那也是没办法的事。”她心不在焉地回答道。今天,就在这里做出了结。让素世死心,和crychic决断,杀死软弱的自己。就在这里。



一团乱麻。



灯的新乐队光是开头就连续失误了三次。那个粉毛她还记得,是叫千早爱音,现在是灯的前后桌。看到爱音和灯在一起的样子让她倍感欣慰。灯甚至知道给紧张的爱音递水了。真不错。虽然这个乐队看上去还很不成熟,但是她们的的确确有着联系彼此的共鸣。除了一个人,除了那个虚与委蛇,满面假笑的贝斯手。她不知道为何自己对于自己昔日的队友会如此厌恶,她已经放弃思考了。也罢。演出又一次开始了,但是灯还是在紧张,声音发不出来。你在干什么?你怎么能够在这种时候失败?你不是想要成为人类吗?你难道不是为了你身边的她们,为了她而歌唱吗?她看向灯,努力地想要捕捉对方的目光。看着我!你的歌声是那么有穿透力,你拥有那样的才华,为何还要恐惧这一切?看着我!



然后,两人目光相交。



那只是一个瞬间的事情,但是在那个瞬间,她意识到了,灯还是一如既往。她脆弱,内心充满恐惧,但是依然会不顾一切地前进。这就是灯,这就是她所亲自选中的主唱。灯的声音响了起来,虽然高音的处理依然不稳定,但是总算是有了支乐队的样子。这样才对。这样就好。没有了她灯依然可以前进,就像她设想的那样。有了新的可以一起奋斗的伙伴,还有往日的不肯松手的羁绊,灯会忘记她,灯可以前进,灯将获得幸福。



这样就好。



“很高兴你愿意过来,那个时候,我没有把歌唱好,害得大家四分五裂,我很难受,很难受……这辈子都不想再有那种感觉了,觉得以后都不会再想组乐队了……”



那么,为什么,为什么灯会突然说那些话?



“但是,虽然这个乐队还没有名字……就算迷失方向,我也想前进。



“今天我一直觉得很害怕,但我有一种被你鼓励了的感觉。



为什么她仿佛在感谢自己?



“我能做的就只有拼命去唱,因为我的歌声是内心的呐喊!



为什么要让她想起自己曾经说过的话?



然后,她听到了,那熟悉的旋律。八六拍的主音吉他前奏,然后是节奏吉他和鼓手,最后是贝斯。她意识到了,是《春日影》。



悴んだ心ふるえる眼差し世界で

不堪重负的心飘忽颤抖的眼神在这世界里

僕はひとりぼっちだった

我总是独自一人

散ることしか知らない春は毎年冷たくあしらう

在只知凋零为何物的春季里每年平淡冷漠地度过暗



为什么要演奏那首她亲自谱写,却再也无法演奏的歌?为什么?为什么?为什么她还会对这一切产生如此强大的感情。为什么自己的眼眶会湿润?为什么自己会不受控制的伸出手?为什么自己还是忘不掉那个愚蠢,幼稚,软弱,快乐,天真,无忧无虑的自己?



がりの中一方通行に

在只通往一个方向的黑暗中

ただただ言葉を書き殴って

只是拼命地书写着文字

期待するだけむなしいと分かっていても

明知道仅仅期待只会落得一场空

救いを求め続けた

却还是不断祈求着救赎



她还记得那一天,在那个下午,羽泽咖啡厅,她第一次拿到这首歌的歌词的时候。只是第一眼,她就为之落泪了。她怎么能不落泪呢?她知道的,她可以理解灯的感情。在孤身一人的世界里,在无尽的黑夜中,孤身一人行走。想要彻底放弃希望,可是又本能地追寻救赎。如同逐火之蛾一般。



せつなくていとおしい

如此悲伤又如此惹人怜爱

今ならば分かる気がする

如今好像能够明白

しあわせでくるおしい

让人幸福又叫人意乱神迷

あの日泣けなかった僕を

照耀着那一天无法哭泣的我

光はやさしく連れ立つよ

光芒温柔地携我一同前行



突然出现的光明,只要是人类就会为之着迷。为之着迷的存在也就成为了人类。那时的灯,不对,那时的crychic就是她们所有人的光。现在灯依然站在台上,闪闪发光,可是她呢?她又算是什么?她和灯的相遇,又算什么?



雲間をぬってきらりきらり

穿过云层的缝隙闪闪发亮地

心満たしては溢れ

我的心绪充盈直至满溢

いつしか頬をきらりきらり

回过神来脸颊也闪闪发光

熱く熱く濡らしてゆく

热泪满溢濡湿了我的脸庞

君の手はどうしてこんなにも温かいの?

为什么你的手会如此地温暖呢

ねぇお願いどうかこのまま離さないでいて

呐拜托你就请一直这样握住永远不要放开



醒过来!你已经没有资格站在她的身边了。你自己的傲慢毁掉了一切,为了你那点可笑的自尊,你伤透了她的心。你只会给她带去痛苦。她已经不需要你了。你也不需要她来提醒自己过去的罪恶。你的存在本身就是一种罪恶。在那一天,你们两个人,你和你身边无辜的共犯,你们一起毁掉了这一切。这就是你的惩罚,你还不明白吗?她是比罪孽深重的你们更好的,更善良,更耀眼的人。她理应拥有更好的人生。再看看你,丰川祥子,满腔怨气,一身疲惫,这样的你还做什么和她在一起玩乐队的美梦?可笑!荒唐!承认吧,你配不上她。她应该过上更好的人生。你知道千早爱音是个什么样的人,她足够体贴,足够耐心,你知道她和灯将会相互吸引,并相互拯救。而在那个未来里,没有你这样的罪人的位置!



她缩回了手,落荒而逃。



縁を結んではほどきほどかれ

人与人的缘分总是系上又解开

誰しもがそれを喜び悲しみながら

谁都是在这样的喜悦和悲伤中

愛を数えてゆく鼓動を確かめるように

细数着一个个爱意为了确认心脏的跳动



是啊,她记得接下来的每一句歌词。她记得这首歌的每一个细节。这是她和灯一起完成的歌。人与人的缘分……她想起了汉学课上听过的那句诗:人有悲欢离合,月有阴晴圆缺,此事古难全。世界上有着那么多人,有着那么多喜悦与悲伤,不缺她一个的。



うれしくてさびしくて

如此喜悦又如此寂寞

今だから分かる気がした

如今好像能够知晓

たいせつでこわくって

如此重要却又叫人害怕

あの日泣けなかった僕を

照耀着那一天无法哭泣的我

光はやさしく抱きしめた

光芒温柔地将我拥进怀抱



过去的她曾经成为了灯的光,正如灯成为了她的光一样。现在的她是阴沟里的老鼠,别说是成为别人的光了,就连灯的光,也过于闪耀,过于刺眼。灯成为了人类,可是她呢?原地踏步,不思进取,软弱无能的人偶罢了!



照らされた世界

在这被光芒照亮的世界



我的世界已经深陷黑暗。



咲き誇る大切な人

盛开怒放的重要之人啊



我根本配不上那个称呼。我既不骄傲绽放,也配不上你的珍重。



あたたかさを知った春は

在知晓了温暖为何物的春日

僕のため君のための涙を流すよ

为了我为了你这泪水倾泻而下



对我们来说,那的确是个美好的春日。但那也终究只是幻影。你还活着,而且过得很好,我很欣慰。可是我……已经身处寒冬,不能再和你一同握手,一同落泪了。



あぁなんて眩しいんだろう

啊啊是多么耀眼夺目



啊,多么耀眼啊。



あぁなんて美しいんだろう

啊啊是多么美丽动人



啊,多么美丽啊。



雲間をぬってきらりきらり

穿过云层的缝隙闪闪发亮地

心満たしては溢れ

我的心绪充盈直至满溢

いつしか頬をきらりきらり

回过神来脸颊也闪闪发光

熱く熱く濡らしてゆく

热泪满溢濡湿了我的脸庞

君の手はどうしてこんなにも温かいの?

为什么你的手会如此地温暖呢

ねぇお願いどうかこのまま離さないでいて

呐拜托你就请一直这样握住永远不要放开

ずっとずっと離さないでいて

永远永远再也不要放手



对不起,灯。只是,我们已经不再是一个世界的存在了。



春日的幻梦,也该结束了。



“拜托了,初华……让我……忘记一切吧……”



\newpage



长崎素世喋喋不休地说着她早已准备好的说辞,那些话恐怕早就不知道排练了多少遍了,虚与委蛇,矫揉造作,令人作呕,就像那些丰川家的大人们一样。软弱的她只能沉迷于过去的幻影当中聊以自慰。



真是难看,就好像在照镜子一样。



“到现在还想着过去,真难看。事到如今,你也该忘记了吧。”她握紧了手。不能软弱!



“……过去大家那么快乐,现在却各奔东西,太奇怪了吧。乐队是命运共同体,当初这么说的人不是小祥吗?”



“命运……”命运,贝多芬第五交响曲。命运,如果她所遭遇的磨难都是命运的注定……可笑至极。她咬住牙齿,让自己回到现在,“那么那个乐队算什么?”让我看看你的觉悟,素世,你自己说了乐队是命运共同体,那你现在的乐队呢?灯呢?她不是你命运的一部分吗?你在这里把她们当成没有思想的棋子来利用,为了你的大义吗?何其崇高!



“你误会了!”果然,只是工具而已吗?



“我有哪里误会了?你的言行是相互矛盾的。crychic已经毁了,绝对不可能再复活了。”



“为什么……我想找回大家的那段快乐时光,小灯和立希都是这么想的。我也打算找小睦和小祥一起回乐队。”小祥,你相信爸爸,这一次,一定会找到工作。啊啊,这个人已经没救了。那么,就让她来成为那个击碎对方幻想的恶人:



“是这样吗?”她看向睦。



“我……”睦垂下眼睛。这是意料之中的。虽然她不能理解为何睦对于素世这样的烂人一直网开一面,但是睦从来不会说谎,这是好事。



“……当初组成crychic的人就是小祥啊。”



“所以我已经亲手将它结束了。”终于,说出来了,带着撕心裂肺的剧痛。不要动摇!不许动摇!在演完你的角色之前你不能示弱!软弱只会招来失败!



“没有结束!我一直为了crychic在努力着!”爸爸一直为了小祥在努力着!



“没有人那样拜托你。”她吸了口气,“这是最后警告,今后别再和我扯上关系了。”结束了,该退场了。



“等等!”她身后传来女性的喘息声,“不要走!不是的!我真的很重视大家!很喜欢大家!”都到这种时候了还在这里表演,是想要骗过自己吗?还是入戏太深呢?那个时候她也说过相同的话吧?“不要!求求你!要是没有小祥你们的话,我就……”她的手腕被抓住了,她转过头,面前的人脸上充满了疯狂和绝望,那个时候自己的脸上也是这种表情吗?



“放开我。”



“要我做什么你们才能回来?只要是我能做的,我什么都愿意做!”素世腿一软,直接跪在了地上。她的脑子里一阵刺痛——



要我做什么你们才不会分开?只要是我能做的,我什么都愿意做!



女人转过头,轻蔑地看着她。你是抱着多大的觉悟说出这种话的?



“你是抱着多大的觉悟说出这种话的?”



你只不过是一个学生,可以背负我或者你爸的人生吗?就因为你的哀求,就要让我和这个废物一直凑活下去吗?



“你只不过是一个学生,可以背负其他人的人生吗?”



“什么都愿意做”就是这样沉重的话。做不来的事情就别随便开口。



“‘什么都愿意做’就是这样沉重的话。做不来的事情就别随便开口。”



可是,我真的……



“可是,我真的……”



女人甩开了她的手:你这个人,“满脑子都只想到你自己呢。”



“其他人都找好了,现在就差你了。”



“我要加入。”睦的语气并没有什么波动,但是她已经知晓了对方的决心。



“……没必要那么着急,你真的考虑好了吗?”



“因为,祥你看上去就要坏掉了。”坏掉?就像那个没用的酒鬼那样?还是说像那个绝情的女人那样?不!她不需要这种同情,没有人可以同情她,没有人可以俯视她,没有人可以拯救她!哪怕是睦也不行!在她反应过来之前,怒火已经喷涌而出:



“真是高高在上呢。”



“抱歉。”睦吓了一跳,还想要再说些什么。但她打断了对方:



“过去那个软弱的我,已经死了。”



月光。据说贝多芬在1801年把这支奏鸣曲献给他的恋人朱丽埃塔,两人最终也没有成为伴侣。那份恋情,在开始之前就无疾而终了。她的手在琴键上飞舞着。她不知道为什么自己会弹这支曲子,但是身体自然而然地行动了起来。过去十二年接受的音乐教育告诉她,她的心境需要这支曲子。她还有最后的事情要做,她还有最后一次道别需要献给那个特别的她。灯推开教师门,她的脸上有了许多自己不再熟悉的情感,勇敢,热切,坚定。不对,这就是灯,是她熟知的,赏识的……深爱的重要之人。她的余光扫向门口,是了,她的同伴也在,千早爱音,那是个比自己更好的,更配得上她的人。那样的话,那样的话……



“小祥……这个!”



她站起身,她是失败的人偶,光是和人类在同一个屋檐下生活就已经是亵渎,更不要说用她那颗残忍冰冷的心玷污人类的赤诚。她不需要在这种时候接受灯的好意,她配不上,也没有余裕。光是和灯在一个教室里就让她的伪装几乎快要彻底破碎,她的歌,她的词,肯定会——不能软弱!就在这里结束一切。就在这里动手!就在这里杀死自己!她拿起书包,径直走过那个拥有着温暖的手的腼腆女孩,走过她无忧无虑一去不返的青春岁月,走过她曾经珍视,热爱,但现在无法拥有的一切:“……祝你幸福。”



然后,一切都结束了。



啊,终于死了。

\end{document}
