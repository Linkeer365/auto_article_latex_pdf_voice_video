\documentclass{article}
\usepackage[utf8]{inputenc}
\usepackage{ctex}
\usepackage{url}

% support Chinese Chars %
\newfontfamily\urlfontfamily{FandolSong-Regular}
\def\UrlFont{\urlfontfamily}

\title{【初祥】追随:后记——以及作者的忏悔录}
\author{sandman}
\date{20231219}

\maketitle
\url{https://nga.178.com/read.php?tid=38561279}
\newpage

\begin{document}
\CJKfamily{zhkai}

% : comm which is only poems edit in-use %


\Large

追随的构思实际上和《脱缰之马》并不是同一时期展开的。最开始的时候是我在群里口嗨说想写一个Mujica四人背着祥子处理尸体成为真正的共犯的故事。当时看了十三集之后我对于初祥关系性有了一些全新的想法,我很希望能够挖掘出初华隐藏的那一面。经过慎重的思考我最后确定了我对于祥子和她周围最重要的三个女人关系的看法:



灯是在祥子坠入深渊的时候在悬崖边上歌唱,希望用绳子把祥子拉回来的人。



睦是和祥子一起坠入深渊,在她身边抱紧她,希望她可以获得些许温暖的人。



初华时跳进深渊,把自己当成垫子也要把祥子推回去的人。



就是这么简单的考虑,决定了这三个人在这篇小说中的角色和做法。但是实际执行的时候依然出了不少问题,我们之后再说。总之这个分类让我确定初华将会是动手的人,而睦则是她的共犯。海铃和喵梦的参与理由似乎并不多,她们对于祥子没有那么多执着。但是看了访谈之后我对这两个人的人设也产生了新的认识——在我的构想里喵梦是和黑魂里的帕奇一样,嘴毒但是关键时刻却会很靠得住的人。海铃和喵梦应该现实当中都有着不小的经济压力,喵梦其实对于音乐大概也会有自己的追求才会加入高难度的Mujica。在设计了喵梦的背景之后我就发现其实只需要说服海铃就好,而海铃并不是那种会大喊大叫或者执着于常人善恶观的角色。只要向她证明这样做最为有效就可以——在这点上她和初华的伦理观不谋而合。



我很快就意识到我想要写一个《罪与罚》风格的故事。初华的位置类似于拉斯柯尔尼科夫,可是到底是怎么样的?如果她在人面前的形象只是一张友好的面具,她私底下的真实又是如何。我的线索不多——她喜欢看星星,会一个人去星象馆,在mana不在的时候会露出很疲惫的落寞神情,看起来交心朋友几乎没有,可能是和灯一样的作词。初华在十三集结尾的反常让我产生怀疑——作为几乎全剧双商顶点,可以开导灯的她怎么会注意不到祥子的异样?我听了几遍something in the way,并且重刷了银翼杀手2049之后意识到了,我想要的是一个如同雷蒙德·钱德勒笔下的菲利普·马洛一般,落魄,忧郁,诗人气息浓重,怀才不遇的侦探形象,但是她同时又是凶手——从第一章到终章她和祥子的地位产生的反转是我尤为得意的设计之一。在敲定了她的形象之后,我开始设计她生活的环境——参考原型很简单,就是2049主角K生活的单人公寓,但是那间公寓里有着虚拟恋人joi,并且空间比较狭窄,不容易感到孤独。我需要一个更大的空间,衬托她的空虚和寂寞,同时里面充满了强调孤身一人的要素:桌子前只有一张的椅子,单人床,只有一个人吃不完的炖肉。她一个人来到东京,经历了复杂的人际关系和巨大的环境变化,却依然几近徒劳地追随着祥子——哪怕她们一年之间都没再有过任何的交流。两人之间除了过去美好的羁绊,应该是有着由此而生,害怕物是人非的微妙的隔阂以及对于接近彼此的恐惧的。这些东西我都希望可以表现出来。同时,在终章的开头,我几乎是复制粘贴了第一章的结构,来体现初华心境上的变化。杂乱的公寓代表她加重的抑郁和恶化的心境——要知道,抑郁症的典型症状正是精力和兴趣的消退。在初华和幽灵对话的那一段,我的灵感完全来自于《透明人间之骨》。这是一部伟大的漫画,推荐大家去看。



第二章和第三章作为过渡章是我在执行上出现了比较大问题的两章。第一章和第二章的结构是一开始就决定的,我希望和mygo本篇一样,在每一章的结尾留下耐人寻味的爆点,同时保证逻辑的流畅,为此我运用了时间线的叙述性诡计来混淆读者对于事情顺序的认知——只有一个读者在评论区发现了这一点。同时我也自信留下了足够多的线索。第二章初华回忆前后的地点发生了变化,同时那座高架桥正是她身处祥子家附近的暗示——祥子在十三集回家时经过的那座高架桥在序章和第一章都被提到了。但是在结尾为了设计爆点,我在不顾人设合理性的情况下想要强行安排当时讨论的很火的祥子跳桥的可能性,但是第三章我怎么写都感觉不对劲。接着我看了@TP02大佬的《唯有河流流淌》,那篇文章给我留下了很大的震撼,让我重新反思自己笔下的祥子这个角色。我意识到其实我最开始就明白祥子只会死在战场上,不可能有意识地自己去紫砂。当我意识到这件事之后我大段重写了第三章,让冲突维持在了一个可控的范围内——我的构想是祥子当时有些意识模糊了,但是在被灯拉住之后就已经意识到自己在做什么。但是那时候她依然很迷茫。和灯的约定成为了新的牵住她的缰绳,睦的支持则让她意识到自己不能再孤身一人下去,否则只会伤害关心她的人。在第三章的她需要经历复活的过程——这是我的设计,但是执行起来感觉由于不敢上强度和人设问题导致表现力没有达到预期的结果。最后初华的独白场面是写到一半的时候自然出现在脑袋里的。我不希望第三章留下一个完全让人安心的happy end,实际上终章也没有这样的感觉。我更喜欢一切都没有完全解决的bittersweet end。



终章初华和祥父生前的最后对白有两个抄袭对象,一个是欧亨利的《提线木偶》,另一个则是芥川龙之介的《罗生门》(不是电影)。初华的罪是完全因为偶然而背负的。这也是我为何选择了局外人引用。初华因为善意和关心在那个时间点抵达了祥子的家,因为祥父偶然的话语而坚定了杀意,因为对方自己的酒精中毒而掌管了对方的生杀大权——这些因素全部都是偶然。但是当她移开手机宣判对方的死刑的的时候,这样的偶然导致了她自己选择的必然的罪恶。原本她只是激情犯罪,但是当睦想要回到现场的时候,她冷静下来,通过和睦的交谈意识到自己必须用其他的罪恶——亵渎尸体——来掩盖过去的罪恶。这并不是为了自己或者那个男人,而是为了让祥子获得小小的喘息,重拾活下去的信念。为了这个目的就算帮助祥子的人不是自己也好。抱着这样的决心她在第三章实行了自己的谋划,压抑住自己对于灯和睦的嫉妒让她们找到祥子,而自己则去背上属于自己的罪——哪怕这些罪恶完全是因为巧合堆积在了她的面前。这就是《罪与罚》的含义——由于自己的选择而不得不背负罪恶感的折磨,把此世化为地狱,把自己与对方之间的距离增加到无穷。但是值得庆幸的是祥子虽然有些对感情迟钝,但是却很敏锐而且聪明。最重要的是她很强大。只要给了她喘息的机会让她恢复,她就比任何人都要强。让这样强大的她在最后拯救初华也是我的小小私心——我受够了单方面付出得不到回报的故事。最开始我就想写一个逆着所有人期待但是逻辑通顺的故事——并不是初华拯救祥子,而是祥子拯救初华。当然另一部分并没有实现,而是被放弃了——灯和睦没有“拯救”祥子,而只是推了她一把,让她缓口气,可以继续在深渊里前进。这个故事的标题“追随”也是双向的——灯睦初三人追随着祥子,但是祥子也追随着她们。如果这点可以传达到就好了。



以上就是我对于本篇目前的所有感想。感谢你可以读完这个故事。欢迎各种有理有据的修改意见。愿此刻读着这段文字的你也可以抓住那颗指引自己的北极星。

\end{document}
