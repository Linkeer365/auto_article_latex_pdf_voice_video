\documentclass{article}
\usepackage[utf8]{inputenc}
\usepackage{ctex}
\usepackage{url}

% support Chinese Chars %
\newfontfamily\urlfontfamily{FandolSong-Regular}
\def\UrlFont{\urlfontfamily}

\title{提线木偶}
\author{欧亨利}
\date{1912}

\maketitle
\url{https://www.99csw.com/book/6063/211031.htm}
\newpage

\begin{document}
\CJKfamily{zhkai}

% : comm which is only poems edit in-use %


\Large

警察站在第二十四号街和一条黑得邪门的胡同的拐角上,高架铁路正好在上面通过。当时是凌晨两点:黎明前的黑暗又浓重,又潮湿,叫人很不舒服。



一个穿着长大衣,帽子拉得很低,手里提着什么东西的男人轻手轻脚地从黑胡同里匆匆走出来。警察迎上前去,态度和蔼,但带着克尽职守的自信。时间、胡同的恶名、行人的匆忙、携带的重物——这一切很自然地构成了“可疑情况”,要求警察干预查明。



“可疑者”立即站住,把帽子往后一推,摇曳的街灯照出的面孔镇定自若,鼻子相当长,深色的眼睛毫不躲闪。他没脱手套就把手伸进大衣口袋,摸出一张名片交给警察。警察凑着晃动的灯光看到名片上印的是“医学博士查尔斯·斯宾塞·詹姆斯”。街道和门牌号码在一个殷实正派的地段,不容产生好奇,更不用说怀疑了。警察的眼光朝下扫去,看到医生手里提的东西:一个漂亮的黑皮白银扣饰的医药包;名片得到进一步的证实。



“请吧,大夫。”警察让开一步,口气和蔼得有些过份。“上面关照要格外小心。最近溜门撬锁、拦路抢劫的案子很多。在这样的夜晚出诊真够呛。不算冷;但是——粘糊糊的。”



詹姆斯医师彬彬有礼地点点头,说了一两句附和警察对天气评价的话,继续匆匆走去。那晚有三个巡警都认为他的名片和神气的医药包足以证明他是正派人,干的是正派事。假如第二天这些警察中间有谁觉得应当去核实一下名片(只要别去得太早,因为詹姆斯医师没有早睡早起的习惯),他将发现一块漂亮的门牌上确有医师的姓名,摆设精致的诊所里确有衣著整饬的医师本人,邻居们都乐意证明两年来医师奉公守法,照顾家庭,业务兴旺。



因此,假如这些热心维护治安的人中有谁能看到那个表面清白的医药包里的东西,准会大吃一惊。包一打开,首先呈现在眼前的是一套最新发明的,“保险箱专家”专用的精巧工具,所谓“保险箱专家”,是如今撬保险箱的窃贼自封的称号。那些工具都是专门设计,特别制作的——短而有力的撬棍,一套奇形怪状的钥匙,在冷铸钢上打孔就象耗子啃乳酪一般轻松的高强度的蓝钢钻头和冲头,能象水蛭那样附着在光滑的保险箱门上,象牙医拔牙那么利索地拔出号码锁的夹钳。“医药”包里的小贴袋中有一瓶四英两装的硝化甘油,用剩了一半。工具下面是一堆皱皱巴巴的钞票和几把金币,总数一共是八百三十元。



詹姆斯医师在他极有限的朋友圈子里被称为“了不起的希腊人”。这个奇特的称呼一半是赞扬他冷静的绅士作风;另一半在帮会黑话里是指头儿和出谋划策的能人,凭他的地址、职业的影响和威望能搞到信息,供哥儿们制订计划,干非法勾当。



这个精干的小圈子的其他成员是斯基采·摩根、根姆·德克尔和利奥波德·普雷茨菲尔德。德克尔是“保险箱专家”,普雷茨菲尔德是城里的珠宝商,负责处理三人工作小组搞来的钻石和其它首饰。他们都是讲朋友义气的好人,守口如瓶,忠实不渝。



合伙人认为那晚的收获并不满意,只能勉强补偿他们花费的气力。一家资金雄厚的经营呢绒的老字号的双层侧栓的老式保险箱,在星期六晚上的存款理应超过两千五百元。但是他们只找到这个数目,三人按照惯例,当场就把钱平分掉。他们本来指望有一万或一万二千元。然而商号股东老板之一办事有点儿过于老派。天黑后,他把大部分现金装在一个衬衫盒里带回家去了。



詹姆斯医师继续沿着杳无行人的第二十四号街走去。经常聚集在这一地区的戏剧界的票友们也早已上床睡觉了。牛毛细雨在铺路的石子间积成小水塘,被弧光灯一照,反射出千百片闪闪发亮的小光点。水汽凝重的寒风从房屋之间的空档里劈头盖脑地一阵阵扑来。



医师刚走近一座高大的砖砌建筑的拐角,这座与众不同的住宅前门猛地打开了,一个嘴里嘀嘀咕咕、脚下踢踢跶跶的黑种女人从台阶下到人行道。她说着什么,很可能是在自言自语——她那个种族的人独自遇到危难时总是采取这类求助的办法。她象是南方旧时的奴仆——多嘴多舌,肆无忌惮,忠心耿耿,却又不服管教;她的外貌说明了这一点——肥胖,整洁,系着围裙,扎着头巾。



詹姆斯医师迎面走去时,这个从沉寂的房屋里突然出现的形象刚走下台阶。她大脑的功能从发音转换到视觉,停止了嘀咕,一对金鱼眼睛死死盯住医师手里的医药包。



“谢天谢地!”她一见到医药包便脱口嚷道。“你是大夫吗,先生?”



“是的,我是大夫。”詹姆斯医师停住脚步说。



“那就请你看在老天的份上去瞧瞧钱德勒先生吧。不知他是犯病还是怎么搞的,象死了似的。艾米小姐派我去找大夫。先生,你不来的话,天知道老辛迪上哪儿才能找到大夫。假如老主人知道这里的情形,就有好戏看了,先生——准会打枪,在地上数好步子,用手枪决斗。那个羔羊般的,可怜的艾米小姐——”



“你要找大夫,就在前面带路。”詹姆斯医师踩上台阶说。“你要找个听你说话的人,我可不奉陪。”



黑女人引他进屋,走上一溜铺着厚地毯的楼梯。他们经过两个光线暗淡的门厅。在第二个门厅里,爬得上气不接下气的引路人拐了弯,在一扇门前站停,打开了门。



“我把大夫请来啦,艾米小姐。”



詹姆斯医师进了屋,朝站在床边的一位年轻太太微微欠身。他把医药包搁在椅子上,脱掉大衣,搭在医药包和椅子背上,镇定自若地向床边走去。



床上躺着一个男人,仍是先前倒下去时的姿势——衣着华丽时髦,鞋子已经脱去;全身松弛,死了似地一动不动。



詹姆斯医师仿佛散发着宁谧、镇定和力量的光环,对他主顾中间软弱失望的人来说简直象是久旱后的甘霖。他在病室的举止风度有某些地方特别使妇女们倾倒。那并不是时髦医师对病人的纵容讨好,而是沉着自信,压倒命运的气魄,对人尊重、保护和献身的态度。他那坚定、明亮的棕色眼睛里有一种清澈的吸引力;和蔼的面相非常适合担任知己和安慰者的角色,冷静而近似牧师的安宁带着潜在的威严。他有时出诊,妇女虽和他初次见面,居然会告诉他,她们为了防止失窃,晚上把钻石藏在什么地方。



詹姆斯医师经验丰富,眼珠不怎么转动,就估出了房间家具摆设的等级和质量,同时也打量了那位年轻太太的外表。她身材瘦小,年纪二十刚出头,容貌有一种迷人的美,但现在蒙上了阴霾。这与其说是意外不幸所引起的,还不如说是由来已久的固定的哀怨。她额头一侧有一道青紫的挫伤,医师根据经验判断,受伤的时间不会超出六小时。



詹姆斯医师伸手去试病人的脉搏。他那双几乎会说话的眼睛在询问年轻女人。



“我是钱德勒太太。”她回答说,带着南方人那种含糊的哭音和腔调。“你来到前十分钟左右,我丈夫突然病了。他以前也犯过心脏病——有几次相当凶险。”病人深更半夜这副打扮促使她作出进一步的解释。“他在外面很晚才回家;我想大概是赴晚宴。”



詹姆斯医师现在把注意力转向病人。不论他从事哪一类“职业”活动,他总是全神贯注地对待“病例”或者“买卖”。



病人年纪有三十左右。面相大胆放荡,但还算端正,一种乐观幽默的神情补救了缺点。他衣服上有一股泼翻了酒的气味。



医师解开他的上衣,用小刀把衬衫的假前胸从领子割破到腰身。清除了障碍之后,他用耳朵贴在病人心口,仔细听着。



“二尖瓣回流?”他站直时轻声说。句子结尾是没有把握的升调。他又俯身听了好久;这次才用确诊的音调说:“二尖瓣闭锁不全。”



“夫人,”他说话的口气曾多次解除过人们的忧虑,“有可能——”当他缓缓朝那位太太转过头去时,只见她脸色惨白,晕了过去,倒在黑老太婆的怀里。



“可怜的小羊羔!可怜的小羊羔!辛迪大妈的宝贝孩子被他们害苦啦!但愿上帝发怒,惩罚那些把她引入迷途,伤了她那颗天使般的心,害她落到这个地步的人——”



“把她的脚抬高。”詹姆斯医师上前去扶持那个晕倒的人。“她的房间在哪里?必须把她抬到床上去。”



“在这儿,先生。”黑老太婆把扎着头巾的脑袋朝一扇门摆摆。“那是艾米小姐的房间。”



他们把她抬进房间,搁在床上。她的脉搏很微弱,但还有规律。她神志没有清醒,从昏迷状态进入了沉睡。



“她体力衰竭。”医师说。“睡眠对她有好处。等她醒来时,给她一杯加热水的酒——再打个鸡蛋在里面,如果她能喝的话。她前额的挫伤是怎么搞的?”



“磕了一下,先生。那个可怜的小羊羔摔了一交——不,先生,”——老太婆那变化不定的种族性格使她突然发作起来——“老辛迪才不替那个魔鬼撒谎呢。是他干的,先生。但愿上帝让他的手烂掉——哎呀,真该死!辛迪答应过她可爱的小羊羔决不讲出来。先生,艾米小姐头上是磕伤的。”



詹姆斯医师向一个精致的灯架走去,把灯光捻低一些。



“你在这儿呆着,太太,”他吩咐道,“别作声,让她睡觉。如果她醒来,就给她喝加热水的酒。如果她情况不好,你就来告诉我。这事有点儿怪。”



“这里的怪事还多着呢。”黑女人正要说下去,医师一反常态,象安抚歇斯底里病人一般专断地吩咐她别出声。他回到另一个房间,轻轻关上门。床上的人没有动弹,但是已睁开了眼睛。他的嘴唇牵动着,仿佛想说什么。詹姆斯医师低下头,只听到微弱的“钱!钱!”



“你听得清我说的话吗?”医师压低嗓门,但十分清晰地说。



病人略微点点头。



“我是医师,是你太太请来的。她们告诉我,你是钱德勒先生。你病得不轻,千万别激动或是慌张。”



病人的眼神仿佛在招唤他。医师弯下腰去听那些仍旧十分微弱的声音。



“钱——两万块钱。”



“钱在哪里?——在银行里吗?”



眼神表示否定。“告诉她”——声音越来越微弱了——“那两万块钱——她的钱”——他的眼光扫视着房间。



“你把钱藏在什么地方了吗?”——詹姆斯医师的声音象塞壬女妖一般急切,想从那个神志逐渐不清的人嘴里掏出秘密——“在这个房间里吗?”



他觉得那对暗淡下去的眼睛里有表示同意的闪动。他指尖能触摸到的脉息细得象一根丝线。



詹姆斯医师的另一门职业的本能在他的头脑和心里出现。他办事敏捷,马上决定要打听出这笔钱的下落,即使知道这肯定会出人命也在所不惜。



他从口袋里掏出一小本空白的处方笺,根据标准的常规做法,开了一张适合病人需要的处方。他到里屋门口,轻声叫那个黑女人出来,把处方交给她,让她去药房配药。



她嘀嘀咕咕地离开后,医师走到钱德勒太太躺着的床边。她仍在沉睡;脉象比先前好一些了;额头除了挫伤发炎的地方以外也不烫了,稍稍有些湿润。没人打扰的话,她可以睡几小时。他找到房门钥匙,出来时随手把门锁上。



詹姆斯医师看看表。有半小时可以归他支配,因为那个老太婆去配药,半小时以内回不了家。他找来水罐和平底酒杯,打开医药包,取出一个盛着硝化甘油的小瓶——他的摆弄手摇曲柄钻的弟兄们把它简单地称做“油”。



他把淡黄色稠厚的液体倒了一滴在酒杯里,然后取出带银套的注射器,安好针头。他根据玻璃管上的刻度细心地抽了几次水,把那滴硝化甘油稀释成将近半酒杯的溶液。



那晚两小时前,詹姆斯医师用同一个针筒把未经稀释的液体注射到他在一个保险箱锁上钻出的窟窿里,一声低沉的爆炸毁坏了控制门栓的机械。现在他打算用同样的方法震撼一个人的主要机械——刺激他的心脏——目的都是为了钱。



同样的方法,但是外表不同。前者是鲁莽粗野,凭借原始动力的巨人,后者是奉承者,但用丝绒和花边掩饰了同样致命的手臂。因为医师用针筒细心地从酒杯里抽取的液体已经成了三硝酸甘油酯,这是医学科学中已知的最厉害的强心剂。二英两能毁坏一扇厚实的保险箱铁门;他现在要用一量滴的五十分之一来使一个活人的复杂机理永远静止。



但不是立即静止。这不符合他的要求。首先要迅速增加身体的活力;给每一个器官和功能以强有力的促进。心脏会勇敢地对致命的鞭策作出反应;静脉里的血液会更快地回到心脏。



詹姆斯医师很清楚,这种心脏病遇到过于强烈的刺激,就象挨了一颗来复枪子弹似的,结果是立即死亡。当血流量在窃贼“油”的作用下骤然增加,管腔本来不畅的动脉会迅速完全阻塞,生命之泉就停止流动了。



医师解开昏迷的钱德勒前胸的衣服,把针筒里的液体熟练地注射到心前区的肌肉里。他干两门行业都干净利落,注射完毕,便仔细擦干针头,把保持针头通畅的细铜丝重新穿好。



三分钟后,钱德勒睁开眼睛,开始说话了,声音虽然微弱,但还能辨清,他问抢救他的是谁。詹姆斯医师再一次解释他怎么会来这儿的。



“我妻子呢?”病人问道。



“她睡着了——由于过度疲劳和担忧。”医师说。“我不愿叫醒她,除非——”



“没有——必要。”钱德勒呼吸短促,说话时常间断。“为了我——去打扰她——她不会——领你情。”



詹姆斯医师拖了一把椅子到床前。时间不容浪费,要抓紧谈话。



“几分钟前,”他以另一门职业的低沉坦率的声调说,“你打算对我说些有关钱的事。我不指望你对我推心置腹,但是我有责任劝告你,焦虑对你的恢复是不利的。假如你心里有什么事——我记得你提到过两万块钱的事——不妨说出来,可以减轻你的精神负担。”



钱德勒脑袋动不了,但他的眼珠转向说话人的方向。



“我说过——这笔钱——在哪里吗?”



“没有。”医师回答说。“我只不过从你模糊不清的话里推测到你十分关心它的安全。如果钱在这个房间里——”



詹姆斯医师住口不说了。他是不是从病人揶揄的脸上看到一丝恍然大悟和起疑的神色?他是不是显得有点儿迫不及待?他是不是说漏了嘴?钱德勒随后说的话使他恢复了自信。



“除了——那个——保险箱以外,”他上气不接下气地说,“还能——藏在哪里呢。”



他用眼光指点房间的一角,医师这才看到窗帘下端半遮着的一个铁制的小保险箱。



他站起身,抓住病人的手腕。他的脉搏宏大,但隔着不祥的间歇。



“抬起胳臂。”詹姆斯医师命令说。



“你知道——我动不了,大夫。”



医师快步走近通向过道的房门,打开门,听听外面有什么声音。一片静寂。他不再旁敲侧击,径直走到保险箱前面,打量了一下。那个保险箱式样古老,设计简单,只能防防手脚不干净的仆人。拿他的技术来说,这只能算是一件玩具,等于是稻草和硬纸板糊的玩意儿。这笔钱可说是已经到手了。他能用夹钳拔出号码盘,钻透制栓,不到两分钟就打开保险箱门。换另一种办法,也许只要一分钟。



他跪在地上,耳朵贴着保险箱门,慢慢转动号码盘。不出他所料,锁门时只用了一个组合暗码。号码盘转动时,他敏锐的耳朵听到轻轻的咔哒一响;他利用暗码组合——把手松动了。他打开了保险箱门。



保险箱里一无所有——空空的铁格子里连一张废纸都看不见。



詹姆斯医师站起来,回到床前。



垂死的人额头汗涔涔的,但嘴角和眼睛露出嘲弄的冷笑。



“我这辈子——从没见过,”他吃力地说,“医药同——盗窃结合!你身兼二职——赚头不坏吧——亲爱的大夫?”



当时的情况十分尴尬,詹姆斯医师的精明强干从没有遇到过比这更严峻的考验。受害者的出了格的幽默感使他陷入既可笑又不安全的处境,但他仍然保持着尊严和清醒的头脑。他掏出表,等那人死去。



“你对——那笔钱——未免——过于猴急了。可是你——亲爱的大夫——根本奈何不了它。它很安全。十分安全。它全部——在赛马——赌注登记人手里。两万块——艾米的钱。我拿去——赛马——输得精光。我是个败家子,贼先生——对不起——大夫,不过我输得光明正大。我可从来没有见过——象你这样——不够格的坏蛋——大夫——对不起——贼先生。给受害者——对不起——给病人喝杯水——是不是违反——你们贼帮的——职业道德?”



詹姆斯医师替他倒了一杯水。他几乎不能吞咽。药物的反应一阵阵袭来,越来越强烈。但他死到临头,还想狠狠地刺痛一下别人。



“赌徒——酒鬼——败家子——我全沾边,可是——医师兼窃贼!”



医师对他刻薄的讽刺只作了一个回答。他俯下身子,盯着钱德勒急剧凝滞的眼光,举手指着那个沉睡的女人的房间,姿势如此严厉而意味深长,以至那个衰竭的人用尽残剩的力量,半抬起头,想看个究竟。他什么也没看到;但听到了医师的冰冷的言语——他临终时听到的最后的声音:



“到目前为止,我可从没有揍过女人。”



企图研究这种人是徒劳的。没有哪一门学问能对他们进行探讨。人们提到某些人时会说“他这也行,那也行”,他们就是这些人的后裔。我们只知道有这种人存在;只知道我们可以观察他们,议论他们的浅显的表现,正如孩子们观看并议论提线木偶戏一样。



然而,这两个人——一个是谋财害命的强盗和凶手,站在受害人面前;另一个虽然没有严重违法,但行为更其恶劣,令人嫌恶,他躺在受他迫害、侮辱和毒打的妻子的房屋里;一个是虎,另一个是狼,他们两人互相憎恨对方的卑劣;尽管大家罪恶昭著,却互相炫耀自己的行为准则(即使不谈荣誉准则)是无可指摘的。



詹姆斯医师的反驳肯定刺伤了对方剩余的羞耻心和男子气概,成了致命的一击。他脸上泛起一阵潮红——临终红斑;钱德勒停止了呼吸,几乎没有颤动,已经一命归天。



他刚咽气,黑老太婆配好药回来了。詹姆斯医师一手轻轻按着死者合上的眼皮,把结果告诉了她。她并不伤心,只带着遗传的,与抽象的死亡友好相处的态度,凄凉地、抽抽搭搭地抱怨说:



“可不是吗!上帝自有安排。他会惩罚有罪的人,帮助落难的人。他现在该帮助我们了。辛迪买这瓶药,把最后一枚硬币都花了,结果药也没用上。”



“难道钱德勒太太没有钱吗?”詹姆斯医师问道。



“钱?先生,你知道艾米小姐为什么晕倒,为什么这么虚弱?是饿成这样的,先生。家里除了一些破饼干以外,三天没有吃的了。那个小天使几个月前就变卖了她的戒指和怀表。这座房子里的红地毯和漂亮家具全是租来的,催租的人凶极了。那个魔鬼——饶恕我,上帝——他已经在你手里遭到了报应——他把家产全败光了。”



医师的沉默使她越说越来劲。他从辛迪杂乱无章的独白中理出了一个古老的故事,其中交织着幻想、任性、灾难、残酷和傲慢。她喋喋不休的话语组成的模糊概貌中,有几幅比较清晰的画面:遥远南方的一个舒适的家庭;草率的,随即后悔的婚事;充满侮辱和虐待的不幸生活;女方最近得到一笔遗产带来了重振家业的希望;狼夺去了那笔钱,两个月不照面,在外面挥霍得精光;一天晚上喝得醉醺醺的又回来了。从一团乱麻似的故事里可以看到一条纯白的线索:黑老太婆的质朴、崇高和始终不渝的爱,不论任何艰难险阻,她都坚定不移地追随着女主人。



她终于住嘴时,医师问她家里有没有威士忌或者任何什么酒。老婆子说有,餐具柜里还有那条豺狼剩下的半瓶威士忌。



“照我刚才对你说的那样,倒些酒,兑些热水,打个鸡蛋在里面。把你的女主人叫醒;让她喝下去,然后告诉她家里出的事。”



十来分钟后,钱德勒太太由老辛迪搀扶着进来了。她睡了一会,喝了热酒,看上去不那么虚弱了。詹姆斯医师已经用床单盖好了床上的死人。



那位太太哀伤和半含惊恐的眼睛朝床上一瞥;向她保护人身边更挨近些。她的眼睛干而发亮。极度的痛苦使她的泪水已经涸竭。



詹姆斯医师站在桌边,他已穿好大衣,手里拿着帽子和医药包。他的神情镇定安详——他的职业使他见惯了人类的痛苦。只有他那闪烁的棕色眼睛里流露出审慎的医师的同情。



他体贴简洁地说,由于时间太晚,请人帮忙肯定有困难,他可以亲自去找合适的人来料理后事。



“最后还有一件事,”医师指着打开的保险箱说。“钱德勒太太,你的丈夫最后知道自己不行了;他把保险箱的组合号码告诉了我,让我打开。如果你要使用,请记住号码是四十一。先朝右拧几圈;再朝左拧一圈;停在四十一这个数字上。他虽然知道自己即将去世,却不让我叫醒你。



“他说他在保险箱里存了一笔数目不大的钱——也够你用来完成他最后的请求了。他请求你回你的老家去,以后日子好过一些的时候,请你原谅他对你犯下的种种罪愆。”



他指指桌子,桌上是一叠整整齐齐的钞票,钞票上面放着两摞金币。



“钱在那儿——如他所说——一共是八百三十元。请允许我留下我的名片,以后有我可以效劳之处,请吩咐。”



他在最后时刻居然顾念到她——并且想得很周到!来得太迟了!但是这个谎话在她认为已经成为一片灰烬和尘埃的地方煽旺了一个柔情的火花。她脱口喊道:“罗勃!罗勃!”转过身,扑在忠诚的仆人怀里,用泪水冲淡她的悲哀。在往后的年月里,凶手的假话象一颗小星星,在爱情的坟墓上空闪烁,给她慰藉,争取她的原谅,这本身就是一件好事。



黑老太婆把她搂在胸口,象哄小孩似地低声安慰她,她终于抬起头——但是医师已经走了。

\end{document}
