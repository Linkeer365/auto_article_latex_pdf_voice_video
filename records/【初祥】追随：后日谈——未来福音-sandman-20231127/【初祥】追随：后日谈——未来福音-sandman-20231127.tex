\documentclass{article}
\usepackage[utf8]{inputenc}
\usepackage{ctex}
\usepackage{url}

% support Chinese Chars %
\newfontfamily\urlfontfamily{FandolSong-Regular}
\def\UrlFont{\urlfontfamily}

\title{【初祥】追随:后日谈——未来福音}
\author{sandman}
\date{20231127}

\maketitle
\url{https://www.pixiv.net/novel/show.php?id=21098279}
\newpage

\begin{document}
\CJKfamily{zhkai}

% : comm which is only poems edit in-use %


\Large

她们去看了演唱会,所有人一起。睦和海铃似乎早已是这里的常客。若麦虽然有些困惑,但是她很知趣地没有多问,转而对于鼓手的专业水平赞叹不已。祥子并没有隐瞒的打算,但是看到若麦装出不感兴趣的样子,便只是简单地解释这是她和睦过去的朋友组成的乐队。在这个过程中,初华一直没有说话。她的目光凝聚在舞台中央,那个腼腆,瘦小,却又执着,炽烈的主唱身上。高松灯一如既往地低着头,看着自己的脚底,直到贝斯手长崎素世提醒她她才从自己的世界中醒来。台下响起了几声友善的轻笑,似乎粉丝们对于她的这一举动早已见怪不怪。灯抬起头,她的视线与初华在空中相遇了。她的脸上出现了一缕惊讶。初华笑了笑,不出声地举起手,指向自己的身边。灯微微转头,看到了初华身边的人。接着,她的眼睛张大了,她的嘴唇颤抖起来。祥子眨了眨眼,竖起食指,冲着灯调皮地笑了一笑。两人并没有说话,但那就已经足够。灯闭上眼,深吸了一口气,再次看向前方:



“今天你能够来,我很惊讶。也很感激。



“那一天,你又一次离开。我很迷茫,也很痛苦,但我最担心的就是你是否会就此消失,杳无音信。



“你身上背负着我所不了解的痛苦。你心中秉持着我无法接近的骄傲。但是,当再次与你相遇的时候,我知道了。你就是你,坚强的,温柔的,哪怕遍体鳞伤也绝不屈服的你。看到这样的你,让我心痛,也让我欣喜。于是我和你立下了约定,希望有一天可以再次让你听到我的歌声,希望有一天我们可以再次一同欢笑,一同哭泣。



“现在,你来了。如同梦境,如同奇迹。离开了幽暗的溪谷,跨越了贫瘠的荒原,带着幸存者的余裕与微笑。你依然站立于此。



“这首歌,献给满身泥泞也不肯放弃挣扎,最终依然能够与我重逢的你,以及这个有你存在的未来——《無路矢》。”



\newpage



“……然后rikki就直接甩下我们去找afterglow了,结果到了跟前又是跟以往一样一句话说不出来。还得soyorin给她圆场。乐奈就又开始缠着我要抹茶糖。那天我抹茶糖正好忘了带了,多亏了小灯救场,否则我真的要被她吃干抹净了。”



“立希同学她确实在那种方面很不擅长呢。不过我感觉她也变得圆滑了不少,大概是因为和你们相处的关系吧。”



“哪有的事。她只有在演出的时候才会给我好脸色看,平时就她和soyorin咬我咬的最凶。我有没有和你说过我们乐队本来打算叫做Anon Tokyo的?结果她俩立刻就拒绝了。真的是一点品味也没有。”



“额……虽然有点对不起爱音同学。我也觉得那个名字实在是太……”



“啊?怎么祥子你也这样?小灯你帮帮我~”



“小爱,其实现在的名字更好听吧……”



“什么?怎么小灯你也背叛了我?算了算了,反正现在的名字有我一半的功劳……”祥子看着喋喋不休的爱音,无奈地叹了口气,转过头,看向还在写天文部日志的灯:



“灯?”



“怎么了,小祥?”



“上次的歌很不一样呢。如同在荒野中不顾一切的呐喊,但是又有着某种不同以往的决绝的部分。你们的那个吉他手,是叫乐奈对吧,她的和声真的很不可思议。初三的学生居然能够有那么稳定的高音吗?曲子也是立希同学做的吧,她总是这样,在我们看不见的地方努力。”祥子抬起头,“能够做到这一步,很不容易呢。”



“……你也一样,小祥。”



她们目送着爱音离开,祥子别过头,发现灯的视线依然停留在爱音的背影上,不由得轻笑起来:“你很喜欢她。”



“小爱是个很了不起的人。我……很尊敬她。”灯哈了口气,“别看她平时那个样子,但是她也有迷茫的时候。是她告诉我迷茫也要前进,不对,迷茫也可以前进。”她转过头,看向祥子,“小祥,之前你在钢琴教室的时候,那时你对我说‘祝你幸福’。我已经有了可以一起迷茫,一起为了一生组乐队努力的伙伴。我已经很幸福了。你呢?”



\newpage



若叶睦抬起手,正打算敲门,门却自己打开了。丰川祥子欣喜地叫道:“睦,你来了。”



“祥。”若叶睦轻声呼唤自己半身的名字,用力跺了跺脚,把脚上的泥和雪踩碎,跟着对方走进门。一股蒸汽裹挟着咖喱的香味扑面而来。初华的声音从她左手边的厨房传来:“小睦你先去和小祥聊会天吧。饭马上就好。”



她无声地点点头,走进客厅。客厅的中间放着一张餐桌,四把椅子。客厅里只开了两盏落地灯,没有开顶灯。桌子边的电暖气正在发出黯淡的红光。除此之外的家具就只有墙上的书架,墙边的一张长沙发,还有窗台前的电子琴。她眨了眨眼睛,注意到自己过去送给祥的人偶依然被她保存在书架上。人偶的衣服似乎是新织的。她没有说话,摘下手套,脱下大衣,走进洗手间洗了洗手,回到桌前坐下。桌子中间摆着一盘橘子。祥子正在剥橘子,见到她出来了就把一个剥好的橘子塞进她手里。她看了看祥子,又把橘子掰成两半,把其中一半塞回了祥子手里。祥子愣了愣,接着笑着和睦一起把橘子吃了下去,然后痛苦地皱起眉头:“好酸!”



睦不禁笑了起来:“确实,很酸。”她想了想,接着补充道:“回甘的部分,很好吃。”



两人有一搭没一搭地聊着天。睦最近和mygo的吉他手要乐奈似乎成为了好友——主要原因是她们的监护人八幡海铃和椎名立希总是一起失踪,她俩就只好被放在一起托管。偶尔爱音会过来投喂乐奈,但是睦依然不太习惯和她相处。素世依然不肯见她,灯告诉她素世还需要一些时间。初华家的新家具是祥子选的。她搬进来第二天就给客厅和卧室铺上了地毯,并且严厉禁止了初华睡地板的想法,跑去买了张双人床。这并不是第一次睦来初华家拜访,两个月前祥子过生日的时候初华就把她和若麦请过来帮忙策划一个惊喜派对。当然最后她们的计划被海铃出卖了——对方的原话是她的效忠对象依然只有祥子本人,因此虽然她被要求保密但是因为祥子单独逼问她所以她很痛快地招了。不过根据她快要扬到天上的眉毛来看她显然乐在其中。最后她们反而被穿着吸血鬼cos服戴着面具的祥子吓了一跳——祥不知怎么回事觉得在自己的生日穿上万圣节装扮是个明智的决定。



“好了,让一让让一让,上菜了!”初华的声音把她拉回现实,祥子已经站起身,帮助初华收拾好桌子,把热气腾腾的咖喱放在了锅垫上。初华又转过身跑去盛米饭。“初华的咖喱可是一绝”——祥子好几次在私下里和她这么吹嘘,看着对方这幅许久未见的兴奋样子,只是让她觉得好笑。上一次祥如此开心的样子是什么时候呢?啊,是那个时候——她摇了摇头,算了,无所谓了。只要现在的祥可以微笑,那就足够了。她站起身,走到初华身边,帮着对方把盛出来的米饭端到桌前。不知为何,她突然想起了在那个陋室里,和初华分开前初华和自己说过的话。在那天之后,初华肉眼可见地消瘦了下去,带着某种她无从知晓的罪恶感与痛苦。她并不知道这是否是单纯因为对方策划了一切,利用了一切,把所有人的命运都牢牢地困在了Ave Mujica这艘贼船上。她也不知道自己去找祥坦白的抉择是否正确。但是当她和祥再次目光相对,当她看着厨房里哼着斯卡布罗集市的初华,她就知道,这一切都不再重要了。



“谢谢。”她低声说道。初华愣了愣,别过头,但是没等她开口,睦就已经转身带着米饭盘离开了。



\newpage



当她走进客厅的时候她发现初华正坐在窗台上,一边喝着水一边望着窗外。她看了看手机,现在是凌晨三点,初华没有开灯,窗外的天空依然一片漆黑,只有不远处的摩天大楼的光火,以及下方地上人造的银河照亮了客厅。睦依然在床上熟睡着。她确信自己没有惊醒睦,那孩子平时太累了,应该让她多休息一点。初华听到响动,转过头看向她:“小祥……”



“你不去多睡会儿?”她指了指卧室,“明明自己打好了地铺却不去睡,这可是浪费。”



“睡不着,”初华打了个哈欠,“……昨天,小睦和我说‘谢谢’了。”



“睦一向很聪明,就算我们不说,她自己估计也会看出来什么的。你不用放在心上。”



“不,我只是……”初华咽了口唾沫,“我不知道。我有些时候会觉得这一切都不是真的,这些都是白日梦。我觉得我并没有做什么配得上她的感谢的事情——”



“她很喜欢你做的咖喱。”祥子打断了她,“她昨天晚上亲口和你说的。”她皱起眉头,“你还会看见他吗?”



“有些时候会。”初华苦笑起来,“他现在就站在你旁边。”



“哦?他在说什么?”



“……我不想说。”



“我来猜猜,”祥子坐到初华身旁,握住她的手,逼着对方直视自己,“说你是个诱骗我的野小子,说我是个没心没肺的不孝女。诸如此类的废话,对吧?”



“倒也没有说的那么重。啊,他走了。我们还是说说Ave Mujica的事情吧。现在若麦小姐已经摘掉了面具,下一个我觉得就让小睦先来吧。毕竟也是时候让她摆脱若叶家的那些人的木偶线了——”



“初华。”



“小祥……”



“我们是共犯。这件事上我们的罪责是同等的。”



“……我明白。”



“我们是Ave Mujica,你们是我的手足,你们将会为我献出一切,直到我残暴的狂想终结。哪怕最终的结果是失败。”



“小祥——”



“但是,我绝对不会死的。”祥子严肃地盯住了初华的双眼,“就像灯说的那样。爬行着,挣扎着,满身泥泞也要活下去。这就是我的战争。”她叹了口气,“真是奇怪,今天睦居然和灯问了我相同的问题。她们问我我幸福吗。”她把头埋进初华的肩膀,享受着对方脖颈的温暖,“我已经很幸福了。因此,作为你的主人,我命令你,你也要幸福地活下去,一直在我的身边陪我参加这场战争。”她抬起头,再次直视着对方的眼睛,“做的到吗?”



“只要是你的命令,小祥……”初华听了这话,轻笑起来。不知不觉间,泪水从她的脸庞滑落,“只要是你的命令……”




\end{document}
