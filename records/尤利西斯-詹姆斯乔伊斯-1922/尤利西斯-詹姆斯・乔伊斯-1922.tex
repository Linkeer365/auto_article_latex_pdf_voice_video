\documentclass{article}
\usepackage[utf8]{inputenc}
\usepackage{ctex}
\usepackage{url}

% support Chinese Chars %
\newfontfamily\urlfontfamily{FandolSong-Regular}
\def\UrlFont{\urlfontfamily}

\title{尤利西斯}
\author{詹姆斯・乔伊斯}
\date{1922}

\maketitle
\url{https://www.99csw.com/book/1121/32652.htm}
\newpage

\begin{document}
\CJKfamily{zhkai}

% : comm which is only poems edit in-use %


\Large

体态丰满而有风度的勃克·穆利根[1]从楼梯口出现。他手里托着一钵肥皂沫,上面交叉放了一面镜子和一把剃胡刀。他没系腰带,淡黄色浴衣被习习晨风吹得稍微向后蓬着[2]。他把那只钵高高举起,吟诵道:



我要走向上主的祭台。



他停下脚步,朝那昏暗的螺旋状楼梯下边瞥了一眼,粗声粗气地嚷道:



“上来,金赤[3]。上来,你这敬畏天主的耶酥会士[4]。”



他庄严地向前走去,登上圆形的炮座。他朝四下里望望,肃穆地对这座塔[5]和周围的田野以及逐渐苏醒着的群山祝福了三遍。然后,他一瞧见斯蒂芬·迪达勒斯就朝他弯下身去,望空中迅速地画了好几个十字,喉咙里还发出咯咯声,摇看头。斯蒂芬·迪达勒斯气恼而昏昏欲睡,双臂倚在楼梯栏杆上,冷冰冰地瞅着一边摇头一边发出咯咯声向他祝福的那张马脸,以及那顶上并未剃光[6]、色泽和纹理都像是浅色橡木的淡黄头发。



勃克·穆利根朝镜下瞅了一眼,赶快阖上钵。



“回到营房去,”他厉声说。



接着又用布道人的腔调说:



“啊,亲爱的人们,这是真正的克里斯廷[7]:肉体和灵魂,血和伤痕。请把音乐放慢一点儿。闭上眼睛,先生们。等一下。这些白血球有点儿不消停。请大家肃静。”



他朝上方斜睨,悠长地低声吹了下呼唤的口哨,随后停下来,全神贯注地倾听着。他那口洁白齐整的牙齿有些地方闪射着金光。克里索斯托[8]。两声尖锐有力的口哨划破寂静回应了他。



“谢谢啦,老伙计,”他精神抖擞地大声说。“蛮好。请你关上电门,好吗?”



他从炮座上跳下来,神色庄重地望着那个观看他的人,并将浴衣那宽松的下摆拢在小腿上。他那郁郁寡欢的胖脸和阴沉的椭圆形下颚令人联想到中世纪作为艺术保护者的高僧。他的唇边徐徐地绽出了榆快的笑意。



“多可笑。”他快活地说。“你这姓名太荒唐了,一个古希腊人[9]。”



他友善而打趣地指了一下,一面暗自笑着,走到胸墙那儿。斯蒂芬·迪达勒斯爬上塔顶,无精打采地跟着他走到半途,就在炮座边上坐下来,静静地望着他怎样把镜子靠在胸墙上,将刷子在钵里浸了浸,往面颊和脖颈上涂起皂沫。



勃克·穆利根用愉快的声调继续讲下去。



“我的姓名也荒唐,玛拉基·穆利根,两个扬抑抑格。可它带些古希腊味道,对不?轻盈快活得正像只公鹿[10]。咱们总得去趟雅典。我要是能从姑妈身上挤出二十镑,你肯一道去吗?”



他把刷子撂在一边,开心地大声笑着说:



“他去吗,那位枯燥乏味的耶酥会士?”



他闭上嘴,仔细地刮起脸来。



“告诉我,穆利根,”斯蒂芬轻声说。



“嗯?乖乖。”



“海恩斯还要在这座塔里住上多久?”



勃克·穆利根从右肩侧过他那半边刮好的脸。



“老天啊,那小子多么讨人嫌!”他坦率地说。“这种笨头笨脑的撒克逊人,他就没把你看作一位有身份的人。天哪,那帮混账的英国人。腰缠万贯,脑满肠肥。因为他是牛津出身呗。喏,迪达勒斯,你才真正有牛津派头呢。他捉摸不透你。哦,我给你起的名字再好不过啦:利刃金赤。”



他小心翼翼地刮着下巴。



“他整宵都在说着关于一只什么黑豹的梦话,”斯蒂芬说,“他的猎枪套在哪儿?”



“一个可悯可悲的疯子!”穆利根说。“你害怕了吧?”



“是啊,”斯蒂芬越来越感到恐怖,热切地说,“黑咕隆咚地在郊外,跟一个满口胡话、哼哼卿卿要射杀一只黑豹的陌生人呆在一块儿。你曾救过快要淹死的人。可我不是英雄。要是他继续呆在这儿,那我就走。”



勃克·穆利根朝着剃胡刀上的肥皂沫皱了皱眉,从坐着的地方跳了下来,慌忙地在裤兜里摸索。



“糟啦,”他瓮声瓮气地嚷道。



他来到炮座跟前,把手伸进斯蒂芬的胸兜,说:



“把你那块鼻涕布借咱使一下。擦擦剃胡刀。”



斯蒂芬听任他拽出那条皱巴巴的脏手绢,捏着一角,把它抖落开来。勃克·穆利根干净利索地揩完剃胡刀,望着手绢说:



“‘大诗人’[11]的鼻涕布。属于咱们爱尔兰诗人的一种新的艺术色彩,鼻涕绿。简直可以尝得出它的滋味,对吗?”



他又跨上胸墙,眺望着都柏林湾。他那浅橡木色的黄头发微微飘动着。



“喏!”他安详地说。“这海不就是阿尔杰所说的吗:一位伟大可爱的母亲[12]?鼻涕绿的海。使人的睾丸紧缩的海。到葡萄紫的大海上去[13]。喂,迪达勒斯,那些希腊人啊。我得教给你。你非用原文来读不可。海!海[14]!她是我们的伟大可爱的母亲。过来瞧瞧。”



斯蒂芬站起来,走到胸墙跟前。他倚着胸墙,俯瞰水面和正在驶出国王镇[15]港口的邮轮。



“我们的强有力的母亲[16],”勃克·穆利根说。



他那双目光锐利的灰色眼睛猛地从海洋移到斯蒂芬的脸上。



“姑妈认为你母亲死在你手里,”他说。“所以她不计我跟你有任何往来。”



“是有人害的她,”斯蒂芬神色阴郁地说。



“该死,金赤,当你那位奄奄一息的母亲央求你跪下来的时候,你总应该照办呀,”勃克·穆利根说。“我跟你一样是个冷心肠人。可你想想看,你那位快咽气的母亲恳求你跪下来为她祷告。而你拒绝了。你身上有股邪气……”



他忽然打住,又往另一边面颊上轻轻涂起肥皂沫来。一味宽厚的笑容使他撇起了嘴唇。



“然而是个可爱的哑剧演员,”他自言自语着。“金赤,所有的哑剧演员当中最可爱的一个。”



他仔细地把脸刮得挺匀净,默默地,专心致专地。



斯蒂芬一只肘支在坑洼不平的花岗石上,手心扶额头,凝视着自己发亮的黑上衣袖子那磨破了的袖口。痛苦——还说不上是爱的痛苦——煎熬着他的心。她去世之后,曾在梦中悄悄地来找过他,她那枯槁的身躯裹在宽松的褐色衣衾里,散发出蜡和黄檀的气味;当她带着微嗔一声不响地朝他俯下身来时,依稀闻到一股淡淡的湿灰气味。隔着槛褛的袖口,他瞥见被身旁那个吃得很好的人的嗓门称作伟大可爱的母亲的海洋。海湾与天际构成环形,盛着大量的暗绿色液体。母亲弥留之际,床畔曾放着一只白瓷钵,里边盛着粘糊糊的绿色胆汁,那是伴着她一阵阵的高声呻吟,撕裂她那腐烂了的肝脏吐出来的。



勃克·穆利根又揩了揩剃刀刃。



“啊,可怜的小狗[17]!”他柔声说,“我得给你件衬衫,几块鼻涕布。那条二手货的裤子怎么样?”



“挺合身,”斯蒂芬回答说。



勃克·穆利根开始刮下唇底下凹陷的部位。



“不是什么正经玩艺儿,”他沾沾自喜地说,“应该叫作二腿货。天晓得是哪个患了梅毒的酒疯子丢下的。我有一条好看的细条纹裤子,灰色的。你穿上一定蛮帅。金赤,我不是在开玩笑。你打扮起来,真他妈的帅。”



“谢谢,”斯蒂芬说,“要是灰色的,我可不能穿。”



“他不能穿,”勃克·穆利根对着镜中自己的脸说,“礼数终归是礼数。他害死了自己的母亲,可是不能穿灰裤子。”



他利利索索地折上剃胡刀,用手指的触须抚摩着光滑的皮肤。



斯蒂芬将视线从海面移向那张有着一双灵活的烟蓝色眼睛的胖脸。



“昨儿晚上跟我一道在‘船记’[18]的那个人,”勃克·穆利根说,“说是你患了痴麻症。他是康内利·诺曼的同事,在痴呆镇工作[19]。痴呆性全身麻痹症。”



他用镜子在空中划了半个圈子,以便把这消息散发到正灿烂地照耀着海面的阳光中去。他撇着剃得干干净净的嘴唇笑了,露出发着白光的齿尖。笑声攫住了他那整个结实强壮的身子。



“瞧瞧你自己,”他说,“你这丑陋的‘大诗人’。”



斯蒂芬弯下身去照了照举在跟前的镜子。镜面上有一道弯曲的裂纹,映在镜中的脸被劈成两半,头发倒竖着。他和旁人眼里的我就是这样的。是谁为我挑选了这么一张脸?这只要把寄生虫除掉的小狗。它也在这么问我。



“是我从老妈子屋里抄来的,”勃克·穆利根说。“对她就该当如此。姑妈总是派没啥姿色的仆人去伺候玛拉基。不叫他受到诱惑[20]。而她的名字叫乌水苏拉[21]。”



他又笑着,把斯蒂芬直勾勾地望着的镜子挪开了。



“凯列班在镜中照不见自己的脸时所感到的愤怒,”[22]他说。“要是王尔德还在世,瞧见你这副尊容,该有多妙。”



斯蒂芬后退了几步,指着镜子沉痛地说:



“这就是爱尔兰艺术的象征。仆人的一面有裂纹的镜子[23]。”



勃克·穆利根突然挽住斯蒂芬的一只胳膊,同他一道在塔顶上转悠。揣在兜里的剃胡刀和镜子发出相互碰撞的丁当声。



“像这样拿你取笑是不公道的,金赤,对吗?”他亲切地说。“老天晓得,你比他们当中的任何人都有骨气。”



又把话题岔开了。他惧怕我的艺术尖刀,正如我害怕他的冷酷无情的钢笔。



“仆人用的有裂纹的镜子。把这话讲给楼下那个牛津家伙[24]听,向他挤出一基尼[25]。他浑身发散着铜臭气,没把你看成有身份的人。他老子要么是把药喇叭[26]根做成的泻药卖给了祖鲁人[27],要么就是靠干下了什么鬼骗局发的家。喂,金赤,要是咱俩通力合作,兴许倒能为本岛干出点名堂来。把它希腊化了[28]。”



克兰利的胳膊[29]。他的胳膊。



“想想看,你竟然得向那些猪猡告帮!我是唯一赏识你的人。你为什么不更多地信任我呢?你凭什么对我鼻子朝天呢?是海恩斯吗?要是他在这儿稍微一闹腾,我就把西摩[30]带来,我们会狠狠地收拾他一顿,比他们收拾克莱夫·肯普索普的那次还要厉害。”



从克莱夫·肯普索普的房间里传出阔少们的喊叫声。一张张苍白的面孔,他们抱在一起,捧腹大笑。唉呀。我快断气啦!要委婉地向她透露这消息,奥布里 [31]!我这就要死啦!他围着桌子一瘸一拐地跑,衬衫被撕成一条条的,像缎带一般在空中呼扇着,裤子脱落到脚后跟上[32],被麦达伦学院那个手里拿着裁缝大剪刀的埃德斯追赶着。糊满了桔子酱的脸惊惶得像头小牛犊。别扒下我的裤子!你们别拿我当呆牛耍着玩!



从敞开着的窗户传出的喧嚷声,惊动了方院的暮色。耳聋的花匠系着围裙,有着一张像煞马修·阿诺德[33]的脸,沿着幽幽的草坪推着割草机,仔细地盯着草茎屑末的飞舞。



我们自己……新异教教义……中心[34]。



“让他呆下去吧,”斯蒂芬说。“他只不过是夜间不对头罢了。”



“那么,是怎么回事?”勃克·穆利根不耐烦地问道。“干脆说吧。我对你是直言不讳的。现在你有什么跟我过不去的呢?”



他们停下脚步,眺望着布莱岬角[35]那钝角形的海岬——它就像一条酣睡中的鲸的鼻尖,浮在水面上。斯蒂芬轻轻地抽出胳膊。



“你要我告诉你吗?”他问。



“嗯,是怎么回事?”勃克·穆利根回答说。“我一点儿也记不起来啦。”



他边说边端详斯蒂芬的脸。微风掠过他的额头,轻拂着他那未经梳理的淡黄头发,使焦灼不安的银光在他的眼睛里晃动。



斯蒂芬边说边被自己的声音弄得很沮丧:



“你记得我母亲去世后,我头一次去你家那天的事吗?”



勃克·穆利根马上皱起眉头,说:



“什么?哪儿?我什么也记不住。我只记得住观念和感觉[36]。你为什么问这个?天哪,到底发生了什么事?”



“你在沏茶,”斯蒂芬说,“我穿过楼梯平台去添开水。你母亲和一位客人从客厅里走出来。她问你,谁在你的房间里。”



“咦?”勃克·穆利根说。“我说什么来看?我可忘啦。”



“你是这么说的,”斯蒂芬回答道,“哦,只不过是迪达勒斯呗,他母亲死得像头畜生。”



勃克·穆利根的两颊骤然泛红了,使他显得更年轻而有魅力。



“我是这么说的吗?”他问道。“啊?那又碍什么事?”



他神经质地晃了晃身子,摆脱了自己的狼狈心情。



“死亡又是什么呢?”他问道,“你母亲也罢,你也罢,我自己也罢。你只瞧见了你母亲的死。我在圣母和里奇蒙[37]那里,每天都看见他们突然咽气,在解剖室里被开膛破肚。这是畜生也会有的那种事情,仅此而已。你母亲弥留之际,要你跪下来为她祷告,你却拒绝了。为什么?因为你身上有可诅咒的耶稣会士的气质,只不过到了你身上就拧啦。对我来说,这完全是个嘲讽,畜生也会有的事儿。她的脑叶失灵了。她管大夫叫彼得·蒂亚泽爵士[38],还把被子上的毛莨饰花拽下来。哄着她,直到她咽气为止呗。你拒绝满足她生前最后的一个愿望,却又跟我怄气,因为我不肯像拉鲁哀特殡仪馆花钱雇来的送葬人那样号丧。荒唐!我想必曾这么说过吧。可我无意损害你母亲死后的名声。”



他越说越理直气壮了。斯蒂芬遮掩着这些话语在他心坎上留下的创伤,极其冷漠地说:



“我想的不是你对我母亲的损害。”



“那么你想的是什么呢?”勃克·穆利根问。



“是对我的损害,”斯蒂芬回答说。



勃克·穆利根用脚后跟转了个圈儿。



“哎呀,你这家伙可真难缠!”他嚷道。



他沿着胸墙疾步走开。斯蒂芬依然站在原地,目光越过风平浪静的海洋,朝那岬角望去。此刻,海面和岬角朦朦胧胧地混为一片了。他两眼的脉搏在跳动,视线模糊了,感到双颊在发热。



从塔里传来朗声喊叫:



“穆利根,你在上边吗?”



“我这就来,”勃克·穆利根回答说。



他朝斯蒂芬转过身来,并说:



“瞧瞧这片大海。它哪里在乎什么损害?跟罗耀拉[39]断绝关系,金赤,下来吧。那个撒克逊征服者[40]早餐要吃煎火腿片。”



他的脑袋在最高一级梯磴那儿又停了一下,这样就刚好同塔顶一般齐了。



“不要成天为这档子事闷闷不乐。我这个人就是有一搭无一搭的。别再那么苦思冥想啦。”



他的头消失了,然而楼梯口传来他往下走时的低吟声:



莫再扭过脸儿去忧虑,



沉浸在爱情那苦涩的奥秘里,



因黄铜车由弗格斯驾驭[41]。



树林的阴影穿过清晨的寂静,从楼梯口悄然无声地飘向他正在眺望着的大海。岸边和海面上,明镜般的海水正泛起一片白色,好像是被登着轻盈的鞋疾跑着的脚踹起来的一般。朦胧的海洋那雪白的胸脯。重音节成双地交融在一起。一只手拨弄着竖琴,琴弦交错,发出谐音。一对对的浪白色歌词闪烁在幽暗的潮水上。



一片云彩开始徐徐地把太阳整个儿遮住,海湾在阴影下变得越发浓绿了。这钵苦水就躺在他脚下。弗格斯之歌,我独自在家里吟唱,抑制着那悠长、阴郁的和音。她的门敞开着,她巴望听到我的歌声。怀着畏惧与怜悯,我悄悄地走近她床头。她在那张简陋的床上哭泣着。为了这一句,斯蒂芬,爱情那苦涩的奥秘。



而今在何处?



她的秘藏:她那上了锁的抽屉里有几把陈旧的羽毛扇、麝香熏过的带穗子的舞会请帖和一串廉价的琥珀珠子。少女时代,她家那浴满阳光的窗户上挂着一只鸟笼。她曾听过老罗伊斯在童话剧《可怕的土耳克》[42]中演唱,而当他这么唱的时候,她就跟旁人一起笑了:



我就是那男孩



能够领略随心所欲地



隐身的愉快。



幻影般的欢乐被贮存起来了,用麝香熏过的。



莫再扭过脸儿去忧虑……



随着她那些小玩艺儿,被贮存在大自然的记忆中了[43]。往事如烟,袭上他那郁闷的心头。当她将领圣体[44]时,她那一玻璃杯从厨房的水管里接来的凉水。在昏暗的秋日傍晚,炉架上为她焙着的一个去了核、填满红糖的苹果。由于替孩子们掐衬衫上的虱子,她那秀丽的指甲被血染红了。



在一个梦中,她悄悄地来到他身旁。她那枯稿的身躯裹在宽松的衣衾里,散发出蜡和黄檀的气味。她朝他俯下身去,向他诉说着无声的密语,她的呼吸有着一股淡淡的湿灰气味。



为了震撼并制伏我的灵魂,她那双呆滞无神的眼睛,从死亡中直勾勾地盯着我。只盯着我一人。那只避邪蜡烛照着她弥留之际的痛苦。幽灵般的光投射在她那备受折磨的脸上。当大家跪下来祷告时,她那嗄哑响亮的呼吸发出恐怖的呼噜呼噜声。她两眼盯着我,想迫使我下跪。饰以百合的光明的司铎群来伴尔,极乐圣童贞之群高唱赞歌来迎尔[45]。



食尸鬼[46]!啖尸肉者!



不,妈妈!由着我,让我活下去吧。



“喂,金赤!”



圆塔里响起勃克·穆利根的嗓音。它沿着楼梯上来,靠近了,又喊了一声。斯蒂芬依然由于灵魂的呼唤而浑身发颤,听到了倾泻而下的温煦阳光以及背后的空气中那友善的话语。



“迪达勒斯,下来吧,乖乖地快点儿挪窝吧。早点做好了。海恩斯为夜里把咱们吵醒的事宜表示歉意。一切都好啦。”



“我这就来,”斯蒂芬转过身来说。



“看在耶稣的面上,来吧,”勃克·穆利根说。“为了我,也为了咱们大家。”



他的头消失了,接着又露了出来。



“我同他谈起你那爱尔兰艺术的象征。他说,非常聪明。向他讨一镑好不好?我是说,一个基尼。”



“今儿早晨我就领薪水了,”斯蒂芬说。



“学校那份儿吗?”勃克·穆利根说。“多少呀?四镑?借给咱一镑。”



“如果你要的话,”斯蒂芬说。



“四枚闪闪发光的金镑,”勃克·穆利根兴高采烈地嚷道。“咱们要豪饮一通,把那些正宗的德鲁伊特[47]吓一跳。四枚万能的金镑。”



他抡起双臂,咚咚地走下石梯,用东伦敦口音荒腔走调地喝道:



啊,咱们快乐一番好吗?



喝威士忌、啤酒和葡萄酒,



为了加冕,



加冕日。



啊,咱们快乐一番好吗?



为了加冕日[48]。



暖洋洋的日光在海面上嬉戏着。镍质肥皂钵在胸墙上发着亮光,被遗忘了。我何必非把它带去不可呢?要么就把它撂在那儿一整天吧,被遗忘的友谊?



他走过去,将它托在手里一会儿,触摸着那股凉劲儿,闻着里面戳着刷子的肥皂沫那粘液的气味。当年在克朗戈伍斯[49]我曾提过香炉[50]。如今我换了个人,可又是同一个人。依然是个奴仆。一个奴仆的奴仆[51]。



在塔内那间有着拱顶的幽暗起居室里,穿着浴衣的勃克·穆利根的身姿,在炉边敏捷地镀来镀去,淡黄色的火焰随之忽隐忽现。穿过高高的堞口,两束柔和的阳光落到石板地上。光线汇合处,一簇煤烟以及煎油脂的气味飘浮着,打着旋涡。



“咱们都快闷死啦,”勃克·穆利根说。“海恩斯,打开那扇门,好吗?”



斯蒂芬将那只刮胡子用的钵撂在橱柜上。坐在吊床上的高个子站起来,走向门道,拉开内侧的两扇门。



“你有钥匙吗?”一个声音问道。



“在迪达勒斯手里,”勃克·穆利根说。“老爷爷,我都给呛死啦。”



他两眼依热望着炉火,咆哮道:



“金赤!”



“它就在锁眼里哪,”斯蒂芬走过来说。



钥匙刺耳地转了两下,而当沉重的大门半开半掩时,怡人的阳光和清新的空气就进来了。海恩斯站在门口朝外面眺望。斯蒂芬把他那倒放着的旅行手提箱拽到桌前,坐下来等着。勃克·穆利根将煎蛋轻轻地甩到身旁的盘子里,然后端过盘子和一把大茶壶,使劲往桌上一放,舒了一口气。



“我都快融化了,”他说,“就像一枝蜡烛在……的时候所说过的。但是别声张。再也不提那事儿啦。金赤,振作起来。面包,黄油,蜂蜜。海恩斯,进来吧。开饭啦。‘天主降福我等,暨所将受于主,普施之惠。’[52]白糖呢?哦,老天,没有牛奶。”



斯蒂芬从橱柜里取出面包、一罐蜂蜜和盛在防融器中的黄油。勃克·穆利根突然气恼起来,一屁股坐下。



“这算是哪门子事呀?”他说。“我叫她八点以后来的。”



“咱们不兑牛奶也能喝嘛,”斯蒂芬说。“橱柜里有只柠檬。”



“呸,你和你那巴黎时尚统统见鬼去吧,”勃克·穆利根说。“我要沙湾牛奶。”



海恩斯从门道里镀了进来,安详地说:



“那个女人带着牛奶上来啦。”



“谢天谢地,”勃克·穆利根从椅子上跳起来,大声说,“坐下。茶在这儿,倒吧。糖在口袋里。诺,我应付不了这见鬼的鸡蛋。”



他在盘子里把煎蛋胡乱分开,然后甩在三个碟子里,口中念诵着:



因父及子及圣神之名[53]。



海恩斯坐下来倒茶。



“我给你们每人两块方糖,”他说。“可是,穆利根,你沏的茶可真酽,呃?”



勃克·穆利根边厚厚地切下好儿片面包,边用老妪哄娃娃的腔调说:



“葛罗甘老婆婆[54]说得好,我沏茶的时候就沏茶,撒尿的时候就撒尿。”



“天哪,这可是茶。”海恩斯说。



勃克·穆利根边沏边用哄娃娃的腔调说:



“我就是这样做的,卡希尔大娘,她说。可不是嘛,老太太,卡希尔大娘说,老天保佑,你别把两种都沏在一个壶里。”



他用刀尖戳起厚厚的面包片,分别递到共餐者面前。



“海恩斯,”他一本正经地说,“你倒可以把这些老乡写进你那本书里。关于登德鲁姆[55]的老乡和人鱼神[56],五行正文和十页注释。在大风年由命运女神姐妹[57]印刷。”



他转向斯蒂芬,扬起眉毛,用迷惑不解的口吻柔声问道:



“你想得起来吗,兄弟,这个关于葛罗甘老婆婆的茶尿两用壶的故事是在《马比诺吉昂》[58]里,还是在《奥义书》[59]里?”



“恐怕都不在,”斯蒂芬严肃地说。



“你现在这么认为吗?”勃克·穆利根用同样的腔调说。“请问,理由何在?”



“我想,”斯蒂芬边吃边说,“《马比诺吉昂》里外都没有这个故事。可以设想,葛罗甘老婆婆跟玛丽·安[60]有血缘关系。”



勃克·穆利根的脸上泛起欣喜的微笑。



“说得有趣!”他嗲声嗲气地说,露出洁白的牙齿,愉快地眨着眼,“你认为她是这样的吗?太有趣啦。”



接着又骤然满脸戚容,一边重新使劲切面包,一边用嘶哑刺耳的声音吼着:



因为玛丽·安老妪,



她一点也不在乎。



可撩起她的衬裙……



他塞了一嘴煎蛋,一边大嚼一边用单调低沉的嗓音唱着。



一个身影闪进来,遮暗了门道。



“牛奶,先生。”



“请进,老太太,”穆利根说,“金赤,拿罐儿来。”



老妪走过来,在斯蒂芬身边停下脚步。“多么好的早晨啊,先生,”她说。“荣耀归于天主。”



“归于谁?”穆利根说着,瞅了她一眼。“哦,当然喽!”



斯蒂芬向后伸手,从橱柜里取出奶罐。



“这岛上的人们,”穆利根漫不经心地对海恩斯说,“经常提起包皮的搜集者[61]。”



“要多少,先生?”老妪问。



“一夸脱[62],”斯蒂芬说。



他望着她先把并不是她的浓浓的白奶倾进量器,随后又倒入罐里。衰老干瘪的乳房。她又添了一量器的奶,还加了点饶头。她老迈而神秘,从清晨的世界踱了进来,兴许是位使者。她边往外倒,边夸耀牛奶好。拂晓时分,在绿油油的牧场里,她蹲在耐心的母牛旁边,一个坐在毒菌上的巫婆,她的皱巴巴的指头敏捷地挤那喷出奶汁的乳头。这些身上被露水打湿、毛皮像丝绸般的牛,跟她熟得很,它们围着她哞哞地叫。最漂亮的牛,贫穷的老妪[63],这是往昔对她的称呼。一个到处流浪、满脸皱纹的老太婆,女神假借这个卑贱者的形象,伺候着她的征服者与她那快乐的叛徒[64]。她是受他们二者玩弄的母王八[65]。来自神秘的早晨的使者。他不晓得她究竟是来伺候的呢,还是来谴责的[66]。然而他不屑于向她讨好。



“的确好得很,老太太,”勃克·穆利根边往大家的杯子里斟牛奶边说。



“尝尝看,先生,”她说。



他按照她的话喝了。



“要是咱们能够靠这样的优质食品过活,”他略微提高嗓门对她说,“就不至于全国到处都是烂牙齿和烂肠子的了。咱们住在潮湿的沼泽地里,吃的是廉价食品,街上满是灰尘、马粪和肺病患者吐的痰。”



“先生,您是医科学生吗?”老妪问。



“我是,老太太,”勃克·穆利根回答说[67]。



斯蒂芬一声不吭地听着,满心的鄙夷。她朝那个对她大声说话的嗓门低下老迈低头,他是她的接骨师和药师; 她却不曾把我看在眼里。也朝那个听她忏悔,赦免她的罪愆,并且除了妇女那不洁净的腰部外,为她浑身涂油以便送她进坟墓的嗓门[68]低头,而妇女是从男人的身上取出来的[69],却不是照神的形象造的[70],她成了蛇的牺牲品[71]。她还朝那个现在使她眼中露着惊奇、茫然神色保持缄默的大嗓门低头。



“你听得懂他在说什么吗?”斯蒂芬问她。



“先生,您讲的是法国语吗?”老妪对海恩斯说。



海恩斯又对她说了一段更长的话,把握十足地。



“爱尔兰语,”勃克·穆利根说。“你有盖尔族[72]的气质吗?”



“我猜那一定是爱尔兰语,”她说,“就是那个腔调。您是从西边儿[73]来的吗,先生?”



“我是个英国人,”海恩斯回答说。



“他是一位英国人,”勃克,穆利根说,“他认为在爱尔兰,我们应该讲爱尔兰语。”



“当然喽,”老枢说,“我自己就不会讲,好惭愧啊。会这个语言的人告诉我说,那可是个了不起的语言哩。”



“岂止了不起,”勃克·穆利根说。“而且神奇无比。再给咱倒点茶,金赤。老太太,你也来一杯好吗?”



“不,谢谢您啦,先生,”老妪边说边把牛奶罐上的提环儿套在手腕上,准备离去。



海恩斯对她说:



“你把帐单带来了吗?穆利根,咱们最好给她吧,你看怎么样?”



斯蒂芬又把三只杯子斟满。



“帐单吗,先生?”她停下脚步说。“喏,一品脱[74]是两便士喽七个早晨二七就合一先令[75]二便士喽还有这三个早晨每夸脱合四个便士三夸脱就是一个先令喽一个先令加一先令二就是二先令二,先生。”



勃克·穆利根叹了口气,并把两面都厚厚地涂满黄油的一块面包皮塞进嘴里,两条腿往前一伸,开始掏起裤兜来。



“清了账,心舒畅,”海恩斯笑吟吟地对他说。



斯蒂芬倒了第三杯。一满匙茶把浓浓的牛奶微微添上点儿颜色。勃克·穆利根掏出一枚佛罗林[76],用手指旋转着,大声嚷道:



“奇迹呀!”



他把它放在桌子面上,朝老妪推送过去,说着:



别再讨了,我亲爱的,



我能给的,全给你啦。[77]



斯蒂芬将银币放到老姻那不那么急切的手里。



“我们还欠你两便士,”他说。



“不着急,先生,”她边接银币边说。“不着急。早安,先生。”



她行了个屈膝礼,踱了出去。勃克·穆利根那温柔的歌声跟在后面:



心肝儿,倘若有多的,



统统献在你的脚前。



他转向斯蒂芬,说:



“说实在的,迪达勒斯,我已经一文不名啦。赶快到你们那家学校去,给咱们取点钱来。今天‘大诗人们’要设宴畅饮。爱尔兰期待每个人今天各尽自己的职责[78]。”



“这么一说我倒想起来了,”海恩斯边说边站起身来,“今天我得到你们的国立图书馆去一趟。”



“咱们先去游泳吧,”勃克·穆利根说。



他朝斯蒂芬转过身来,和蔼地问:



“这是你每月一次洗澡的日子吗,金赤?”



接着,他对海恩斯说:



“这位肮脏的‘大诗人’拿定主意每个月洗一次澡。”



“整个爱尔兰都在被湾流[79]冲洗着,”斯蒂芬边说边听任蜂蜜淌到一片面包上。



海恩斯在角落里正松垮垮地往他的网球衫那宽松领口上系领巾,他说:



“要是你容许的话,我倒想把你这些说词儿收集起来哩。”



他在说我哪。他们泡在澡缸里又洗又擦。内心的苛责。良心。可是这儿还有一点污迹[80]。



“关于仆人的一面有裂纹的镜子就是爱尔兰艺术的象征那番话,真是太妙啦。”



勃克·穆利根在桌子底下踢了斯蒂芬一脚,用热切的语气说:



“海恩斯,你等着听他议论哈姆莱特吧。”



“喏,我是有这个打算,”海恩斯继续对斯蒂芬说着。“我正在想这事儿的时候,那个可怜的老家伙进来啦。”



“我能从中赚点儿钱吗?”斯蒂芬问道。



海恩斯笑了笑。他一面从吊床的钩子上摘下自己那顶灰色呢帽,一面说道:



“这就很难说啦。”



他漫步朝门道踱了出去。勃克·穆利根向斯蒂芬弯过身去,粗声粗气地说:



“你这话说得太蠢了,为什么要这么说?”



“啊?”斯蒂芬说。“问题是要弄到钱。从谁身上弄?从送牛奶的老太婆或是从他那里。我看他们两个,碰上谁算谁。”



“我对他把你大吹了一通,”勃克·穆利根说,“可你却令人不快地斜眼瞟着,搬弄你那套耶酥会士的阴郁的嘲讽。”



“我看不出有什么指望,”斯蒂芬说,“老太婆也罢,那家伙也罢。”



勃克·穆利根凄惨地叹了口气,把手搭在斯蒂芬的胳膊上。



“我也罢,金赤,”他说。



他猛地改变了语调,加上一句:



“千真万确,我认为你说得对。除此之外,他们什么也不称。你为什么不像我这样作弄他们呢?让他们统统见鬼去吧。咱们从这窝里出去吧。”



他站起来,肃穆地解下腰带,脱掉浴衣,认头地说:



“穆利根被强剩下衣服[81]。”



他把兜儿都掏空了,东西放在桌上。



“你的鼻涕布就在这儿,”他说。



他一边安上硬领,系好那不听话的领带,一边对它们以及那东摇西晃的表链说着话,责骂它们。他把双手伸到箱子里去乱翻一气,并且嚷着要一块干净手绢。内心的苛责。天哪,咱们就得打扮得有点特色。我要戴深褐色的手套,穿绿色长统靴。矛盾。我自相矛盾吗?很好,那么我就是要自相矛盾[82]。能言善辩的 [83]玛拉基。正说着的当儿,一个黑色软东西从他手里嗖地飞了出来。



“这是你的拉丁区[84]帽子,”他说。



斯蒂芬把它拾起来戴上了。海恩斯从门道那儿喊他们:



“你们来吗,伙计们?”



“我准备好了,”勃克·穆利根边回答边朝门口走去。“出来吧,金赤,你大概把我们剩的都吃光了吧。”



他认头了,一面迈着庄重的脚步踱了出去,一面几乎是怀着悲痛,严肃地说:



“于是他走出去,遇见了巴特里[85]。”



斯蒂芬把木手杖从它搭着的地方取了来,跟在他们后面走出去。当他们走下梯子时,他就拉上笨重的铁门,上了锁。他将很大的钥匙放在内兜里。



在梯子脚下,勃克·穆利根问道:



“你带上钥匙了吗?”



“我带着哪,”斯蒂芬边说边在他们头里走着。



他继续走着。他听见勃克·穆利根在背后用沉甸甸的浴巾抽打那长得最高的羊齿或草叶。



“趴下,老兄。放老实点儿,老兄。”



海恩斯问道,



“这座塔,你们交房租吗?”



“十二镑,”勃克,穆利根说。



“交给陆军大臣,”斯蒂芬回过头来补充一句。



他们停下步来,海恩斯朝那座塔望了望,最后说:



“啊,冬季可阴冷得够呛。你们管它叫作圆形炮塔吧?”



“这些是比利·皮特[86]叫人盖的,”勃克·穆利根说,“当时法国人在海上[87]。然而我们那座是中心。”



“你对哈姆莱特有何高见?”海恩斯向斯蒂芬问道。



“不,不,”勃克·穆利根烦闷地嚷了起来,“托巴斯·阿奎那[88]也罢,他用来支撑自己那一套的五十五个论点也罢,我都甘拜下风。等我先喝上几杯再说。”



他一边把淡黄色背心的两端拽拽整齐,一边转向斯蒂芬,说:



“金赤,起码得喝上三杯,不然你就应付不了,对吧?”



“既然都等这么久了,”斯蒂芬无精打采地说,“不妨再等一阵子。”



“你挑起了我的好奇心,”海恩斯和蔼可亲地说,“是什么似非而是的怪论吗?”



“瞎扯!”勃克·穆利根说。“我们早就摆脱了王尔德和他那些似非而是的怪论了。这十分简单。他用代数运算出,哈姆莱特的孙子是莎士比亚的祖父,而他本人是他亲爹的亡灵。”



“什么?”海恩斯说着,把指头伸向斯蒂芬。“他本人?”



勃克·穆利根将他的浴巾像祭带[89]般绕在脖子上,纵声笑得前仰后合,跟斯蒂芬咬起耳朵说:“噢,老金赤[90]的阴魂!雅弗在寻找一位父亲哪![91]”



“每天早晨我们总是疲倦的,”斯蒂芬对海恩斯说,“更何况说也说不完呢。”



勃克·穆利根又朝前走了,并举起双手。



“只有神圣的杯中物才能使迪达勒斯打开话匣子,”他说。



“我想要说的是,”当他们跟在后面走的时候,海恩斯向斯蒂芬解释道,“此地的这座塔和这些悬崖不知怎地令我想到艾尔西诺。濒临大海的峻峭的悬崖之巅[92]——对吧?”



勃克·穆利根抽冷子回头瞅了斯蒂芬一眼,然而并没吱声。光天化日之下,在这沉默的一刹那间,斯蒂芬看到自己身穿廉价丧服,满是尘埃,夹在服装华丽的二人之间的这个形象。



“那是个精采的故事,”海恩斯这么一说,又使他们停下脚步。



他的眼睛淡蓝得像是被风净化了的海水,比海水还要淡蓝,坚毅而谨慎。他这个大海的统治者[93],隔着海湾朝南方凝望,一片空旷,闪闪发光的天边,一艘邮船依稀冒着羽毛形的烟,还有一叶孤帆正在穆格林沙洲那儿抢风掉向航行。



“我在什么地方读过从神学上对这方面的诠释,”他若有所思地说,“圣父与圣子的概念。圣子竭力与圣父合为一体。”



勃克·穆利根的脸上立刻绽满欢快的笑容。他望着他们,高兴地张开那生得很俊的嘴唇,两眼那股精明洞察的神色顿然收敛,带着狂热欢快地眨巴着。他来回晃动着一个玩偶脑袋,巴拿马帽檐颤动着,用安详、欣悦而憨朴的嗓门吟咏起来:



我这小伙子,无比地古怪,



妈是犹太人,爹是只鸟儿[94]。



跟木匠约瑟,我可合不来,



为门徒[95]和各各他[96]干一杯。



他伸出食指表示警告:



倘有人认为,我不是神明,



我造出的酒,他休想白饮。



只好去喝水,但愿是淡的,



可别等那酒重新变成水[97]。



为了表示告别,他敏捷地拽了一下斯蒂芬的木手杖,跑到悬崖边沿,双手在两侧拍动着,像鱼鳍,又像是即将腾空飞去者的两翼,并吟咏道:



再会吧,再会,写下我说的一切,



告诉托姆、狄克和哈利,我已从死里复活[98]。



与生俱来的本事,准能使我腾飞,



橄榄山[99]和风吹——再会吧,再会!



他朝着前方的四十步潭[100]一溜烟儿地蹿下去,呼扇着翅膀般的双手,敏捷地跳跳蹦蹦。墨丘利[101]的帽子迎着清风摆动着,把他那鸟语般婉转而短促的叫声,吹回到他们的耳际。



海恩斯一直谨慎地笑着,他和斯蒂芬并肩而行,说:



“我认为咱们不该笑。他真够亵渎神明的。我本人并不是个信徒,可以这么说。然而他那欢快的腔调多少消除了话里的恶意,你看呢?他管这叫什么来看?《木匠约瑟》?”



“那是《滑稽的耶稣》[102]小调,”斯蒂芬回答说。



“哦,”海恩斯说,“你以前听过吗?”



“每天三遍,饭后,”斯蒂芬干巴巴地说。



“你不是信徒吧?”海恩斯问,“我指的是狭义上的信徒,相信从虚无中创造万物啦,神迹和人格神[103]啦。”



“依我看,信仰一词只有一种解释,”斯蒂芬说。



海恩斯停下脚步,掏出一只光滑的银质烟盒,上面闪烁着一颗绿宝石。他用拇指把它按开,递了过去。



“谢谢,”斯蒂芬说着,拿了一支香烟。



海恩斯自己也取了一文,啪的一声又把盒子关上,放回侧兜里,并从背心兜里掏出一只镍制打火匣,也把它按开,自己先点着了烟,随即双手像两扇贝壳似的拢着燃起的火绒,伸向斯蒂芬。



“是啊,当然喽,”他们重新向前走着,他说。“要么信,要么不信,你说对不?就我个人来说,我就容忍不了人格神这种概念。你也不赞成,对吧?”



“你在我身上看到的,”斯蒂芬闷闷不乐地说,“是一个可怕的自由思想的典型。”



他继续走着,等待对方开口,身边拖着那棍棒木手杖。手杖上的金属包头沿着小径轻快地跟随着他,在他的脚后跟吱吱作响。我的好搭档跟着我,叫着斯蒂依依依依依芬。一条波状道道,沿着小径。今晚他们摸着黑儿来到这里,就会踏看它了。他想要这把钥匙。那是我的。房租是我交的。而今我吃着他那苦涩的面包 [104]。把钥匙也给他拉倒。一古脑儿。他会向我讨的。从他的眼神里也看得出来。



“总之,”海恩斯开口说……



斯蒂芬回过头去,只见那冷冷地打量着他的眼色并非完全缺乏善意。



“总之,我认为你是能够在思想上挣脱羁绊的。依我看,你是你自己的主人。”



“我是两个主人的奴仆,”斯蒂芬说,“一个英国人,一个意大利人。”



“意大利人?”海恩斯说。



一个疯狂的女王[l05],年迈而且爱妒忌:给朕下跪。



“还有第三个[106],”斯蒂芬说,“他要我给他打杂。”



“意大利人?”海恩斯又说,“你是什么意思?”



“大英帝国,”斯蒂芬回答说,他的脸涨红了,“还有神圣罗马使徒公教会[107]。”



海恩斯把沾在下唇上的一些烟叶屑抹掉后才说话。



“我很能理解这一点,”他心平气和地说。“我认为一个爱尔兰人一定会这么想的。我们英国人觉得我们对待你们不怎么公平。看来这要怪历史[108]。”



堂堂皇皇而威风凛凛的称号勾起了斯蒂芬对其铜钟那胜利的铿锵声的记忆,信奉独一至圣使徒公教会,礼拜仪式与教义像他本人那稀有着的思想一般缓慢地发展并起着变化,命星的神秘变化。《马尔塞鲁斯教皇[109]弥撒曲》[110]中的使徒象征[111],大家的歌声汇在一起,嘹亮地唱着坚信之歌;在他们的颂歌后面,富于战斗性的教会那位时刻警惕着的使者[112]缴了异教祖师的械,并加以威胁。异教徒们成群结队地逃窜,主教冠歪歪斜斜;他们是佛提乌 [112]以及包括穆利根在内的一群嘲弄者;还有为了证实圣子与圣父并非一体而毕生展开漫长斗争的阿里乌[114],以及否认基督具有凡人肉身的瓦伦廷 [115];再有就是深奥莫测的非洲异教始祖撒伯里乌[116],他主张圣父本人就是他自己的圣子。刚才穆利根就曾用此活来嘲弄这位陌生人[117]。无谓的嘲弄。一切织风者最终必落得一场空[118]。他们受到威胁,被缴械,被击败;在冲突中,来自教会的那些摆好阵势的使者们,米迦勒的万军,用长矛和盾牌永远保卫教会。



听哪,听哪。经久不息的喝采。该死!以天主的名义![119]



“当然喽,我是个英国人,”海恩斯的嗓音说,“因此我在感觉上是个英国人。我也不愿意看到自已的国家落入德国犹太人的手里[120]。我认为当前,这恐怕是我们民族的问题。”



有两个人站在悬崖边上眺望着,一个是商人,另一个是船老大。



“她正向阉牛港[121]开呢。”



船老大略带轻蔑神情朝海湾北部点了点头。



“那一带有五[]深,”他说,“一点钟左右涨潮,它就会朝那边浮去了。今儿个已经是第九天[122]啦。”



淹死的人。一只帆船在空荡荡的海湾里顺风改变着航向,等待一团泡肿的玩艺儿突然浮上来,一张肿胀的脸,盐白色的,翻转向太阳。我在这儿哪。



他们沿着弯曲的小道下到了湾汊。勃克·穆利根站在石头上,他穿了件衬衫,没有别夹子的领带在肩上飘动。一个年轻人抓住他附近一块岩石的尖角,在颜色深得像果冻般的水里,宛若青蛙似地缓缓踹动着两条绿腿。



“弟弟跟你在一起吗,玛拉基?”



“他在韦斯特米思。跟班农[123]一家人在一起。”



“还在那儿吗?班农给我寄来一张明信片。说他在那儿遇见了一个可爱的小姐儿。他管她叫照相姑娘[124]。”



“是快照吧,呃?一拍就成。”



勃克·穆利根坐下来解他那高腰靴子的带子。离岩角不远处,抽冷子冒出一张上岁数的人那涨得通红的脸,喷着水。他攀住石头爬上来。水在他的脑袋以及花环般的一圈灰发[125]上闪烁着,沿着他的胸脯和肚子流淌下来,从他那松垂着的黑色缠腰市里往外冒。



勃克·穆利根闪过身子,让他爬过去,瞥了海恩斯和斯蒂芬一眼,用大拇指甲虔诚地在额头、嘴唇和胸骨上面了十字[126]。



“西摩回城里来啦,”年轻人重新抓住岩角说,“他想弃医从军呢。”



“啊,随他去吧!”勃克·穆利根说。



“下周就该受熬煎了。你认识卡莱尔家那个红毛丫头莉莉吗?”



“认得。”



“昨天晚上跟他在码头上调情来看。她爸爸阔得流油。”



“她够劲儿吗?”



“这,你最好去问西摩。”



“西摩,一个嗜血的军官,”勃克·穆利根说。



他若有所思地点点头,脱下长裤站起来,说了句老生常谈:



“红毛女人浪起来赛过山羊。”



他惊愕地住了口,并摸了摸随风呼扇着的衬衫里面的肋部。



“我的第十二根肋骨没有啦,”他大声说。“我是超人[127]。没有牙齿的金赤和我都是超人。”



他扭着身子脱下衬衫,把它甩在背后他堆衣服的地方。



“玛拉基,你在这儿下来吗?”



“嗯。在床上让开点儿地方吧。”



年轻人在水里猛地向后退去,伸长胳膊利利索索地划了两下,就游到湾汊中部。海恩斯坐在一块石头上抽着烟。



“你不下水吗?”勃克·穆利根问道。



“呆会儿再说,”海恩斯说,“刚吃完早饭可不行。”



斯蒂芬掉过身去。



“穆利根,我要走啦,”他说。



“金赤,给咱那把钥匙,”勃克·穆利根说,“好把我的内衣压压平。”



斯蒂芬递给了他钥匙。勃克·穆利根将它撂在自己那堆衣服上。



“还要两便士,”他说,“好喝上一品脱。就丢在那儿吧。”



斯蒂芬又在那软塌塌的堆儿上丢下两个便士。不是穿,就是脱。勃克·穆利根直直地站着,将双手在胸前握在一起,庄严地说:



“琐罗亚斯德如是说[128]:‘偷自贫穷的,就是借给耶和华……’[129]”



他那肥胖的身躯跳进水去。



“回头见,”海恩斯回头望着攀登小径的斯蒂芬说,爱尔兰人的粗扩使他露出笑容。



公牛的角,马的蹄子,撒克逊人的微笑[130]。



“在‘船记’酒馆,”勃克·穆利根嚷道。“十二点半。”



“好吧,”斯蒂芬说。



他沿着那婉蜒的坡道走去。



饰以百合的光明的



司铎群来伴尔,



极乐圣童贞之群……[131]



壁龛里是神父的一圈灰色光晕,他正在那儿细心地穿上衣服[132]。今晚我不在这儿过夜。家也归不得。



拖得长长的、甜甜的声音从海上呼唤着他。拐弯的时候,他摆了摆手,又呼唤了。一个柔滑、褐色的头,海豹的,远远地在水面上,滚圆的。



篡夺者[133]。



\newpage





{\centering\section*{第一章:注释}}





[1]据理查德・艾尔曼的《詹姆斯,乔伊斯》(牛津大学出版社1983年版,第117页),穆利根的原型系爱尔兰作家、爱尔兰文艺复兴运动的参加者奥利弗・圣约翰・戈加蒂(1878一1957)。



[2]这里,穆利根在模仿天主教神父举行弥撤时的动作。他手里托着的那钵肥皂沫,就权当圣餐杯。镜子和剃胡刀交叉放着,呈十字架形。淡黄色浴衣令人联想到神父做弥撒时罩在外面的金色祭披。下文中的“我要……台“,原文是拉丁文。



[3]金赤是穆利根给斯蒂芬・迪达勒斯起的外号。他把斯蒂芬比作利刃,用金赤来模仿其切割声。



[4]耶酥会是天主教修会之一,一五三四年由西班牙贵族依纳爵・罗耀拉(1491-1556)所创。会规严格,要求会士必须绝对服从会长。



[5]指坐落在都柏林郊外的港口区沙湾(音译为桑迪科沃)的圆形炮塔。这是一八0三至一八0六年间为了防备拿破仑率领的法军入侵,而在爱尔兰沿岸修筑的碉堡的一座。其造型仿效法属科西嘉岛的马铁洛岬角上的海防炮塔,故名马铁洛塔。



[6]某些修会的天主教神父将头顶剃光,周围只留一圈头发。参看本章注[125]。穆利根只是装出一副神父的样子,故未剃发。



[7]这里原应作“圣餐”(Eucharist),作者却写成了女子名克里斯廷(Christine)。二词中均含有基督(Christ)一名。其用意是便它同第十五章末尾玛拉基・奥弗林神父在卧于圣女芭巴拉的祭台上的那个女人身上做黑弥撒的场面相呼应。参看该章注[956]及有关正文。耶酥和门徒(据《新约・马太福音》第l0章第l节,耶酥收了彼得、约翰等十二个门徒)吃筵席时,曾把饼和酒祝福后递给他们,说那是自己的身体和血(见《新约・路加福音》第22章第19-20节)。后世举行弥撒时,神父饮的葡萄酒即代表耶酥的血,教徒领的圣体(面饼)则代表耶酥的躯体。“血和伤痕”是中世纪的一句诅咒“天主的血和伤痕”的简称。



[8]克里索斯托(约347一407),古代基督教希腊教父,名叫约翰。三九八年任君士坦丁堡大主教后,锐意进行改革。但操之过急,开罪于豪富权门,曾被禁闭。死后得以昭雪,被封为圣约翰。他善于传教讲经,长于词令,因而通称“金口约翰”。



[9]据《新约・使徒行传》第6、7章,最早的殉教者斯蒂芬(?一约35)是个受过希腊文教育的犹太人。迪达勒斯(Dedalus)一姓来自神话传说中的希腊建筑师和雕刻家Daedalus。有史时期的希腊人把无法溯源的建筑和雕像都算作是出自迪达勒斯之手。



[10]指他的教名Buck,意译为公鹿。勃克・玛拉基・穆利根是全名。勃克是教名(即洗礼名或第一个名字)。玛拉基是纪念其父亲或家属中其他人的名字。穆利根是姓。通常只称作勃克・穆利根,中间的名字就省略了。



[11]原文作bad,原意吟游诗人。因含有挖苦口吻,故译为大诗人,并加上引号,以示区别。下同。



[12]阿尔杰是阿尔杰农的爱称。这里指英国诗人、文学批评家查理・阿尔杰农・斯温伯恩(1837-1909)。“伟大可爱的母亲”一语出自他的长诗《时间的胜利》1866)。“伟大”是根据海德版翻译的,诸本均作“灰色”。



[13]原文为希腊文。荷马的《奥德修纪》(杨宪益译,上海译文出版社1979年饭苇23页)有“强劲的西风歌啸着,吹过葡萄素的大海”一语。



[14]原文为希腊文。语出自希腊历史学家色诺芬(公元前431一前35O以前)的《远征记》。写作者跟随与胞兄波斯王争夺王位的小居鲁士远征。失败后,他率领万名希腊雇佣军且战且退,公元前四00年回到黑海之滨的希腊城市特拉佩祖斯。这是他们见到海时发出的吹呼。



[15]国王镇(丹莱里的旧称)是都柏林的一个海港区。有东西两个大码头伸入海中,构成一道人造港湾。



[16]语出自拉塞尔(参看第三章注[109]的《宗教与爱情》)。他在这篇散文中阐明“强有力的母亲”指的是“大自然的精神面貌”。穆利根紧接着所说的“姑妈……你手里”一语,当天上午在海边(见第三章注[943]以及当夜(见第十五章注[688])重新浮现在斯蒂芬的脑际。



[17]原文作“dog'sbody"。在凯尔特族(参看第二章注[48])的神话中,狗含有“严加保密”意,所以穆利根用此词来称呼性格内向的斯蒂芬。



[18]“船记”是斯蒂芬等人经常去的酒馆的店名。



[19]康内利・诺曼(1858-1908),爱尔兰精神病学家。痴呆镇指里奇蒙精神病院,自一八八六年起诺曼在那里任院长。



[20]此处套用《天主经》中“不叫我们受到诱惑”一语,但将“我们”改成了“他”。见《路加福音》第11章第4节。



[21]女仆与四世纪的圣女乌尔苏拉同名。据传匈奴人入侵东南欧洲时,科隆(今穗国境内)有一万一千名童贞女殉教。乌水苏拉是她们的领袖。



[22]凯列班是莎士比亚的戏剧《暴风雨》(1611)中一个丑陋而野性的奴隶。语出自爱尔兰诗人、小说家奥斯卡・王水德(1854一1900)的长篇小说《道林・格雷的肖像》(1891)的序言。在该文中,王尔德表达了自己为艺术而艺术的美学观点。原话是:“十九世纪人们对现实主义的厌恶,是凯列班在镜中照得见自己的脸时所感到的愤怒。十九世纪人们对浪漫主义的厌恶,是凯列班在镜中照不见自己的脸时所感到的愤怒。”这里,穆利根把斯蒂芬比作凯列班。



[23]语出自王尔德的论文集《意图》中的《谎言的衰退》(1889)。全句是:“我完全明白你反对把艺术当作一面镜子。你认为,这样一来就把天才降低到有裂纹的镜子的境地了。然而,你无意说,人生是艺术的模仿。人生其实就是一面镜子,艺术才是真实的,对吧?”



[24]牛津家伙指正在搜集爱尔兰格言的海恩斯。



[25]基尼是旧时英国金币,一基尼合二十一先令。



[26]药喇叭,又名球根牵牛;根部可以用来制做泻药。



[27]祖鲁人是非洲东南部班图族的一支土著。



[28]这里的希腊化指的是使爱尔兰开化。都柏林市不同于近代化的大都会,有着当年希腊城邦的性质。正如奥德修由于离乡多年,初回伊大嘉时未认出那是什么地方一样,斯蒂芬回到故里后也觉得格格不入。因此他听了穆利根所说的使爱尔兰“希腊化”的话,并不曾引起共鸣。



[29]在乔伊斯的另一部长篇小说《艺术家年轻时的写照》第5章里,克兰利(参看第九章注[13])曾和斯蒂芬挽臂而行。克兰利参加了爱尔兰独立运动。斯蒂芬则说:“我不愿意去为我已经不再相信的东西卖力,不管它把自己叫作我的家、我的祖国或我的教堂都一样,我将试图在……某种艺术形式中……表现我自己,并仅只使用我能容许自己使用的那些武器来保卫自己――那就是沉默、流亡和机智。”(见黄雨石译本第297页,外国文学出版社1988年版。)



[30]西摩是英国牛津大学麦达伦学院的学生。



[31]“要委婉……息”出自美国人查理・哈里斯所作通俗歌曲《向母亲透露这消息》(1897)。写一个战士临终前嘱咐道,向母亲透露自己阵亡的消息时,要说得委婉一些。奥布里是斯蒂芬迁居到都柏林之前,住在布莱克罗克镇时的一个游伴,见《艺术家年轻时的写照》第2章。



[32]剑桥、牛津等大学的学生们当中时兴的一种捉弄同学的办法:把对方的裤子剥下来,用剪子将衬衫铰成一条条的。



[33]马修・阿诺德(1822-1888),英国诗人、评论家。



[34]“我们自己”是十九世纪九十年代开展的复兴爱尔兰语言文化的运动所提出的口号。意思是:“爱尔兰人的爱尔兰。”“中心”,原文为希腊文。马修・阿诺德提出的文化理想是建立在个人主义之上的古稀腊人文主义与建立在社会伦理上的希伯来主义的统一。斯蒂芬从阿诺德的这一理想联想到要求爱尔兰民族独立的自救口号。他又进一步想到把异教与基督教相调和而成的新异教教义。最后才联想到omphalos一词。此词的意思是中心,指位于雅典西北一百英里处的帕耳那索斯山麓峡谷里的一块圣石,转义为人体的中心部位:肚脐。这里隐啥斯蒂芬等人所住的这座圆塔,乃是爱尔兰艺术的发祥地。



[35]布莱岬角位于沙湾以南七英里处。



[36]这里,穆利根借用了英国哲学家戴维・哈特利(1705一1757)的观点。哈特利的主要著作有《对人及其结构、职责和期望的观察》(两卷本,1749)等。他认为,真正存在于记忆中的只有观念和感觉。



[37]圣母是仁慈圣母玛利亚医院的简称。这是由天主教仁慈会修女所开办的都柏林市最大的一家医院。里奇蒙是里奇蒙精神病院的简称。



[38]彼得・蒂亚泽爵士是生于爱尔兰的英国戏剧家理查德・布林斯利・谢里丹(1751-1816)所作喜剧《造谣学校》(1777)中的一个人物。这位爵士晚年与一个年轻活泼的农村姑娘结了婚。



[39]指耶酥会的创始人,依纳爵・罗耀拉。



[40]撒克逊征服者,原文为爱尔兰语。



[41]这是爱尔兰诗人威廉・巴特勒・叶芝(1865一1939)所作《谁与弗格斯同去》一诗的第7至9行。弗格斯是据传于五世纪从爱尔兰移去的第一位苏格兰国王。下文中的“树林的阴影”和“朦胧的海洋那雪白的胸脯”,出自该诗的第10、l1行。



[42]老罗伊斯指英国喜剧演员爱德华・威廉・罗伊斯(1841一?)。《可怕的土耳克》(1873)是爱尔兰作家埃德温・汉密尔顿(1849一1919)根据英国童话剧《神奇的玫瑰》(1868)改编的。土耳克王由老罗伊斯扮演。当他发现神奇的玫瑰能教会他隐身术时,便高兴地唱起下面这首歌。



[43]英国通神论者艾尔弗雷德・珀西・辛尼特(1840-1921)在《灵魂的成长》(1896)一书中提出,一切事件和思根都贮存在宇宙的记忆中。参看第七章注[224]。



[44]天主教徒领圣体前,自午夜起禁止饮食。



[45]原文为拉丁文。这是信徒弥留之际助善终者在一旁为他(她)念的临终祷文中的两句。斯蒂芬的母亲是一位虔诚的信徒。她死前,斯蒂芬却不曾满足她的愿望,拒绝为她祷告。



[46]这是斯蒂芬责备自己的话。他意识到在母亲生前,他对罗马天主教会的怀疑和不满曾使母亲深深苦恼,故以东方神话中的食尸鬼自喻。



[47]这是英国旧时的一种金币,每枚值一英镑。因上面镌有国王(或女王)像,所以俗称“君主”。



[48]德鲁伊特是古代凯尔特人中有学识者,通常担任祭司、教师和法官。德鲁伊特的家庭里,竟连圣诞节的蛋糕都禁止吃。



[49]出自庆祝爱德华七世加冕(1901年1月22日)的歌曲《加冕日》。“加冕日”又指发薪日,因为工资可折合成克朗。Crown(意即王冠)是旧时的一种镌有王冠图案的硬币,每枚值五先令。



[50]即克朗戈伍斯森林公学。在《艺术家年轻时的写照》一书中,斯蒂芬曾就读于这家小学。下文中的“提过香炉”指神父做弥撒时,斯蒂芬曾担任助祭。



[51]据《旧约・创世记》第7至9章,挪亚一家人乘方舟逃避水灾后,一天挪亚喝醉了酒睡在帐棚里。二儿子含看见父亲赤身露体,便出去告诉了哥哥闪和弟弟雅弗。闪和雅弗替父亲盖上长袍。挪亚洒醒后说:“迦南[含的儿子]当受咒诅,必给他弟兄作奴仆的奴仆。”



[52]这是《饭前祝文》,引自《圣教日课》。



[53]原文为拉丁文。这是《圣号经》的下半段,引自《圣教日课》。



[54]葛罗甘老婆婆是爱尔兰歌曲《内德・葛罗甘》中的人物。



[55]登德鲁姆有两个。(一)位于都柏林市以北六十五英里的港口。(二)都柏林近郊的村。



[56]人鱼神是古代腓力斯人和腓尼基人所信奉的半人半鱼的神。



[57]命运女神姐妹原指《麦克白》中的三女巫,这里则影射爱尔兰诗人叶芝的姐妹伊丽莎白和莉莉。一九0三年,伊丽莎白在登德鲁姆村创立了邓恩・埃默出版社,并为叶芝出版《在七座树林中》一书。该书的版权页上写着,完成于“大风年七月十六日,一九0 三”。按一八三九年爱尔兰曾遭受一场空前的大风灾。从此,“大风年”一词便流行开来。



[58]《马比诺吉昂》是中世纪十一则威尔士故事的总称,以神话、民间故事和英雄传说为基础,记载十二世纪下半叶至十三世纪末的口传故事。



[59]《奥义书》是印度教古代吠陀教义的思辨作品,用散文或韵文写成。自公元前六百年起次第成书,为后世各派印度哲学所依据。



[60]玛丽・安是一八四三年左右为了吓唬苛吏而在爱尔兰民间组织起来的秘密团体。成员以妇女为主,也有乔装成妇女的男子。因此,后来又用此词来影射同性恋者。关于玛丽・安,流传着一些歌曲,而梅布尔・沃辛顿找到的那个版本的末句是:“像男人那样撒尿。”与下文中穆利根所唱的三句歌词刚好凑成一段。



[61]包皮的搜集者,指耶和华。犹太教徒有行割礼(割除阴茎包皮)的传统。参看《创世记》第17章第10至14节。



[62]夸脱是液量单位,一夸脱为一・一四升。



[63]毛皮像绢丝般的牛、最漂亮的牛和贫穷的老妪均为爱尔兰古称。



[64]征服者指英国人,这里,以海恩斯为代表。快乐的叛徒指满足于现状的爱尔兰人,这里,以勃克・穆利根为代表。



[65]母王八,原文为cuckquean,指其丈夫姘上了其他女人。



[66]在《奥德修纪》卷一中,女神雅典娜替奥德修说情,于是,主神宙斯表示同意让奥德修回国。女神便扮成外乡人的模样,到伊大嘉岛来鼓励奥德修的儿子帖雷马科。这里斯蒂芬把送牛奶的老妪比作雅典娜女神,他怀疑她是为了谴责自己不曾满足母亲最后的愿望而来的。



[67]下文中,海德版多一行,[“瞧,真是的,”她说。]其他诸本部没有。



[68]那个嗓门指神父。天主教徒临终前,神父在他(她)身上涂清香油,以便减轻肉体上的痛苦,并给心灵以慰藉。这叫作终傅礼。但据《旧约・利未记》第12章,天主曾通过摩西说,妇女分娩后以及月经期间不洁,因此不在阴部周围涂油。



[69]见《创世记》第2章第22至23节:“耶和华神就用从那人身上所取的肋骨,造成一个女人……那人说:‘她是从男人的身上取出来的。’”



[70]同上,第1章第27节有“神就照着自己的形象造人……造男造女”一语。



[71]同上,第3章:夏娃在蛇的引诱下偷吃禁果,并给她丈夫亚当吃。作为惩罚,耶和华将二人逐出伊甸园。



[72]盖尔语是苏格兰高地人和古代爱尔兰盖尔族的语言。“你有盖尔族的气质吗?”是爱尔兰西部农民的口头用语,意思是:“你会讲爱尔兰话吗?”十九世纪初叶,爱尔兰民族主义的发展使人们重新对爱尔兰的语言、文学、历史和民间传说发生兴趣。当时,除了在偏僻的农付,盖尔语作为一种口语已经衰亡,英语成为爱尔兰的官方和民间通用语言。后来语言学家找到了翻译古代盖尔语手稿的方法,人们这才得以阅读爱尔兰的古籍。



[73]西边儿指爱尔兰西部的偏僻农村。那里的人们依然说爱尔兰语。



[74]品脱是液量名,一品脱合0・五七升弱。



[75]先令是英国当时通用的货币单位。二十先令为一英镑,一先令为十二便士。英币改为十进制后,合十便士。



[76]佛罗林是十三世纪时意大利开始铸造的一种银币。一八四九年以来在英国通用,一佛罗林合两先令。



[77]这是斯温伯恩的长诗《日出前的歌》(1871)“贡献”一节中的第1、2行。下文中的“心肝儿……你的脚前”见同一节的第3、4行。



[78]这里套用一八0五年英国海军统帅纳尔逊(1758一1805)在特拉法尔加角与法、西军舰进行殊死战时对英国海军的训话。 只是把原话“英国期待每人今天各尽自己的职责”中的“英国”改成了“爱尔兰”。



[79]即墨西哥清流。它流向东北,在加拿大纽芬兰岸外与北大西洋漂流汇合,继续朝东北流向不列颠群岛以及北海和挪威海。



[80]语出自莎士比亚的悲剧《麦克白》第5章第1场。麦克白夫人怂恿丈夫把苏格兰国王邓肯杀死后,在梦游中不断地擦手,并且说:“可是这儿还有一点污迹。”



[81]天主教为了纪念耶酥受难,在教堂里设十四座十字架,教徒沿着一座座十字架,边念经边朝拜。“被恶人强剥下衣服”是在第十座十字架前念的经文中的一句。这里,不信教的穆利根戏谑地以耶稣自况。



[82]“我自……矛盾”是美国诗人沃尔特・惠特曼(1819-1892)的长诗《自己之歌》(1855)第51首第6、7行诗句。



[83]“能言善辩的”,也可以译为“墨丘利般的”,参看本章注[101]。



[84]拉丁区是巴黎塞纳河南岸的地区。有不少大学及文化设施,历来是学生和艺术家麇集之地。



[85]按当时都柏林郊区有两个叫作莫里斯・巴特里的农民。《路加福音》第22章第26节作:“于是彼得出去痛哭。”这是文字游戏,“metButterly”(遇见了巴特里)与“weptbitterly”(痛哭)谐音。



[86]比利是威廉的昵称。威廉・皮特(1759一1806),英国首相。



[87]“法国人在海上”一语出自《贫穷的老妪》。这首十八世纪末叶的爱尔兰歌谣表达了“贫穷的老妪”(爱尔兰古称)对越海而来的法国支援者的期待心情。一七九六至一七九七年间,法国人曾两次派出远征军支援爱尔兰革命,均未能到达。一七九八年法国人虽登了陆,却被迫投降。下文中的“中心”,原文为希腊文。



[88]托马斯・阿奎那(1225一1274),意大利神学家、诗人。他区分了自然领域与超自然领域之后,将希腊哲学家亚里士多德和柏拉图的思想,以及奥古斯丁和其他早期教父的思想加以综合,发展成为一套复杂而富有特色的思想体系。



[89]祭带是神父做弥撒时所挂的细长带子,从脖颈垂到胸前。



[90]老金赤指斯蒂芬的父亲。



[91]指英国海军军官弗雷德里克・马里亚特(1792一1848)所写的一部以寻父为主题的小说(1836)。弃儿雅弗千方百计找到的生父,却原来是东印度群岛上的一名脾气暴躁的军官。据《创世记》,挪亚喝醉后,他的儿子闪和雅弗曾去找他,见本章注[51]。斯蒂芬的父亲也是个酒鬼。这里,穆利根把斯蒂芬比作雅弗。



[92]艾尔西诺是丹麦的谢兰岛上一军港。莎士比亚的悲剧《哈姆莱特》即以此港为背景。“濒临……之颠”一语引自《哈姆莱特》第1幕第4场中霍拉旭对哈姆莱特所说的话。



[93]“大海的统治者”指一九一四年以前英国海军和商船在海上称霸。



[94]据(路加福音)第1章,犹太童贞女玛利亚已许配给木匠约瑟,但未成婚前,因圣灵降临到她身上而怀孕,遂生下耶稣。圣灵通常以鸽子的形象出现,故有“鸟儿”一说。《马可福音》第1章第l0节有云:“圣灵仿佛鸽子,降在她身上。”



[95]指耶酥的十二门徒。



[96]各各他是耶稣被钉十字架的地方。



[97]据《约翰福音》第2 章,耶稣和他的门徒在加利利的迦拿应邀赴婚筵时,酒用尽了。那儿摆着六口缸。耶稣对佣人说:“把缸倒满了水。”他们就倒满了,漫到缸口。舀出来一尝,水已变成了酒。这是耶稣所行的头一件神迹。这首打油诗的最后一句指喝下去的酒变成了尿。



[98]语出《路加福音》第24章第46节:“第三日从死里复活。”



[99]橄榄山在耶路撒冷以东,耶稣经常偕同门徒到此。



[100]四十步潭是沙湾的一座专供男子洗澡的天然浴场。



[1O1]墨丘利是罗马神话中众神的信使,相当于希腊神话中的赫耳墨斯。穆利根与《旧约全书》末卷《玛拉基书》里的先知玛拉基(活动时期公元前约460)同名。该名是希伯来语“我的使者”的音译,所以这里把他与墨丘利相比。



[102]勃克・穆利根所唱的《滑稽的耶稣》是根据奥利弗・圣约翰・戈加蒂所作的讽刺诗《快活的耶稣之歌》改编的。



[1O3]人格神是指神也具有人格,而神子耶稣基督乃是人格的楷模。



[104]典出自《神曲・天堂》第17篇。但丁的高祖卡却基达对他说:“你将懂得别人家的面包是多么苦涩,别人家的楼梯是多么难以攀上攀下。”



[1O5]指维多利亚女王(1819一1901),她统治英国达六十四年之久(1837一1901)。



[106]第三个,指穆利根。



[107]指罗马天主教会。



[108]后文中,斯蒂芬借用了海恩斯这句话(见第十五章注[860]及有关正文)。下段中的“独一至圣使徒公教会”,原文为拉丁文。



[109]即马尔塞鲁斯二世(1501一1555),意大利籍教皇,原名塞维尼。即位后仅二十二天即逝世。



[110]《马尔塞鲁斯教皇弥撒曲》系意大利作曲家乔瓦尼・皮耶路易吉・帕莱斯特里纳(1525一1594)所作。这支弥撒曲曾于一八九八年在都柏林的圣女德肋撒教堂被人重新演奏。



[111]指《使徒信经》。传统上,《信经》中的十二个信条分别由十二名使徒来象征,故名。如“我信全能者,天主父,化成天地。”(彼得) “我信其唯一子,耶稣基利斯督我等主。”(约翰)



[112]“教会的使者。指天使长米迦勒。



[113]佛提乌(816一891),原系在俗学者,由拜占廷皇帝米恰尔三世任命为拜占廷教会君士坦丁堡牧首,受到罗马教皇尼古拉一世的反对。在君士坦丁堡会议(867年)上,佛提乌谴责尼古拉,从而形成对立,史称佛提乌分裂局面。



[114]阿里乌(约250-336),利比亚人,埃及亚历山大里亚基督教司铎。尼西亚公会议(325年)公布《尼西亚信经》,指明基督(圣子)与天主(圣父)同样具有神性。阿里乌拒绝签名。他倡导阿里乌主义,认为基督是被造的(made, 指系天主所造,因而不具有完全的神性),而不是受生的(begotten,指由天主所生,因而具有完全的神性)。这种理论被早期教会宣布为异端。



[115]瓦伦廷是公元二世纪的宗教哲学家,出生于埃及, 为诺斯替教罗马派和广大利派的创始人。公元一四0年前后曾谋求罗马主教之职位而失败, 遂脱离基督教。瓦伦廷的早期理论与保罗的神秘神学相似,强调基督死后复活,信徒因而得救。



[116]撒伯里乌(?一270),可能曾任罗马教会长老。他反对天主教会关于三位一体(谓天主本体为一,但又是圣父、圣子耶稣基督和圣灵三位)的教义,而主张天主是单一的,而有三种功能,圣父创造天地,圣子救赎罪人,圣灵使人成圣。因此,被斥为异端邪说。



[117]陌生人是爱尔兰人对英国人(侵略者与霸主)的称呼。



[118]这里套用英国诗人约翰・韦伯斯特(约1580一约1625)的《魔鬼的诉讼》(1623)的词句:“国王野心一场空……织网只为了捕风。”



[119]原文为法语。这是斯蒂芬从冥想中醒过来后暗自说的话。



[120]指德裔犹太富豪罗斯蔡尔德家族。当时他们控制着英国经济。



[121]她指船。阉牛港位于都柏林湾东南方的岬角。下文中的噚是测量水深用的长度单位,一噚合一·八九八米。



[122]它指溺尸。民间迷信:失去踪影的沉尸会在第九天浮上来。



[128]韦斯特米思位于都柏林市以西四十英里处,是爱尔兰伦斯特省一郡。亚历克·班农是个学生,参看第四章中米莉来信和第十四章注[146]及有关正文。



[124]指本书另一主人公利奥波德·布卢姆的女儿米莉。她在韦斯特米思郡穆林加尔市的照相馆工作。该市距都柏林五十英里。



[125]这个泅水者的头顶剃光了,只留下一圏灰发,说明他是个天主教神父。直到一九七二年,这一习俗才由教皇保罗六世下令废除。



[126]这是基督教会自古流行的一种对天主三位一体(圣父=额头,圣子=嘴唇,圣神=胸部)表示尊崇的手势。天主教神父举行弥撒时,在诵读经文前以及仪式结束后,照例要划十字。



[127]原文为德语。《创世记》第2章第21节有天主抽掉亚当一根肋骨的记载。这里,穆利根以亚当自况,说他的“第十二根肋骨没有了”,这样,他就成了“超人”。



[128]琐罗亚斯德(约公元前628一约前551),穆斯林先知、琐罗亚斯德教创始人。古波斯语作查拉图斯特拉。《琐罗亚斯德如是说》(1883-1885)是德国哲学家尼采(1844一1900)的一部谶语式的格言著作。他在其中借琐罗亚斯德来鼓吹自己的“超人”哲学(即认为“超人”是历史的创造者,有权奴役群众,而普通人只是“超人”实现自己权力意志的工具)。



[129]这里,勃克·穆利根故意篡改了《箴言》第19章第17节“怜悯贫穷的,就是借给耶和华……”一语,借以挖苦说,尼采是个极端的利己主义者,以别人为踏脚石来达到自己的目的,在本世纪初,西欧曾流行过这种论点。



[130]意思是说,这三者都是危险的,不能掉以轻心。



[131]原文是拉丁文。



[132]本章以勃克·穆利根假装举行弥撒为开端[见本章注[2]],结尾处又把一位真正的神父出浴后在湾汊的岩洞中穿衣服比作弥撒结束后神父在更衣,并将神父那圈灰发描述成圣徒头后的光晕。壁龛指岩洞。



[133]篡夺者指从斯蒂芬手里讨走钥匙的勃克·穆利根。在《奥德修纪》卷1、2中,帕雷马科也曾指责那些求婚子弟们掠夺他的家财;哈姆莱特王子则对霍拉旭说,叔叔克劳狄斯“篡夺了我嗣位的权利”,参看《哈姆莱特》第5幕第2场。

\end{document}
