\documentclass{article}
\usepackage[utf8]{inputenc}
\usepackage{ctex}
\usepackage{url}

% support Chinese Chars %
\newfontfamily\urlfontfamily{FandolSong-Regular}
\def\UrlFont{\urlfontfamily}

\title{【初祥】追随:第三章——燃烧}
\author{sandman}
\date{20231014}

\maketitle
\url{https://www.pixiv.net/novel/show.php?id=20843015}
\newpage

\begin{document}
\CJKfamily{zhkai}

% : comm which is only poems edit in-use %


\Large

人类与人偶的差别究竟是什么?



“小睦真可爱啊,就像人偶一样漂亮。”她已经习惯了这样的赞美。父母似乎很喜欢在出门会客时带上自己。每周三两小时钢琴课,周四三小时芭蕾。周六两小时茶道和一小时插花。周日还有家教指导礼仪。国色天香,行为得体的“若叶睦”就这样被制造出来。吃东西时要闭住嘴,喝水时不出声,被别人看着的时候要垂下眼睑表示谦卑,没人看时更要三缄其口,如同干花一般隐入背景。这样才是父母的好女儿,会客厅里一道亮丽的好风景。守规矩才会被人所爱,不听话就会被怒脸相迎。这就是她的人生。不要张扬地表达欢喜,因为那并非淑女;不要恣意地显露悲伤,因为那会惹人心烦。久而久之,若叶睦也忘记了什么是表情。



她睁开眼睛,她们依然置身于黑暗,静谧,无边无际的树林之中。这是个新月之夜,照亮她们道路的只有苍白的星光与山脚下城市的残辉。她的脚踝依然在疼,身体好像变得更冷了。身下的少女喘着粗气,后颈上满是汗水。她轻轻呼唤背负自己的人的名字:“祥。”



“怎么了,睦?”祥子头也不回地说道,“又有哪里不舒服吗?坚持一下,我们马上就出去了。”



她们本来不应该在这里的。不应该让两个不到八岁的小姑娘,在晚上,自己在树林里玩的。也许是祥的一时兴起,也许是注意到了她渴望的眼神,也许单纯地是想要让她开心一点,祥提出了这个主意。她的脚在从一块石头上跳下的时候崴到了,直到那时她们才意识到自己已经深入树林太深,距离安全太远。山脚下的灯火看上去就像暴风雨中的灯塔,缥缈而遥远。祥没有说些什么,只是默默地走到她身边,背起她,颤颤巍巍地向着山下走去。



“放下我。我只会拖累你。祥你自己回去,再去找大人。”



“我不要。”



“为什么?”



“因为……”祥子停下了脚步,喘了口气,果然自己对于她还是太重了,“睦你会害怕的。而且我们不是说好了吗?不管发生什么事都要一起分担。我和睦可是命运共同体哦。”



“祥……”她闭上眼睛,胃里感觉好奇怪,好像有股暖流在里面翻涌,这是第一次有这种感觉,她要死了吗?“我只是个人偶,你听到他们说的了。你不应该和人偶这么亲密。”



“睦才不是什么人偶!”祥的语气突然激烈起来,“睦可和人偶不一样。睦有自己的想法,会自己行动,最重要的是,睦和我一样,有着一颗跳动的心啊。”



拥有自己的想法就不是人偶吗?



“睦也是会笑的哦。”



不对。



可以自己行动的就不是人偶吗?



“素世她……不知道该怎么办才好了。”



不对。



拥有自己的心就不是人偶吗?



“这不是任何人的错。”



不对!



一条狗也会有自己的想法,一座上了发条的钟也会自己行动,一个自动人偶也有着自己的机械之心。这些都不是区分人和人偶的关键!



丰川祥子走进录音室,外面在下雨,而她没有打伞。若叶睦看着水滴从祥子湿透的衣服和发梢上滴落,砸在地上,如同沉重的定音鼓般在她的耳中回荡。祥子抱着左臂,目光扫过众人,唯有在看到灯的时候多看了几眼,最后停留在地上。睦看着他,两人不需要任何多余的话语,甚至不需要目光交流,睦便已经理解了一切。祥在痛苦,祥很脆弱,但是祥会一如既往地自己扛下这一切。然后祥子开口了:“……我要退出Crychic。”



人偶是被人类制造的。人因为有了心而被自由诅咒,人因为拥有自由而在这世上犯下罪恶。人因罪恶而蒙尘,人因罪恶而定义自己。人类制造的人偶则恰好相反——不论犯下何等暴行,不论沾染多少鲜血,这一切罪恶却终究属于,而且只能属于人偶的制造者。而人偶是无罪的。



“小睦,你要记住,你的名字寓意和睦与幸福,是很好的名字。むつみ,也就是無(む)罪(つみ)。这是因为爸爸妈妈希望你一辈子都不要被罪恶沾染,好好地过上幸福的生活。”



若叶睦讨厌自己的名字。



罪孽深重的若叶睦是无罪的,在若叶睦看来,这简直就是黑色幽默——她的无罪正是因为有人替她背负了自己的罪。她可以在月之森和大小姐们为伍,在贵安中愉快地安享温暖的被褥,丰富的美食,宽敞的房子,亲切的友人;而祥却只能栖身于那小小的出租屋中,与蟑螂,垃圾,燥热为伴,承担所有的指责和恶意,一天天被生活锤烂。祥是无辜的,却背负了所有的罪恶,睦罪孽深重,却是无罪的。这不公平。



若叶睦抬起头,不远处,一个留着黑色长发,戴着黑框眼镜的女性正向着她走来。女人走到她身边,停下脚步,接着三角初华那熟悉的关切的声音从她身边响起了:“小睦,你真的想好了吗?”



这全都是小睦你的错!



也许吧。长崎素世,也许只有她看穿了自己。但现在也已经不重要了。素世已经获得了新的归宿和幸福。笼罩在素世身上的易碎感已经消失了。那样的她也不需要自己的关心了。到头来还是没有帮上她们之中任何一个的忙。她早该料到的,试图调和两种互不相容的背道而驰的力量,结果就是谁也不能拯救。中立已经没有意义了,于是她选择了选边站,可是这依然不够。



“嗯,”她点点头,看向身旁通向出租屋的台阶,“走吧,初华。”如果人偶注定是无罪的话,那么沾染罪恶的她能否终于成为人类呢?带着这样的疑问,她推开了房门。“已经不能让祥继续痛苦下去了。”



\newpage





{\centering\section*{第三章——燃烧}}





祇园精舍钟声响,诉说世事本无常;

娑罗双树花失色,盛者转衰如沧桑。

骄奢淫逸不长久,恰如春夜梦一场;

强梁霸道终覆灭,好似风中尘土扬。



高松灯站在桥上,看着不远处旁若无人地起舞的蓝发少女。少女的身形比她印象中更加瘦削了,脸上的颧骨上挂着一层皮,青白色的筋几乎要从手臂上爆裂出来。少女歌唱着,舞动着,仿佛自己并非身处人群之中,铁路桥上,而是在一间宽敞明亮的舞厅,独自一人。在那里,没有饥饿,没有忧虑,也没有悲伤。晚秋已至,一个小小的气旋牵着尘埃与落叶从灯身边掠过,下沉的夕阳把人群镀上了一层蜡黄。丰川祥子继续哼唱着那首只有她和灯知道的歌,旋转着,跳跃着,仿佛在追随着某种只有她才能看见的东西。恍惚之间,灯仿佛回到了那个春天,那时也是在这座桥上,她追随着花瓣,然后遇到了祥子。那时樱花盛开,温暖的春日洒在她们的肩头。那时的她依然孤独,那时的祥子依然可以展露笑容。



在她接到那个电话的时候,她正在整理自己的笔记本。她看着手机上陌生的号码,心中不知为何有了些许不安,胸口仿佛有一只仓鼠在啮咬着她的心,但她还是接通了电话,一个清冷的女声响起了,片刻之后,她才意识到那是三角初华的声音。这位善解人意的萍水相逢的观星者对于她的恩情她还铭记于心,只是对方的声音失去了往日的随和与温柔,取而代之的则是某种无法言说的焦躁:“小灯……小祥失踪了。”



灯的眼睛因为惊讶张大了,她虽然对于新秀乐队Ave Mujica的成员身份有了一些自己的猜测,但是她并没有预料到会从对方口中直接听到那个名字:“你说的小祥……是指丰川祥子吗?”



电话那边的人沉默了,不知过了多久,也许是一分钟,也许是一个世纪,初华的声音又一次响起了,这次带上了些许颤抖:“是的,她之前因为营养不良被送进了医院,但是却从医院里逃跑了。我们现在找不到她,没有人知道她去了哪里。我和小睦都不知道她的去向,我就在想,如果是你,也许会知道些什么……”



“对不起,初华小姐,我还在家,并没有见到她。但是……”她闭上眼睛——



花和生命,究竟哪个重要呢?



从今往后,我们就是一起演奏的命运共同体了。



灯的歌词,是你内心的呐喊。



好想要成为人类啊!



今天我一直觉得很害怕,但我有一种被你鼓励了的感觉。



祝你幸福。



“我会去找到她的!因为小祥对我来说也是十分重要的人!我不会让她有事的!”



电话那边的声音又一次沉默了,终于,初华苦涩的声音艰难地从话筒中传来,这一次每一个字仿佛都要耗尽她全身的力气:“是吗……那么就拜托你了……请你一定要找到她……”电话挂断了。



她并不知道为什么自己会有信心找到祥子。也许只是一时的意气用事,也许只是因为想要还对方一个人情,也许到头来,她依然相信自己和祥子拥有着某种神秘的联系。她直接去了那里,千登世桥,她和祥子第一次相遇的地方。然后,她看到了,蓝色头发的少女如同人偶般翩翩起舞,樱花树上颤抖着的落叶在她脸上印下暧昧的光斑,少女的身体如此瘦小,仿佛一个来自异界的幻影一般缥缈不定。祥子神情恍惚地转了个圈,舞姿变得愈发激烈了,她高高地抬起右腿,然后,向着栏杆跳去——



“不可以!小祥——”



\newpage



八幡海铃挂掉电话,抬起头。丰川祥子的状态不对劲,这点从Ave Mujica成立伊始是个明眼人就看得出来。但从医院逃跑……她把手机放进口袋,用力吸干盒中最后一点橘子牛奶,把盒子扔进垃圾桶。丰川祥子会去哪里?对于这位她们这群一丘之貉的领袖,她了解并不多。丰川祥子与三角初华和若叶睦都是青梅竹马,她似乎认识立希,而且与立希之前那支乐队的主唱有过一段孽缘。这就是她所知道的全部。丰川祥子的身上似乎总是有一种拒人于千里之外的疏离感,如同一匹受了伤的独狼,冲着所有关心她的人龇牙咧嘴,暗地里却一个人舔着伤口。八幡海铃理解距离感在人类交往中的重要性。她知道对于他人想要隐瞒的秘密刨根问底往往是不和的开端。于是她闭上嘴,视而不见,听而不闻,遵守着雇佣兵的道德,做好自己的分内之事——弹贝斯,安抚祐天寺若麦,做若叶睦的保镖,和初华搞好关系,直到大厦崩塌。



她并没有预料到会在这里遇到若叶睦。这是一个普通的周末,她虽然按照祥子的要求,已经不再接Ave Mujica之外的临时工作,但她还是会来Ring这里,看看过去的那些说不上多么亲密的老朋友们的演出。据她所知,若叶睦虽然也会来这里,但是基本只有MyGO演出的时候才会来。虽然睦似乎当初送出的礼物被对方拒绝了,但她还是风雨无阻地去听了之后的每一场演出。每次演出结束的时候她都会和台上的贝斯手有一些无声的眼神交流,每一次那个叫长崎素世的贝斯手都会皱起眉头,别开眼睛。今天MyGO并没有演出,也许她应该修改自己对于睦的观察。睦神色匆忙地从地下室走出,在她一向波澜无惊的脸庞上,海铃第一次看到了可以说是焦急的神情。她想要伸出手招呼对方,但是却鬼使神差地缩回了手。也许是因为好奇,也许是因为长时间不闻不问导致的补偿机制,也许单纯是因为那个下午,阳光过于刺眼,她什么也没有说,而是沉默地跟在睦的身后。睦走出了LiveHouse,没有向往常一样去找自己的司机,而是径直向着电车站走去。海铃跟着她走上电车,然后看着她在那个陌生的站台下了车。赤羽站。她默默记下站牌,然后在关门前的最后一刻跳下电车,保持着距离。她看着若叶睦停在了那间不起眼的合租屋门口。过了一会儿,一个其貌不扬的黑发女人背着双肩背靠近若叶睦,然后两人一起走上了楼梯,消失无踪。五分钟过去了,十分钟过去了,二十分钟过去了,八幡海铃终于无法忽视内心的不安,走向那间小房子,爬上楼梯。一股阴湿的尘土气息扑面而来,坑坑洼洼长满了霉菌的木地板在她脚下发出垂死的呻吟,一只苍蝇在她身边徘徊了几圈,越过这位不速之客,一头扎进了她刚刚离开的秋风萧瑟的十月天。右侧的第一间房门开着,一股垃圾腐败的酸臭伴随着刺鼻的酒精味从门口飘出。她屏住呼吸,一步步踱到门口,眼前的景象有如异界一般——



若叶睦和黑发女人站在房间中间,地板上堆满了空易拉罐和酒瓶,在她们的正前方,失去了气息的男人一动不动地卧倒在房间中间。夕阳透过窗外建筑物的反射洒入屋内,给一切染上了一层妖异的紫红色。电视机里的女人正在用尖细的声音推销一款除臭剂。一个穿着红色洋装的人偶坐在门口的柜子上,静静地俯瞰着这小小斗室中的悲欢离合,喜怒哀乐。黑发女人听到脚步声,向她瞥了一眼,接着看向睦。八幡海铃皱起眉头,她一向对于人世间的种种不义有所了解,故而很少对身边的荒唐之事感到惊讶。但是今天,也许她头一次感到了某种可以称之为惊讶的感情:“睦……你在做什么?”



听到她的话,黑发女人的肩膀颤抖起来,她张开嘴,想要说些什么,但是睦却仿佛预料到了她的问题一般,抢先开口了:“没事的,初华。我确认过了,只有海铃跟着我。”



听到睦的话,黑发女人仿佛下定了什么决心,举起手,摘掉眼镜,把头发用力一扯——黑色的假发套随之脱落,三角初华那头标志性的金色短发显现在她眼前。三角初华虽然戴了美瞳,但是眼神里的冰冷依然仿佛要溢出来:“八幡同学,你觉得呢?”



八幡海铃眯起眼睛:“难道说……”她看向睦,“你从一开始就发现了,对吗?”



没等睦回答,脚步声又一次响起了,一个轻浮的声音从她背后传来:“嗨,大家好啊,今天喵梦亲要给大家解密的是喵梦亲担任鼓手的Ave Mujica最为神秘的成员,Oblivionis的住处。真是人不可貌相啊,居然住在这种其貌不扬的出租屋。啊嘞?”她看向盯住她的三人,“你们怎么也在?”



\newpage



她的身体重重地撞进了祥子,把祥子扑倒在地。在她背后,爱音发出了一声惊呼,接着赶紧向着周围的路人解释她们这是在排练话剧。祥子既没有喊叫,也没有挣扎,只是呆呆地看着她,仿佛看到了某种镜花水月一般。膝盖很疼,大概是蹭破皮了吧,但她没有时间在意那些:“小祥……”



祥子眨了眨眼,接着皱起眉头,终于认出了她。一股力量猛地从她身上爆发出来,把灯用力甩开。灯被她的爆发吓了一跳,摔倒在地,周围的人群中传出了一声惊呼。祥子跌跌撞撞地爬起身,想要拨开人群离开。灯赶忙挣扎着站起身,膝盖处立刻传来一阵针扎般的刺痛,让她倒吸一口冷气:“小祥!”



也许是再也无法忽视她的叫喊,也许是因为听到了她的吸气声,对方的脚步停下了。祥子的肩膀微微颤抖起来:“……你……带着创可贴吗?”



“小祥……”



祥子叹了口气,面无表情地转过身,扶住灯:“那边有个长椅,这里人太多了,我们处理一下你的伤口。”



她乖巧地点点头,一瘸一拐地跟着祥子向着长椅走去。她感到自己的脸有些发热,可能是因为尴尬,可能是因为惊喜,当然最有可能的是她久违的剧烈运动。她在长椅上坐下,安静地从口袋里抽出企鹅创可贴递给祥子。祥子则一言不发地蹲下身,把创可贴贴在她还在流血的膝盖上。她抬起头,看向在北风中颤抖的树叶,如同群马一般在天上奔腾的白云,还有被残阳的余晖染红的天空。她低下头,祥子已经贴好了创可贴,正要起身离开。她赶忙伸出手,抓住了对方的袖子:“小祥!”



“……怎么了?”祥子低着头,让她看不清对方脸上的表情。她一时间感到口干舌燥,是啊,自己要说些什么呢?问她为什么会在这里跳舞吗?问她刚刚想要做什么吗?告诉她自己依然很想念她吗?她咬了咬牙:



“可以……陪我坐会儿吗?如果你方便的话……”



出乎她的意料,祥子并没有拒绝,而是轻声说了句“失礼了”,坐在了长椅的另一端。然后,两人便一言不发地看着天空中变换的流云,看着大街上熙熙攘攘的人群,看着逐渐亮起的路灯。时间不知道过去了多久,祥子一直没有说话,也一直没有离开,只是静静地坐在那里。两人之间隔了一个人的距离,却心照不宣地都没有再靠近对方一点。灯用自己的余光偷偷观察祥子,发现对方的脸上依然面无表情,仿佛刚刚在桥上的一切都没有发生过一般。她轻轻哈了口气,看着翻卷的白烟,终于下定了决心:“Ave Mujica的演出,很好看呢。虽然不是我习惯的类型……小睦的吉他真厉害啊。”



“是吗?你看了啊。”祥子淡淡地回答道。



“小祥……你为什么在这里?”



“我为什么在这里呢?”祥子仰起头,“我也不知道啊。大概是因为发现自己并没有自己想象中那么强大,所以逃跑了?”



沉默。



“你……迷路了吗?”



“迷路吗?”祥子自嘲地笑了,“也许吧。在无意识中抵达了这里。”她转过头,看向灯,“你还真是喜欢这个词啊,就连你的新乐队都是这个名字。”



“小祥,那个是……”



“不是很好吗?迷路的孩子们,一直迷失的人。”祥子低声说道,“要是我也只是迷路就好了。”



沉默。



“……我,真的成为人类了吗?”



“成为了哦,有着值得为自己流泪的重要事物,在这件事上灯毫无疑问是成为人类了。”



“小祥,那你呢?”



“……我不再想成为人类了。我也没办法成为人类了。”



再次沉默。



“……之前想要给你的歌,立希已经写好曲子了,”她咽了口唾沫,试探着开口了,“如果哪天你方便了,希望你可以……不对,你一定要来听一下。”



祥子皱起眉头,沉默了。她的心漏跳了半拍,自己一定是又说错话了,不对,不要这么想——自己又伤害了她,她也不会怪罪你的——她又会离开自己了,她没有那么脆弱——



“好啊,”她惊讶地转过头,出乎她的意料,祥子的眉头舒展开来了,她的表情依然冷漠,但是眼睛里却仿佛有一丝若有若无的笑意,“我答应你,一定会去听的。不过在那之前,这大概是我们最后一次见面了。”她站起身,走了几步,接着扭过头,再次看向灯,“时间也不晚了,你也该回家了。你还有什么分别的话吗?”



然后,高松灯意识到了。丰川祥子不会被拯救,也从来都不需要拯救。祥子是个坚强而又骄傲的人,她无法忍受别人的同情,别人的怜悯,别人的善意,那对她来说比杀死她更为痛苦。她甚至不需要“你已经很努力了”这样虚无的肯定。到头来,她所能够做的,其实只是在对方的背上轻轻拍上一把:



“……小祥,我是个凡人,我其实只会唱歌和写歌,我也知道我的歌既不能抚平你溃烂的伤口,也无法改变你灰暗的现实,更无法把你从绝望中拯救。这样的我唱着的歌到头来也只是自我满足罢了,根本帮不到你。你离开crychic的时候,我真的很痛苦,很痛苦。感觉心仿佛第二次沉入了海底一样。不知道该怎么做。但是和你在一起的那些时光,曾经是我最温暖的一段日子……但是……但是——即便如此,不管你变成什么样子,做过什么,身处何方,我的心都会和你一道。所以……”她闭上眼睛,低声说道,“加油啊!”



她睁开眼,接着,自从那个雨天以来,她第一次看到丰川祥子的笑容:“谢谢你,灯。”



\newpage



“所以说,现在怎么办?”三角初华转过头,乐队的贝斯手八幡海铃两手抱胸立在门口,面无表情。在她的身后,祐天寺若麦用手遮住了口鼻,不知是在阻挡气味还是在掩盖她脸上若有若无的笑意。初华站起身,若叶睦站在她的身旁一动不动,一如既往地捉摸不透。在她们四个人面前,狭小逼仄的起居室里散落着空酒瓶和易拉罐,电视机的屏幕一闪一闪的,偶尔会照亮男人苍白的脸庞。男人的头不自然地歪到一边,右边的额角不自然地肿了起来。他胸口的衣服被口中溢出的白沫和呕吐物覆盖,散发出一股酸味。一只顽强的苍蝇不知何时飞进了屋子,盘旋了几圈之后停留在了男人涣散的瞳孔上,爬行着,探求着。初华转过身,目光扫过她的队友们,最后停留在了门口的储物柜上。一个穿着红色洋装的人偶坐在那里静静地看着她们,无神的眼睛里似乎有微光在闪烁。太阳已经沉入了地底,晚秋的夜空中浓云密布。远方依稀可以听见火车在高架桥上的低吟声,在三角初华听来,那声音却仿佛雷鸣般震耳欲聋。八幡海铃再一次开口了:“要报警吗?”



“不行。”没等她开口,睦抢先回答了这个问题,“祥……不需要知道这些。”



三角初华蹲下身,男人的手机就放在他右手边不远处的茶几上,她拿起手机,没有未接来电,也没有新的通话记录。男人曾经是个才华横溢的钢琴家,虽然不能说红极一时,也可以说是小有名气。她曾经在电视上看到男人的演奏,那时的她对对方只有崇敬。而现在,男人就这样悄无声息地死去了,孤身一人,在这个肮脏的出租屋里,身边唯有酒瓶和蟑螂陪伴,连求救的水花也没有溅起一点。



“那么的话……你打算靠自己来处理他吗?”八幡海铃皱起眉头,“那丰川同学那边你又怎么说?这毕竟是她的父亲。”



“说实话,那种人就算死了也不会有人在意吧?”所有人都没有预料到祐天寺若麦会在这个时候开口。初华第一次在若麦脸上看到如此激烈的表情,若麦瞪着男人死不瞑目的尸体,她的脸因鄙夷和不屑而扭曲了,“那种人……那种渣滓也可以被称为父亲也未免太过荒唐了。那种人根本就不配!”



“但这不是你我可以决定的。”海铃冷静地反驳道,“破坏尸体可是犯罪。”



“是的,那是犯罪,破坏尸体是错误的。”初华放下手机,站起身,她不知道自己的脸上是怎样的表情,大概不是什么适合偶像的表情,但她也不需要什么适合偶像的表情。苍蝇又一次从男人的身体上起飞了,盘旋着,摇摆着,一头扎入了门外的黑暗之中。她看着那身着红衣的一言不发的人偶,仿佛可以透过对方看到祥子在这小小巢穴里的生活,“但是你应该明白的吧,海铃,在这种事情上已经没有考虑对错的余地了。我想要保护小祥,因此我必须做有效的事情。这个结局是真实的,但是我们,不,我相信她应该配的上更好的结局。你呢?我知道这个时机很糟糕,但是你必须做出选择。”



“……”海铃眯起眼睛,良久,她叹了口气,“毕竟是我要堵上职业生涯的乐队,和命案粘上关系也确实很麻烦。但是我得警告你,如果警察盘问到我的头上,别指望我会撑多久。三角同学你最好能够想出一个好一点的解决方案。若麦子你呢?”



“我还以为我刚刚的态度已经够明显了呢。”若麦又一次换上了她的那副营业微笑,仿佛什么也没有发生过一般,“要是这么邪恶的坏事不带上喵梦亲的话,我可是会记恨你们一辈子的哦。”



“好吧,那就这么定了。”初华努力按压下胃中的翻涌,屋子里没开暖气,桌上的残羹冷炙上流淌着一层令人反胃的油光。她又看了一眼男人那双无神的眼睛,一股寒意伴随着反胃爬上了她的脊背,可是没有退路了,“我们四个人就是共犯,在这件事上是共同保守秘密的一丘之貉。”她看了眼手机,接着看向睦,“是灯的短信,她们找到小祥了!”



\newpage



她听到脚步声,转过头,若叶睦那双与自己无比相似的金色眸子看着自己。她皱起眉头:“怎么找到我的?”



“灯告诉我的。我推测你之后会来这里,我们以前总是来这里。”若叶睦在她身边坐下。对方心照不宣地没有问她为什么在桥上,她也心照不宣地没有提起“初华……很担心你。”接着,她犹豫了一下,“我也一样。”



“这里的光污染比较少,可以看到星星,就是云有点多。”她皱起眉头,初华……“你是来拉我回去的,对吧?”



“祥……你应该休息。”



“是啊,医生也这么说。”她舔了舔嘴唇,她突然觉得这场面有点可笑,睦就像在对一个任性的小孩子说话,但她的确是一个任性的小孩子,“但我不知道。也许,灯是对的。我的确是有点迷失了。但是……”



“祥,你累了。”睦斩钉截铁地说道,“你已经坏了。”



“你说话还是一如既往地高高在上呢。”祥子咬住嘴唇,“之前找素世的时候也是因为这个没有收到好眼色吧。”



“抱歉……”她看着睦害怕的表情,心中一紧,不由得皱起眉头。你又让她伤心了。



“……别说那种话了。”她咬咬牙,伸出手,轻轻握住睦的手,“睦实在不会说谎,那又不是你的错,你不应该和我道歉。是啊,我大概是累了。我也的确坏了。但是,接下来怎么办呢?”



“所以……祥……回去休息。”睦吐出每一个字都仿佛需要极大的努力。是了,她是知道的,睦是比灯更加不善言谈的人,但是睦却依然说了下去,“坏了就修好。然后……继续前进。祥还不可以停下。那是你,不,我们一起决定的事情。在那之前,我都会在祥的身边。”睦轻轻用力,想要把她拉起,“走吧,我们回去。”



“……你变了啊,睦。”一股怪异的感觉堵住了她的喉咙,她摸了摸自己的脸颊,她突然意识到自己在流泪,可她并不知道为什么自己在流泪。真是可笑,原本已经以为自己舍弃了那种软弱的。实际上感觉并没有预想的那么糟糕,倒不如说轻松了许多。她试着站起身,却脚底一软,腿使不上力气,是因为饥饿还是因为单纯地躺了太久呢?睦已经在她摔倒前一把拉住了她:



“你也一样,祥。”睦转过身,蹲下身子,向祥子露出自己的后背。祥子看着眼前的睦,突然想起来小时候,也是在这里,也是在相同的夜空下,也是相同的姿势。当时的自己处于睦的位置,自说自话地背负起了对方。而现在,一切都被反转了过来。她想要说些什么,却意识到已经没有什么可说的,于是顺从地俯下身,把自己的重量压在对方的身上。然后睦站起身,一言不发地把她背起,向着山下走去。



\newpage



八幡海铃看着三角初华扛着男人下了车,深一脚浅一脚地向着不远处的公园走去。她看了看驾驶座上的祐天寺若麦,扬起眉毛:“我可不知道你有驾照。”



“我没有。”看到海铃脸上惊讶的神情,若麦露出了一个得意的笑容,“好吧,其实我有,但是我的年龄是谎报的。怎么样?喵梦亲的又一个小秘密,有没有觉得我更加神秘了呢?话说回来,初子真是坏心眼啊。”



“你是在说她利用你的车来运送尸体,借机把我们全都拉下水的行为吗?”海铃耸耸肩,“我无所谓的,她毕竟是演艺圈摸爬滚打出来的人,有这样的心机倒也是预料之内。倒是你,若麦子,你不也挺坏心眼的吗?明明是早有准备却装作不期而遇,你进门的时候手机根本就没开录像功能,在那么暗的环境里基本上一眼就会被发现吧?”



“哦呀哦呀,还是躲不过海子你的火眼金睛。这不是很好吗?用绝对的共犯关系捆绑在一起的乐队,一辈子的承诺没有这种重量也只是纸上谈兵罢了。”喵梦眯起眼睛,“你不好奇吗?我为什么当时会说那种话?”



“如果你不想说的话我就不会问。”海铃闭上眼睛,“这是我的职业道德。”



“也就是说你其实还是想知道对吧?”喵梦嘟起嘴,“真狡猾,我可不喜欢这种给人把柄的事情。不过也没什么,只不过是看到和自己家一样的混蛋老爹有点共情罢了。喵梦亲我呢,不像小祥子那样有着好的出身,小时候就过惯了苦日子,学也没上几天就退了,一直以来都是为了自己活着。但是看到小祥子这样要强而且执着的人,都忍不住想要为她喊几声加油。”喵梦舔了舔嘴唇,玩味地笑了,“呀嘞呀嘞,真的是被她耍的团团转呢。”



“……你觉得,她们真的动手了吗?”



“你是在说初子和睦子对吗?”



“在我到达之前,她们就已经进去了很久了,我进去的时候,丰川同学的父亲就已经那个样子了……”



“我觉得不太可能哦。”



“为什么这么说?”



“喵梦亲我虽然说不上见多识广,但也算得上身经百战。你觉得那两个人在今天之前杀过人吗?没有吧?第一次杀完人的人不可能会那么冷静,而是会紧张的要死,手上全都是手汗。那两个人的反应虽然有些惊讶和紧张,但是没有那种质疑自我的表现。说话也很流畅,除非是天生的反社会人格,否则不可能那样的。”



“但那也只是反应——”



“而且,你在下楼的时候也摸了那具尸体了吧?”若麦自顾自地继续说着,“喵梦我恰好对法医学略知一二。尸体温度有点冷,大概只有三十度出头。刚刚死的人不会那么冷的,起码得有四五个小时了。而且在我们进去的时候他的脖子那里已经有了尸斑,那是死后半小时之后才会有的哦。我们可没在那里待那么久哦。他额头上虽然有受到击打的痕迹,但是没有严重的淤血,更像是因为喝多了摔倒导致的外伤。从面色的青紫色和呕吐物来看,更像是酒精中毒导致的呼吸系统麻痹。怎么样?现在心服口服了吧?初子和睦子绝对不可能在当时杀人的。”



海铃眯起眼睛,总是感觉有些违和感,但是却不知道为什么:“……但愿如此。”



\newpage



他们和她说这种情况叫做共情泛滥。他们告诉她这种情况是一种心理障碍。他们和她说这种情况十分少见,但是过去也有过同类病例的记载。但她知道,事实并非如此。



从始至终,她的心都只会与一人共鸣。



不管能否见面,不管相隔多远,她的心都随着祥的心而跳动。祥微笑时她的心也跟着欢喜,祥落泪时她的心也一道伤痛。她不曾问过祥是否有过相同的感受,但是她只要和祥目光对视就会明白,她们是一样的。她不需要聆听便可以知道祥子的痛苦,她不需要看到便可以了解祥子的迷茫,她不需要开口便可以明白祥的恐惧。恐惧被同情,恐惧被遗忘,恐惧悲伤,恐惧恐惧本身,恐惧被爱,恐惧死亡。她知道祥因为恐惧而从医院逃跑,她知道祥因为迷茫而在千登世桥无意识地起舞,她也知道,祥因为痛苦而无法沉沦,保持着骄傲。不应产生关联的双子,不应共享命运的半身,这就是她们的存在方式。



直到祥自作主张地把一切不幸揽到自己身上为止。而她,软弱的她,无能的她,如同人偶一般缺少自我的她,就这么理所当然地承受了一切的幸福。



这就是若叶睦的罪。



她喘了口气,领口早已被汗水浸透,她并没有预料到背人会这么累,她也没有预料到自己居然还可以行走。祥的身体并不温暖,但是两人贴在一起时的热量足够让她抵抗逐渐呼啸起来的西风。她咬咬牙齿,想要继续向前,祥子的声音从她身后传来:“够了,放下我吧,我可以自己走。”



她没有回答,只是用力把祥子往上推了推,确保对方的状态依然稳固,然后迈开腿。从耳边传来了一声微不可闻的叹息。祥子没有再说话。她迟疑了一会儿,开口了:“祥,你找到了你想要的东西了吗?”



“……你已经很久没有问过我问题了呢,睦。”祥的头发缠绕在她的脖子上,有些发痒,“不过不用担心,我已经不会再逃跑了。”



祥并没有回答她的问题,但是这个答案足够让她满足。于是她再一次闭上了嘴,接下来的路程便在两人的无言中度过。过了许久,久到若叶睦几乎以为这是她们这辈子最后一次谈话的时候,耳旁终于再一次响起了祥子的声音:“对不起,睦。”



\newpage



三角初华躺在草地上,手机屏幕亮了起来,她转过头,看向手机,睦已经把小祥送回了医院,灯也让小祥的状态稳定下来了,这样就好。她转过头,重新看向天空。男人的尸体就在不远处,这样处理的话就算被发现也不会引起小祥的注意。对此她有足够的信心,只是……



晚秋将尽。今天的天气并不是很好,但是云层已经被傍晚的西风吹散了不少,公园周围的光污染也没那么严重,可以看到星星。此时已近九点,东京,这座容纳了1400万人的庞大都市,进入了夜晚。争红斗艳的霓虹灯,熙熙攘攘的车水马龙,叽叽喳喳的俊男靓女,这些常见的光景在这座小小的公园里都无迹可寻。唯有萤火虫的微光与猫头鹰的哀嚎与她为伴。话说回来,自从来到东京之后,她大多数情况下都满足于星象馆提供的表演,也很久没有看到过真正的星星了。她举起手,伸向天空,上一次一个人看星星是什么时候呢?对了,是在岛上的时候,那时的她如同现在一般思念着祥子,那时的她如同现在一般孤身一人。看着那些来自过去的遥远天体的光芒,她却丝毫感不到丝毫的温暖。秋风吹过,她打了个哆嗦,天气已经不知何时变冷了。寒气渗入了她的四肢,让她身体僵硬起来。不要紧的,自己做的事情是有效的,她对自己说道,这是唯一可以帮到小祥的方法。



不要动摇。



她的脑海中又一次浮现出了科特柯本那首熟悉的歌,她戴上耳机,轻声哼唱起来:



Underneath the bridge, tarp has sprung a leak

And the animals I've trapped have all become my pets

And I'm living off of grass, and the drippings from my ceiling

It's OK to eat fish 'cause they don't have any feelings

Something in the way, hmm-mmm

Something in the way, yeah, hmm-mmm

Something in the way, hmm-mmm

Something in the way, yeah, hmm-mmm

、



星光闪烁,她仿佛置身于浩瀚无垠的宇宙空间之中。这里依然如同她儿时一般,空无一人,只有那些遥远的冰冷的火球在散发着光与热,可是那些光却永远无法照亮她的身体。在这里她可以享受孤独,在这里她永远停留在黑暗中,在这里她只与自己的心一道。怀念着这久违的感触,她不由自主地露出微笑:



“哈哈。”

\end{document}
