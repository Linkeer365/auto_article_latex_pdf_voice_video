\documentclass{article}
\usepackage[utf8]{inputenc}
\usepackage{ctex}
\usepackage{url}

% support Chinese Chars %
\newfontfamily\urlfontfamily{FandolSong-Regular}
\def\UrlFont{\urlfontfamily}

\title{【初祥】追随:终章——排气}
\author{sandman}
\date{20231021}

\maketitle
\url{https://www.pixiv.net/novel/show.php?id=20884836}
\newpage

\begin{document}
\CJKfamily{zhkai}

% : comm which is only poems edit in-use %


\Large

在她回到医院之后,睦和她心照不宣地隐瞒了她的行程。她们都没有提及她在桥上的举动,也没有说她为什么要逃跑。医生似乎已经习惯了这种情况,于是也没有打探。在静养了三天之后她就被宣判允许出院。在她出院的前一天她收到了来自父亲的短信:



小祥:



当你看到这条信息的时候我已经离开了家。我知道自己不是个好的父亲,甚至不是一个好的父亲。如果不是因为我,你原本不必如此。你有着出色的才能,坚韧的性格,高贵的品德,我不希望这一切因为我而埋没。经过深思熟虑之后,我想,也许离开你对于我们两个都是比较好的选择。我很抱歉过去对你造成的伤害以及让你经受的一切不幸。我在家里的桌子上留下了一个信封,里面有着我用来接收抚养费的银行卡,密码是你的生日。我知道你可能还记恨着她,但我希望你能明白她也有她的苦衷。之前其实我偶尔打零工也赚了点钱,刨去还债用的还有大概两万多,我也放在了那个信封里。忘掉我,好好生活。记住——



你应当获得幸福。



她打开房门,眯起眼睛。金黄色的夕阳透过对面的建筑物反射进房间,照亮了客厅中心塞满了酒瓶的大袋子。男人似乎在临走前还是决定留下一个体面的后续,收齐了瓶子,并且带走了湿垃圾。衣柜里那套发白的西装和门口那双不知道放了多久的皮鞋不见了。屋子里似乎显得前所未有的寂静,接着她意识到男人终于在离开之前关掉了电视机。她叹了口气。这就是结局。养育了她十五年的男人就这样自说自话地消失了。她脱下鞋子,走进房间。客厅的小桌子上放着一个旧信封,大概是男人过去用来取款时拿到家里来的。信封里正如男人所说,放着一笔现金和一张银行卡。她坐在桌前,轻轻抚摸着那张银行卡。真是奇怪,她的心里什么感觉也没有。



她原本应该感到悲伤的。



小祥,来吧,跟着我一起弹。啊,你可真是个小天才,你喜欢钢琴吗?好吧,那就跟我一起学吧。



她原本应该感到失落的。



不许你那样看我!你凭什么那样看我?就因为你有才华就可以这样看你爹?我养你这么大也只是养了个白眼狼!



她原本……应该感到感激的。



小祥,你要记住,人穷不能志短。



“终于……结束了。”



她轻笑了起来,真是荒唐,自己的父亲失踪了,却丝毫没有怀念和报警的想法。也许她一直以来都在期盼着这个结局也说不定。她站起身,拨通了电话:“喂,睦。你能来一趟我家吗?帮我个忙?



“对的,我打算打扫一下屋子。我也想买点新家具了。”



\newpage





{\centering\section*{终章——排气}}





她掏出钥匙,打开防盗门,走进屋子,把帽子挂在门口的挂钩上。她把墨镜从头上摘下,随手扔在门口的高柜上。接着掏出手机,是经纪人的短信:Ave Mujica的专辑收录快要进入尾声了,周末还要去录音棚。她划掉信息,打开选歌界面,轻轻点击了几下。约翰尼·凯许沙哑的嗓音便开始从她的耳机中传来:



I hurt myself today

To see if I still feel

I focus on the pain

The only thing that's real

The needle tears a hole

The old familiar sting

Try to kill it all away

But I remember everything



她没有开灯,作为观星者,她的双眼早已适应了黑暗。借着窗外的纸醉金迷的余光,她一个人驾轻就熟地穿过昏暗的房间,踩过散落在地面上的缴费单。走进厨房。厨水池里依然泡着几天前用完的锅碗瓢盆,她眯起眼睛看了看在黑暗中泛着油光的炖锅,决定继续无视它的存在。她走到冰箱前,取出一听零度可乐。她知道所谓的“零度”只是聊以自慰的宣传语,而自己的嗓子其实根本无法负担碳酸饮料带来的刺痛,但这一切都无所谓了。她走出厨房,进入客厅,进入这间孤独,冷寂,阴暗的死亡之所,坐到窗前。窗台上摆着加缪的《局外人》,她随便拿起书,翻了两三页:



我想我只要一转身,事情就完了。可是整个海滩在陽光中颤动,在我身后挤来挤去。我朝水泉走了几步,阿拉伯人没有动。不管怎么说,他离我还相当远。也许是因为他脸上的陰影吧,他好像在笑。我等着,太陽晒得我两颊发烫,我觉得汗珠聚在眉峰上。那太陽和我安葬妈妈那天的太陽一样,头也像那天一样难受,皮肤下面所有的血管都一齐跳动。我热得受不了,又往前走了一步。我知道这是愚蠢的,我走一步并不能逃过太陽。但是我往前走了一步,仅仅一步。这一次,阿拉伯人没有起来,却抽出刀来,迎着陽光对准了我。刀锋闪闪发光,仿佛一把寒光四射的长剑刺中了我的头。就在这时,聚在眉峰的汗珠一下子流到了眼皮上,蒙上一幅温吞吞的,模模糊糊的水幕。这一泪水和盐水搀和在一起的水幕使我的眼睛什么也看不见。我只觉得铙钹似的太陽扣在我的头上,那把刀刺眼的刀锋总是隐隐约约地对着我。滚烫的刀尖穿过我的睫毛,挖着我的痛苦的眼睛。就在这时,一切都摇晃了。大海呼出一口沉闷而炽热的气息。我觉得天门洞开,向下倾泻着大火。我全身都绷紧了,手紧紧握住槍。槍机扳动了,我摸着了光滑的槍柄,就在那时,猛然一声震耳的巨响,一切都开始了。我甩了甩汗水和陽光。我知道我打破了这一天的平衡,打破了海滩上不寻常的寂静,而在那里我曾是幸福的。这时,我又对准那具尸体开了四槍,子弹打进去,也看不出什么来。然而,那却好像是我在苦难之门上短促地叩了四下。



她恼火地把书扔到一边的柜子上,在那里还杂七杂八地散落着几本遭遇了相同命运的倒霉蛋。她的手机振动起来,她拿出手机,发现是真奈的短信。真奈说她知道初华最近嗓子状态不太好,希望她可以多休息,不要勉强自己,接着问她周末有没有安排,要不要一起出去玩玩。她皱起眉头,把手机扔到一边。



What have I become

My sweetest friend

Everyone I know goes away

In the end

And you could have it all

My empire of dirt

I will let you down

I will make you hurt



她仰起头,把易拉罐里的可乐一饮而尽,把罐子随手扔到了厨房的回收袋里。明天要收可燃垃圾,真麻烦。她捡起手机,转过身走进卧室。卧室里的东西并不多,一张单人床,一个大衣柜,一座书柜,还有窗户边上的那架天文望远镜。那是她在sumimi第一次公演之后用余下的报酬买给自己的礼物,现在那只是一坨盖了块布子积满了灰的铁块。她已经忘记自己上一次用那东西观星是什么时候了,她甚至想不起来装目镜的箱子被自己扔到哪里去了。虽然觉得自己应该掀开布子看看,但是也没有什么心情。她看着床上卷成一团的被单,叹了口气,放任自己摔进那团带着些许馊味的柔软。她又一次打开手机,随便刷了刷新闻,有一股寒流将在下周抵达东京,会有明显降温,除此以外没什么引起她兴趣的,于是再次把手机扔到一边。应该洗个澡换身衣服的,算了,无所谓了。她闭上眼睛。冥想,放空自己,让炽热的大脑冷静下来,不要在意那些——



她睁开眼,接着苦笑起来:“果然,你还在啊。”



I wear this crown of thorns

Upon my liar's chair

Full of broken thoughts

I cannot repair

Beneath the stains of time

The feelings disappear

You are someone else

I am still right here



她坐起身。站在墙角的男人一如她最后一次看到对方时的样子。男人稀疏花白的头发如同杂草一样贴在头上,脸因为饥饿和酗酒浮肿起来,右眼周围还有着暗红色的尸斑。他的眼窝深深地凹陷下去,阴影吞噬了他的瞳孔。男人依然穿着那件破旧的毛衣,领口处沾满了呕吐物和白沫。佝偻着背,双腿微曲,神色木然,如同一只刚刚睡醒的黑猩猩。虽然看不到男人的眼睛,但是初华却可以知道男人是在看着自己的。男人举起手,无声地指向初华,接着指向自己,随后他的表情扭曲起来。愤怒和怨恨爬上了他的五官,他的眉头拧紧了,如同粗糙的麻绳,他张开嘴,什么声音也没有发出,可是初华却已经听到了他的大喊。初华咬了咬牙,低声说道:“对的,我不后悔。”



男人抱紧双臂,接着摇了摇头,冲着她向下伸出大拇指。



“我做了有效的事情,不正确,但是有效。“



男人生气地用力跺脚,再次指向她。



“我已经说的很清楚了。你只是我脑中的幻觉。”



男人的手颤抖起来,接着他用力拍打自己的胸脯,指向自己。



“你什么也不明白。我拒绝你的审判。”



男人的身形变得模糊不定了,他的身体开始从脚部向上瓦解,但他依然恶狠狠地发出了最后的诅咒。



“啊,没错,地狱见。”



What have I become

My sweetest friend

Everyone I know goes away

In the end

And you could have it all

My empire of dirt

I will let you down

I will make you hurt



她瘫倒在床上,喘着粗气,心脏努力地想要挣脱胸腔的束缚,她几乎想要撕开自己的胸脯,把那颗躁动的心挖出。身下的床单和被罩散发出汗水的酸臭味,地板上到处都是灰尘聚集成的毛团,不远处洗手间的水池当中满是黄色的水垢。这是她自己造成的,这就是她选择生活其中的世界。别去理他,他只是你的幻想,他已经死了,你看着他死的。她深呼吸了几口气,努力想要让自己平静下来。想想开心的事情,想想Ave Mujica,想想小祥。



她用力擦干自己脸上的咸水,这就是她选择的结局,哪怕失去了站在她身边的资格,只要能够看到她再次微笑——



“好脏啊。”



\newpage



她看了看表,距离这一次排练结束还有半个小时的时间。丰川祥子拍了拍手,叫停了她的队友们:“各位辛苦了,我们先休息十分钟。过一会儿再来。睦,你的速弹越来越厉害了。若麦,海玲,一如既往地Pro。初华你今天状态比前几天好多了,继续顺着这个势头。”她停顿了一下,接着顶着若麦惊讶的眼光开口了,“下周就是平安夜了,如果各位没有什么安排的话,我觉得我们可以补办一下Ave Mujica的初次庆功宴,一起好好休息一下。”



“真的吗?好耶!最爱你了,祥子!”若麦兴奋地大叫道,“可以请我们吃和牛吗?我好久之前就想吃了!”



“我的话平安夜晚上有安排,但是中午还是没问题的。”海铃抱起双臂,转过头,“睦和初华怎么说?”



“我没问题。”睦点点头。



“那么初子呢?”



“……抱歉,各位。”初华脸上挤出一个苦笑,双手合十,“我那天已经有安排了,是sumimi的商务活动。所以不能一起去。不过你们不用管我,就当替我吃了就好。我事后一定找机会补上。”说完,她转过身,踉踉跄跄地向着后台跑去。祥子看着她的背影,眯起眼睛。



她一定以为自己掩饰的很好。



自从祥子出院已经过了接近两个月的时间。Ave Mujica举行了她们的第四次演出,第一张专辑也已经开始录制。TGW物产发来了一封邮件,以戏谑的方式祝贺她的过家家一切顺利,并且表示既然那个男人已经不在了,而祥子也证明了自己的能力,他们随时欢迎她回去。她删掉了那封邮件,不置可否。她曾经试着给那个男人打过电话,但不出意外,手机关机了。第二天她在自己家里醒来的时候发现睦送给自己的人偶的裙子被人用剪刀剪了个稀碎。她叹了口气,那些老人们一定是老糊涂了才会觉得这种程度的威胁对她有用。她把人偶的衣服自己用线缝好,结果因为生疏还扎破了两次手指。不过整体来说,一切都在往好的方向发展,除了一件事。



初华的脸色变得愈发苍白了,她已经忘记自己上次看见对方真心的笑容是什么时候。初华虽然在演出的时候还可以依靠底力应付,但是专业人士都听得出她的失误越来越多,嗓子的状态也每况愈下。她也曾经试着让对方休息或者询问对方发生了什么,但是每次对方都用假笑的城墙把她挡开。初华再也没提过和她一起回家的事,甚至就连去星象馆约会也失去了兴趣。她看着这一切发生,她知道每一个有眼睛的人都可以看出来初华不对劲,然而海铃只是表示她爱莫能助,若麦则暧昧地笑着,说每个人都有自己的秘密。只有睦一直一言不发。结论已经很明显了。



初华和她们合谋瞒着自己什么。



有人拍了拍自己的肩膀,她回过头,发现是睦:“姐姐大人,结束之后,和我来。”



“好的。”她点点头,虽然不知道睦在想些什么,不过可以感觉到对方明显比之前开心了不少。尽管她已经告诉睦没有必要在休息的时候也叫她姐姐大人,但是睦似乎对这个称呼很满意。于是她也顺水推舟默许了。初华已经从后台回来了。她看向初华,然后,两个人的眼神在空中相遇了。



那只是一瞬间的事情,可她看清楚了,初华美丽的紫色眼睛里满是痛苦和不舍,如同在寒风中摇摆不定的篝火。初华别开了脸,匆匆走上台。在那之后没有再说哈。排练结束之后,初华一声不响地和若麦一起坐车离开了。她给海铃叫了一辆专车,只留下她和睦两个人:“说吧,你知道些什么?”



睦闭上眼睛,吸了口气,然后开口了:“10月25日,你住院的第二天,我和初华去了祥的家。”



\newpage



她走上楼梯,初华一言不发地跟在她身后。她走到右手第一间房门口,停下脚步。她掏出事先配好的钥匙,打开房门。初华紧跟着她走近屋子。在房间的正中央,失去了气息的男人歪坐在那里。房间里低下的温度麻痹了人的嗅觉,把酸败的腐朽气息锁在了男人的尸体周围。睦的脑子里一片混乱,在她想要动手之前,男人就已经以这种意料之外情理之中的方式无声地死在了这间小小的陋室里。曾经的音乐新秀,备受瞩目的钢琴家,以这种平平无奇、不负责任的方式结束了自己的一生。他的心力衰竭、饱经生活折磨的女儿则从医院里逃走,此刻依然下落不明。



不知不觉间,她握紧了拳头。



“怎么办?小睦?”初华的脸隐藏在阴影里,让睦看不穿她脸上的表情。“已经结束了。”



“……不应该这样结束。”



“是的,不应该这样结束。”



“她配得上更好的结局。”



“那么,我们就需要编造一个故事。”初华看向睦。



\newpage



“没事的,初华。我确认过了,只有海铃跟着我。”



八幡海铃眯起眼睛:“难道说……”她看向睦,“你从一开始就发现了,对吗?”



\newpage



初华蹲下身,男人的手机就放在他右手边不远处的茶几上,她拿起手机,没有未接来电。八幡海铃开口了:“那么的话……你打算靠自己来处理他吗?”八幡海铃皱起眉头,“那丰川同学那边你又怎么说?这毕竟是她的父亲。”



“说实话,那种人就算死了也不会有人在意吧?”



\newpage



“我们四个人就是共犯,在这件事上是共同保守秘密的一丘之貉。”初华看了眼手机,接着看向睦,“是灯的短信,她找到小祥了!在千登世桥!小睦,你最了解她,你去把小祥带回医院。灯说她虽然和小祥分开了,但是她知道小祥离开的大致方向。我把信息转发给你。”



“初华你们呢?”



“我们来处理这边的事情。不用担心,我会按照之前商讨的计划进行,而且喵梦亲应该有车,这样又解决了一个问题。”



她点点头,既然初华这么说了,那她只好相信对方。况且,祥还需要她。她转过身,向着门口走去,初华的声音却又一次响起了:“对了,小睦?”



“怎么了?”她回过头,初华的双眼在黑暗中如同星辰般闪烁,仿佛要把她刺穿:



“你就算没有罪孽也可以作为人类活下去。”



她愣了愣,然后闭上眼:“谢谢。”



\newpage



丰川祥子走出餐厅。睦似乎又在吃饭的时候睡着了,她确实太勉强自己了,不过海铃会送她回家,不用担心。若麦则提前离开了,据她说她今天还得直播营业,毕竟是平安夜。天空已经飘起了细细的雨点,她撑起雨伞,看着摩天大楼上的巨大液晶屏在黑夜中闪现出异样的光彩,看着忙碌的车流嘶鸣着吐出污浊的废气,看着来来往往的行人说笑喧闹,叹了口气,一个人走向电车站。



她下了车之后又走了十五分钟。她原本可以打车过来的,但她不想花多余的钱,她也需要一点时间独处来思考和谋划。她走到那座高耸的公寓楼前,按响了门铃:“初华,是我。”



“小祥?你怎么来了?”初华的声音听起来慌张而又疲惫,她捏紧了手中的背带,果然……



“作为总指挥有必要来看望一下我生病的主唱罢了。怎么?不欢迎我吗?”



“没有没有。”初华解开了电子锁,她走进电梯,看着指示牌上的数字变换,心跳也逐渐加快。她走出电梯,初华给她开了门。她径直越过地上的缴费单,穿过垃圾袋构成的迷宫,走进这间黑暗,脏乱,冷清的公寓。下定决心,她告诉自己。她坐在窗台上,转过身,开口了:



“睦已经告诉我了。”



初华先是一愣,接着露出了如释重负的表情:“是吗?果然啊……”



“但那并不是全部的真相。”祥子打断了她,“在前一天晚上,睦问过你有没有去过我家,那时你回答没有。睦并不觉得你在说谎,但是第二天你和她一起去的时候,你是第二次去我家,对吗?”



初华低下头,她的肩膀剧烈地颤抖起来,她想要说些什么,声音却被哽咽堵住,消失在了嗓子里。终于,她嗫嚅着开口了:“对不起,小祥……”



\newpage



“不用,会有人来接。”睦摇摇头,“初华,去过祥家了吗?”



初华微微一愣,在她和睦那场并不能说友好的交谈之后,两人都心照不宣地对当天的谈话内容避而不谈。她并没有预料到睦会主动提起此事:“……还没有,我想还不是时候。”



睦先是沉默了一小会儿,接着再次开口了:“你应该去一次。”



她做了个梦,梦中她回到了那个她出身的小岛。当她醒来的时候她发现自己的枕头已经不知何时被泪水浸透。她爬起身,穿上衣服,抓起睦给她的备用钥匙,出了门。她首先去了银行取了笔现金,然后在机器上买了单程电车票,随后坐上了前往赤羽的电车。



她一个人去了那座屋子。几只硕大的蟑螂在她推开门的时候从门上跑了出来。她不耐烦地把这些顽强的生灵拨到一边,走进屋子。虽然是在白天,没有开灯的出租屋里依然一片昏暗,但她的眼睛瞬间就看清了门口堆积如山的易拉罐和酒瓶。虽然瓶瓶罐罐不少,但是地板竟然意外的干净——主人显然花了不少力气来维持这个小家形式上的整洁。她抬起脚,小心地绕开地上的陷阱,走近客厅。在她的左手边,几张薄薄的毯子铺在地上,一条简易的帘子组成了居住者与外界唯一的阻隔。右手边的柜子上,一个身着红色洋装的人偶正盯着她露出诡异的笑容,人偶的衣服虽然有些磨损,但是依然整洁。显然人偶的主人依然对它关照有加。她从来搞不懂为何小祥对那个人偶情有独钟,小祥说它很像小睦,但是在初华看来睦的内心可比人偶炽热无数倍。她的目光扫过地上堆积的垃圾,茶几上散落的餐盘,闪烁不止的电视机,最终停留在了紧闭的窗帘前匍匐着的男人身上。



男人的呼吸很浅,仿佛失去了意识一般。初华还记得男人,在过去,在她和小祥第一次在岛上相遇的时候,男人就在小祥家的屋子里。男人是个开朗的人,喜欢音乐,爱开玩笑,对自己的老婆和女儿都一往情深。她还记得男人牵着自己和小祥的手在山上奔跑,自己教会男人用叶子吹口哨,男人给自己吹了一首舒伯特的小夜曲。那是多么欢快的时光啊!那是多么美好,转瞬即逝,一去不返的她和小祥的少女时代啊!此刻的男人在地上瘫成一团烂泥,从喉咙里不时发出可怕的呼噜声。男人的眼眶和四肢都因为缺少运动和过度饮酒浮肿起来,皮肤也皱皱巴巴的,失去了弹性。地上的酒瓶在诉说着男人无可救药的堕落,而空气中断断续续的喘息声则是男人沉沦的挽歌。三角初华闭上眼睛,这就是小祥身处的世界,空气当中弥漫着尘埃,霉菌,以及垃圾腐败的臭味,如同甜腻的泔水,令人作呕。孤身一人躺在这坚硬的地板上,与蟑螂,呕吐物,火车的汽笛声为伴,甚至找不出第二套可以更换的私服,更不要说在乎自己的外貌。她的人生就要和这样的一个酒鬼一点点消磨掉,浪费在每天的打工,收拾垃圾,帮男人洗衣服,让男人活下去上。她曾经那么的才华横溢,意气风发,现在的她却栖身在这样一件狭窄,阴暗的斗室之中,和这个男人一样,一点点死去。她本不必这样的……她本不应这样的!



不知何时,三角初华的拳头握紧了。她华是个偶像,但在那之前她是个乡下长大的野孩子。她喜欢运动,喜欢看星星,喜欢爬树,喜欢游泳。她的视力可以轻易地在东京找到一颗四等星,而她的身体足够让她背着五公斤重的书包一口气跑完十公里。她伸出手,眼前的男人衰老而脆弱,只要她用自己的双手环绕住对方的脖子,一用力,他甚至没有反抗的机会,就会在睡眠中死去。



到底该怎么做?



男人的身体突然颤抖起来,如同失去了头的苍蝇,如同垂死挣扎的鱼。一种可怕的呼噜声从男人的喉咙中爬出,起初是低沉的口哨,接着变成了粗重的汽笛,最后化为一台濒临散架的管风琴,奏起了不和谐音。男人咳嗽了几声,吧嗒了几下嘴,睁开眼。他那双浑浊的金色眸子盯紧了眼前的人,眨了眨,眯起眼睛。困惑在他那双深深凹陷进去的小黑窟窿里荡漾着,他又咳嗽了几声,终于眼睛逐渐明亮起来:“是初华啊……你……什么时候来的?”



“丰川叔叔……”她舔了舔嘴唇,俯下身,坐在对方面前。



“算了,你为什么会在这里?”男人揉了揉自己的脑袋,颤颤巍巍地伸出手,想要抓住身旁一罐开着的啤酒,却手一抖,将其打翻在地。男人不满地砸了咂嘴,伸出手重新从身边的箱子里拿了一罐新的打开,猛地灌了一大口,“小祥呢?她昨天晚上怎么没回来?”



“小睦告诉我的这个地方。”她眯起眼睛,鄙夷地打量着男人。他的女儿正在医院里因为疲劳和营养不良昏迷不醒,而他却依然在这里不以为然,一罐接一罐地喝着酒,甚至不知道他的女儿的状况。她咬住牙齿,“她因为营养不良和过度劳累病倒了,在医院里。”



男人先是愣了愣,接着低下头,看向自己手中的易拉罐:“我对不起她……她是个好姑娘,可是我却把她的人生都毁了。要是我当时争点气,她也就不用跟着我一起被扫地出门。她可是个十指不沾阳春水大小姐啊,在她上高中之前,她连饮料机都不会用。一夜之间,什么都没了。可她一点也没有气馁,还在那里安慰我……”他捂住脸,干号起来,“我没有用啊,让她跟着我受这份罪。我还记得她第一次找到工作的时候回到家里,眼睛里好像在发光。她和我说:‘爸爸,不用担心,您教导我人穷不能志短,我都记着呢。您看,我现在也可以靠着自己的双手自食其力,分担一点家里的经济负担。等到您找到了学生,生活稳定下来,我们就换个好点的房子,不在这种地方受气’。我看着她高兴,我也真心觉得高兴。可是我根本就配不上她的好心。我是个无可救药的酒鬼,软蛋,赌狗!她妈妈打给我的抚养费,我每次都想着存起来,等她上大学再用。结果每次回过神来,我都已经喝光了。现在她变成这个样子,我该怎么去和她妈交代……”



“叔叔……你有考虑过戒酒吗?”



“我试过啊,没有用的。”男人说着,把罐子扔到一边,又打开了一听新的,“我已经没救了,我就是这样的人。我也曾经想过只要我能够戒酒,我就会改头换面,做个好人,好好生活,好好对她。可是没有用。如果不喝这东西我根本活不下去。我当然知道喝这玩意是不对的,可我有什么办法?我每天醒过来,看着这样没用的自己,这样没有希望的日子。只要喝一点这东西,这一切就可以没那么难以忍受。”他抽噎起来,“都变成这种样子了,活着还有什么指望呢?”



“她很尊敬你。”她看着泪水从男人的面颊上留下,却只觉得恶心,就连公园里的山田起码有点自尊,“她之所以这么努力,就是因为相信了你的教导。她过去就和我说过,‘人穷不能志短’这句话是你教给她的。她也肯定不希望看到你这么放弃自己。你曾经是她的导师,我也曾经崇拜过你。”



“那都是过去的虚影罢了。我们其实都心知肚明,我完蛋了。”男人咳嗽了两声,“我是个混蛋,咳咳,我对不起她。你还能指望我怎么样?你难道觉得我不痛苦吗?你难道觉得我愿意这样吗?”他激动地指向周围的瓶瓶罐罐,脸色变青了,“咳咳咳……这就是人生!她是个好姑娘,所以她在受苦;而我是个混蛋,所以我可以继续这样浑浑噩噩地活下去咳咳咳……”



男人的呼吸变得愈发粗重了,他的额头上布满了细密的汗珠。易拉罐不受控制地从他手中脱落,摔在了地上。她意识到了,男人正在体验酒精中毒的症状。男人挣扎起来,接着目光转向了不远处的地面上,不知何时被他扔到那里的手机。他笨拙地伸出手,想要够到它,却控制不住身体,摔倒在地。他抬起头,看向她:“初华……帮忙……手机……”



她转过身,缓慢地伸出手,抓起手机。她闭上眼睛,用力握紧那块冰冷的金属。男人的喘息依然在屋子里回响着,一只苍蝇在她耳边盘旋着,发出恼人的嗡嗡声。她睁开眼,转过身,把手机放在了距离男人两步之外的桌子上。



男人的眼睛因惊讶和困惑而张大了,他的五官被迷茫吞噬。接着他意识到了,愤怒扭曲了他的眉头:“原来……你来这里……就是为了这个吗?”



初华没有理会男人的话语,只是默默地看着他,如同看着一条臃肿的蠕虫在脱水之后的垂死挣扎。男人的脸因为缺氧和兴奋涨红了,白沫和呕吐物从他的嘴角流出,但他依然没有停止说话:“所以说……你是来审判我的?亏你还有脸去当什么偶像……是,酒鬼,人渣,垃圾,这些我都承认……可你又是什么?出卖青春供死宅幻想的商品?卖弄色相伪装成清纯的碧池?偶像兼杀人犯……我还真是开了眼界!”



三角初华走近男人,一把抓住对方已经失去控制的四肢,把对方扶正,让对方坐直。她严厉地看着男人,伸出手,指向门口。男人顺着她的手指,却只看到了门口橱柜上的人偶。三角初华开口了:“我可没有厚颜无耻到依靠自己的女儿养活的同时,在这里大言不惭地用自己的痛苦来为自己的堕落开脱。我可没有靠夸耀自己的苦难来博取别人的同情。我可没有心安理得地对自己重要的人的不幸视而不见!况且,”她凑近男人的耳边,“丰川叔叔自己不是说了吗:‘都变成这种样子了,活着还有什么指望呢?’”



她站起身,俯视着男人。男人的脸因为憎恨扭曲了。他想要伸出手,可是他的四肢已经没有力气了,他想要说些什么,但他的喉咙却发不出声音。他挣扎着,诅咒着,蠕动着。白沫从他的嘴角流出,污染了他的领口。正当她以为这就是结束的时候,男人的声音最后一次响起了:



“卡……柜子里……密码是0214……”她睁大眼睛,看着回光返照的男人,男人脸上的愤怒消失了,取而代之的是某种她几乎以为不会在对方脸上看到的平和,“祥……就拜托你了……”



然后,男人垂下头,停止了呼吸。



三角初华转过身,走出大门,离开了公寓,向着电车站走去。她的步伐歪歪扭扭,仿佛失去了灵魂一般,终于,她跪倒在地,发出无声的嚎叫,在她看来,这叫声几乎要撕碎自己的心脏。



这就是结局。



她杀了他。



她杀死了小祥的父亲。



她是个罪人了。



她回忆起男人在屋子里和自己用树叶合奏小夜曲的时光。她看着男人和小祥其乐融融,亲密无间,心里满是羡慕和祝福。而现在……



电话响了,她从回忆中抬起头,打开手机:“喂?这里是三角。你说什么?她从医院里失踪了?”



一辆列车从她身边的高架桥上飞驰而过,哐当作响,在三角初华听来,这是她世界粉碎的声音。



\newpage



初华蹲下身,男人的手机就放在他右手边不远处的茶几上,她拿起手机,没有未接来电。不远处若麦和海铃依然在争执着。睦看到了她的举动,没有说话。这样就好。擦掉手机上的指纹太过于招摇,不如让所有人看到她是如何在上面留下指纹的。她打发走了睦,因为睦不需要参与到这些事情里来。睦不需要背负更多。她和若麦麻利地脱掉男人的外套,给男人换上柜子里那套仅存的西装。她找到了男人所说的银行卡,接着告诉海铃自己从睦那里得知了密码。海铃帮她编辑完了短信。她确保每个人都参与到这件事里,这样她们就都知道她的罪恶感是因为破坏尸体。然后她和海铃一起把男人扶到车上,开往她每天晨跑时会经过的那座公园。



流浪者们已经习惯了这里来来往往的人。我要做的很简单,我不需要藏匿尸体,我只需要让尸体被以另外一种方式被发现。公园里多出了一个不知何时而来,不知何时死去,没有过去没有名字的人。这个人在角落里默默地死去了。公园里的长期住户对他一无所知。就算有人注意到了她的行为,他们也都会自然帮她掩饰。没有人报案,也就没有人认领尸体。男人就只是东京每天死去的几十个人当中的一个无名氏,彻底与祥子的生活断绝了联系。她算计到了一切,唯一没有算计到的,就是自己的罪恶感。



\newpage



“然后,我一直关注着新闻,每天也照常晨跑,直到第三天,古川告诉我说他的尸体被警察带走了。之后也再没有人来调查过。我就知道我成功了。”初华低着头,断断续续地说着,努力不让自己的哽咽打断自己的叙述。祥子的脸上阴晴不定,在整个过程中一言不发,只是听着她的叙述,“这就是全部。我杀了你的父亲。我根本不是你平时见到的那个样子。你爸爸说的对,我既不阳光,也不开朗,更不配做什么偶像。你不应该让我在你的身边的,我是个罪人,根本配不上你——”



“要来跳舞吗?”



“什么?”她惊讶地抬起头,发现祥子并没有在看她,而是在看着窗外的瓢泼大雨。祥子又重复了一遍:



“要来跳舞吗?去外面?”她转过头,站起身,看向初华,“我们已经很久没有跳过舞了。我想去外面和初华跳舞。”



“……在雨里吗?”



“不行吗?我们小时候就在雨里跳过吧?”没等她回答,祥子已经强硬地抓住了她的手,把她拉出门,顺手拿上了她的蓝牙耳机放进兜里,按了电梯。初华没有反抗,恐惧使她的手脚变软了,让她无法拒绝祥子。话说回来,她本来也拒绝不了祥子。她跟着祥子走进电梯,低着头,躲避着对方的视线,祥子叹了口气:“真是自我中心呢。



“本来谁能够在我身边就是只有我能决定的事情。你这家伙却只顾着你自己的想法,想要逃跑对吧?



“但是话说回来,我也没好到哪里去呢。”祥子轻笑起来,电梯抵达了一层,她牵起初华的手,引着初华走出电梯,初华抬起头,却发现祥子只是看着前方:“还有,刚才初华说错了一件事哦。在那天之前,我的父亲就已经死了。”



“小祥……”没等她说完,祥子已经转过身,把一枚耳机塞进了她的左耳,另一枚耳机则进入了祥子的右耳。祥子掏出手机操作了几下,耳朵里便传来了连接成功的确认音。祥子转过身,推开大门,率先跑入了雨中:



“来吧,初华,来跳舞吧!”



I'm singing in the rain

Yes, singing in the rain

What a glorious feeling

And I'm happy again



她迟疑了一下,也跟着祥子跑进了雨中。她的衣服立刻被雨水浸透,贴在皮肤上,如同一条条蛇,束缚着,牵扯着,纠缠着她。祥子身上的洋装也早已因雨水变得半透明起来,可对方却仿佛丝毫没有在意一般。祥子一把抓住她的手,转了个圈,接着松开她,跳起了轻快的芭蕾。初华也跟着对方的引导,一同舞动起来。水珠在她们脚下飞溅,布料在她们身边飞旋。祥子开口了:“初华,你还记得我小时候去你的岛上你告诉我关于星星的光来自过去的事情吗?”



“小祥,你刚刚说你的父亲……”



“我的父亲就像那些星星一样,其实已经死了。剩下的只有过去的美好的虚像,以及那具失去了灵魂的行尸走肉。他并不是被你杀死的。而是被自己,被过去杀死的。”



I'm laughing at clouds

So dark up above

The sun's in my heart

And I'm ready for love



“你刚刚说自己并不是那样阳光开朗的人。这种事情我早就知道了。我当然知道我其实在你身上寻求的只不过是一个过去的虚像。我并不了解真实的你。但是我其实也一样,不是吗,初华?我在和你见面的时候隐瞒了自己的现状,想要让你相信我依然是过去那个自信的,温柔的,不谙世事的大小姐。如果初华说自己罪孽深重的话,我在这种事情上不是和初华一样吗?”祥子再一次牵住了初华的手,初华顺从地在对方的引导下转了个圈。两人猛地分开,又被牵住的手拉住,再次靠近了彼此。



Let the stormy clouds chase

Everyone from the place

Come on with the rain

I have a smile on my face



“况且,初华你告诉我这些事,你希望我怎么做呢?我可不是那种忘恩负义的人。你说的事情,就作为我们两个人共同的秘密吧。”祥子贴近她的身体,轻轻抱住了她,抚摸着她的头,“所以说,初华你不要再一个人承担了。在这件事上,我们可是一丘之貉的共犯啊。”



“小祥……”她再一次无法控制自己,眼泪决堤而出,“我是个嫉妒心重的胆小鬼!我来了东京之后一直想要见到你,可是我一直不敢去见你,因为害怕你知道我是个这样的软弱的人。”



“嗯。”



“我嫉妒灯!我知道她和你相互理解,在我不在的时间里你们是无话不谈的知己。”



“我知道。”



“我也嫉妒睦!我知道她和你亲密无间,心意相通,是无法割舍的半身。”



“我知道。”



“可是,即便如此……即便如此,我也想要在你的身边支持你。为此,我说谎了。”



“是啊,我也说谎了。”祥子叹了口气,看向她的眼睛,“但是即便如此,我也想要去了解真正的初华。我还想和初华继续前进。”她露出一个恶作剧的笑容,“那么,请多指教了,三角初华小姐。”



“……请多指教,丰川祥子小姐。”



I walk down the lane

With a happy refrain

Just singin', singin' in the rain

Dancing in the rain, da-da-dada



“小祥,我……”



然后,一切被雨声淹没。



\newpage



I'm happy again

I'm singing and dancing in the rain

I'm dancing and singing in the rain




\end{document}
