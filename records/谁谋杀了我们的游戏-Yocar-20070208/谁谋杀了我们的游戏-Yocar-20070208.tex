\documentclass{article}
\usepackage[utf8]{inputenc}
\usepackage{ctex}
\usepackage{url}

% support Chinese Chars %
\newfontfamily\urlfontfamily{FandolSong-Regular}
\def\UrlFont{\urlfontfamily}

\title{谁谋杀了我们的游戏}
\author{Yocar}
\date{20070208}

\maketitle
\url{https://youxiputao.com/article/26721}
\newpage

\begin{document}
\CJKfamily{zhkai}

% : comm which is only poems edit in-use %


\Large



{\centering\section*{01:引文}}





失败!继续下一个失败!



就在此时,大量的游戏研发团队正在走向失败,闭上眼睛想象一下这个画面,然后睁开眼睛看看周围。它正发生在你的身边吗?



在2006年,国内有超过60款自主研发的网络游戏面世,最后活下来的并能够盈利的,不超过15款。不用怀疑这一点:超过75%的项目都直接失败了或远未达到预期!在竞争日益激烈的中国网游市场中,这个比例甚至还在提高。



大盘新出炉还冒着热气的游戏,等不到大规模宣传,就在各种级别的内部测试中无声无息地被倒进了泔水桶。剩下的呢?你试图回忆上个月在17173上大张旗鼓宣布公测万人空巷的国产巨作叫什么名字,却怎么也记不起来。昙花一现,用来形容今天的大部分新网游,毫不过分。



到底出了什么错?



是谁谋杀了我们的游戏?



为什么抬眼四望,到处只见血淋淋的成品?



四年前“一边睡觉一边印钞”的黄金产业消失了?



可爱的玩家怎么突然都变得这么始乱终弃面目狰狞?



本文并非要分析外部环境,市场竞争,文化积累,用户心理,游戏内容或者各种人品问题,而只是想从游戏开发者内部一个小小方面——也就是我所从事的游戏策划的角度,来看待游戏自主研发的失败原因。



本文不是有关如何做好一个策划的指南,只是较为感性地表达作者个人的观点,并列举与此相关的实际教训,希望对那些还没有彻底失败的项目有所警示。





{\centering\section*{02:死婴}}





十月怀胎,胎死腹中。



这样的说法有点残忍,但如果你也在某个最后不得不以解散告终的游戏项目组里呆过,应该明白这个比喻恰如其分。



不少游戏都呈现出这种惊人的相似性:研发一年左右,然后在大部分玩家见到之前迅速挂掉。



这和投资商的急功近利有关,也可以和决策失误,混乱的管理,市场口味变换,团队经验缺乏,或印尼海啸都扯上一点关系,可是我们无法确定最根本的原因:是精子活力不强么?是怀的时间不够长么?是营养跟不上么?是负责剖腹产手术的医生水平低劣么?为什么我们总是不能顺利生产出健康的宝宝(产品)呢?



成功的游戏往往被人忽视的一点是:它们之所以活下来,并不是因为它们比多数死掉的同胞消耗了更多的资源,花费了更久的时间。



人们总是喜欢不自觉地夸大游戏(或者其他产品)研发过程中“精英化”“勤奋”和“拖延”的成分,他们举出很多例子,如“程序们彻夜讨论物理引擎的改良”“美术使用256色调色板反复尝试终于调出逼真的尾焰”“策划否决了近30种不同的BOSS方案”“经理拼命说服董事会再次推迟发行一年”等等来证实这些观点。



事实真的是这样吗?



不可否认,工作态度和外部环境都对最后的成功起着关键性的作用,但如果大家的注意力都放在慢工出细活,十年磨一剑的少数伟大作品上时,就会形成了一种回避,一种对自身进行全方位检讨的回避。



这样做的结果,是导致出现了一个如何做好游戏的“大方法论”,把做游戏看成命题作文,公司里所有有关如何做好的讨论,都演变成如何更接近这个“大方法论”的研讨会。在我看来,这种对成功者千篇一律的单一诠释,实在是居心叵测,它一定程度抵消了对游戏制作者自身根本素质的质疑,而没有质疑,更不会有反省和改善。



回到现实,看看市场上那些正在大赚其钱的那些游戏都是抄袭哪个统一模版生产出来的。《街头篮球》是一个很好的例子,这个游戏是在一年内创造的杰作,一个没有太多经验的团队(他们的第一款作品甚至失败了)仓卒中完成了它,后来它成了市场占有率最高的体育类网络游戏。这足以提醒我们去寻找一些和“杰出领袖+精英齐聚+多年心血”不一样的东西,一些有关游戏人本身相关的东西。



好吧,让我们回忆一下,在最后令人悲痛万分的难产发生之前,策划都干过些什么事情。





{\centering\section*{03:被诅咒的团队}}





我总在怀疑,是不是一部分失败的项目,一开始就输在了起跑线上?如果把这个问题集中到游戏策划一个点上,不难发现,有些东西,从一个游戏的酝酿之初就在某些人脑中萌芽了,可以把它描述为一种越来越强烈心理暗示—— “我是如此深刻地洞察了它的固有缺陷,我已经预感到了它最后的必然失败。”



一旦你们的策划也这样想,很不幸,这个团队被这无敌的谶言给诅咒了。



请再次展开你的记忆,这次的场景里包含很多个屏幕——没错,是闲暇时同事们五颜六色的显示器,在你脑中应该很快可以浮现出一些画面:重叠的聊天窗口,18禁的图片,新的美剧,Blizzard或者Valve的陈年旧作……



你看到项目组正在测试的那款游戏了么?没有。



下次再留心一下,如果情况真的是这样,这便是被诅咒后最典型的表现。



为什么不玩你自己的游戏?



“为什么不玩玩咱们自己的游戏?”



挨个去问你的同事这个问题。大部分人可能会嘲笑你,不屑于回答;也可能有人老实一点告诉你:“有什么好玩的,天天工作就是干这个,还不腻么?”如果这话是出自程序和美术之口,你可以无视,他们即使不怎么爱玩自己的游戏,也一样可以把本职工作做好,只是不够卓越而已。但如果类似的回答来自项目的策划,创意总监,策划主管之类的人,那么很不幸,也许最糟糕的情况发生了:



项目的策划,尤其是主策划不热衷玩自己的游戏,是游戏研发中极端危险的征兆。



这个说法有老生常谈之嫌,很像“你做的游戏自己都不喜欢,更不用指望别人”的翻版,不过对于策划而言,我认为这个经验值得被经常提起。因为更常见的情况是,大家早对此视而不见了。



我之前所在的公司曾开发过一个有趣的小型消除类游戏。几乎所有人,不只是开发人员,都在这个游戏测试期成了它的忠实玩家,我们在下班后兴致勃勃地组队进行比赛,为游戏中的胜利和新的等级头衔得意洋洋。



我当时负责它的音效部分,为此我制作了3套不同的音效换着体验,包括一套黑人RAP风格的音效,仅仅是自己为了好玩。这个游戏正式上线后,它的在线人数很快超过了我们的预期。



请记住两件事情:



第一,热衷不是喜欢。没有任何人会强迫你喜欢某样东西,总有一些人不喜欢自己的想法变成现实。问题是,如果策划连为自己游戏投入大量业余时间的耐性都没有的话(这就叫热衷),他如何能发现这个游戏真正的可玩性所在呢?他如何理解那些为这个游戏乐此不疲的玩家呢?他如何知道下一步该做什么才会让用户满意呢?



第二,一个好的游戏是任何时候都值得玩的。如果如果策划用“玩伤了”作为理由来搪塞,他等于是在说,我已经放弃了,我看不到它还有任何可以让我兴奋的地方(虽然我没怎么玩它),天哪!别让我继续体验这个烂东西了,我天天都在伤脑筋,难道还不比你清楚?它没有办法改进了,一点也没有!



认真想一想,不觉得崩溃么?



当项目组最核心的成员,那些专职负责思考游戏性,不断发掘游戏新乐趣,制定游戏未来开发方向的“先知”们,在项目还根本没有失败的时候,竟似乎获得了神启,看到了未来自己作品的惨淡收场,然后为此感到无可救药的沮丧,决定不再碰自己的游戏,不仅不碰,甚至开始憎恨它为自己带来了那么多的挫败感。



这好比士兵们还对战争的前景抱有希望的时候,指挥官们竟在悄悄准备着缴械投降,还有什么这更糟的吗?在一次对组内成员玩自己游戏情况的观察中,我注意到策划的游戏平均等级并不比程序和美术更高。得分最多的一位策划拥有的角色经验值,击杀数和游戏局数,均不到得分最多的一位客户端程序的1/5。



而美术里得分最多的那个人,他的游戏分数大约相当于所有策划的分数之和。



另一个可怕的事实是,多数策划以为这种情绪不会被人察觉——没错,看上去他们一直在加班,对每项工作细致跟进,跑来跑去积极进行多方沟通,和其他人开着毫无幽默感的玩笑。可是真相往往就是这么简单和难堪:



他们真的,真的很少在玩自己的游戏。



所以不用再遮遮掩掩了,当策划对自己所在的游戏项目流露出悲观时,这种态度就会像春雨一样潜入所有人的心田,迅速在整个团队蔓延。哪怕是最迟钝的成员,也会很快被这种情绪感染,然后你将看到前文所说的情况:越来越少的同事在玩这个游戏了。



消极的策划对团队的破坏力就是这样巨大,他们对团队带来的绝望是如此深刻。



他们本该是一群最有动力,热情和主动性的人,但现在最主动的工作却变成环境所迫的被动行为;他们本是推动变革的领袖,他们却失去了最起码的胆量;他们不再愿意去探索这个游戏还有什么可玩之处,他们惧怕任何大的改变;他们没有野心,没有信心,更不用提为“我们的宝宝”规划远景和绘制蓝图;他们会把每个阶段的失败推诿到各种“正确的原因”上,但绝不会提及自己是如何毁灭了一个游戏开发团队的精神根基——我们在做好玩的游戏。



如果你留意看过他们的眼睛,也许就明白了这一切,那是一潭死水,里面根本看不到理想主义的灼灼火焰。



离策划越近,离玩家越远。





{\centering\section*{04:要不要把玩家当牲口看?}}





别被吓着,在国内(网络游戏)策划圈内,类似这种主题的讨论司空见惯,毫不夸张。如果硬要换个温和点的说法,可以描述为:



狗日的网络游戏产业,催生出一帮像我这样的狗东西,天天琢磨下面五个命题:



1.如何让玩家一直沉迷 



2.如何让玩家吐出更多的人民币 



3.如何让玩家拉帮结伙 



4.如何让玩家相互仇视 



5.如何实现隐性的现金赌博和金币交易



请相信我,几乎所有做网络游戏研发的公司,都会要求策划设计了大量的功能模块来实现以上5点。衡量一个策划,尤其是数值策划是否优秀的标准,便是看以上几点在游戏实际运营中贯彻的是否彻底,当然,不同类型的游戏会有不同侧重点。



结果,在相当数量的游戏研发团队里,策划的工作焦点,不是研究如何让游戏更好玩更丰富,而是研究如何让玩家成瘾,让他们习惯党同伐异,谩骂虐杀,以及进行更安全的在线现金活动(赌博,虚拟物品交易等)。



顺理成章的,网络游戏涌现出了很多独有的东西:先是源源不断的新地图/新怪物/新等级/新装备,然后是这个转生,那个飞升;接着是双倍经验,家族系统,小喇叭,PK榜,踢人权,防踢权;还有抽奖卡,金币区,10倍金币区,50倍金币区……相比那些过时的,传统的单机元素,这些新玩意儿在经济效益上获得了明显的,甚至是空前的成功。



于是我们弹冠相庆,为自己的创造力喝彩,为探索出了一条有中国特色的网络游戏事业兴旺之路手舞足蹈。



这真是网络游戏研发界最奇特的现象:我们成了终日分析某个级数通项是否合理,不停做曲线积分解微分方程的数学家;我们成了研究如何提高患者药物依赖程度,不断改进提纯工艺的职业医师;我们成了鼓励人们无视现实规则,恣意发泄个人情绪,激化各种矛盾的职业鼓动家和武器提供商;我们成了地下赌场的庄家和各种黑市交易的中间人。



我们成了,富有经验的游戏策划。



我上一个项目的主策划是一个对数值异常执着的人。他擅长对和游戏相关的,所有包含数字的地方进行再改造,包括游戏分数,动力参数,一次活动的奖励比例等等。他总是能够敏锐发现每一个不当之处,然后重新编制全新的神奇公式改善这些不当,并为此花费大量的时间进行测试和完善。可是每当他醉心于这些所谓“平衡”和“合理”的时候,玩家却因为游戏内容的贫乏和玩法的单调悄悄流失掉了。





{\centering\section*{05:插曲:从反沉迷说起}}





中国要出《网络游戏防沉迷系统》,这不是什么新鲜事儿。但为什么日本没有,韩国没有,欧洲没有,连网络普及率最高的美国,也只使用了游戏分级制度,而不是强制限时这么死板的做法来控制受众面?



为什么只有中国,会出台这种看似严重伤害新兴市场的法规?



定有人要跳出来了:“中国的政策制定一向这么粗暴。”



我只能说你too naive,too simple,too young!



是因为国情决定了一切。只有中国,具备了如此大量的“失意人群”,在依靠市场本身已经无法做出正确调控时,国家有必要使用行政手段拨乱反正。



何谓失意人群?



我的定义是,在现实中无法获得足够的成就感,在现行教育体制下彷徨无措,在激烈的社会竞争中感到不安和失落的人群。



失意人群的特性,决定了他们是网络天生的最佳用户。在这个本身就人口极度过剩,社会正处于转型期的国度,这一人群的数量之巨,直接导致了中国在短短几年内间便成为了第一网民大国和第一网游大国。



回忆一下网吧如何一夜之间大街小巷梨花开,网瘾如何成为人人皆知的社会公害,便不难理解这一点。



何谓市场本身无法正确调控?一位英国的经济学家说过一段我们都耳熟能详的话:



“资本害怕没有利润或利润太少,就像自然界害怕真空一样。一旦有适当的利润,资本就胆大起来。如果有10%的利润,它就保证到处被使用;有20%的利润,它就活跃起来;有50%的利润,它就铤而走险;有100%的利润,它就敢践踏一切人间法律;有300%的利润,它就敢犯任何罪行,甚至冒绞首的危险。如果动乱和纷争能带来利润,它就敢鼓励动乱和战争。”



网游游戏运营商的背后是什么?是资本。



网络游戏的本质的是什么?是虚拟的存在感和成就感。



中国庞大的失意人群在资本眼里是什么?是最好最鲜美的待宰羊群;是世界任何地方都寻不到的超级金矿;是完美的,未开垦的,最肥沃的处女地。



现在我们可以把经典语录稍微改一改了:



“网络游戏运营商害怕没有利润或利润太少,就像他们的服务器害怕机房断电一样。一旦有适当的利润,网络游戏运营商就忘记了游戏的原罪。如果有10%的利润,它就保证到处去宣传;有20%的利润,它就开始谎称自己的良善和玩网络游戏的诸多益处;有50%的利润,它就铤而走险,玩弄人性的弱点,只为让用户沉迷自己的产品;有100%的利润,它就敢制作任何违法的内容,践踏一切现实的规则,哪怕民怨滔天;有300%的利润,它就敢于策动玩家做出最变态最疯狂的事情,甚至冒取缔的危险。如果一代人的垮掉,能带来利润,它就敢鼓励他们垮掉。”



试问,这样的国情之下,如何指望这只“看不见的手”实行有效的调控呢?



如果国家还不出手阻止,它会丧心病狂到怎样的地步呢?



请记住,在资本眼里,是永远不会看到那些哭喊的父母和猝死的玩家的。尸体是它的佳肴,眼泪是它的佐料,它以此为生,乐此不疲。



所以,不要听说几家大网游公司在北京鼓捣了个《北京宣言》,说坚决支持反沉迷,不会影响网游盈收云云就真的相信了,那是典型中国特色“政府搭台,企业唱戏”的闹剧。一旦《网络游戏防沉迷系统》明天就实施,我保证几个老总如丧考妣午夜泪奔:)





{\centering\section*{06:原力的黑暗面}}





《星球大战》里,原力是生命所能掌握的宇宙间最强大的能量,它分为两面,光明面和黑暗面,就像光与影。



光明面诞生了绝地武士,黑暗面则造就了黑武士。绝地武士用自己的力量守卫公义和所有生命的平等权益,而黑武士只为满足自己的欲望不择手段。



如果我们把原力比喻为今天的网络游戏。把原力的光明面,看作是玩家从游戏中获得的健康乐趣;而把原力的黑暗面,理解为游戏运营商背后资本的无边贪婪。那么我们游戏策划,就像年轻的阿纳金?天行者,他的原力是如此强大——他若信念坚定,世间的原力便能保持平衡和稳定;他若堕落,则整个银河将陷入万劫不复。



为了了解你们项目组的策划是否已经走入原力的黑暗面,请马上去问他这样一个问题:“你做过的所有工作中,真正能改善游戏性的部分,和只考虑赚钱而与游戏乐趣无关的部分,各占了多少?”



一个狡猾的策划会反过来教育你说,任何和游戏性看上去无关的工作,其实都会一定程度增加玩家的乐趣。



很遗憾,他已经被黑暗原力侵蚀太深了。永远不要轻信这样的谎言,就如同网络游戏运营商永远不会承认“你越宅越废材,我越高兴越HIGH”一样。



回到前面的话题,通过对整个产业一次浅显的批判,应该可以初步解释为什么策划会离玩家越来越远。最根本的原因,乃是资本异化了网络游戏的制作初衷,网络游戏被首先定位成一项能够持续赚钱的服务性业务。全部的工作,被要求围绕 “持续盈利”和“让用户在里面呆上成百上千个小时”而展开。而原来的初衷仅仅是“创造有趣的东西”。



这不是为策划开脱,导致网络游戏成为众矢之的更深层面原因,的确是来自资本的黑暗力量。



我只是想提醒诸位另一种危险:在这样强大的黑暗原力的感召下,势单力薄的策划们,开始呈现整体堕落的趋势。我们正在逐渐形成一种新的游戏策划指导思想,它的核心不是关于如何制作出“有趣的,让玩家获得快乐的”游戏,而演变为如何设计出一个成功的互联网圈套。



更严重的是,在国内有相当数量的优秀策划已经站在了这一黑暗面,为此推波助澜,他们不断地,卖力地补充着大量来自实践的经验,并运用心理学和统计学的知识为将其升华为各种定律和理论。



哪些特征可以表现出这种堕落的趋势?请对照你的项目组看看是否符合下面的八条:



游戏原始模型的创新被压缩到几乎为零;



策划很少做前瞻性的思考,他们更多在做的是类比、修饰和抄袭;



网络游戏作为单机游戏的成分,如人物情感,世界观,任务剧情,音乐音效的完成度要求显著降低;



玩家被当成数学模型,在所有决策中,个体玩家的感受可以被完全忽略;



策划普遍具备了凌驾玩家之上的心态,他们对热爱自己游戏的“上帝”毫无虔诚可言;



如果不是工作要求,策划大都不愿意和玩家做主动的,直接的,频繁的交流,更不愿意他们干扰到自己的私人时间;



资深(数值)策划的衡量标准是设计出能够强力成瘾的系统,他们以此为荣;



头头们经常说的话是“我只关心它能不能为我赚到钱”。



请务必重视这个预测。



不仅因为资本的贪婪本性,也因为我国恰好正处在这样一个缺乏单机游戏文化沉淀,网络游戏市场独大的畸形生态中。在这样的大环境下,任何急功近利的火花都比其他时候更容易燎原!



也许真的会有一天——那时的网络游戏,不再是游戏,它们是华丽画面和千锤百炼的数值体系组成的阴谋。那时的网络游戏玩家,不再是传统意义上的游戏玩家,他们和药物依赖者别无二致。欢迎来到NHK!



2006年上半年,我接手了一项旨在使得游戏内玩家追求多样化的工作,为此我考察了多个单机游戏,分析它们是如何提高重玩率的。



在考察中,我对如特殊荣誉,额外奖励之类的设定做了大量的记录。最后,我为我们的游戏写出了一份繁复冗长的特别勋章表。



这项工作带来的一个后果是,我看待单机游戏的眼光不知不觉发生了改变,以后每接触一个新游戏,我总是能先敏锐地找到它们做了哪些事情,是让玩家在通关后还可以继续玩更长的时间。



可我一定是忘记了,那些东西只是点缀和彩蛋,并不是它们流行的原因。



从资本发源,再到策划合谋,一个网络游戏这时才可以说味道彻底变坏了。它不再是为带给玩家快乐而制造的冰淇淋,它是一颗量身定做精确制导的糖衣炮弹,它的使命纯粹而残忍——榨干他们所有的钱,哪怕摧毁他们的意志和身体。



本文尽管着重于探讨策划为游戏失败所付的责任,但在这里我不得不悲观地指出:虽然大部分“纯利润”导向的网络游戏被迅速揭穿并鞭尸;虽然有思考能力的成熟玩家和有责任感的媒体都敏锐地察觉到一危险;虽然现在的市场上不把玩家当人看的游戏,生存的可能性已经微乎其微。



但是。那些自以为高明的,老练的,邪恶的游戏策划们并没有因此醒悟,他们仍然在悄悄酝酿着更多的阴谋。



他们是一群内心真正信奉“网络游戏赚钱只能靠沉迷”“做网游就是做电子鸦片”的黑武士,他们是资本鼎力推崇的业内精英,他们身经百战意志坚强,他们占了大多数。连我自己,也不过是其中一个尚未被黑暗吞噬全部良心的小辈。



2006年年初,当时《征途》刚刚崭露头角,有人在公司发起了一个内部讨论群,策划们就《征途》的收费装备,代练人偶等一系列“歪门邪道”的做法,和到底史玉柱做游戏是不是赚了大钱进行了激烈的辩论。开始还有各种不同的声音,但到了最后,讨论的主题变成了“我们是不是也要像征途那般黑”。我亲爱的玩家,我听见了背后你们凄厉的哀嚎,可是资本总在前方对我妩媚微笑啊。





{\centering\section*{07:运营策划的尴尬}}





“你们策划都是吃屎的?白痴才会去参加这种SB活动!”



听到这样的话,你或许会感到委屈,官网上的活动日历不是排的满满的么?今天攻城,明天抽奖,周末是双倍经验,下周还有“XX天使”评选决赛阶段正式开启投票。可那帮不知足的家伙在论坛上一点情面都不给,总是无休止的抱怨现在的活动千篇一律毫无新意。



《网络游戏开发》一针见血地指出,网络游戏的一半是服务。



这服务,落实到策划头上,基本就相当于运营策划的工作。千万不要轻视他们对游戏成败的影响力,如果说前期的游戏策划决定了什么人会来玩,那么后期的运营策划则决定着有多少玩家会留下。



浏览各大网游公司的招聘广告,不难发现,运营策划的要求明显较其他策划为低。运营策划的要求通常是“文笔优美,能够承受压力,吃苦耐劳,有一款以上的网游经验”,游戏策划的要求则是“熟悉历史;精通奇幻文学,AD&D体系;深刻了解市场同类产品;擅长写作和表达”。



简直一个是体力劳动,一个脑力劳动!



造成这种差别最初的原因,也许是来自人类潜意识里的某种劣根性——我们固执地认定原始的思想者和创造者,胜过在此基础上继续发展的生产者和经营者。在单机时代,团队里没有运营策划的概念,从那时起我们慢慢养成了一个习惯,认为一个游戏的成功,伟大的游戏策划功不可没,但从不提及优秀的运营策划。



对于网络游戏这个新生事物,也能够这么简单的理解吗?



我刚进入游戏公司时,为一个正在运营的MMORPG做运营策划,我获得的第一个任务,是“三个月内写一个活动”。



后来,这个要求逐渐变为“小活动不见断,大活动每月一个”,于是我不得不制作了一些模版,来应付如此多的需求。



我从没问过,为什么要写这么多活动。也从没人问过我,你对游戏的下个版本有什么意见。至少我所了解现状是,运营策划的任务,在大部分时候可以被简单的表述为“不要让玩家闲着”。



在多数情况下,策划主管不会要求手下的运营策划对每个活动提案涉及的敏感人群,投入产出,可能风险,长远影响等因素进行预测和分析;也不会对已经结束的活动做效果总结,得失记录和横向比较。



长此以往,因为缺乏有效的参照物和系统的标准,任何一个活动究竟上不上,只取决于决策者感性上的可行或不可行。



运营策划,本该是最了解玩家需求的一群策划,现实却是他们和那些真正能够改善游戏性的,高高在上的“核心策划”如隔参商。他们也的确曾在恶劣条件下了办出了让玩家叫好的活动,但因为长期的意见被忽视加之繁多而枯燥的需求,更多的活动变得模式化严重,草率和不负责任。



对游戏内某个技能的伤害数值一丝不苟,对某次活动明显的不严谨不公平置若罔闻。这种在策划上重设计轻运营的思想,这种对运营策划的“非策划级”的要求标准,对网络游戏,尤其是一个已经运营拥有一定数量玩家群的网络游戏而言,无疑是潜伏的定时炸弹。



不能免俗,我还是试图找出了一些不成熟的,感性的运营策划经验,仅供参考。



哪些活动事后让玩家怨声载道?



1.要求玩家不断砸钱的活动



2.容易导致作弊,刷分的活动



3.黑箱操作决定奖品最终归属的活动



4.难于报名,过程繁琐的活动



5.过于简单粗糙的赠送类活动



6.单调,重复,形式长期不变的活动



7.不能给与全部玩家公平待遇的活动



8.易于引起玩家间矛盾的活动



哪些活动容易受到玩家欢迎?



1.免费,方便,轻松参加的活动



2.体现游戏技术含量的活动



3.提倡玩家团队合作的活动



4.提供超级特殊奖励的活动



5.提供全新游戏内容的活动



6.鼓励玩家相互交流的活动



7.系统自动刷新获奖结果的活动



8.玩家参与构建游戏世界的活动



9.配合现实节日主题的活动



10.丰富和多样化的任务



11.紧扣游戏新版本的活动



12.针对玩家热点的活动



13.向游戏内恶意行为宣战的活动



14.与游戏主要目标大相径庭的活动(如小游戏,答题等)



15.两性主题的活动



哪些活动应该谨慎举办?



1.开支庞大的线下比赛活动



2.各种不伦不类的赞助活动



3.需投入大量人力监督的活动



4.得不到足够重视的调研活动



5.事前准备不充分的包机活动



6.和社会公益结合的慈善活动



7.公开选拔玩家明星的选秀活动





{\centering\section*{08:何苦做策划}}





说了这么多策划的坏话,最后想为策划鸣冤几声,决没有要翻案的意思。



先点出一个事实:即使是在我们一向认为重视创意的欧美,程序员/程序主管的收入仍然比同级别策划/主策划多出将近30%,美术则大概多出5%~10%。这一点和广告行业可能不同,在游戏业,产品必须经过程序才能真正产生。



道理很简单:程序需要创意,也需要专业技术。美术需要创意,也需要专业技术。策划需要创意,然后需要敏锐的嗅觉,丰富的经验,良好的表达技巧,但没人把这些当作专业。



如果缺程序,产品=0。如果缺美术,产品必定惨不忍睹。如果缺策划,产品仍可顺利诞生。程序对易用性的理解未必比策划差。美术对爽快感的表现未必比策划差。策划说,我对游戏性有深刻理解,然后程序和美术都笑了。



我承认,伟大的游戏总是来自伟大的思想,伟大的思想通常来自伟大的游戏设计师。但技术决定游戏的时代并没有完全过去,E3展每年层出不穷的FPS和新硬件技术就很能说明问题。在游戏性已经积累了厚厚的基本规则后,技术高低仍是影响游戏是否热卖的决定性因素。



何况,没人相信你能设计出“伟大的游戏”。



所以事实上,程序美术策划重要程度的差别是如此显而易见无处不在。也许只有日本的“游戏制作人”是个例外,但宫本茂们有超过20年的游戏制作经验,他们是整个游戏行业的启蒙者。亦只有这种积累,才让他们获得了超越一般分工的更高存在。



说到这里,还是回到本节标题——如果你总是积极地写文档和设计数值,其他人会认为你很合格而且尊重你,但决不会认为因此你可以拿到和他们相同甚至更高的薪水。如果你能7天做一个原型,能提出解决当前疑难的更高效算法,能直接设计并拿出某个UI的效果图,其他人会尊重并佩服你。



在不长的工作经历中,我基本可以证明以上的话是真实的。



别忘了,咱们是策划,老板还巴望着咱们在创意/经验/文案/眼光/兴趣/沟通/外语等等方面超过其他人呢。



这么看来,尽管很多人觉得策划好混,但就前途发展而言,的确是不适合很多人的职业呢。



突然想起,CSI第一季里,老大Gil说,你的悲哀就是把它当成了工作。



路还很长,风还很邪,妖气正冲天。

\end{document}
